
\section{Related Work and Discussion}\label{sec:discussion}

\subsection{Language-parametric verification}

The idea of a Circularity rule is already present 
in the existing literature on reachability logics \cite{RosuS12oopsla,RosuSCM13lics,StefanescuCMMSR19},
where it was used to generalize the notion of \emph{loop invariant}.
However, RL can not be used to reason about $k$-safety hyperproperties and does not perform lockstep execution.
The literature on language-independent equivalence checking also contains the idea of a Circularity rule.
In~\cite{CiobacaLRR16}, circularities are used to synchronize executions of different programs (possibly in different languages);
however, the proof system of~\cite{CiobacaLRR16} can be used only to reason about full program equivalence
and not about $k$-safety hyperproperties.
What makes the proof system of~\cite{CiobacaLRR16} really unusable in our situation is the fact that its
Circularity rule
\begin{align*}
  & \prftree[l]{}
    { \mathcal{S} \vDash \varphi_1 \Rightarrow^+ \varphi^\prime_1 }
    { \mathcal{S} \vDash \varphi_2 \Rightarrow^+ \varphi^\prime_2 }
    { \mathcal{S} \vdash_{\mathit{equiv}} \langle \varphi^\prime_1, \varphi^\prime_2 \rangle \Downarrow^{\infty} E}
    { \mathcal{S} \vdash_{\mathit{equiv}} \langle \varphi_1, \varphi_2 \rangle \Downarrow^{\infty} E}
\end{align*}
requires all the components to make progress (denoted by the ``plus'' sign on top of the arrows) before the circularity can be used,
which would prohibit RL-style reasoning when one component has already finished execution.
%The need for such reasoning is there even in our simple running example, as we illustrate in ???.

We believe that all the mentioned proof systems that use a variant of the Circularity rule are instances of a more general
framework, known as \emph{Circular coinduction} in the literature~\cite{RosuL09CircularCoinduction}.



\subsection{Relation to Cartesian Hoare logic}

\subsubsection{(Non)determinism}
We base our work on the \emph{one-path} variant of reachability logic.
Consequently, CRL inherits a known limitation of one-path reachability logic: that the tight correspondence between
one-path RL and Hoare logics is limited to deterministic languages.
This is because the CRL semantics we existentially quantify
over reachable configurations, while in CHL, target states are quantified
universally. %However, this distinction has no effect when working with
%deterministic languages
Despite that, we can prove the following theorem (the proof is technical and can be found in \Cref{app:reltoCHL}).
\begin{theorem}\label{thm:chlCRLrelation}
  CRL extends CHL on the deterministic fragment of the CHL-supported language.
  That is, assume a sound reachability-logic formalization (that is, a reachability system) $\mathcal{S}_{\mathit{IMP}}$ of the CHL's imperative language.
  Then there exist translation functions $\mathit{tr}$ and $\mathit{end}$ such that
  given any statement $P$ in the deterministic fragment of the CHL's imperative language
  and any first order formulas $\varphi, \psi$,
  \begin{equation*}
    \vDash_{\mathit{CHL}} ||\varphi||\ P\ ||\psi||
    \iff
    \mathcal{S}_{\mathit{IMP}} \vDash_\CRL \mathit{tr}(P, \varphi) \Rightarrow^{c\exists} \mathit{end}(P, \psi) \, .
  \end{equation*}
\end{theorem}


\subsubsection{Similarities}

Our understanding of the inner workings of CHL is based on the extended, unpublished version (\cite{SousaD16Extended})
of \cite{SousaD16}.
There, the authors define a \emph{linearization operation} on lists of programs, which roughly corresponds to our
\emph{star extension} of the language's semantics.
Then, the authors prove lemmas saying that a CHL triple with a list of programs inside is derivable
in the CHL proof system
if and only if 
a Hoare triple having the same list of programs but linearized inside, is derivable;
the "only if" implication corresponds to our \Cref{lem:CRLalmostSoundness}, where we construct
an RL proof from a CRL proof,
while the "if" implication corresponds to our Reduce rule.
Furthermore, in the proof of soundness of CHL, the authors assume soundness of the self-composition technique;
self-composition corresponds to our \emph{star extension} and its soundness to our \Cref{thm:CRLandRLcorrespondence}.
Finally, \cite{SousaD16Extended} assumes soundness and relative completeness of the underling Hoare logic;
similarly, we assume soundness and relative completeness of reachability logic.
However, for completeness, we had to prove that the \emph{star extension} preserves decidability of validity.

\subsubsection{Differences}
There are also differences between the CHL and CRL techniques.
First, our proof system has only 8 rules, and they do not mention any programming language construct,
while CHL has 17 rules (not counting the Expand rule), half of which are specific to the underlying language.

Second, there is a redundancy between the language-specific CHL rules and the Hoare logic rules of the
programming language: for example, the conditional statement ("if") has (1) a rule in the formal semantics of the language,
(2) a rule in the Hoare logic (not shown in the paper), and (3) a rule in CHL.
When considering that the three rules have to play nice together (that is, CHL and Hoare logic rules have to be sound with respect to the semantics), someone had to think about the conditional statement at least five times.
We consider this to be highly uneconomical
- and the situation
is even worse for the looping construct ("while-with-breaks"), which is supported using additional CHL rules.
In contrast, in the CRL/RL framework, it is enough to design each language construct once - when giving
its operational semantics.


Third, in CHL the support for lockstep reasoning is hard-wired into the rules for loops,
while in our framework, lockstep reasoning is not limited to loops, but can support arbitrary sources
of circular behavior - including loops, recursion, gotos.
%Therefore, while we are inspired by the \emph{idea} of lockstep execution from \cite{SousaD16},
%for the \emph{realization} of this idea we turn to the literature on \emph{equivalence checking}.

\subsection{Other Related Work}

In \cite{DOsualdoFD22}, the authors develop a \emph{logic for hyper-triple composition (LHC)}
that allows reasoning about $k-safety$ properties compositionally.
Similarly to CHL, LHC targets a particular small imperative language.
We believe that compositionality is orthogonal to language-parametricity,
and thus we would like to generalize their work to language-parametric settings in future work.

A game-based technique for verifying software hyperproperties beyond $k$-safety
has been developed in \cite{BeutnerF22}.
This technique works with \emph{symbolic transition systems},
so it already is language-independent \emph{in some sense}. However, it is not clear how to use the technique
with an arbitrary language $L$, without writing a \emph{compiler} from $L$ to symbolic transition systems first.
