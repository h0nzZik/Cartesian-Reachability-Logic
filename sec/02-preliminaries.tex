\section{Preliminaries}

%\subsection{$k$-safety hyperproperties}

\subsection{Cartesian Hoare logic}

Introduced in~\cite{SousaD16}, Cartesian Hoare logic is a formalism for specifying and reasoning about
a class of hyperproperties known as \emph{$k$-safety hyperproperties}, or simply
\emph{$k$-safety properties}.
A $k$-safety property is a hyperproperty whose violation can be witnessed by a $k$-tuple of execution traces.
Many \emph{security policies} - for example, noninterference (requiring that sensitive or privileged data
do not influence insensitive or unprivileged computations) - are $k$-safety hyperproperties.
Similarly, many functional correctness properties are actually $k$-safety hyperproperties;
for example, \emph{transitivity} (which needs to by satisfied e.g. by comparators when managing data
in collections), associativity (important in the map/reduce paradigm), or monotonicity.
Cartesian Hoare logic (CHL) can be used for reasoning about these on the level of the source code,
similarly to Hoare logic.

In a Hoare logic, one specifies safety properties by means of so-called \emph{Hoare triples}.
These have the shape $\{ \varphi \} S \{ \psi \}$, with the meaning that
the formula $\psi$ holds in any state after the termination (if any) of the program $S$,
executed from a state satisfying $\varphi$.
In Cartesian Hoare logic, the situation is similar: one can specify a triple
$\langle \Phi \rangle (S_1 * \cdots * S_k) \langle \Psi \rangle$
with the following meaning: for any $k$-tuple $(\sigma_1,\ldots,\sigma_k)$ of states
satisfying the formula $\Phi$, if we execute each program $S_i$ in the respective state $\sigma_i$ and they all terminate,
then the $k$-tuple $(\sigma_1^\prime,\ldots,\sigma_k^\prime)$ of resulting program states satisfies $\Psi$ \footnote{An
important technical assumption here is that every program $S_i$ operates on its own set of program variables,
distinct from variables of other programs $S_j$ (for $i \not = j$) - otherwise the formulas $\Phi$ and $\Psi$
would not be able to distinguish between program variables of different programs.}.
%To make specifications more compact, one can also consider triples $|| \Phi ||\ S \ || \Psi ||$ 

As an example, consider the program $P(x,y) \equiv \texttt{while(x > 0) \{ x--; y++; \} }$ and the 2-safety property
of \emph{monotonicity}; that is, intuitively: with growing input $x$, the resulting $y$ also grows.
In CHL, this can be formalized as
\begin{equation*}
\langle x_1 \leq x_2 \land y_1 = y_2  \rangle (P(x_1, y_1) * P(x_2, y_2)) \langle y_1 \leq y_2 \rangle \, .
\end{equation*}
The main idea here is that the formulas can \emph{relate} variables from different executions.

Cartesian Hoare logic is equipped with a proof system that allows one to prove validity of CHL triples.
The proof system contains rules like
\begin{align*}
    & \prftree[l]{If}
      { \prfStackPremises
        { \langle \Phi \land c \rangle (\texttt{B1; S}) * R \langle \Psi \rangle }
        { \langle \Phi \land \neg c \rangle (\texttt{B2; S}) * R \langle \Psi \rangle }
      }
      { \langle \Phi \rangle (\texttt{if (c) {B1} {B2} ; S}) * R \langle \Psi \rangle }
\end{align*}
that replicate standard Hoare-logic reasoning, and is sound and complete.
However, what is more interesting, the proof system allows one to perform \emph{lockstep reasoning},
even for loops. This is achieved by means of the rule
\begin{align*}
  & \prftree[l]{Fusion}
    { \prfStackPremises
    { \langle I \land c_1 \land c_2 \rangle (\texttt{B1} * \texttt{B2}) \langle I \rangle }
    { \{ I \land \neg c_1 \}\ \texttt{while (}c_2\texttt{) {B2}}\ \{ \Psi \} }
    { \{ I \land \neg c_2 \}\ \texttt{while (}c_1\texttt{) {B1}}\ \{ \Psi \} }
  }
  { \langle \Phi \rangle ((\texttt{while (}c_1\texttt{) {B1}}) * (\texttt{while (}c_2\texttt{) {B2}}))  \langle \Psi \rangle }
\end{align*}
(which we present here in a simplified form, for Cartesian claims of arity 2 only).
This rule assumes an invariant $I$, which is a \emph{relational invariant} - meaning that it can relate variables
from both execution.
The rule breaks reasoning about a pair of loops into three cases: the case where both loop conditions hold,
and the two cases where one of the conditions does not hold.
In the first case, one needs to prove that the relational invariant $I$ is preserved by execution of the two loop bodies,
assuming that the loop condition holds.
In the other two cases, one needs to handle the case when one of the loops finishes its execution.
In that situation, one can assume that the loop condition of the finished loop does not hold,
and has to prove that executing the other loop in a state satisfying the invariant results in a state
satisfying the postcondition. This can be done by means of standard Hoare logic reasoning.

For an example, consider the program
\begin{equation}\label{eqn:CounterProgram}
P(x,y) \equiv \texttt{while(}x\texttt{ > 0)\{ }y\texttt{++; }x\texttt{--;\}}
\end{equation}
and the property that $P$ is monotone with respect to the initial values of $x$ and $y$ and resulting value of $y$.
This can be formalized as the CHL triple
\begin{equation*}
\langle x_1 \leq x_2 \land y_1 \leq y_2  \rangle (P(x_1, y_1) * P(x_2, y_2)) \langle y_1 \leq y_2 \rangle
\end{equation*}
Intuitively, this hyperproperty holds because we can synchronize the two executions until the point when the first one terminates;
then, the second execution may continue for a while, making the difference betwen $y_2$ and $y_1$ even greater.
Formally, we can apply the Fusion rule, with the precondition being also the relational invariant.
This results in three subgoals:
\begin{equation*}
    \begin{aligned}
        & \langle x_1 \leq x_2 \land y_1 \leq y_2 \land x_1 > 0 \land x_2 > 0  \rangle\ ( (y_1\texttt{++} ; x_1\texttt{--}); * (y_2\texttt{++} ; x_2\texttt{--}))\ \langle x_1 \leq x_2 \land y_1 \leq y_2 \rangle \\
        & \{ x_1 \leq x_2 \land y_1 \leq y_2 \land \neg (x_1 > 0) \}\ P(x_2, y_2)\ \{ y_1 \leq y_2 \} \\
        & \{ x_1 \leq x_2 \land y_1 \leq y_2 \land \neg (x_2 > 0) \}\ P(x_1, y_1)\ \{ y_1 \leq y_2 \} \\
    \end{aligned}
\end{equation*}
The first one requires one to show that the body of the loop preserves the invariant,
the second (third) represent the cases when the first (second) execution terminated.
To prove the second subgoal, one needs to find a (non-relational) invariant of the (single) loop;
the third subgoal is trivial, as its precondition implies the negation of the loop condition.

This lockstep reasoning is a powerful tool, because the relational invariants it requires are often very simple.
To prove the above example without lockstep reasoning, one would need to find (non-relational) loop invariants
strong enough to summarize the whole loop.
However, lockstep-reasoning rules become more complicated as one adds other features into the language
- for example, the \texttt{break} statement.
It is not clear how to extend this approach to handle, e.g., recursion, \texttt{continue}, or \texttt{goto}.
We also observe that a single language feature (\texttt{while} loops) needed to be considered five times
in order for CHL to soundly support it: the semantics of \texttt{while} is present in the operational semantics of
the target language, in the Hoare logic for that language, in the Cartesian Hoare logic for that language,
and in the proofs of soundness of the logics.


Our aim in this paper is to make the above ideas available for any deterministic languge.
Therefore, in the following subsections we review some tools from recent literature
on language-independent program verification.

\subsection{($\mu$-free) Matching Logic}

We work with a variant of matching logic described in
\cite{StefanescuCMMSR19,RosuSCM13lics}.
This particular variant of matching logic is used for reasoning about static properties of program configurations.
There exist newer and more expressive variants of matching logic (\cite{MmL, MLexplained});
we used the older variant in order to be compatible with the literature on reachability logic which uses this variant.

Matching logic \emph{formula} (aka \emph{pattern}) is a first-order logic (FOL) formula which allows terms,
over a signature $\Sigma$, with variables, as nullary predicates.
A typical example of a matching logic formula is $\varphi_{\mathit{example}}$, defined as
\begin{equation}\label{eqn:exampleMLPattern}
\texttt{<k> x--; <k><st> x} \texttt{ |-> } X\texttt{ </st>} \land (X \texttt{ >Int } 1 = \mathit{true})
\end{equation}
which, when interpreted in a model of a particular programming language,
denotes the set of program configurations in which the code \texttt{x--} is to be executed
and in which the program variable $\texttt{x}$ has a value $X$ that is greater than $1$.
In this example, the part
\begin{equation*}
    \texttt{<k> x--; <k><st> x} \texttt{ |-> } X\texttt{ </st>}
\end{equation*}
is the term-as-predicate, with $X$ being the only free FOL variable.
The program variable $\texttt{x}$ is not a FOL variable, but a constant symbol from the signature of the programming language.
The subterm $\texttt{x} \texttt{ |-> } X\texttt{ </st>}$ says that the program variable $\texttt{x}$
has the value $X$, and the $X \texttt{ >Int } 1 = \mathit{true}$ part then says that the realization
of the function symbol $\_ \texttt{ >Int } \_$ returns the boolean value $\mathit{true}$ when given $X$ and $1$
as arguments.

The satisfaction relation $(M, \gamma, \rho) \vDash \varphi$ between a model $M$, a model element $\gamma \in M$,
an $M$-valuation $\rho$, and a pattern $\varphi$, is defined inductively on the structure of $\varphi$.
The definition is as in FOL; the main difference is the semantics of terms-as-predicates, which is defined by
\begin{equation*}
    (M, \gamma, \rho) \vDash t \iff \gamma = \rho(t) \text{ if t is a term}
\end{equation*}
(where $\rho(t)$ is the homomorphic extension of $\rho$ applied to the term $t$).
For example, we might have a matching logic model $M$ containing (concrete) program configurations
of a particular programming language.
One such configuration might be $\gamma_{\mathit{example}}$:
\begin{equation*}
    \texttt{<k> x--; <k><st> x} \texttt{ |-> } 3\texttt{ </st>} \, .
\end{equation*}
Then, we have that $(M, \gamma_{\mathit{example}}, \rho) \vDash \varphi_{\mathit{example}}$
for any valuation $\rho$ satisfying $\rho(X) = 3$, and we say that
$\varphi_{\mathit{example}}$ \emph{matches} $\gamma_{\mathit{example}}$ in $\rho$.


A pattern $\varphi$ is \emph{valid in $M$}, written $M \vDash \varphi$, iff $(M, \gamma, \rho) \vDash \varphi$
for every $\gamma$ and $\rho$.
We observe that validity of a structureless pattern (that is a pattern without terms-as-predicates) does not depend on the selected model element.
Also, validity of any pattern does not depend on those variables which the pattern does not mention.
A more formal treatment of matching logic is to be found in the Appendix.


\subsection{One-path Reachability Logic}
Reachability logic \cite{RosuS12oopsla, StefanescuCMMSR19} (RL) is a formalism for
defining formal semantics of programming languages,
and also for specifying and reasoning about partial correctness properties
of programs in those languages.
On the formal semantics side, a programming language is modelled as a \emph{reachability system}
$\mathcal{S} = (\mathcal{T}, S)$, where $\mathcal{T}$ is a $\Sigma$-algebra
and $S$ is a set of \emph{reachability rules} of the shape $\varphi \Rightarrow^\exists \varphi^\prime$,
where $\varphi$ and $\varphi^\prime$ are matching logic patterns over $\Sigma$.
For example, one can have a rule
\begin{equation}\label{eqn:ruleIfTrue}
    \begin{aligned}
    & \texttt{<k> if (} \mathit{true} \texttt{) }P_1\texttt{ else } P_2 \texttt{</k><st>} S \texttt{</st>} \\
    & \Rightarrow \texttt{<k> }P_1 \texttt{</k><st>} S \texttt{</st>}
    \end{aligned}
\end{equation}
saying that the \texttt{if} construct of the particular language takes the first branch ($P_1$)
whenever the condition is $\mathit{true}$.
(Typically, there would be another rules governing evaluation of the condition.)

The meaning of reachability rules is the following.
A reachability system $\mathcal{S} = (\mathcal{T}, S)$ (together with a $\Sigma$-sort $\mathit{Cfg}$)
induces
a \emph{transition system}
$(\Tcfg , \Rightarrow_{\mathcal{S}})$,
where $\gamma \Rightarrow_{\mathcal{S}} \gamma^\prime$
for $\gamma, \gamma^\prime \in \Tcfg$
iff there is some rule $\varphi \Rightarrow^\exists \varphi^\prime \in S$
and some valuation $\rho : \Var \to \mathcal{T}$ with $(\gamma, \rho) \vDash \varphi$
and $(\gamma^\prime , \rho) \vDash \varphi^\prime$.
The intuition is that when taking a transition in the resulting transition system,
some rule $\varphi \Rightarrow^\exists \varphi^\prime \in S$ is selected,
then the current configuration is pattern-matched against the rule's left-side pattern $\varphi$,
resulting in a valuation $\rho$ which is then used to instantiate the right-side $\varphi^\prime$ of the rule,
forming a new configuration.
For example, the rule in \Cref{eqn:ruleIfTrue} induces (among others) the transition
\begin{equation}\label{eqn:ruleIfTrue}
    \begin{aligned}
    & \texttt{<k> if (} \mathit{true} \texttt{) x++; else x--; </k><st>x} \texttt{ |-> } 3\texttt{</st>} \\
    & \Rightarrow_{\mathcal{S}} \texttt{<k> x++; </k><st>x} \texttt{ |-> } 3\texttt{</st>} \, .
    \end{aligned}
\end{equation}

On the partial correctness side, RL reuses the notion of a \emph{reachability rule}.
For example, one can specify that the program \texttt{while(x > 0) x--;}
can, if it terminates at all, reach a configuration
where the program variable \texttt{x} has a non-positive value
by means of the reachability rule
\begin{equation*}
    \begin{aligned}
        & \texttt{<k> while( x > 0 ) x--; </k> <st> x |-> } V \texttt{</st>} \\
        & \Rightarrow^\exists \exists V^\prime.\, \texttt{<k> . </k> x |-> } V^\prime \texttt{</st>} \land (V^\prime \texttt{ <=Int } 0 = \mathit{true})
    \end{aligned}
\end{equation*}
Assuming the language is deterministic, this is equivalent to saying that if the program terminates,
the resulting configuration will have non-positive value of \texttt{x}.
Formally, a rule of the shape $\varphi \Rightarrow^\exists \varphi^\prime$
is \emph{satisfied}
in a reachability system $\mathcal{S} = (\mathcal{T}, S)$,
written $\mathcal{S} \vDash_\RL \varphi \Rightarrow^\exists \varphi^\prime$,
iff for every $\gamma \in \Tcfg$
such that $\gamma$ terminates in $(\Tcfg, \Rightarrow_{\mathcal{S}})$
and for any valuation $\rho : \Var \to \mathcal{T}$
such that $(\gamma, \rho) \vDash \varphi$,
there exists some $\gamma^\prime \in \Tcfg$
such that
$\gamma \Rightarrow^{*}_{\mathcal{S}} \gamma^\prime$
and $(\gamma^\prime, \rho) \vDash \varphi^\prime$.


Reachability logic is equipped with a proof system which derives sequents of the shape
$A, C \vdash_\RL \varphi \Rightarrow^\exists \varphi^\prime$;
the proof system is sound and complete: a RL claim is satisfied in $\mathcal{S}$
iff $\mathcal{S}, \emptyset \vdash_\RL \varphi \Rightarrow^\exists \varphi^\prime$.
The set $C$, initially empty, contains so-called \emph{circularities},
which are claims postulated to hold but not justified yet.
Circularities, which correspond to the notion of \emph{loop invariants} of Hoare logic,
enable one to reason about repetetive behavior of programs.
The proof system contains a rule
\begin{align*}
    & \prftree[l]{Circularity}
      { A, C \cup \{ \varphi\Rightarrow^\exists \varphi^\prime \} \vdash_\RL \varphi \Rightarrow^\exists \varphi^\prime }
      { A, C \vdash_\RL \varphi \Rightarrow^\exists \varphi^\prime }
\end{align*}
which adds to current claim to the set of circularities.
When a progress is made (by means of other rules, essentially performing a symbolic execution),
the claim is moved from circularities to axioms and can be reused, similarly to the way one assumes a loop invariant
in order to prove it again.
We refer an interested reader to~\cite{RosuS12oopsla} for more details.


%For illustration purposes, we show one rule of the RL proof system:
%\begin{align*}
%    & \prftree[l]{Case Analysis}
%      %{ \prfStackPremises
%        { A, C \vdash_\RL \varphi_1 \Rightarrow^\exists \varphi }
%        { A, C \vdash_\RL \varphi_2 \Rightarrow^\exists \varphi }
%      %}
%      { A, C \vdash_\RL \varphi_1 \lor \varphi_2 \Rightarrow^\exists \varphi }
%\end{align*}


\begin{remark}\label{rem:shapeOfReachabilityRules}
For simplicity, we restrict the class of reachability systems we work with to those whose reachability rules
have the shape
\begin{equation*}
    \phi \land P \Rightarrow^\exists \phi^\prime \land P^\prime
\end{equation*}
where $\phi,\phi^\prime$ are terms-as-predicates, and $P,P^\prime$ contain no terms-as-predicate.
\end{remark}

\begin{remark}\label{rem:noEmptySteps}
We work only with $\epsilon$-free reachability systems.
A reachability system $(\mathcal{T}, S)$ is \emph{$\epsilon$-free}
iff for any two configurations $\sigma, \sigma^\prime \in \mathcal{T}_{\mathit{Cfg}}$, if
$\sigma \Rightarrow_{\mathcal{S}} \sigma^\prime$, then $\sigma \not = \sigma^\prime$;
\end{remark}