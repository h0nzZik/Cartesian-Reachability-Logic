\section{Preliminaries}

\subsection{($\mu$-free) Matching Logic}

We work with a variant of matching logic described in
\cite{StefanescuCMMSR19,RosuSCM13lics}.
This particular variant of matching logic is used for reasoning about static properties of program configurations.
There exist newer and more expressive variants of matching logic (\cite{MmL, MLexplained});
we used the older variant in order to be compatible with the literature on reachability logic which uses this variant.

Matching logic \emph{formula} (aka \emph{pattern}) is a first-order logic (FOL) formula which allows terms,
over a signature $\Sigma$, with variables, as nullary predicates.
A typical example of a matching logic formula is $\varphi_{\mathit{example}}$, defined as
\begin{equation}\label{eqn:exampleMLPattern}
\texttt{<k> x--; <k><st> x} \texttt{ |-> } X\texttt{ </st>} \land (X \texttt{ >Int } 1 = \mathit{true})
\end{equation}
which, when interpreted in a model of a particular programming language,
denotes the set of program configurations in which the code \texttt{x--} is to be executed
and in which the program variable $\texttt{x}$ has a value $X$ that is greater than $1$.
In this example, the part $\texttt{<k> x--; <k><st> x} \texttt{ |-> } X\texttt{ </st>}$
is the term-as-predicate, with $X$ being the only free FOL variable.
The program variable $\texttt{x}$ is not a FOL variable, but a constant symbol from the signature of the programming language.
The subterm $\texttt{x} \texttt{ |-> } X\texttt{ </st>}$ says that the program variable $\texttt{x}$
has the value $X$, and the $X \texttt{ >Int } 1 = \mathit{true}$ part then says that the realization
of the function symbol $\_ \texttt{ >Int } \_$ returns the boolean value $\mathit{true}$ when given $X$ and $1$
as arguments.

The satisfaction relation $(M, \gamma, \rho) \vDash \varphi$ between a model $M$, a model element $\gamma \in M$,
an $M$-valuation $\rho$, and a pattern $\varphi$, is defined inductively on the structure of $\varphi$.
The definition is as in FOL; the main difference is the semantics of terms-as-predicates, which is defined by
\begin{equation*}
    (M, \gamma, \rho) \vDash t \iff \gamma = \rho(t) \text{ if t is a term}
\end{equation*}
(where $\rho(t)$ is the homomorphic extension of $\rho$ applied to the term $t$).
For example, we might have a matching logic model $M$ containing (concrete) program configurations
of a particular programming language.
One such configuration might be $\gamma_{\mathit{example}}$:
\begin{equation*}
    \texttt{<k> x--; <k><st> x} \texttt{ |-> } 3\texttt{ </st>} \, .
\end{equation*}
Then, we have that $(M, \gamma_{\mathit{example}}, \rho) \vDash \varphi_{\mathit{example}}$
for any valuation $\rho$ satisfying $\rho(X) = 3$, and we say that
$\varphi_{\mathit{example}}$ \emph{matches} $\gamma_{\mathit{example}}$ in $\rho$.


A pattern $\varphi$ is \emph{valid in $M$}, written $M \vDash \varphi$, iff $(M, \gamma, \rho) \vDash \varphi$
for every $\gamma$ and $\rho$.
We observe that validity of a structureless pattern (that is a pattern without terms-as-predicates) does not depend on the selected model element.
Also, validity of any pattern does not depend on those variables which the pattern does not mention.
A more formal treatment of matching logic is to be found in the Appendix.


\subsection{One-path Reachability Logic}
Reachability logic \cite{StefanescuCMMSR19,RosuS12oopsla} is a formalism for reasoning about partial correctness
of programs in any programming language which has a RL-based formal semantics. The programming language is modelled as a \emph{reachability system}
$\mathcal{S} = (\mathcal{T}, S)$, where $\mathcal{T}$ is a $\Sigma$-algebra
and $S$ is a set of \emph{reachability rules} of the shape $\varphi \Rightarrow^\exists \varphi^\prime$,
where $\varphi$ and $\varphi^\prime$ are matching logic patterns over $\Sigma$.
A reachability system $\mathcal{S} = (\mathcal{T}, S)$ (together with a $\Sigma$-sort $\mathit{Cfg}$)
induces
          a \emph{transition system}
          $(\Tcfg , \Rightarrow_{\mathcal{S}})$,
          where $\gamma \Rightarrow_{\mathcal{S}} \gamma^\prime$
          for $\gamma, \gamma^\prime \in \Tcfg$
          iff there is some rule $\varphi \Rightarrow^\exists \varphi^\prime \in S$
          and some valuation $\rho : \Var \to \mathcal{T}$ with $(\gamma, \rho) \vDash \varphi$
          and $(\gamma^\prime , \rho) \vDash \varphi^\prime$.
A one-path reachability rule $\varphi \Rightarrow^\exists \varphi^\prime$ is \emph{satisfied}
          in a reachability system $\mathcal{S} = (\mathcal{T}, S)$,
          written $\mathcal{S} \vDash_\RL \varphi \Rightarrow^\exists \varphi^\prime$,
          iff for every $\gamma \in \Tcfg$
          such that $\gamma$ terminates in $(\Tcfg, \Rightarrow_{\mathcal{S}})$
          and for any valuation $\rho : \Var \to \mathcal{T}$
          such that $(\gamma, \rho) \vDash \varphi$,
          there exists some $\gamma^\prime \in \Tcfg$
          such that
          $\gamma \Rightarrow^{*}_{\mathcal{S}} \gamma^\prime$
          and $(\gamma^\prime, \rho) \vDash \varphi^\prime$.
A reachability system $(\mathcal{T}, S)$ is \emph{$\epsilon$-free}
          iff for any two configurations $\sigma, \sigma^\prime \in \mathcal{T}_{\mathit{Cfg}}$, if
          $\sigma \Rightarrow_{\mathcal{S}} \sigma^\prime$, then $\sigma \not = \sigma^\prime$;
Reachability logic is equipped with a proof system which derives sequents of the shape
$\mathcal{S}, A, C \vdash_\RL \varphi \Rightarrow^\exists \varphi^\prime$;
the proof system is sound and complete: a RL claim is satisfied in $\mathcal{S}$
iff $\mathcal{S}, \emptyset, \emptyset \vdash_\RL \varphi \Rightarrow^\exists \varphi^\prime$.


\begin{remark}\label{rem:shapeOfReachabilityRules}
For simplicity, we restrict the class of reachability systems we work with to those whose reachability rules
have the shape
%\begin{equation*}
%    \bigvee_{i=1}^{m} (\phi_i \land P_i) \Rightarrow^\exists \exists \vec{Y}.\, \bigvee_{j=1}^{n} (\phi^\prime_j \land P^\prime_j) \, .
%\end{equation*}
\begin{equation*}
    \phi \land P \Rightarrow^\exists \phi^\prime \land P^\prime
\end{equation*}
where $phi \land P$ and $\phi^\prime \land P^\prime$ are constrained terms.
\end{remark}

\begin{remark}\label{rem:noEmptySteps}
We work only with $\epsilon$-free reachability systems.
\end{remark}