\section{Cartesian Reachability Logic}
This section introduces \emph{Cartesian Reachability logic (CRL)} - a
language-parametric logic for reasoning about $k$-safety hyperproperties.  Our
aim with CRL is to make reasoning in the style of \emph{Cartesian Hoare logic
  (CHL)}~\cite{SousaD16} available for any deterministic language for which a
reachability-logic semantics $S$ is available.  For that purpose, we define the
language of CRL and its semantics, and demonstrate the logic's expressiveness
on a couple of examples.  Finally, we give CRL a sound proof system, which is the
main contribution of this paper.


\subsection{Syntax and Semantics}
\label{sec:CRLsemanticsComparisonToCHL}

Cartesian reachability logic is an extension of (one-path) reachability
logic. To express k-safety properties we extend reachability rules $\varphi
\Rightarrow^{\exists} \varphi'$ to  \emph{cartesian reachability claims} of the form
\begin{equation*}
  [\varphi_1,\ldots,\varphi_k] \land P
  \Rightarrow^{c\exists} \exists \vec{Y}.\, [\varphi^\prime_1,\ldots,\varphi^\prime_k] \land P^\prime
\end{equation*}

The intuitive meaning of such a claim is as follows: there are $k$ programs
embedded into $k$ source configurations, with $i$-th source configuration matching $\varphi_i$,
and $k$ target configurations matching $\varphi'_i$s. Additionally, the FOL
formula $P$ can relate the source configurations, and the FOL formula $P'$ the target
configurations. We call formulas of the form
$ \exists \vec{Y}.\,[\varphi_1,\ldots,\varphi_k] \land P$ (where $\vec{Y}$ may
be an empty vector) \emph{existentially-quantified constrained list patterns}
(ECLP).


For example, let us again consider the program
\begin{equation*}
  P \equiv \texttt{while(x > 0)\{ x--; y++; \}} \, .
\end{equation*}
Additionally, let $\xi(Q, X, Y)$ be a pattern matching those configurations of
program $Q$ where the program variable $\texttt{x}$ has value $X$ and the program
variable $\texttt{y}$ has value $Y$:
\begin{equation}\label{eqn:CQXY}
 \xi(Q, X, Y) \equiv\ \tap{$Q$}{\texttt{x}\mapsto X, \texttt{y}\mapsto Y }  \, .
\end{equation}
Then, the claim $\Omega_{\textit{mono}}$, defined as
\begin{align*}
&[\xi(P, X_1, Y_1),\xi(P, X_2, Y_2)] \land X_1 \leq X_2 \land Y_1 \leq Y_2
\\ \Rightarrow^{c\exists}\ &
\exists X^\prime_1, Y^\prime_1, X^\prime_2, Y^\prime_2.\,  [\xi(\epsilon, X^\prime_1, Y^\prime_1), \xi(\epsilon, X^\prime_2, Y^\prime_2)] \land Y^\prime_1 \leq Y^\prime_2   
\end{align*}
(where $\epsilon$ denotes the empty program)
expresses the property that the program $P$ is \emph{monotone}.
That is, when we start an execution (using the semantics of the particular language)
from some configuration $\gamma_1$ matching $\xi(P, X_1, Y_1)$
and a second execution from some configuration $\gamma_2$ matching $\xi(P, X_2, Y_2)$,
if $X_1 \leq X_2$ and $Y_1 \leq Y_2$,
we end up in configurations $\gamma_1^\prime,\gamma_2^\prime$ matching
$\xi(\epsilon, X^\prime_1, Y^\prime_1)$ and $\xi(\epsilon, X^\prime_2, Y^\prime_2)$
for some $X^\prime_1,Y^\prime_1,X^\prime_2,Y^\prime_2$
satisfying $Y^\prime_1 \leq Y^\prime_2$.
%
Note that we need the existential quantification on the right-hand side of
claims to be able to talk about the new values of program variables (whereas
in CHL, we have original values in the precondition and new values in the postcondition).

We formally define the semantics of a CRL claim as follows:
\begin{definition}[CRL semantics]\label{def:opCRLsemantics}
    A claim
%    \begin{equation*}
     $[\varphi_1,\ldots,\varphi_k] \land P
     \Rightarrow^{c\exists} \exists \vec{Y}.\, [\varphi^\prime_1,\ldots,\varphi^\prime_k] \land P^\prime$
%    \end{equation*}
    is \emph{valid} in a reachability system $\mathcal{S} = (\mathcal{T}, S)$,
    written
    \begin{equation*}
        (\mathcal{T}, S) \modelsCRL [\varphi_1,\ldots,\varphi_k] \land P
     \Rightarrow^{c\exists} \exists \vec{Y}.\, [\varphi^\prime_1,\ldots,\varphi^\prime_k] \land P^\prime \, ,
    \end{equation*}
    iff for all configurations $\gamma_1,\ldots,\gamma_k \in \Tcfg$
    which terminate in $(\Tcfg, \Rightarrow_{\mathcal{S}})$
    and any $\mathcal{T}$-valuation $\rho$,
    whenever $(\mathcal{T}, \gamma_i,\rho) \modelsML \varphi_i \land P$ for all $i \in \{ 1, \ldots, k \}$,
    then there exist configurations $\gamma_1^\prime,\ldots,\gamma_k^\prime \in \Tcfg$
    such that $\gamma_i \Rightarrow^{*}_{\mathcal{S}} \gamma_i^\prime$ for all $i \in \{ 1, \ldots, k \}$,
    and there also exists an $\mathcal{T}$-valuation $\rho^\prime$
    satisfying $\rho(v) = \rho^\prime(v)$ for any $v \in \mathit{Var} \setminus \vec{Y}$,
    and
    $(\mathcal{T}, \gamma_i^\prime,\rho^\prime) \modelsML \varphi^\prime_i \land P^\prime$ for all $i \in \{ 1, \ldots, k \}$.
\end{definition}

%\subsection{Comparison to CHL}\label{sec:CRLsemanticsComparisonToCHL}

As can be seen from the example above, CRL is more verbose than CHL. This is
partly because of the need to specify patterns matching the whole program
configurations and partly because of the need to existentially quantify those
variables on the right side whose value is not determined by the left side. To
alleviate this problem, we introduce the following notation, which we inherit
from RL:

\paragraph{Notation}
Variables whose names start with a question mark are implicitly considered
existentially quantified on the right side.  Also, an underscore is used to
denote anonymous variables, whose values we are not interested in.  For
example, we can write the claim $\Omega_{\textit{mono}}$ as
\begin{align*}
[\xi(P, X_1, Y_1),\xi(P, X_2, Y_2)] \land X_1 \leq X_2 \land Y_1 \leq Y_2
 \Rightarrow^{c\exists} [\xi(\epsilon, ?\_, ?Y_1), \xi(\epsilon, ?\_, ?Y_2)] \land ?Y_1 \leq ?Y_2 \, .
\end{align*}

\subsection{CRL as an extension of Reachability Logic}

We want to point out that one cannot handle k-safety properties directly
in RL, by simply replacing a CRL claim with $k$ components by $k$ RL
claims. The reason is that in CRL formulas, one can relate variables from different components. 


%Consider, for example, a reachability system $S_{\mathit{IMP}}$ representing a simple imperative language,
%and the same program as in \Cref{eqn:CounterProgram}; that is, let
%\begin{equation*}
%  P \equiv \texttt{while(x > 0)\{ y++; x--;\}} \, .
%\end{equation*}
%We can let $C(Q, X,Y)$ represent a configuration of a program $Q$ where the program variable $\texttt{x}$ has the value $X$
%and the program variable $\texttt{y}$ has the value $Y$;
%for example,
%\begin{equation*}
% C(Q, X, Y) \equiv \texttt{<k>} Q \texttt{</k><st>(x |-> } X \texttt{)(y |-> } Y \texttt{)</st>}    \, .
%\end{equation*}
%Then, the proposition
%\begin{align*}
% S_{\mathit{IMP}} \modelsCRL
%&[C(P, X_1, Y_1),C(P, X_2, Y_2)] \land X_1 \leq X_2
%\\ \Rightarrow^{c\exists} &
%\exists X^\prime_1, Y^\prime_1, X^\prime_2, Y^\prime_2.\,  [C(\epsilon, X^\prime_1, Y^\prime_1), C(\epsilon, X^\prime_2, Y^\prime_2)] \land Y^\prime_1 \leq Y^\prime_2   
%\end{align*}
%holds iff the program $P$ is monotone (when considering the variable $x$ to be an input, and $y$ to be an output; $\epsilon$ represents an empty program).
In CRL, one can localize the ``global'' constraints; for example, the claim $\Omega_{\textit{mono}}$
is equivalent to
\begin{align*}
[\xi(P, X_1, Y_1),\xi(P, X_2, Y_2) \land X_1 \leq X_2 \land Y_1 \leq Y_2] 
\Rightarrow^{c\exists}
[\xi(\epsilon, ?\_, ?Y_1), \xi(\epsilon, ?\_, ?Y_2) \land ?Y_1 \leq ?Y_2] \, .
\end{align*}
However, if one were to ``split'' the CRL claim into two, the resulting claims might express
a different property than monotonicity.
For example, the two claims
\begin{align}
& \xi(Q, X_1, Y_1) \land X_1 \leq X_2 \land Y_1 \leq Y_2 & \Rightarrow^{\exists} \quad & \xi(\epsilon, ?X_1, ?Y_1) \\
& \xi(Q, X_2, Y_2) \land X_1 \leq X_2 \land Y_1 \leq Y_2 & \Rightarrow^{\exists} \quad  & \xi(\epsilon, ?X_2, ?Y_2) \land ?Y_1 \leq ?Y_2
\end{align}
hold for any reasonable program $Q$ (meaning that $Q$ either executes fully or diverges),
because $?Y_1$ in the second claim is unrelated to $?Y_1$ in the first claim and thus one can set it to
the value of $?Y_2$.
On the other hand, if in the second claim we renamed $?Y_1$ to $Y_1^\prime$ (without a question mark),
the claim would require that $?Y_2$ is greater than or equal to \emph{any} integer (because $Y_1^\prime$ is not present on the left side),
which clearly cannot hold.

On the other hand, CRL is an extension of RL in the following natural sense:
If we restrict configuration lists in CRL claims to contain a single
configuration each, validity of CRL and RL claims neatly coincide
(the proof of the following proposition can be found in Appendix~\ref{app:CRL}):


%\begin{remark}
%\Cref{def:opCRLsemantics} extends \Cref{def:oprlSemantics} of \Cref{def:basics}:
%if we fix $k=1$, then
\begin{proposition}\label{prop:opCRLopRL}
  $ (\mathcal{T}, S) \modelsCRL [\varphi] \land \top \Rightarrow^{c\exists}
  [\varphi^\prime] \land \top \iff (\mathcal{T}, S) \modelsRL \varphi
  \Rightarrow^{\exists} \varphi^\prime \, .  $
\end{proposition}
%\end{remark}

%\begin{definition}[All-Path Cartesian Reachability Rule]\label{def:apCRLsemantics}
%An all-path Cartesian reachability rule
%$(\varphi_1,\ldots,\varphi_k) \land \varphi \Rightarrow^{c\forall} (\psi_1,\ldots,\psi_k) \land \psi$
%of arity $k$
%is \emph{valid} in a reachability system $\mathcal{S} = (\mathcal{T}, S)$,
%written
%$\mathcal{S} \modelsCRL (\varphi_1,\ldots,\varphi_k) \land \varphi \Rightarrow^{c\forall}
%(\psi_1,\ldots,\psi_k) \land \psi$,
%iff for all configurations $\sigma_1,\ldots,\sigma_k \in \Tcfg$ \traian{Why $\sigma$ instead of $\gamma$?}
%and any $\mathcal{T}$-valuation $\rho$,
%whenever $(\sigma_1, \rho) \modelsML \varphi_1 \land \varphi$ and \ldots
%and $(\sigma_k, \rho) \modelsML \varphi_k \land \varphi$,
%then for every $k$-tuple of complete paths $(\pi_1, \ldots, \pi_k)$
%such that
%$\sigma_1 = \pi_1(0) \land \ldots \land \sigma_k = \pi_k(0)$,
%there exist indices $i_1, \ldots, i_k$ such that
%$(\pi_1(i_1), \rho) \modelsML \varphi_1 \land \psi$ and \ldots and $(\pi_k(i_k), \rho) \modelsML \varphi_k \land \psi$.
%\end{definition}

\subsection{A language-independent alternative to self-composition}
\label{sec:self-composition}


%Now we present a novel, general technique called \emph{star extension} that is reminiscent of.

% To prove the soundness of our proof system for CRL, it would be advantegous to
% be able to apply the 

Before presenting a proof system for CRL, we must first discuss a concept
crucial to proving its soundness. Self-composition~\cite{BartheDR04,DarvasHS05}
is a technique where a program $P$ together with a $k$-safety hyperproperty is
reduced to a sequential composition of $P$ with itself (with renamed
variables) together with a safety property.  This technique allows one to use
tools and techniques for verification of safety properties to perform
verification of $k$-safety hyperproperties.  The challenge here is to
generalize self-composition to work with any deterministic language, even if
we do not know in advance how the language implements sequential composition,
if at all.

To address this issue, we present a novel, general technique, which we call
\emph{star extension}. The main idea is to transform a CRL claim into a RL
claim over an extended reachability system, whose configurations are lists of
configurations of the original system.  The function $\mathit{flatten}$ then
converts an ECLP (which contains a meta-level list of matching logic patterns)
into a matching logic formula (containing an object-level list) that matches
lists inside the model.  The transformation is quite straightforward but
technical, and we refer an interested reader to the
\Cref{app:CRLandRLcorrespondence}; here we state the main theorem:
\begin{theorem}\label{thm:CRLandRLcorrespondence}
  There exist a function $\_^*$ on matching logic signatures,
  a (equally-named) function $\_^*$ from reachability systems over $\Sigma$ to reachability systems over $\Sigma^*$,
  and a function $\mathit{flatten}$ from ECLPs over $\Sigma$ to matching logic $\Sigma^*$-formulas,
  such that
  \begin{equation*}
  \mathcal{S} \modelsCRL \Psi \Rightarrow^{c\exists} \Psi^\prime
    \iff \mathcal{S}^* \modelsRL \mathit{flatten}(\Psi) \Rightarrow^{c\exists} \mathit{flatten}(\Psi^\prime)
  \end{equation*}
\end{theorem}


The price paid for self-composition is that the property of the self-composed program is often hard to reason about.
Therefore, in~\cite{SousaD16}, the authors do not apply self-composition directly, but only use its soundness to justify
their technique - namely, the soundness of their proof system, which avoids the explicit construction
of self-composed programs.
We use the star extension for the same purpose.



\subsection{Proof System for CRL}
\label{sec:proof-system}

We are now ready to give Cartesian Reachability Logic a proof system to
facilitate mechanical reasoning.  While the intuition behind the semantics of
CRL is similar to that of CHL, with only a few minor differences, coming up
with a proof system for CRL is not straightforward.  One could attempt to reuse the proof
system of CHL and modify it \emph{somehow} to be independent of the particular
programming language.  Since many CHL rules (e.g., its \texttt{If} rule shown
in \Cref{chlRule:If}) are simply Hoare logic rules acting on a particular
component of a tuple of formulas (that is, symbolic states), one could, for
example, lift rules of reachability logic (RL) to the tuple-context and be
done.  That would indeed work, and the resulting proof system might even be
complete.

However, the distinguishing feature of Cartesian Hoare Logic is \emph{not} its
completeness but its ability to simplify reasoning by performing lockstep
execution of loops, because that can \textquote[\cite{SousaD16}]{greatly
  simplify the verification task (e.g., by requiring simpler invariants)}.
To achieve that, it is not enough to 
just lift RL rules to tuple-context. And it is not entirely obvious how
to support lockstep reasoning involving construct with repetitive behavior: in
order to support \texttt{while} cycles with \texttt{break}, Cartesian Hoare logic itself uses
five additional fairly complex rules (besides the lifted Hoare logic rules) 
that need to reflect the precise semantics of this construct.
% The task for
%us is even harder since we do not know in advance what kinds of constructs with
%repetitive behavior does the supplied language support.

We provide a single proof system consisting of only eight rules (Figure~\ref{fig:CRLproofsystem});
the proof system is agnostic about the kinds of constructs with repetetive behavior that the supplied language supports,
but enables lockstep reasoning about such constructs.
The proof system derives claims of the shape
\begin{equation*}
(\mathcal{T}, S) \provesCRL \Phi \Downarrow_{C,E} \Psi
\end{equation*}
where $\Phi,\Psi$ are of the shape $\exists \vec{X}.\, (\varphi_1, \ldots, \varphi_k) \land P$.
One can think about $\Phi$ as representing a \emph{premise}, while
$\Psi$, which propagates through the proof rules unchanged, as representing a
\emph{conclusion}.  For each $i$, $\varphi_i$ is a matching logic
pattern representing a particular component, and $P$ is a FOL
formula (\emph{global constraint}) relating variables from different
components.  The sets $C$ and $E$ contain \emph{synchronization points} and \emph{enabled
  synchronization points}, respectively.  Together, they implement the concept of an
\emph{invariant} relating different components.  In particular, set $C$
represents the invariants that were postulated \emph{right now}, while set
$E$ represents those postulated in the past and ready to be used.
Initially, the proof search starts with $E = C = \emptyset$.

Similarly to \emph{circularities} of reachability logic, \emph{synchronization points}
allow us to reason coinductively\footnote{A formal treatement of the relationship between coinduction and reachability logic
is to be found in \cite{MoorePR18}.} about (cartesian) reachability claims:
one can introduce them at any point of the proof, but they can be used only after progress has been made
by means of symbolic execution.
The progress requirements resemble the concept of \emph{productivity} of co-recursive definitions.

\begin{figure}
    \centering
    \begin{align*}
    & \prftree[l]{Reflexivity}{(\mathcal{T}, S) \provesCRL \Psi \Downarrow_{\emptyset,E} \Psi}
    \end{align*}
    \begin{align*}
    & \prftree[l]{Axiom}{\Psi \in E}{(\mathcal{T}, S) \provesCRL \Psi \Downarrow_{C,E} \Psi^\prime}
    \end{align*}
    \begin{align*}
    & \prftree[l]{Reduce}
      {(\mathcal{T}^*, S^* \cup \mathit{flatten}^\exists(E, \Psi^\prime)), \emptyset \vdash_\RL
        \mathit{flatten}^\exists(\Psi, \Psi^\prime) }
      {(\mathcal{T}, S) \provesCRL \Psi \Downarrow_{C,E} \Psi^\prime}
    \end{align*}
    \begin{align*}
    & \prftree[l]{Case}
    { \prfStackPremises
      {(\mathcal{T}, S) \provesCRL [\varphi_1, \ldots, \varphi_{i-1}, \varphi_i, \varphi_{i+1}, \ldots, \varphi_k] \land P^\prime \Downarrow_{C, E} \Psi^\prime }
      {(\mathcal{T}, S) \provesCRL [\varphi_1, \ldots, \varphi_{i-1}, \psi_i, \varphi_{i+1}, \ldots, \varphi_k] \land P^\prime \Downarrow_{C, E} \Psi^\prime }
    }
    {(\mathcal{T}, S) \provesCRL [\varphi_1, \ldots, \varphi_{i-1}, (\varphi_i \lor \psi_i), \varphi_{i+1}, \ldots, \varphi_k] \land P^\prime \Downarrow_{C, E} \Psi^\prime}
    \end{align*}
    \begin{align*}
    & \prftree[l]{Step}
    { \prfStackPremises
       {\varphi \Rightarrow^\exists \varphi^\prime \in S}
       {\mathcal{T} \modelsML \varphi_i \leftrightarrow \varphi \land P^\prime}
       {P^\prime \mbox{ is a FOL formula}}
       {  (\mathcal{T}, S) \provesCRL [\varphi_1, \ldots, \varphi_{i-1}, \varphi^\prime \land P^\prime, \varphi_{i+1}, \ldots, \varphi_k]
          \land P
          \Downarrow_{\emptyset, (C \cup E)} \Psi^\prime
      }
    }
    {(\mathcal{T}, S) \provesCRL [\varphi_1, \ldots, \varphi_{i-1}, \varphi_i, \varphi_{i+1}, \ldots, \varphi_k] \land P \Downarrow_{C, E} \Psi^\prime}
    \end{align*}
    \begin{align*}
    & \prftree[l]{Circularity}
      { (\mathcal{T}, S) \provesCRL \Psi \Downarrow_{C \cup \{ \Psi \} , E} \Psi^\prime}
      { (\mathcal{T}, S) \provesCRL \Psi \Downarrow_{C, E} \Psi^\prime}
    \end{align*}
    \begin{align*}
    & \prftree[l]{Conseq}
      { \prfStackPremises
        { (\mathcal{T}, S) \provesCRL \Phi^\prime \Downarrow_{C, E} \Psi^\prime}
        { \mathcal{T}^* \modelsML \mathit{flatten}(\Phi) \rightarrow \mathit{flatten}(\Phi^\prime) }
      }
      { (\mathcal{T}, S) \provesCRL \Phi \Downarrow_{C, E} \Psi^\prime}
    \end{align*}
%    \begin{align*}
%    & \prftree[l]{Conseq2}
%      { \prfStackPremises
%        { (\mathcal{T}, S) \provesCRL (\varphi_1, \ldots, \varphi_k) \land P \Downarrow_{C, E} \Psi^\prime}
%        { \mathcal{T} \modelsML \varphi_i \leftrightarrow \psi_i }
%      }
%      { (\mathcal{T}, S) \provesCRL (\varphi_1^\prime, \ldots, \varphi_{i-1}, \varphi_{i}, \varphi_{i+1}, \ldots, \varphi_k^\prime) \land P^\prime \Downarrow_{C, E} \Psi^\prime}
%    \end{align*}

    %\begin{align*}
    %& \prftree[l]{Abstract}
      %{ \prfStackPremises
        %{ X \not\in \mathit{FV}(\Psi^\prime, \varphi_1, \ldots, \varphi_{i-1}, \varphi_{i+1}, \ldots,\varphi_k)
        %}
        %{(\mathcal{T}, S) \provesCRL (\varphi_1, \ldots, \varphi_{i-1}, \varphi_i, \varphi_{i+1}, \ldots, \varphi_k) \land P \Downarrow_{C, E} \Psi^\prime}
      %}
      %{(\mathcal{T}, S) \provesCRL (\varphi_1, \ldots, \varphi_{i-1}, \exists X.\, \varphi_i, \varphi_{i+1}, \ldots, \varphi_k) \land P \Downarrow_{C, E} \Psi^\prime}
    %\end{align*}
    
    \begin{align*}
    & \prftree[l]{Abstract}
      { \prfStackPremises
        { X \not\in \mathit{FV}(\Psi^\prime)
        }
        {(\mathcal{T}, S) \provesCRL \exists \vec{Y}.\, [\varphi_1, \ldots, \varphi_k] \land P \Downarrow_{C, E} \Psi^\prime}
      }
      {(\mathcal{T}, S) \provesCRL \exists X, \vec{Y}.\, [\varphi_1, \ldots, \varphi_k] \land P \Downarrow_{C, E} \Psi^\prime}
    \end{align*}
    \caption{A proof system for CRL}
    \label{fig:CRLproofsystem}
\end{figure}

The rules of our system are fairly simple, and general in the sense that none of them has the semantics of any particular language construct
hard-wired into it.
We now explain the proof rules of CRL (shown in~\Cref{fig:CRLproofsystem}) one by one.
\begin{itemize}
  \item The Circularity rule is a key rule which allows lockstep reasoning about arbitrary program constructs.
        It allows the user to postulate the validity of the current claim by means of adding the \emph{current premise}
        into the set of \emph{synchronization points}, from which a $k$-tuple of program configurations satisfying the postcondition
        is claimed to be reachable. Once progress is made (by means of the Step rule), the added synchronization points are
        \emph{enabled} and can be used to finish the proof using the Axiom rule.
  \item The Step rule performs symbolic execution on a selected component $i$ (represented by $\varphi_i$)
        using the semantic reachability rule $\varphi \Rightarrow^\exists \varphi^\prime \in S$.
        For this rule to apply, its left side ($\varphi$) has to match all the program configurations matching $\varphi_i$.
        Therefore, the rule decomposes $\varphi_i$ into $\varphi$ and an additional constraint $P^\prime$,
        which can be thought of as a part of \emph{path condition} that is \emph{local} to the component $i$.
        This local path condition $P^\prime$ is then used to constrain the right-side $\varphi^\prime$ of the selected rule.
        This proof rule also enables the synchronization points from $C$ by adding them to $E$.
  \item The Axiom rule uses an enabled synchronization point to finish the proof.
  \item The Reflexivity rule can be used to finish a proof when the premise corresponds to the conclusion.
  \item The Case rule implements case analysis on a selected component $i$.
  \item The Conseq rule is used to weaken (or generalize) the premise. It can also be used
        to propagate information between components and the global constraint.
  \item The Abstract rule can be used to remove existential quantifiers from the premise.
        Intuitively, this corresponds to a proof step in first-order logic that replaces
        an existential quantifier on the left side of an implication
        with a universal quantifier over the implication, assuming that the variable
        bound by the existential quantifier does not occur free in the right side.
        In our setting, the typical way of \emph{obtaining} a top-level existential quantifier in the premise
        is by means of the Conseq rule.
  \item The Reduce rule is a way to get completeness into our proof system:
        it reduces the goal to reachability logic reasoning.
        This rule also provides a way to prove properties that do not benefit
        from lockstep reasoning.
\end{itemize}

The soundness of the proof system, as stated by the following theorem, is the main
technical result of this paper.

\begin{theorem}[Proof system soundness]\label{thm:proofsystemSoundness}
\begin{equation*}
    (\mathcal{T}, S) \provesCRL \Psi \Downarrow_{\emptyset,\emptyset} \Psi^\prime \implies
    (\mathcal{T}, S) \modelsCRL \Psi \Rightarrow^{c\exists} \Psi^\prime
\end{equation*}
\end{theorem}

In order to  prove this theorem, we will need (beside
Theorem~\ref{thm:CRLandRLcorrespondence}, which we discuss in the previous
section) a technical lemma stating that one can generate an RL proof on a star-extended system
from a CRL proof. This lemma is the second major component of the soundness
proof of CRL and its proof can be found in \Cref{app:crlsoundness}.
\begin{lemma}\label{lem:CRLalmostSoundness}
    \begin{align*}
        & (\mathcal{T}, S) \provesCRL \Psi \Downarrow_{C,E} \Psi^\prime \implies \\
        &
        (\mathcal{T}^*, S^* \cup \mathit{flatten}^\exists(E, \Psi^\prime)), \mathit{flatten}^\exists(C, \Psi^\prime) \vdash_\RL
          \mathit{flatten}^\exists(\Psi, \Psi^\prime) 
    \end{align*}
\end{lemma}

\noindent With all the technical tools is place, the proof itself is a straightforward affair:
\begin{proof}[Proof of \Cref{thm:proofsystemSoundness}]
Assume $(\mathcal{T}, S) \provesCRL \Psi \Downarrow_{\emptyset,\emptyset} \Psi^\prime$.
By \Cref{lem:CRLalmostSoundness}, we have $(\mathcal{T}, S) \vdash_\RL \mathit{flatten}^\exists(\Psi, \Psi^\prime)$.
By soundness of reachability logic, we have $(\mathcal{T}, S) \modelsRL \mathit{flatten}^\exists(\Psi, \Psi^\prime)$.
By \Cref{thm:CRLandRLcorrespondence}, 
we have $(\mathcal{T}, S) \modelsCRL \Psi \Rightarrow^{c\exists} \Psi^\prime$ and we are done.
\end{proof}

% \begin{figure}

%   \begin{align*}
%     & \prftree[l]{ImplConclusion}
%       {\mathcal{T}^* \modelsML \mathit{flatten}(\Phi) \rightarrow \mathit{flatten}(\Psi^\prime)}
%       {(\mathcal{T}, S) \provesCRL \Phi \Downarrow_{C, E} \Psi^\prime}
%   \end{align*}

%   \begin{align*}
%     & \prftree[l]{ImplEnabled}
%       { \prfStackPremises
%         { \Phi^\prime \in E }
%         {\mathcal{T}^* \modelsML \mathit{flatten}(\Phi) \rightarrow \mathit{flatten}(\Phi^\prime)}
%       }
%       {(\mathcal{T}, S) \provesCRL \Phi \Downarrow_{C, E} \Psi^\prime}
%   \end{align*}

%   \begin{align*}
%     & \prftree[l]{Contradiction}
%       {\mathcal{T}^* \modelsML \mathit{flatten}(\Phi) \rightarrow \bot }
%       {(\mathcal{T}, S) \provesCRL \Phi \Downarrow_{C, E} \Psi^\prime}
%   \end{align*}

%   \caption{Selected derived rules}
%   \label{fig:CRLderivedRules}
% \end{figure}

The proof system is also \emph{relatively complete} (with respect to an oracle deciding validity
in the underlying matching logic model $\mathcal{T}$).
\begin{theorem}[Relative completeness]\label{thm:relativeCompleteness}
  \begin{equation*}
      (\mathcal{T}, S) \modelsCRL \Psi \Rightarrow^{c\exists} \Psi^\prime \implies
      (\mathcal{T}, S) \provesCRL \Psi \Downarrow_{\emptyset,\emptyset} \Psi^\prime
  \end{equation*}
  \end{theorem}
  \begin{proof}[Proof of \Cref{thm:relativeCompleteness}]
  Assume $(\mathcal{T}, S) \modelsCRL \Psi \Rightarrow^{c\exists} \Psi^\prime$.
  By \Cref{thm:CRLandRLcorrespondence}, 
  we obtain
  \begin{equation*}
    (\mathcal{T}^*, S^*) \modelsRL
    \mathit{flatten}^\exists(\Psi, \Psi^\prime) \, .
  \end{equation*}
  By relative completeness of reachability logic, we obtain
  \begin{equation*}
    (\mathcal{T}^*, S^*) \vdash_\RL
    \mathit{flatten}^\exists(\Psi, \Psi^\prime) \, ,
  \end{equation*}
  and we conclude the proof using the Inherit rule.
  Note that to apply relative completeness of RL, we need to have an oracle for deciding validity in the extended model.
  A construction of such oracle from the oracle for deciding validity in $\mathcal{T}$ is in \Cref{app:completeness}.
  \end{proof}
Our completeness result is similar to the completeness result of CHL in the sense that the completeness
does not involve features used for lockstep reasoning.
It would be interesting to investigate whether we can have some complete proof system without the Reduce rule;
we leave this for future work.


%We give Cartesian Reachability Logic a proof system to faciliate mechanical reasoning.
One may ask the question, ``why give CRL a new proof system if one can perform the same reasoning
by means of \Cref{thm:CRLandRLcorrespondence} and the existing proof system of reachability logic?''.
The answer is the following.
When one reduces a CRL goal into RL using \Cref{thm:CRLandRLcorrespondence},
the function $\mathit{flatten}$ appears in the goal.
To perform reasoning using the RL proof system, one then has to either simplify the goal
by unfolding the definition of $\mathit{flatten}$,
or use (and prove) some helper lemmas about the effect of applying RL proof rules
%and other frequently used steps
on RL goals containing $\mathit{flatten}$.
In the first case one lowers the abstraction level and have to reason about matching logic formulas
containing translations of other matching logic formulas into FOL, which then makes RL reasoning difficult.
In the second case, however, one may end up proving lemmas which, when combined, result in a proof system for CRL.
Indeed, the soundness of our proof system
is established by a (meta-)proof which constructs a RL proof from a CRL one (\Cref{lem:CRLalmostSoundness}).

\section{An example proof involving lockstep reasoning}\label{sec:example}

We now present an example proof sketch using our proof system; the proof involves lockstep reasoning.
To ease the notation, we write simply $t_b$ instead of $t_b = \mathit{true}$ for
boolean-sorted side conditions $t_b$.
Consider again the claim $\Omega_{\mathit{mono}}$ from \Cref{sec:CRLsemanticsComparisonToCHL}:
\begin{align*}
  [\xi(P, X_1, Y_1),\xi(P, X_2, Y_2)] \land X_1 \leq_{\mathit{Int}} X_2 \land Y_1 \leq_{\mathit{Int}} Y_2
   \Rightarrow^{c\exists} [\xi(\epsilon, ?\_, ?Y_1), \xi(\epsilon, ?\_, ?Y_2)] \land ?Y_1 \leq_{\mathit{Int}} ?Y_2 \, .
\end{align*}
Let $\Psi^\prime_{\mathit{mono}}$ denote the right side of the $\Rightarrow^{c\exists}$ above.
We want to prove
\begin{align*}
  \mathcal{S}_{\mathit{imp}} \provesCRL [\xi(P, X_1, Y_1),\xi(P, X_2, Y_2)] \land X_1 \leq_{\mathit{Int}} X_2 \land Y_1 \leq_{\mathit{Int}} Y_2
  \Downarrow_{\emptyset, \emptyset} \Psi^\prime_{\mathit{mono}} \, .
\end{align*}
To do so, we first want to add the current premise as a synchronization point.
However, we need to make the synchronization point more general than the current claim, as we will see later in the proof.
Therefore, we first apply the Conseq rule to change the goal to
\begin{align*}
  \mathcal{S}_{\mathit{imp}} \provesCRL \exists X_1,Y_1,X_2,Y_2.\, [\xi(P, X_1, Y_1),\xi(P, X_2, Y_2)] \land X_1 \leq_{\mathit{Int}} X_2 \land Y_1 \leq_{\mathit{Int}} Y_2
  \Downarrow_{\emptyset, \emptyset} \Psi^\prime_{\mathit{mono}} \, ,
\end{align*}
then apply the Circularity rule, followed by application of the Abstract rule, which basically changes the goal
back, except that now we have a general synchronization point. The goal is now
\begin{align*}
  \mathcal{S}_{\mathit{imp}} \provesCRL [\xi(P, X_1, Y_1),\xi(P, X_2, Y_2)] \land X_1 \leq X_2 \land Y_1 \leq Y_2
  \Downarrow_{\Phi_{\mathit{sync}}, \emptyset} \Psi^\prime_{\mathit{mono}} \, ,
\end{align*}
where
\begin{align*}
  \Phi_{\mathit{sync}} \equiv \mathcal{S}_{\mathit{imp}} \provesCRL \exists X_1,Y_1,X_2,Y_2.\, [\xi(P, X_1, Y_1),\xi(P, X_2, Y_2)] \land X_1 \leq_{\mathit{Int}} X_2 \land Y_1 \leq_{\mathit{Int}} Y_2 \, .
\end{align*}
Now we perform symbolic execution on both components, using repeated applications of the Step rule,
enabling the synchronization points.
The exact details depend on the exact rules of $\mathcal{S}_{\mathit{imp}}$, but assuming that the semantics of
the \texttt{while} statement is defined by unrolling into the \texttt{if} statement,
we end up with a goal like
\begin{align*}
  & \mathcal{S}_{\mathit{imp}} \provesCRL
  [ \xi(\texttt{if (}X_1 >_{\mathit{Int}} 0 \texttt{)\{y++;x--;P\}}, X_1, Y_1),  
    \xi(\texttt{if (}X_2 >_{\mathit{Int}} 0 \texttt{)\{y++;x--;P\}}, X_2, Y_2) ] \\
  & \land X_1 \leq_{\mathit{Int}} X_2 \land Y_1 \leq_{\mathit{Int}} Y_2
  \Downarrow_{\emptyset, \Phi_{\mathit{sync}}} \Psi^\prime_{\mathit{mono}} \, ,
\end{align*}
Now we want to perform case analysis. To do so, we first use the Conseq rule to
obtain disjunctions of patterns at the respective components;
then we repeatedly apply the Case rule, leading to four goals;
finally, we use the Conseq rule to propagate the constraints from the components to the global constraint.
Then, the four goals will be as follows
(the differences are shown in {\color{blue}blue}):
\begin{equation}\label{ex:case1}
\begin{aligned}
  & \mathcal{S}_{\mathit{imp}} \provesCRL
  [ \xi(\texttt{if (}{\color{blue}\mathit{true}}\texttt{)\{y++;x--;P\}}, X_1, Y_1),  
    \xi(\texttt{if (}{\color{blue}\mathit{true}}\texttt{)\{y++;x--;P\}}, X_2, Y_2) ] \\
  & \land X_1 \leq_{\mathit{Int}} X_2 \land Y_1 \leq_{\mathit{Int}} Y_2 \land {\color{blue} X_1 >_{\mathit{Int}} 0 \land X_2 >_{\mathit{Int}} 0 }
  \Downarrow_{\emptyset, \Phi_{\mathit{sync}}} \Psi^\prime_{\mathit{mono}} \, ,
\end{aligned}
\end{equation}
\begin{equation}\label{ex:case2}
\begin{aligned}
  & \mathcal{S}_{\mathit{imp}} \provesCRL
  [ \xi(\texttt{if (}{\color{blue}\mathit{true}}\texttt{)\{y++;x--;P\}}, X_1, Y_1),  
    \xi(\texttt{if (}{\color{blue}\mathit{false}}\texttt{)\{y++;x--;P\}}, X_2, Y_2) ] \\
  & \land X_1 \leq_{\mathit{Int}} X_2 \land Y_1 \leq_{\mathit{Int}} Y_2 \land {\color{blue} X_1 >_{\mathit{Int}} 0 \land X_2 \leq_{\mathit{Int}} 0 }
  \Downarrow_{\emptyset, \Phi_{\mathit{sync}}} \Psi^\prime_{\mathit{mono}} \, ,
\end{aligned}
\end{equation}
\begin{equation}\label{ex:case3}
\begin{aligned}
  & \mathcal{S}_{\mathit{imp}} \provesCRL
  [ \xi(\texttt{if (}{\color{blue}\mathit{false}}\texttt{)\{y++;x--;P\}}, X_1, Y_1),  
    \xi(\texttt{if (}{\color{blue}\mathit{true}}\texttt{)\{y++;x--;P\}}, X_2, Y_2) ] \\
  & \land X_1 \leq_{\mathit{Int}} X_2 \land Y_1 \leq_{\mathit{Int}} Y_2 \land {\color{blue} X_1 \leq_{\mathit{Int}} 0 \land X_2 >_{\mathit{Int}} 0 }
  \Downarrow_{\emptyset, \Phi_{\mathit{sync}}} \Psi^\prime_{\mathit{mono}} \, ,
\end{aligned}
\end{equation}
\begin{equation}\label{ex:case4}
\begin{aligned}
  & \mathcal{S}_{\mathit{imp}} \provesCRL
  [ \xi(\texttt{if (}{\color{blue}\mathit{false}}\texttt{)\{y++;x--;P\}}, X_1, Y_1) ,  
    \xi(\texttt{if (}{\color{blue}\mathit{false}}\texttt{)\{y++;x--;P\}}, X_2, Y_2) ] \\
  & \land X_1 \leq_{\mathit{Int}} X_2 \land Y_1 \leq_{\mathit{Int}} Y_2 \land {\color{blue} X_1 \leq_{\mathit{Int}} 0 \land X_2 \leq_{\mathit{Int}} 0 }
  \Downarrow_{\emptyset, \Phi_{\mathit{sync}}} \Psi^\prime_{\mathit{mono}} \, .
\end{aligned}
\end{equation}
\begin{itemize}
\item The case in \Cref{ex:case4} represents the situation when both loops have finished their execution.
We can solve this case by symbolically executing both programs (using the Step rule) to the end.
It is easy to see that then the premise implies the conclusion; therefore, we finish this case
by generalizing the premise (using the Conseq rule) to be exactly the conclusion, and then applying
the Reflexivity rule.
\item The case in \Cref{ex:case2} represents the situation when the first loop continues execution and the second is finished.
      This can never happen - we see that the side condition is contradictory.
      We finish this case using the Reduce rule.
\item The case in \Cref{ex:case3} represents the complementary situation - the first loop has finished its execution,
      while the second loop continues.
      This requires inventing an invariant capturing the idea that the execution of the second loop can only increase
      the difference between the values of \texttt{y}.
      We can Reduce this subgoal to simple RL reasoning. We could also prove this case without Reduce,
      using the other rules, but lockstep reasoning does not help there.
\item The case in \Cref{ex:case1} is the one when we utilize the ability to perform lockstep reasoning.
      We symbolically execute both components until their program parts become the \texttt{while} loops again;
      that is, $P$.
      The goal is now
      \begin{align*}
        & \mathcal{S}_{\mathit{imp}} \provesCRL
        [ \xi(P, X_1 -_{\mathit{Int}} 1, Y_1 +_{\mathit{Int}} 1 ) ,  
          \xi(P, X_2 -_{\mathit{Int}} 1, Y_2 +_{\mathit{Int}} 1) ] \\
        & \land X_1 \leq_{\mathit{Int}} X_2 \land Y_1 \leq_{\mathit{Int}} Y_2 \land {X_1 \leq_{\mathit{Int}} 0 \land X_2 \leq_{\mathit{Int}} 0 }
        \Downarrow_{\emptyset, \Phi_{\mathit{sync}}} \Psi^\prime_{\mathit{mono}} \, .
      \end{align*}
      which implies the synchronization point $\Phi_{\mathit{sync}}$ which is already enabled.
      We can therefore conclude the proof using Conseq and Axiom.
      %their program parts
      %become $\xi(P, X_1, Y_1)$ and $\xi(P, X_2, Y_2)$, respectively.
\end{itemize}
We observe that the proof is rather low-level.
On the other hand, the proof system itself is very simple,
and one can prove derived rules that raise the abstraction level, as we show in \Cref{app:derivedRules}.

%%% Local Variables:
%%% mode: latex
%%% TeX-master: "../main"
%%% End:
