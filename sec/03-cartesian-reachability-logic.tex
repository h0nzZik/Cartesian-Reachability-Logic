\section{Cartesian Reachability Logic}
In this section we introduce \emph{Cartesian Reachability logic (CRL)} - a language-parametric logic for reasoning
about $k$-safety hyperproperties.
Our aim with CRL is to make reasoning in the style of \emph{Cartesian Hoare logic (CHL)}~\cite{SousaD16} available for any
deterministic language with given reachability-logic semantics $S$.
For that purpose we define the language of CRL and its semantics, and demonstrate the logic's expressiveness
on a couple of examples.
Then we give CRL a sound proof system, which is the main contribution of this paper.


\subsection{Syntax and Semantics}

Cartesian reachability logic is an extension of (one-path) reachability
logic. In this extension, we replace reachability rules $\varphi
\Rightarrow^{\exists} \varphi$ with  \emph{reachability claims} of the form
\begin{equation*}
  [\varphi_1,\ldots,\varphi_k] \land P
  \Rightarrow^{c\exists} \exists \vec{Y}.\, [\varphi^\prime_1,\ldots,\varphi^\prime_k] \land P^\prime
\end{equation*}

The intuitive meaning of such a claim should be clear enough: There are $k$
programs, the set of source configurations of $i$-th program matching $\varphi_i$
and target configurations matching $\varphi'_i$. Additionally, the (TODO -
matching logic) formula $P$ can relate source configurations, and $P'$ the
target configurations. We call formulas of the form   $ [\varphi_1,\ldots,\varphi_k] \land P$
\emph{existentially-quantified constrained list patterns} (ECLP). 


For example, consider the same program as in \Cref{eqn:CounterProgram}; that is, let
\begin{equation*}
  P \equiv \texttt{while(x > 0)\{ y++; x--;\}} \, .
\end{equation*}
Additionally, let $C(Q, X, Y)$ represent a configuration of a program $Q$
where the program variable $\texttt{x}$ has the value $X$
and the program variable $\texttt{y}$ has the value $Y$;
for example,
\begin{equation}\label{eqn:CQXY}
 C(Q, X, Y) \equiv \texttt{<k>} Q \texttt{</k><st>(x |-> } X \texttt{)(y |-> } Y \texttt{)</st>}    \, .
\end{equation}
Then, the claim $\Omega_{\textit{mono}}$, defined as
\begin{align*}
&[C(P, X_1, Y_1),C(P, X_2, Y_2)] \land X_1 \leq X_2 \land Y_1 \leq Y_2
\\ \Rightarrow^{c\exists} &
\exists X^\prime_1, Y^\prime_1, X^\prime_2, Y^\prime_2.\,  [C(\epsilon, X^\prime_1, Y^\prime_1), C(\epsilon, X^\prime_2, Y^\prime_2)] \land Y^\prime_1 \leq Y^\prime_2   
\end{align*}
(where $\epsilon$ denotes the empty program)
expresses the property that the program $P$ is monotone.
That is, when we start an execution (using the semantics of the particular language)
from some configuration $\gamma_1$ matching $C(P, X_1, Y_1)$
and a second execution from some configuration $\gamma_2$ matching $C(P, X_2, Y_2)$,
if $X_1 \leq X_2$ and $Y_1 \leq Y_2$,
we end up in configurations $\gamma_1^\prime,\gamma_2^\prime$ matching
$C(\epsilon, X^\prime_1, Y^\prime_1)$ and $C(\epsilon, X^\prime_2, Y^\prime_2)$
for some $X^\prime_1,Y^\prime_1,X^\prime_2,Y^\prime_2$
satisfying $Y^\prime_1 \leq Y^\prime_2$.

We formally define the semantics of a CRL claim as follows:
\begin{definition}[CRL semantics]\label{def:opCRLsemantics}
    A claim
    \begin{equation*}
     [\varphi_1,\ldots,\varphi_k] \land P
     \Rightarrow^{c\exists} \exists \vec{Y}.\, [\varphi^\prime_1,\ldots,\varphi^\prime_k] \land P^\prime
    \end{equation*}
    is \emph{valid} in a reachability system $\mathcal{S} = (\mathcal{T}, S)$,
    written
    \begin{equation*}
        \mathcal{S} \vDash_\CRL [\varphi_1,\ldots,\varphi_k] \land P
     \Rightarrow^{c\exists} \exists \vec{Y}.\, [\varphi^\prime_1,\ldots,\varphi^\prime_k] \land P^\prime \, ,
    \end{equation*}
    iff for all configurations $\gamma_1,\ldots,\gamma_k \in \Tcfg$
    which terminate in $(\Tcfg, \Rightarrow_{\mathcal{S}})$
    and any $\mathcal{T}$-valuation $\rho$,
    whenever $(\gamma_1,\rho) \vDash \varphi_1 \land P$ and \ldots
    and $(\gamma_k,\rho) \vDash \varphi_k \land P$,
    then there exist configurations $\gamma_1^\prime,\ldots,\gamma_k^\prime \in \Tcfg$
    such that $\gamma_1 \Rightarrow^{*}_{\mathcal{S}} \gamma_1^\prime$
    and \ldots and $\gamma_k \Rightarrow^{*}_{\mathcal{S}} \gamma_k^\prime$,
    and there also exists an $\mathcal{T}$-valuation $\rho^\prime$
    satisfying $\rho(v) = \rho^\prime(v)$ for any $v \in \mathit{Var} \setminus \vec{Y}$,
    and
    $(\gamma_1^\prime,\rho^\prime) \vDash \varphi^\prime_1 \land P^\prime$ and \ldots and $(\gamma_k^\prime, \rho^\prime) \vDash \varphi^\prime_k \land P^\prime$.
\end{definition}

\subsection{Comparison to CHL}\label{sec:CRLsemanticsComparisonToCHL}
CRL is more verbose then CHL. This is partly because of the need to specify patterns matching the whole program configurations,
and partly because of the need to existentially quantify those variables on the right side whose value is not
determined by the left side.
The second reason is related to the fact that in CRL, as well as in RL, a variable occuring in both sides
has the same value in both sides.
Therefore, in (C)RL, one does not need to introduce ghost variables for remembering values from the precondition.
We inherit a notation commonly used in RL, that variables whose names start with a question mark are implicitly
considered existentially quantified in the right side.
Also, we use underscore for a variable with unknown name, different between all occurences -
that is, to represent a variable whose value we are not interested in.
For example, we can write the claim $\Omega_{\textit{mono}}$ as
\begin{align*}
[C(P, X_1, Y_1),C(P, X_2, Y_2)] \land X_1 \leq X_2 \land Y_1 \leq Y_2
 \Rightarrow^{c\exists} [C(\epsilon, ?\_, ?Y_1), C(\epsilon, ?\_, ?Y_2)] \land ?Y_1 \leq ?Y_2 \, .
\end{align*}

A deeper distinction is that in the CRL semantics we existentially quantify over reachable configurations,
while in CHL, target states are quantified universally.
However, this distinction has no effect when working with deterministic languages.

\subsection{Comparison to Reachability logic}


We note here that in general, having one valid CRL claim is different from having just $k$ valid RL claims,
because in CRL, the variables are shared across components.
%Consider, for example, a reachability system $S_{\mathit{IMP}}$ representing a simple imperative language,
%and the same program as in \Cref{eqn:CounterProgram}; that is, let
%\begin{equation*}
%  P \equiv \texttt{while(x > 0)\{ y++; x--;\}} \, .
%\end{equation*}
%We can let $C(Q, X,Y)$ represent a configuration of a program $Q$ where the program variable $\texttt{x}$ has the value $X$
%and the program variable $\texttt{y}$ has the value $Y$;
%for example,
%\begin{equation*}
% C(Q, X, Y) \equiv \texttt{<k>} Q \texttt{</k><st>(x |-> } X \texttt{)(y |-> } Y \texttt{)</st>}    \, .
%\end{equation*}
%Then, the proposition
%\begin{align*}
% S_{\mathit{IMP}} \vDash_\CRL
%&[C(P, X_1, Y_1),C(P, X_2, Y_2)] \land X_1 \leq X_2
%\\ \Rightarrow^{c\exists} &
%\exists X^\prime_1, Y^\prime_1, X^\prime_2, Y^\prime_2.\,  [C(\epsilon, X^\prime_1, Y^\prime_1), C(\epsilon, X^\prime_2, Y^\prime_2)] \land Y^\prime_1 \leq Y^\prime_2   
%\end{align*}
%holds iff the program $P$ is monotone (when considering the variable $x$ to be an input, and $y$ to be an output; $\epsilon$ represents an empty program).
In CRL, one can localize the ``global'' constraints; for example, the claim $\Omega_{\textit{mono}}$
is equivalent to
\begin{align*}
[C(P, X_1, Y_1),C(P, X_2, Y_2) \land X_1 \leq X_2] 
\Rightarrow^{c\exists}
[C(\epsilon, ?\_, ?Y_1), C(\epsilon, ?\_, ?Y_2) \land ?Y_1 \leq ?Y_2] \, .
\end{align*}
However, if one were to ``split'' the CRL claim into two, the resulting claims might express
a different property than monotonicity.
For example, the two claims
\begin{align}
& C(Q, X_1, Y_1) & \Rightarrow^{\exists} \quad & C(\epsilon, ?X_1, ?Y_1) \\
& C(Q, X_2, Y_2) \land X_1 \leq X_2 & \Rightarrow^{\exists} \quad  & C(\epsilon, ?X_2, ?Y_2) \land ?Y_1 \leq ?Y_2
\end{align}
hold for any reasonable program $Q$ (meaning that $Q$ either executes fully or diverges),
because $?Y_1$ in the second claim is unrelated to $?Y_1$ in the first claim and thus one can set it to
the value of $?Y_2$.
On the other hand, if in the second claim we renamed $?Y_1$ into $Y_1^\prime$ (without a question mark),
the claim would require that $?Y_2$ is greater than or equal to \emph{any} integer (because $Y_1^\prime$ is not present in the left side),
which clearly cannot hold.

Still, reachability logic is a special case of CRL.
\begin{remark}
%\Cref{def:opCRLsemantics} extends \Cref{def:oprlSemantics} of \Cref{def:basics}:
%if we fix $k=1$, then
$
    (\mathcal{T}, S) \vDash_\CRL
    [\varphi] \land \top  \Rightarrow^{c\exists}
    [\varphi^\prime] \land \top
    \iff
    (\mathcal{T}, S) \vDash_\RL \varphi \Rightarrow^{\exists} \varphi^\prime \, .
$
\end{remark}

%\begin{definition}[All-Path Cartesian Reachability Rule]\label{def:apCRLsemantics}
%An all-path Cartesian reachability rule
%$(\varphi_1,\ldots,\varphi_k) \land \varphi \Rightarrow^{c\forall} (\psi_1,\ldots,\psi_k) \land \psi$
%of arity $k$
%is \emph{valid} in a reachability system $\mathcal{S} = (\mathcal{T}, S)$,
%written
%$\mathcal{S} \vDash_\CRL (\varphi_1,\ldots,\varphi_k) \land \varphi \Rightarrow^{c\forall}
%(\psi_1,\ldots,\psi_k) \land \psi$,
%iff for all configurations $\sigma_1,\ldots,\sigma_k \in \Tcfg$ \traian{Why $\sigma$ instead of $\gamma$?}
%and any $\mathcal{T}$-valuation $\rho$,
%whenever $(\sigma_1, \rho) \vDash \varphi_1 \land \varphi$ and \ldots
%and $(\sigma_k, \rho) \vDash \varphi_k \land \varphi$,
%then for every $k$-tuple of complete paths $(\pi_1, \ldots, \pi_k)$
%such that
%$\sigma_1 = \pi_1(0) \land \ldots \land \sigma_k = \pi_k(0)$,
%there exist indices $i_1, \ldots, i_k$ such that
%$(\pi_1(i_1), \rho) \vDash \varphi_1 \land \psi$ and \ldots and $(\pi_k(i_k), \rho) \vDash \varphi_k \land \psi$.
%\end{definition}

Now we present a novel, general technique called \emph{star extension} that is reminiscent of self-composition~\cite{BartheDR11}.
Self-composition is a technique where a program $P$ together with a $k$-safety hyperproperty is reduced to
a sequential composition of $P$ with itself (with renamed variables) together with a safety property.
This technique allows one to use tools and techniques for verification of safety properties
to perform verification of $k$-safety hyperproperties.
The challenge here is to generalize self-composition to work with any deterministic language,
even if we do not know in advance how the language implements sequential composition, if at all.

The main idea of \emph{star extension} is to transform a CRL claim into a RL claim over an extended reachability system,
where configurations of the extended reachability system are lists of configurations of the original system.
The transformation is quite straightforward but technical, and we refer an interested reader to the \Cref{app:CRLandRLcorrespondence};
here we only present the main theorem.
\begin{theorem}\label{thm:CRLandRLcorrespondence}
  There exist a function $\_^*$ on matching logic signatures,
  a (equally-named) function $\_^*$ from reachability systems over $\Sigma$ to reachability systems over $\Sigma^*$,
  and a function $\mathit{flatten}$ from ECLPs over $\Sigma$ to matching logic $\Sigma^*$-formulas,
  such that
  \begin{equation*}
  \mathcal{S} \vDash_{\CRL} \Psi \Rightarrow^{c\exists} \Psi^\prime
    \iff \mathcal{S}^* \vDash_\RL \mathit{flatten}(\Psi) \Rightarrow^{c\exists} \mathit{flatten}(\Psi^\prime)
  \end{equation*}
\end{theorem}

The price paid for self-composition is that the property of the self-composed program is often hard to reason about.
Therefore, in~\cite{SousaD16}, the authors do not apply self-composition directly, but only use its soundness to justify
their technique - namely, the soundness of their proof system, which avoids explicit construction
of the self-composed programs.
We use the star extension for the same purpose.



\subsection{Proof System}




%We give Cartesian Reachability Logic a proof system to faciliate mechanical reasoning.
%One may ask the question, ``why give CRL a new proof system if one can perform the same reasoning
%by means of \Cref{thm:CRLandRLcorrespondence} and the existing proof system of reachability logic?''.
%The answer is the following.
%When one reduces a CRL goal into RL using \Cref{thm:CRLandRLcorrespondence},
%the function $\mathit{flatten}$ appears in the goal.
%To perform reasoning using the RL proof system, one then has to either simplify the goal
%by unfolding the definition of $\mathit{flatten}$,
%or use (and prove) some helper lemmas about the effect of applying RL proof rules
%%and other frequently used steps
%on RL goals containing $\mathit{flatten}$.
%In the first case one lowers the abstraction level and have to reason about matching logic formulas
%containing translations of other matching logic formulas into FOL, which then makes RL reasoning more complex.
%In the second case, however, one may end up proving lemmas which, when combined, result in a proof system for CRL.
%Indeed, the soundness of our proof system (\Cref{fig:CRLproofsystem})
%is established by a (meta-)proof which constructs a RL proof from a CRL one (\Cref{lem:CRLalmostSoundness}).


We give Cartesian Reachability Logic a proof system to faciliate mechanical reasoning.
While the intuition behind the semantics of CRL is similar to that of CHL, with only a few minor differences,
giving CRL a proof system is not straightforward.
One could attempt to reuse the proof system of CHL and modify it \emph{somehow} to be independent
of the particular programming language.
Since many CHL rules (e.g., its \texttt{If} rule shown in \Cref{chlRule:If}) are simply Hoare logic rules acting on a particular component of
a tuple of formulas (that is, symbolic states),
one could for example lift rules of reachability logic (RL) to the tuple-context and be done.
That would indeed work, and the resulting proof system might even be complete.

However, the distinguishing feature of Cartesian Hoare logic is \emph{not} its completeness,
but its ability to simplify reasoning by performing lock-step execution of loops,
because that can \textquote[\cite{SousaD16}]{greatly simplify the verification task (e.g., by requiring simpler invariants)}.
Simply (re)using RL rules lifted to the tuple-context does not provide any support for that.
And it is not entirely obvious how to support lock-step reasoning about construct with repetetive behavior:
in order to support while-with-breaks, Cartesian Hoare logic itself uses additionally (besides the lifted Hoare logic rules)
five fairly complex rules that need to understand the precise semantics of this construct.
The task for us is even harder, since we do not know in advance what constructs with repetetive behavior does
the supplied language support: \texttt{goto}s? Mutual recursion? 

Yet, we provide a single proof system (consisting of only eight rules) which enables lockstep reasoning about such constructs.
The proof system derives claims of the shape
\begin{equation*}
(\mathcal{T}, S) \vdash_\CRL \Phi \Downarrow_{C,E} \Psi
\end{equation*}
where $\Phi$ is of the shape
$(\varphi_1, \ldots, \varphi_k) \land P$
and $\Psi$ is of the shape
$\exists \vec{Y}.\, (\varphi^\prime_1, \ldots, \varphi^\prime_k) \land P^\prime$.
One can think about $\Phi$ as representing a \emph{premise}, while $\Psi$, which propagates through
the proof rules unchanged, as representing a \emph{conclusion}.
Every $\varphi_i, \varphi^\prime_i$ is a matching logic pattern representing a particual component,
and $P$ and $P^\prime$ are FOL formulas (\emph{global constraints}) relating variables from different components.
The sets $C$ and $E$ contain \emph{cutpoints} and \emph{enabled cutpoints}, respectively.
Ttogether, they implement the concept of a \emph{relational invariant}.
In particular, the set $C$ represents the invariants that were postulated \emph{right now},
while the set $E$ represents those that were postulated in past and are ready to be used.
Initially, the proof search starts with $E = C = \emptyset$.

\begin{figure}
    \centering
    \begin{align*}
    & \prftree[l]{Reflexivity}{(\mathcal{T}, S) \vdash_\CRL \Psi \Downarrow_{\emptyset,E} \Psi}
    \end{align*}
    \begin{align*}
    & \prftree[l]{Axiom}{\Psi \in E}{(\mathcal{T}, S) \vdash_\CRL \Psi \Downarrow_{C,E} \Psi^\prime}
    \end{align*}
    \begin{align*}
    & \prftree[l]{Reduce}
      {(\mathcal{T}^*, S^* \cup \mathit{flatten}^\exists(E, \Psi^\prime)), \emptyset \vdash_\RL
        \mathit{flatten}^\exists(\Psi, \Psi^\prime) }
      {(\mathcal{T}, S) \vdash_\CRL \Psi \Downarrow_{C,E} \Psi^\prime}
    \end{align*}
    \begin{align*}
    & \prftree[l]{Case}
    { \prfStackPremises
      {(\mathcal{T}, S) \vdash_\CRL [\varphi_1, \ldots, \varphi_{i-1}, \varphi_i, \varphi_{i+1}, \ldots, \varphi_k] \land P^\prime \Downarrow_{C, E} \Psi^\prime }
      {(\mathcal{T}, S) \vdash_\CRL [\varphi_1, \ldots, \varphi_{i-1}, \psi_i, \varphi_{i+1}, \ldots, \varphi_k] \land P^\prime \Downarrow_{C, E} \Psi^\prime }
    }
    {(\mathcal{T}, S) \vdash_\CRL [\varphi_1, \ldots, \varphi_{i-1}, (\varphi_i \lor \psi_i), \varphi_{i+1}, \ldots, \varphi_k] \land P^\prime \Downarrow_{C, E} \Psi^\prime}
    \end{align*}
    \begin{align*}
    & \prftree[l]{Step}
    { \prfStackPremises
       {\varphi \Rightarrow^\exists \varphi^\prime \in S}
       {\mathcal{T} \vDash_\ML \varphi_i \leftrightarrow \varphi \land P^\prime}
       {P^\prime \mbox{ is a FOL formula}}
       {  (\mathcal{T}, S) \vdash_\CRL [\varphi_1, \ldots, \varphi_{i-1}, \varphi^\prime \land P^\prime, \varphi_{i+1}, \ldots, \varphi_k]
          \land P
          \Downarrow_{\emptyset, (C \cup E)} \Psi^\prime
      }
    }
    {(\mathcal{T}, S) \vdash_\CRL [\varphi_1, \ldots, \varphi_{i-1}, \varphi_i, \varphi_{i+1}, \ldots, \varphi_k] \land P \Downarrow_{C, E} \Psi^\prime}
    \end{align*}
    \begin{align*}
    & \prftree[l]{Circularity}
      { (\mathcal{T}, S) \vdash_\CRL \Psi \Downarrow_{C \cup \{ \Psi \} , E} \Psi^\prime}
      { (\mathcal{T}, S) \vdash_\CRL \Psi \Downarrow_{C, E} \Psi^\prime}
    \end{align*}
    \begin{align*}
    & \prftree[l]{Conseq}
      { \prfStackPremises
        { (\mathcal{T}, S) \vdash_\CRL \Phi^\prime \Downarrow_{C, E} \Psi^\prime}
        { \mathcal{T}^* \vDash_\ML \mathit{flatten}(\Phi) \rightarrow \mathit{flatten}(\Phi^\prime) }
      }
      { (\mathcal{T}, S) \vdash_\CRL \Phi \Downarrow_{C, E} \Psi^\prime}
    \end{align*}
%    \begin{align*}
%    & \prftree[l]{Conseq2}
%      { \prfStackPremises
%        { (\mathcal{T}, S) \vdash_\CRL (\varphi_1, \ldots, \varphi_k) \land P \Downarrow_{C, E} \Psi^\prime}
%        { \mathcal{T} \vDash_\ML \varphi_i \leftrightarrow \psi_i }
%      }
%      { (\mathcal{T}, S) \vdash_\CRL (\varphi_1^\prime, \ldots, \varphi_{i-1}, \varphi_{i}, \varphi_{i+1}, \ldots, \varphi_k^\prime) \land P^\prime \Downarrow_{C, E} \Psi^\prime}
%    \end{align*}

    %\begin{align*}
    %& \prftree[l]{Abstract}
      %{ \prfStackPremises
        %{ X \not\in \mathit{FV}(\Psi^\prime, \varphi_1, \ldots, \varphi_{i-1}, \varphi_{i+1}, \ldots,\varphi_k)
        %}
        %{(\mathcal{T}, S) \vdash_\CRL (\varphi_1, \ldots, \varphi_{i-1}, \varphi_i, \varphi_{i+1}, \ldots, \varphi_k) \land P \Downarrow_{C, E} \Psi^\prime}
      %}
      %{(\mathcal{T}, S) \vdash_\CRL (\varphi_1, \ldots, \varphi_{i-1}, \exists X.\, \varphi_i, \varphi_{i+1}, \ldots, \varphi_k) \land P \Downarrow_{C, E} \Psi^\prime}
    %\end{align*}
    
    \begin{align*}
    & \prftree[l]{Abstract}
      { \prfStackPremises
        { X \not\in \mathit{FV}(\Psi^\prime)
        }
        {(\mathcal{T}, S) \vdash_\CRL \exists \vec{Y}.\, [\varphi_1, \ldots, \varphi_k] \land P \Downarrow_{C, E} \Psi^\prime}
      }
      {(\mathcal{T}, S) \vdash_\CRL \exists X, \vec{Y}.\, [\varphi_1, \ldots, \varphi_k] \land P \Downarrow_{C, E} \Psi^\prime}
    \end{align*}
    \caption{Proof System}
    \label{fig:CRLproofsystem}
\end{figure}

The rules itself are fairly simple, and none of them has the semantics of any particular language construct
hard-wired into it.
We now explain the proof rules of CRL (shown in~\Cref{fig:CRLproofsystem}) one by one.
\begin{itemize}
  \item The Circularity rule is the key that allows lockstep reasoning about arbitrary program constructs.
        It allows the user to postulate validity of the current claim by means of adding the \emph{current premise}
        into the set of \emph{cutpoints}, from which a $k$-tuple of program configurations satisfying the postcondition
        is claimed to be reachable. Once progress is made (by means of the Step rule), the added cutpoints are
        \emph{enabled} and can be used to finish the proof using the Axiom rule.
  \item The Step rule performs symbolic execution on a selected component $i$ (represented by $\varphi_i$)
        using the semantic rule $\varphi \Rightarrow^\exists \varphi^\prime \in S$.
        For this rule to apply, its left side ($\varphi$) has to match all the program configurations matching $\varphi_i$.
        Therefore, the rule decomposes $\varphi_i$ into $\varphi$ and an additional constraint $P^\prime$,
        which can be thought of as a part of \emph{path condition} that is \emph{local} to the component $i$.
        This local path condition $P^\prime$ is then used to constrain the right-side $\varphi^\prime$ of the selected rule.
        This proof rule also enables the cutpoints from $C$ by adding them to $E$.
  \item The Axiom rule uses an enabled cutpoint to finish the proof.
  \item The Reflexivity rule can be used to finish a proof when the premise corresponds to the conclusion.
  \item The Case rule implements case analysis on a selected component $i$.
  \item The Conseq rule is used to weaken (or generalize) the premise. It can also be used
        to propagate information between components and the global constraint.
        TODO explain flatten
  \item The Abstract rule can be used to remove existential quantifiers from the premise.
        Intuitively, this corresponds to a proof step in firstorder logic that replaces
        an existential quantifier on the left side of an implication
        with a universal quantifier over the implication, assuming that the variable
        bound by the existential quantifier does not occur free in the right side.
        In our setting, the typical way of \emph{obtaining} a toplevel existential quantifier in the premise
        is by means of the Conseq rule.
  \item The Reduce rule is a way to get completeness into our proof system:
        it reduces the goal to reachability logic reasoning.
        This rule also provides a way to prove properties that do not benefit
        from lockstep reasoning.

\end{itemize}


The following lemma says that one can generate an RL proof on a star-extended system
from a CRL proof.
This lemma is a major component of the soundness proof of CRL and can be found in \Cref{app:crlsoundness}.
\begin{lemma}\label{lem:CRLalmostSoundness}
    \begin{align*}
        & (\mathcal{T}, S) \vdash_\CRL \Psi \Downarrow_{C,E} \Psi^\prime \implies \\
        &
        (\mathcal{T}^*, S^* \cup \mathit{flatten}^\exists(E, \Psi^\prime)), \mathit{flatten}^\exists(C, \Psi^\prime) \vdash_\RL
          \mathit{flatten}^\exists(\Psi, \Psi^\prime) 
    \end{align*}
\end{lemma}

Now we can state and prove the soundness theorem, which is the main result of this paper.
\begin{theorem}[Proof system soundness]\label{thm:proofsystemSoundness}
\begin{equation*}
    (\mathcal{T}, S) \vdash_\CRL \Psi \Downarrow_{\emptyset,\emptyset} \Psi^\prime \implies
    (\mathcal{T}, S) \vDash_{\CRL} \Psi \Rightarrow^{c\exists} \Psi^\prime
\end{equation*}
\end{theorem}
\begin{proof}[Proof of \Cref{thm:proofsystemSoundness}]
Assume $(\mathcal{T}, S) \vdash_\CRL \Psi \Downarrow_{\emptyset,\emptyset} \Psi^\prime$.
By \Cref{lem:CRLalmostSoundness}, we have $(\mathcal{T}, S) \vdash_\RL \mathit{flatten}^\exists(\Psi, \Psi^\prime)$.
By soundness of reachability logic, we have $(\mathcal{T}, S) \vDash_\RL \mathit{flatten}^\exists(\Psi, \Psi^\prime)$.
By \Cref{thm:CRLandRLcorrespondence}, 
we have $(\mathcal{T}, S) \vDash_{\CRL} \Psi \Rightarrow^{c\exists} \Psi^\prime$ and we are done.
\end{proof}

% \begin{figure}
%   \begin{align*}
%     & \prftree[l]{GenWithCirc}
%       { \prfStackPremises
%         { \vec{X} = \mathit{FV}(\Phi) \setminus \mathit{FV}(\Psi^\prime)
%         }
%         {(\mathcal{T}, S) \vdash_\CRL \Phi \Downarrow_{C \cup \{ \exists \vec{X}.\, \Phi  \}, E} \Psi^\prime}
%       }
%       {(\mathcal{T}, S) \vdash_\CRL \Phi \Downarrow_{C, E} \Psi^\prime}
%   \end{align*}

%   \begin{align*}
%     & \prftree[l]{ImplConclusion}
%       {\mathcal{T}^* \vDash_\ML \mathit{flatten}(\Phi) \rightarrow \mathit{flatten}(\Psi^\prime)}
%       {(\mathcal{T}, S) \vdash_\CRL \Phi \Downarrow_{C, E} \Psi^\prime}
%   \end{align*}

%   \begin{align*}
%     & \prftree[l]{ImplEnabled}
%       { \prfStackPremises
%         { \Phi^\prime \in E }
%         {\mathcal{T}^* \vDash_\ML \mathit{flatten}(\Phi) \rightarrow \mathit{flatten}(\Phi^\prime)}
%       }
%       {(\mathcal{T}, S) \vdash_\CRL \Phi \Downarrow_{C, E} \Psi^\prime}
%   \end{align*}

%   \begin{align*}
%     & \prftree[l]{Contradiction}
%       {\mathcal{T}^* \vDash_\ML \mathit{flatten}(\Phi) \rightarrow \bot }
%       {(\mathcal{T}, S) \vdash_\CRL \Phi \Downarrow_{C, E} \Psi^\prime}
%   \end{align*}

%   \caption{Selected derived rules}
%   \label{fig:CRLderivedRules}
% \end{figure}

The proof system is also relatively complete (with respect to an oracle for deciding validity
in the underlying matching logic model $\mathcal{T}$).
\begin{theorem}[Relative completeness]\label{thm:relativeCompleteness}
  \begin{equation*}
      (\mathcal{T}, S) \vDash_{\CRL} \Psi \Rightarrow^{c\exists} \Psi^\prime \implies
      (\mathcal{T}, S) \vdash_\CRL \Psi \Downarrow_{\emptyset,\emptyset} \Psi^\prime
  \end{equation*}
  \end{theorem}
  \begin{proof}[Proof of \Cref{thm:relativeCompleteness}]
  Assume $(\mathcal{T}, S) \vDash_{\CRL} \Psi \Rightarrow^{c\exists} \Psi^\prime$.
  By \Cref{thm:CRLandRLcorrespondence}, 
  we obtain
  \begin{equation*}
    (\mathcal{T}^*, S^*) \vDash_\RL
    \mathit{flatten}^\exists(\Psi, \Psi^\prime) \, .
  \end{equation*}
  By relative completeness of reachability logic, we obtain
  \begin{equation*}
    (\mathcal{T}^*, S^*) \vdash_\RL
    \mathit{flatten}^\exists(\Psi, \Psi^\prime) \, ,
  \end{equation*}
  and we conclude the proof using the Inherit rule.
  Note that to apply relative completeness of RL, we need have an oracle for deciding validity in the extended model.
  A construction of such oracle from the oracle for deciding validity in $\mathcal{T}$ is in \Cref{app:completeness}.
  \end{proof}
Our completeness result is similar to the completeness result of CHL in the sense that the completeness
does not involve features used for lockstep reasoning.
It would be interesting to investigate whether a proof system without the Reduce rule would still be complete;
we leave this for a future work.

\section{An example proof involving lockstep reasoning}

We now present an example proof using our proof system; the proof involves lockstep reasoning.
Consider again the claim $\Omega_{\mathit{mono}}$ from \Cref{sec:CRLsemanticsComparisonToCHL}:
\begin{align*}
  [C(P, X_1, Y_1),C(P, X_2, Y_2)] \land X_1 \leq X_2 \land Y_1 \leq Y_2
   \Rightarrow^{c\exists} [C(\epsilon, ?\_, ?Y_1), C(\epsilon, ?\_, ?Y_2)] \land ?Y_1 \leq ?Y_2 \, .
\end{align*}
Let $\Psi^\prime_{\mathit{mono}}$ denote the right side of the $\Rightarrow^{c\exists}$ above.
TODO


%%% Local Variables:
%%% mode: latex
%%% TeX-master: "../main"
%%% End: