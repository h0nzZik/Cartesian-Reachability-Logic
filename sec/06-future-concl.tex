\section{Future Work and Conclusion}

We have presented Cartesian Reachability logic - a logic for reasoning about $k$-safety hyperproperties
in any deterministic language equipped with a RL-based operational semantics.
The logic has a simple, sound, and complete proof system and allows lockstep reasoning
similar to Cartesian Hoare logic.
Instantiating CRL with a new language does not require any changes to the soundness proof;
therefore, CRL has the potential to significantly reduce the costs of the development of tools and techniques for $k$-safety verification.

In the future, we want to develop a variant of CRL that would not require the language to be deterministic.
We believe this to be viable because (1) CHL has some support for nondeterminism
and because (2) reachability logic, on which we base our work, has a newer variant that supports nondeterminism, too.
On the theoretical side, we would like to know whether our proof system would be complete even in the absence of
the Reduce rule.
An orthogonal line of future research is compositionality: we would like to enable compositional reasoning
using the technique developed in \cite{DOsualdoFD22}.
Finally, we plan to develop a practical, language-parametric tool implementing CRL,
using the \K{} semantic framework, and use it for verification of smart contracts
(typically written in deterministic languages);
we already have an early prototype\footnote{available at \url{https://github.com/h0nzZik/crl-tool/}}
capable of using lockstep reasoning for verification of the example from \Cref{sec:example}.
