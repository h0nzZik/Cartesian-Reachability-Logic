\section{Introduction}
Recent years have witnessed an increased interest in formal verification of
\emph{hyperproperties}~\cite{ClarksonS08}. Unlike properties, whose validity depends on a single
execution trace, hyperproperties can relate multiple program executions. A
particularly interesting class of hyperproperties are \emph{$k$-safety
  (hyper-)properties}~\cite{FinkbeinerHT19CanonicalKsafety,SousaD16,AgrawalB16RuntimeKSafetyHLTL,ClarksonS08}.
A $k$-safety property is a hyperproperty whose violation can be witnessed by a
$k$-tuple of execution traces.  Many \emph{security policies} - for example,
\emph{noninterference} (requiring that sensitive or privileged data do not influence
insensitive or unprivileged computations)
or
\emph{observational determinism} 
- are $k$-safety hyperproperties~\cite{ClarksonFKMRS14,ClarksonFKMRS14TR,ClarksonS08}.
Similarly, many functional correctness properties are actually $k$-safety
hyperproperties; for example, \emph{transitivity} (which needs to by satisfied
e.g. by comparators when managing data in collections), associativity
(important in the map/reduce paradigm), or monotonicity~\cite{SousaD16}.


Techniques and tools for verifying hyperproperties of (finite-state) hardware
\cite{CoenenFST19,FinkbeinerRS15}, as well as (infinite-state) software
systems have been developed.  For verification of software systems in
particular, \emph{Cartesian Hoare logic} (CHL), introduced in \cite{SousaD16},
extends Hoare logic in order to allow reasoning about \emph{$k$-safety
  hyperproperties}. In \cite{SousaD16} the authors not only managed to give
CHL a sound and relatively complete proof system, but also successfully used
their logic to analyse several natural $k$-safety properties of Java
programs. (The verification algorithm was even implemented in a fully automated
tool.) To formalize Cartesian Hoare Logic, \cite{SousaD16} uses a simple
imperative language, whose looping constructs are while loops with
breaks. However, extending the approach of~\cite{SousaD16} to
other constructs affecting control flow, or indeed other programming
languages, can be both highly non-trivial and time consuming.


On the other hand, there have been recent developments in the area of
\emph{language-paremetric} software verification.  \emph{Reachability logic}
(RL) \cite{RosuS12oopsla,RosuSCM13lics,StefanescuCMMSR19} is a formalism for
reasoning about partial correctness of software, in the spirit of Hoare
logics.  Being implemented in the \K{} framework \cite{KVision}, its biggest
advantage is that reachability logic is \emph{language parametric}: its proof
system can be used unchanged to reason about programs in any language, as long
as the language has a formal semantics in RL.  Therefore, researchers no
longer need to think about a particular language construct three times (once
for the operational semantics, once for axiomatic, and once for the
correspondence); additionally, a single researcher or an architect of a tool does not
need to understand both the precise (and often intricate) semantics of a
programming language, \emph{and} formal verification techniques, which makes
\emph{division of labour} possible.  Through \K{}, reachability logic has been
used to build verifiers for real-world languages, such as C (\cite{RVMatch}),
Java (\cite{StefanescuPYLR16VerifiersForAll}), JavaScript
(\cite{StefanescuPYLR16VerifiersForAll}), and EVM
(\cite{KevmVerificationTool}).

In this paper we argue that we can indeed have the best of both worlds.  We
propose a new logic called \emph{Cartesian Reachability logic (CRL)}, which
properly extends reachability logic to allow reasoning about $k$-safety
hyperproperties. Similarly to CHL, CRL has a sound and relatively complete
proof system. A major advantage of CRL against CHL is that it works with any
deterministic
language for which RL works; that is, with any deterministic language which has a RL-based
formal semantics.  However, CRL does \emph{not} extend CHL, because the two
logics give different semantics to properties of nondeterministic programs;
despite this distinction, CRL extends CHL on the deterministic fragment of the
CHL-supported language.  We elaborate on this relation in
\Cref{sec:discussion}.  We draw our inspiration from the literature on
language-independent verification of partial correctness
(\cite{RosuS12oopsla,RosuSCM13lics,StefanescuCMMSR19}) and program equivalence
(\cite{CiobacaLRR16,CiobacaLRR14}).  


\paragraph{Contributions} The approach of our paper can be summarized as follows:
\begin{itemize}
\item We propose Cartesian Reachability Logic, an extension of reachability
  logic for reasoning about k-safety properties along the lines of Cartesian
  Hoare Logic.
\item We define a \emph{language-agnostic equivalent of self-composition}
  (\cite{BartheDR11}) (Section~\ref{sec:self-composition}) and establish a relation between CRL validity of the
  original and RL validity of the composed system.
\item We give CHL sound and relatively complete \emph{proof system}
  (Section~\ref{sec:proof-system}). The proofs in this proof system can be
  translated to ordinary RL proofs of the composed system (for soundness), but
  also such that it allows relatively high-level reasoning about
  \emph{circular behaviour} and \emph{lockstep execution} (for ease of
  verification and simplicity of invariants).
\end{itemize}



%%% Local Variables:
%%% mode: latex
%%% TeX-master: "../main"
%%% End:
