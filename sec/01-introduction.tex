\section{Introduction}
Recent years have witnesssed an increased interest in formal verification of \emph{hyperproperties}.
A hyperproperty is a set of \emph{sets of} execution traces, with the following semantics:
a system $S$ (a set of traces) satisfies a hyperproperty $H$ iff $S \in H$.
Hyperproperties are strictly more expressive then properties; for example, many security policies
of interest are hyperproperties but not properties.

Techniques and tools for verifying hyperproperties of (finite-state) hardware \cite{CoenenFST19,FinkbeinerRS15},
as well as (infinite-state) software systems have been developed.
For verification of software systems in particular, \emph{Cartesian Hoare logic} (CHL, \cite{SousaD16}) extends Hoare logic
to reason about \emph{$k$-safety hyperproperties}
(a hyperproperty is a \emph{$k$-safety hyperproperty} if its violation can be witnessed
by a set of traces of size $k$).
However, it is not clear how to use these techniques with arbitrary programming languages,
since CHL is tied to a particular imperative language.

On the other hand, there have been recent developments in the are of \emph{language-independent} software verification.
\emph{Reachability logic} (RL) \cite{RosuS12oopsla,RosuSCM13lics,StefanescuCMMSR19}
is a
formalism for reasoning about partial correctness of software, in the spirit of Hoare logics.
Being implemented in the \K{} framework \cite{KVision},
its biggest advantage is that reachability logic is \emph{language independent}:
its proof system can be used unchanged to reason about programs in any language,
as long as the language has a formal semantics in RL.
Therefore, researchers no longer need to think about a particular language construct three times
(once for the operational semantics, once for axiomatic, and once for the correspondence);
additionally, a single researcher or a tool-builder does not need to understand both the precise (and often intricate)
semantics of a programming language, \emph{and} formal verification techniques,
which makes \emph{division of labour} possible.
Through \K{}, reachability logic has been used to build verifiers for real-world languages,
such as C (\cite{RVMatch}), Java (\cite{StefanescuPYLR16VerifiersForAll}), Javascript (\cite{StefanescuPYLR16VerifiersForAll}),
and EVM (\cite{KevmVerificationTool}).

Can we have the best of both worlds? We think it is possible.
In this paper we develop a logic named \emph{Cartesian Reachability logic (CRL)}, which properly extends reachability logic
to allow reasoning about $k$-safety hyperproperties. Simiarly to CHL, CRL has a sound
and relatively complete proof system. A major advantage of CRL against CHL is that it works with any language
for which RL works; that is, with any language which has a RL-based formal semantics.
However, CRL does \emph{not} extend CHL, because the two logics give different semantics to properties of nondeterministic programs;
despite this distinction, CRL extends CHL on the deterministic fragment of the CHL-supported language.
We elaborate on this relation in \Cref{sec:discussion}. 

The idea behind soundness and completeness of our approach is similar to the one of \cite{SousaD16} for CHL.
We define a language-agnostic equivalent of self-composition (\cite{BartheDR11}) and establish a relation
between CRL validity of the original and RL validity of the composed system.
Then we construct a proof system such that its proofs can be translated to ordinary RL proofs of the composed system
(for soundness),
but also such that it allows relatively high-level reasoning about \emph{circular behavior} and \emph{lock-step execution}
(for ease of verification and simplicity of invariants).
We draw our inspiration from the literature on language-independent verification of partial correctness
(\cite{RosuS12oopsla,RosuSCM13lics,StefanescuCMMSR19})
and program equivalence (\cite{CiobacaLRR16,CiobacaLRR14}).
Again, we elaborate on the similarities and differences in \Cref{sec:discussion}. 