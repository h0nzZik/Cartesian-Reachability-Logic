\documentclass{easychair}
\usepackage[utf8]{inputenc}
\usepackage{amsmath,amssymb,mathtools}
\usepackage{amsthm}
\usepackage{thmtools}
\usepackage{hyperref}
% `cleveref` has to be loaded after `hyperref`
\usepackage{cleveref}
\usepackage{stackengine}
%\usepackage{minted}
\usepackage{tcolorbox}
\usepackage{xcolor}
\usepackage{multicol}
\usepackage{prftree}
\usepackage{fancyvrb}
\usepackage{csquotes}
\usepackage{appendix}
\usepackage{enumitem}
\usepackage{todonotes}


\usepackage{amssymb}

\newcommand{\traian}[1]{\todo[author=Traian]{#1}}
\newcommand{\jt}[1]{\todo[author=Jan]{#1}}

\title{Cartesian Reachability Logic: A Language-Parametric Logic for Verifying $k$-Safety Properties}
\author{
  Jan Tu\v{s}il \inst{1}
  \and Traian Florin Șerbănuță  \inst{2}
  \and Jan Obdržálek \inst{1}
}

\institute{
  Masaryk University,
  Brno, Czech Republic\\
  \email{jan.tusil@mail.muni.cz,obdrzalek@fi.muni.cz}
\and
   University of Bucharest,
   Bucharest, Romania\\
   \email{traian.serbanuta@unibuc.ro}\\
 }

\authorrunning{Tušil, Șerbănuță, and Obdržálek}
\titlerunning{CRL: A Language-Independent Logic for $k$-Safety}

%\date{\today}

\declaretheorem[]{example}
\declaretheorem[]{definition}
\declaretheorem[]{lemma}
\declaretheorem[]{theorem}
\declaretheorem[]{remark}
\declaretheorem[]{proposition}
\declaretheorem[]{notation}
\declaretheorem[]{assumption}

\newcommand{\K}{$\mathbb{K}$}
\newcommand{\RL}{\mathsf{RL}}
\newcommand{\ML}{\mathsf{ML}}
\newcommand{\CRL}{\mathsf{CRL}}
\newcommand{\FOL}{\mathsf{FOL}}
\newcommand{\FOLeq}{\FOL_{=}}
\newcommand{\Var}{\mathit{Var}}
\newcommand{\Tcfg}{\mathcal{T}_{\mathit{Cfg}}}
\newcommand{\Pattern}{\mathsf{Pattern}}

% We cannot do this, because in the appendix we consistently use \phi to denote the 'structural' part of a pattern
%\renewcommand{\phi}{\varphi}


\newcommand\oast{\stackMath\mathbin{\stackinset{c}{0ex}{c}{0ex}{\ast}{\bigcirc}}}

\newenvironment{proofenv}
  {
    \VerbatimEnvironment\begin{tcolorbox}[colback=black!0!white] % 5 is the default
  }
  {
   \end{tcolorbox}
  }


% CHL macros
\newcommand{\chl}[3]{\langle #1\rangle\ #2\ \langle#3\rangle}
\newcommand{\st}{\circledast}

\newcommand{\impConfig}[2]{\ll#1\mid #2\gg}
% term-as-predicate
\newcommand{\tap}[2]{\impConfig{\texttt{#1}}{#2}}
%\newcommand{\tap}[2]{\ll\texttt{#1} \mid #2\gg}



  
\begin{document}

\maketitle


\begin{abstract}

  In recent years, formal verification of hyperproperties has become an
  important topic in the formal methods community.  An interesting class of
  hyperproperties is known as $k$-safety properties, which express the absence
  of a \emph{bad} $k$-tuple of execution traces.  Many security policies, such
  as \emph{noninterference}, and functional properties, such as commutativity,
  monotonicity, and transitivity, are $k$-safety properties. A prominent example
  of a logic which can reason about $k$-safety properties of software systems
  is Cartesian Hoare Logic (CHL). However, CHL targets a specific, small
  imperative language. In order to use it for sound verification of programs
  in a different language, one needs to extend it with the desired features,
  or to hand-craft a translation. Both these approaches require a lot of
  tedious, error-prone work.

  In this paper we introduce a sound and relatively complete calculus called
  Cartesian Reachability Logic, a proper extension of Reachability Logic.
  Unlike CHL, Cartesian Reachability Logic is language-parametric: it can be
  instantiated with any deterministic language whose operational semantics is
  given by Reachability Logic rules such as, e.g.,
  EVM --- the language powering the Ethereum blockchain.
  This approach
  % JT: I would not mention K in such way, since it might suggest that we have implemented
  % CRL in K. (We have, but we do not want to draw attention to it yet.)
  %, used by tools like \K{} framework,
  can significantly reduce
  the development costs of tools and techniques for sound $k$-safety
  verification.
  
    %Its soundness theorem is proved once and for all, with no need to adapt or re-prove it
  % for different languages or their variants.

  
  % Uhas a sound and complete proof system and a performant verification algorithm.
  % However, CHL targets only a specific, tiny imperative language, and to use it for sound verification
  % of programs in a different language $L$, one needs to extend it with the desired features,
  % or to hand-craft a translation.
  % Both these approaches require a lot of tedious, error-prone work.
  % An even worse alternative is to manually implement the verification algorithm,
  % designed to work with the tiny language, with ad-hoc modifications targeting $L$:
  % that gives the user of the implementation no guarantee at all.

\end{abstract}

% \begin{abstract}
%   In recent years, formal verification of hyperproperties has become an important topic
%   in the formal methods community.
%   An interesting class of hyperproperties is known as $k$-safety:
%   a $k$-safety property prohibits the existence of a \emph{bad} $k$-tuple of execution traces.
%   Many security policies, such as \emph{noninterference}, 
%   and functional properties, such as \emph{commutativity}, are $k$-safety.
  
%   One logic for reasoning about $k$-safety hyperproperties of software
%   is Cartesian Hoare logic (CHL).
%   CHL has a sound and complete proof system and a performant verification algorithm.
%   However, CHL targets only a specific, tiny imperative language, and to use it for sound verification
%   of programs in a different language $L$, one needs to either extend it with the desired features,
%   or to hand-craft a translation. Both these approaches require a lot of tedious, error-prone work.
%   An even worse alternative is to manually implement the verification algorithm,
%   designed to work with the tiny language, with ad-hoc modifications targeting $L$:
%   that gives the user of the implementation no guarantee at all.
  
%   Therefore we introduce Cartesian Reachability logic (CRL): a new program logic for reasoning
%   about $k$-safety hyperproperties.
%   The distinguishing feature of CRL is that it can be instantiated with any deterministic language
%   whose model can be specified in a particular "operational semantics"-like formalism.
%   Its soundness theorem is proved once and for all, with no need to adapt or re-prove it
%   for different languages or their variants.
%   This powerful instrument can significantly reduce the development costs of tools and techniques
%   for sound $k$-safety verification.
% \end{abstract}


%\begin{abstract}
%  Crafting a program logic targeting a specific language can be a challenging and time-consuming task.
%  In this paper we introduce a program logic for reasoning about $k$-safety hyperproperties,
%  with the distinguishing feature that our logic works with any deterministic language
%  that has a Reachability logic-based formal semantics.
%  This powerful instrument can significantly reduce development costs of tools and techniques for k-safety verification.
%\end{abstract}

\section{Introduction}
Recent years have witnessed an increased interest in formal verification of
\emph{hyperproperties}~\cite{ClarksonS08}. Unlike properties, whose validity depends on a single
execution trace, hyperproperties can relate multiple program executions. A
particularly interesting class of hyperproperties are \emph{$k$-safety
  (hyper-)properties}~\cite{FinkbeinerHT19CanonicalKsafety,SousaD16,AgrawalB16RuntimeKSafetyHLTL,ClarksonS08}
(first introduced in \cite{ClarksonS08}).
A $k$-safety property is a hyperproperty whose violation can be witnessed by a
$k$-tuple of execution traces.  Many \emph{security policies} - for example,
\emph{noninterference} (requiring that sensitive or privileged data do not influence
insensitive or unprivileged computations)
or
\emph{observational determinism} 
- are $k$-safety hyperproperties~\cite{ClarksonFKMRS14,ClarksonFKMRS14TR,ClarksonS08}.
Similarly, many functional correctness properties are actually $k$-safety
hyperproperties; for example, \emph{transitivity} (which needs to by satisfied
e.g. by comparators when managing data in collections), associativity
(important in the map/reduce paradigm), or monotonicity~\cite{SousaD16}.


Techniques and tools for verifying hyperproperties of (finite-state) hardware
\cite{CoenenFST19,FinkbeinerRS15}, as well as (infinite-state) software
systems have been developed.  For verification of software systems in
particular, \emph{Cartesian Hoare logic} (CHL), introduced in \cite{SousaD16},
extends Hoare logic in order to allow reasoning about \emph{$k$-safety
  hyperproperties}. In \cite{SousaD16} the authors not only managed to give
CHL a sound and relatively complete proof system, but also successfully used
their logic to analyse several natural $k$-safety properties of Java
programs. (The verification algorithm was even implemented in a fully automated
tool.) To formalize Cartesian Hoare Logic, \cite{SousaD16} uses a simple
imperative language, whose looping constructs are while loops with
breaks. However, extending the approach of~\cite{SousaD16} to
other constructs affecting control flow, or indeed other programming
languages, can be both highly non-trivial and time consuming.


On the other hand, there have been recent developments in the area of
\emph{language-paremetric} software verification.  \emph{Reachability logic}
(RL) \cite{RosuS12oopsla,RosuSCM13lics,StefanescuCMMSR19} is a formalism for
reasoning about partial correctness of software, in the spirit of Hoare
logics.  Being implemented in the \K{} framework \cite{KVision}, its biggest
advantage is that reachability logic is \emph{language parametric}: its proof
system can be used unchanged to reason about programs in any language, as long
as the language has a formal semantics in RL.  Therefore, researchers no
longer need to think about a particular language construct three times (once
for the operational semantics, once for axiomatic, and once for the
correspondence); additionally, a single researcher or an architect of a tool does not
need to understand both the precise (and often intricate) semantics of a
programming language, \emph{and} formal verification techniques, which makes
\emph{division of labour} possible.  Through \K{}, reachability logic has been
used to build verifiers for real-world languages, such as C (\cite{RVMatch}),
Java (\cite{StefanescuPYLR16VerifiersForAll}), JavaScript
(\cite{StefanescuPYLR16VerifiersForAll}), and EVM
(\cite{KevmVerificationTool}).

In this paper we argue that we can indeed have the best of both worlds.  We
propose a new logic called \emph{Cartesian Reachability logic (CRL)}, which
properly extends reachability logic to allow reasoning about $k$-safety
hyperproperties. Similarly to CHL, CRL has a sound and relatively complete
proof system. A major advantage of CRL against CHL is that it works with any
deterministic
language for which RL works; that is, with any deterministic language which has a RL-based
formal semantics. 

CRL does \emph{not} extend CHL, because the two
logics give different semantics to properties of nondeterministic programs;
despite this distinction, CRL extends CHL on the deterministic fragment of the
CHL-supported language.  We elaborate on this relation in
\Cref{sec:discussion}.  We draw our inspiration from the literature on
language-independent verification of partial correctness
(\cite{RosuS12oopsla,RosuSCM13lics,StefanescuCMMSR19}) and program equivalence
(\cite{CiobacaLRR16,CiobacaLRR14}).  


\paragraph{Contributions} The approach of our paper can be summarized as follows:
\begin{itemize}
\item We propose Cartesian Reachability Logic, an extension of reachability
  logic for reasoning about k-safety properties along the lines of Cartesian
  Hoare Logic.
\item We define a \emph{language-agnostic equivalent of self-composition}
  (\cite{BartheDR11}) (Section~\ref{sec:self-composition}) and establish a relation between CRL validity of the
  original and RL validity of the composed system.
\item We give CHL sound and relatively complete \emph{proof system}
  (Section~\ref{sec:proof-system}). The proofs in this proof system can be
  translated to ordinary RL proofs of the composed system (for soundness), but
  also such that it allows relatively high-level reasoning about
  \emph{circular behaviour} and \emph{lockstep execution} (for ease of
  verification and simplicity of invariants).
\end{itemize}



%%% Local Variables:
%%% mode: latex
%%% TeX-master: "../main"
%%% End:

\section{Preliminaries}

%\subsection{$k$-safety hyperproperties}


\subsection{Cartesian Hoare logic}

Introduced in~\cite{SousaD16}, Cartesian Hoare Logic is a formalism for
specifying and reasoning about k-safety properties, in a similar way as Hoare
logic is used to reason about (safety) properties. In Hoare logic, properties are
specified by means of so-called \emph{Hoare triples}.  These have the shape
$\{ \varphi \} S \{ \psi \}$, with the meaning that the formula $\psi$ holds
in any state after the termination (if any) of the program $S$, executed from
a state satisfying $\varphi$.  In Cartesian Hoare logic, the situation is
similar: one can specify a triple
$\langle \Phi \rangle\ S_1 \st \cdots \st S_k\ \langle \Psi \rangle$ with the
following meaning: for any $k$-tuple $(\sigma_1,\ldots,\sigma_k)$ of states
satisfying the formula $\Phi$, if we execute each program $S_i$ in the
respective state $\sigma_i$ and they all terminate, then the $k$-tuple
$(\sigma_1^\prime,\ldots,\sigma_k^\prime)$ of the resulting program states
satisfies $\Psi$ \footnote{An important technical assumption here is that
  every program $S_i$ operates on its own set of program variables, distinct
  from variables of other programs $S_j$ (for $i \not = j$) - otherwise, the
  formulas $\Phi$ and $\Psi$ would not be able to distinguish between program
  variables of different programs.}.
%To make specifications more compact, one can also consider triples $|| \Phi ||\ S \ || \Psi ||$ 

As an example, consider the program $P(x,y) \equiv \texttt{while(x > 0) \{ x--; y++; \} }$ and the 2-safety property
of \emph{monotonicity} stating, intuitively: with growing inputs $x$ and $y$, the resulting $y$ also grows.
In CHL, this can be formalized as
\begin{equation}\label{eq:mono_formula}
\langle x_1 \leq x_2 \land y_1 \leq y_2  \rangle\ P(x_1, y_1)\st{}P(x_2, y_2)\  \langle y_1 \leq y_2 \rangle \, .
\end{equation}
The main idea here is that the formulas in the precondition and
postcondition can \emph{relate} variables from different executions.
%
Cartesian Hoare logic is equipped with a proof system that allows one to prove the validity of CHL triples.
This proof system contains rules\footnote{We have changed the notation
  slightly compared to the original paper.} like
\begin{align}\label{chlRule:If}
    % & \prftree[l]{If}
    %   { % \prfStackPremises
    %     { \langle \Phi \land c \rangle (\texttt{B1; S}) \st R \langle \Psi
    %   \rangle } \qquad
    %     { \langle \Phi \land \neg c \rangle (\texttt{B2; S}) * R \langle \Psi \rangle }
    %   }
    %   { \langle \Phi \rangle (\texttt{if (c) {B1} {B2} ; S}) * R \langle \Psi \rangle }
    & \prftree[l]{If}
      { % \prfStackPremises
        { \chl{\Phi \land c}{(\texttt{B1; S}) \st R}{\Psi} } \qquad
        { \chl{\Phi \land \neg c}{(\texttt{B2; S}) \st R}{\Psi} }
      }
      { \chl{\Phi}{(\texttt{if (c) {B1} {B2} ; S}) \st R}{\Psi} }
\end{align}
that replicate standard Hoare-logic reasoning, and is sound and complete. 
However, what is more interesting, the proof system allows one to perform \emph{lockstep reasoning},
even for loops. This is achieved by means of the following rule (version for two executions):
\begin{align*}
  & \prftree[l]{Fusion}
    {
    \chl{I \land c_1 \land c_2}{\texttt{B1} \st \texttt{B2}}{I} \quad
    \chl{I \land \neg c_2}{\texttt{while (}c_1\texttt{) {B1}}}{\Psi} \quad
    \chl{I \land \neg c_1}{\texttt{while (}c_2\texttt{) {B2}}}{\Psi}    
  }
  { \chl{\Phi}{(\texttt{while (}c_1\texttt{) {B1}}) \st (\texttt{while (}c_2\texttt{) {B2}})} {\Psi}}
\end{align*}
%(which we present here in a simplified form, for Cartesian claims of arity 2 only).
Note that the invariant $I$, assumed by this rule, can relate variables from \emph{both} executions.
%
The rule breaks reasoning about a pair of loops into three cases: the case
where both loop conditions hold and two cases where one of the conditions
does not hold.  In the first case, both loops are executed ``in lockstep'',
performing one iteration each, and their execution must preserve the
invariant. In the remaining two cases, only one of the loops executes (in a
state satisfying the invariant and negation of the other loop condition),
resulting in a state satisfying the postcondition.


% This can be done by means of standard Hoare logic reasoning.

% For an example, consider the program
% \begin{equation}\label{eqn:CounterProgram}
% P(x,y) \equiv \texttt{while(}x\texttt{ > 0)\{ }y\texttt{++; }x\texttt{--;\}}
% \end{equation}
% and the property that $P$ is monotone with respect to the initial values of $x$ and $y$ and resulting value of $y$.
% This can be formalized as the CHL triple
% \begin{equation*}
% \langle x_1 \leq x_2 \land y_1 \leq y_2  \rangle (P(x_1, y_1) * P(x_2, y_2)) \langle y_1 \leq y_2 \rangle
% \end{equation*}
% Intuitively, this hyperproperty holds because we can synchronize the two executions until the point when the first one terminates;
% then, the second execution may continue for a while, making the difference betwen $y_2$ and $y_1$ even greater.
% Formally, we can apply the Fusion rule, with the precondition being also the relational invariant.
% This results in three subgoals:
% \begin{equation*}
%     \begin{aligned}
%         & \langle x_1 \leq x_2 \land y_1 \leq y_2 \land x_1 > 0 \land x_2 > 0  \rangle\ ( (y_1\texttt{++} ; x_1\texttt{--}); * (y_2\texttt{++} ; x_2\texttt{--}))\ \langle x_1 \leq x_2 \land y_1 \leq y_2 \rangle \\
%         & \{ x_1 \leq x_2 \land y_1 \leq y_2 \land \neg (x_1 > 0) \}\ P(x_2, y_2)\ \{ y_1 \leq y_2 \} \\
%         & \{ x_1 \leq x_2 \land y_1 \leq y_2 \land \neg (x_2 > 0) \}\ P(x_1, y_1)\ \{ y_1 \leq y_2 \} \\
%     \end{aligned}
% \end{equation*}
% The first one requires one to show that the body of the loop preserves the invariant,
% the second (third) represent the cases when the first (second) execution terminated.
% To prove the second subgoal, one needs to find a (non-relational) invariant of the (single) loop;
% the third subgoal is trivial, as its precondition implies the negation of the loop condition.

This lockstep reasoning is a powerful tool because the required invariants (relating
different executions) are often very simple. For example, consider the
program $P(x,y)$ and the formula (\ref{eq:mono_formula}) above.
% If one chooses
%the invariant to be the same as the precondition (i.e. $I \equiv x_1 \leq x_2 \land y_1 \leq y_2$),
%the precondition of the second premise becomes
%$\langle x_1 \leq x_2 \land y_1 \leq y_2 \land \neg (x_2 > 0) \rangle$, which implies
%$x_1 \leq 0$ and therefore the loop condition in $P(x_1, y_1)$ is not
%satisfied.
In this case it is enough to choose the invariant to be the same as the precondition (i.e. $I \equiv x_1 \leq x_2 \land y_1 \leq y_2$).
To prove the above example without lockstep reasoning, one must find (non-relational) loop invariants
strong enough to summarize the whole loop.

Unfortunately, lockstep reasoning rules become more complicated as one adds other features into the language
-- for example, the \texttt{break} statement (as done in \cite{SousaD16}).
It is not immediately obvious how to extend this approach to handle, e.g., recursion, \texttt{continue}, or \texttt{goto}.
We also observe that a single language feature (\texttt{while} loops) needed to be considered five times
in order for CHL to support it soundly: the semantics of \texttt{while} is present in the operational semantics of
the target language, in the Hoare logic for that language, in the Cartesian Hoare logic for that language,
and in the proofs of soundness for both of the logics.

This paper aims to make the above ideas available for any deterministic language.
Therefore, we review some tools from recent literature
on language-parametric program verification in the following subsections.

%\subsection{($\mu$-free) Matching Logic}
\subsection{Matching Logic}

Before introducing reachability logic, we must first talk about matching
logic, on top of which reachability logic is built. We work with the variant of matching logic described in
\cite{StefanescuCMMSR19, RosuSCM13lics}. (There are other variants of matching
logic, e.g. \cite{MmL, MLexplained}, which are of no particular interest for this paper.)


% This particular variant of matching logic is used for reasoning about static properties of program configurations.
% There exist newer and more expressive variants of matching logic (\cite{MmL, MLexplained});
% we used the older variant in order to be compatible with the literature on reachability logic which uses this variant.

% Matching logic \emph{formula} (commonly known as a \emph{pattern}) is a
% first-order logic (FOL) formula which additionally allows terms (with variables)
% over some signature $\Sigma$ as nullary predicates.
% A typical example of a matching logic formula is $\varphi_{\mathit{example}}$,
% defined as

% \begin{equation}\label{eqn:exampleMLPattern}
% \texttt{<k>x--<k><st>x} \texttt{|->} X\texttt{</st>} \land (X \texttt{ >Int } 1 = \mathit{true})
% \end{equation}

% which, when interpreted in a model of a particular programming language,
% denotes the set of program configurations in which the code \texttt{x--} is to be executed
% and in which the program variable $\texttt{x}$ has a value $X$ that is greater than $1$.
% In this example, the part
% \begin{equation*}
%     \texttt{<k>x--<k><st> x} \texttt{|->} X\texttt{ </st>}
% \end{equation*}
% is a nullary term used as a predicate (term-as-predicate), with $X$ being the only free FOL variable.
% Here $\texttt{x}$ is not a FOL variable, but a constant symbol from the signature of the programming language.
% The subterm $\texttt{<st>x} \texttt{ |-> } X\texttt{</st>}$ says that the program variable $\texttt{x}$
% has the value $X$, and the $X \texttt{ >Int } 1 = \mathit{true}$ part then says that the realization
% of the function symbol $\_ \texttt{ >Int } \_$ returns the boolean value $\mathit{true}$ when given $X$ and $1$
% as arguments.

%ALTERNATIVE NOTATION

A matching logic \emph{formula} (commonly known as a \emph{pattern}) is a
first-order logic (FOL) formula which additionally allows terms (with variables)
over some signature $\Sigma$ as nullary predicates (we refer to these as ``terms-as-predicates'').
To enable reasoning about programming language syntax and semantics,
the signature often contains the syntactical constructs of a programming language of interest.
A typical example of a matching logic formula is $\varphi_{\mathit{example}}$,
defined as\footnote{In the syntax of the \K{} framework this formula would look more like
$\texttt{<k>x--<k><st>x |-> X</st> } \land \texttt{ (X >Int  1 = true)}$.}


\begin{equation}\label{eqn:exampleMLPattern}
    \varphi_{\mathit{example}} \equiv\quad \tap{x--;}{\texttt{x}\mapsto X}  \land\ (X >_\text{Int} 1 = \mathit{true})
\end{equation}


which, when interpreted in a model of a particular programming language,
denotes the set of program configurations in which ``\texttt{x--;}'' is the
code to be executed next, and the program variable $\texttt{x}$ has a value $X$
that is greater than $1$.  In this example, the subformula
$\tap{x--;}{\mathtt{x}\mapsto X}$ is a nullary term used as a predicate
(term-as-predicate), with $X$ being the only free FOL variable.  ($\texttt{x}$
is not a FOL variable but a constant symbol from the signature of the
programming language.)  The subterm $\texttt{x}\mapsto X$ states that the
program variable $\texttt{x}$ has value $X$, and the
$X >_\text{Int} 1 = \mathit{true}$ part then says that the realization of the
function symbol ``$>_\text{Int}$'' returns the boolean value $\mathit{true}$
when given $X$ and $1$ as arguments.

The satisfaction relation $(M, \gamma, \rho) \vDash \varphi$ for a model $M$, a model element $\gamma \in M$,
an $M$-valuation $\rho$, and a pattern $\varphi$, is defined inductively on the structure of $\varphi$.
The definition is as in FOL; the main difference is the semantics of
terms-as-predicates, which is given as
\begin{equation*}
    (M, \gamma, \rho) \vDash t \iff \gamma = \rho(t) \text{ if t is a term}
\end{equation*}
(where $\rho(t)$ is the homomorphic extension of $\rho$ applied to the term $t$).
For example, we might have a matching logic model $M$ containing (concrete) program configurations
of a particular programming language.
One such configuration might be:
\begin{equation*}
  \gamma_{\mathit{example}} \equiv\quad \tap{x--;}{\mathtt{x}\mapsto 3}
    % \texttt{<k> x--; <k><st> x} \texttt{ |-> } 3\texttt{ </st>} \, .
\end{equation*}
Then, we have that $(M, \gamma_{\mathit{example}}, \rho) \vDash \varphi_{\mathit{example}}$
for any valuation $\rho$ satisfying $\rho(X) = 3$, and we say that
$\varphi_{\mathit{example}}$ \emph{matches} $\gamma_{\mathit{example}}$ in $\rho$.


A pattern $\varphi$ is \emph{valid in $M$}, written $M \vDash \varphi$, iff $(M, \gamma, \rho) \vDash \varphi$
for every $\gamma$ and $\rho$.
We observe that the validity of a structureless pattern (a pattern without terms-as-predicates) does not depend on the selected model element.
Also, the validity of any pattern does not depend on those variables the pattern does not mention.
A more formal treatment of matching logic is to be found in \Cref{app:MLandRL}.


\subsection{One-path Reachability Logic}
Reachability logic \cite{RosuS12oopsla, StefanescuCMMSR19} (RL) is a formalism for
both a) defining formal semantics of programming languages,
and b) specifying and reasoning about partial correctness properties
of programs in those languages.
On the formal semantics side, a programming language is modeled as a \emph{reachability system}
$\mathcal{S} = (\mathcal{T}, S)$, where $\mathcal{T}$ is a $\Sigma$-algebra
and $S$ is a set of \emph{reachability rules} of the shape $\varphi \Rightarrow^\exists \varphi^\prime$,
where $\varphi$ and $\varphi^\prime$ are \emph{matching logic} patterns over $\Sigma$
describing sets of \emph{source} and \emph{target} program configurations.

Each reachability system naturally induces a \emph{transition system}
$(\Tcfg , \Rightarrow_{\mathcal{S}})$, whose states are program configurations
and transitions $\Rightarrow_{\mathcal{S}}$ are defined as follows: for
$\gamma, \gamma^\prime \in \Tcfg$ we have
$\gamma \Rightarrow_{\mathcal{S}} \gamma^\prime$ iff there is some rule
$\varphi \Rightarrow^\exists \varphi^\prime \in S$ and some valuation
$\rho : \Var \to \mathcal{T}$ such that $(\gamma, \rho) \vDash \varphi$ and
$(\gamma^\prime , \rho) \vDash \varphi^\prime$.

As an example, consider the following reachability rule

\begin{equation}\label{eqn:ruleIfTrue}
    \tap{if ($true$) then $P_1$ else $P_2$}{S} \ \Rightarrow^{\exists}\ \tap{$P_1$}{S}
\end{equation}

saying that the \texttt{if} construct of the particular language takes the first branch ($P_1$)
whenever the condition is $\mathit{true}$. (Typically, there would be additional rules
governing the evaluation of the condition.)
%
This rule induces (among others) the transition
\begin{equation}\label{eqn:ruleIfTrue}
  \tap{if ($true$) then x++ else x--}{\mathtt{x}\mapsto 3}\
  \Rightarrow_{\mathcal{S}}\  \tap{x++}{\mathtt{x}\mapsto 3} \,.
\end{equation}



% Typically, \emph{program configurations} contain some program (or a fragment of it) that is to be executed,
% together with a state of the program.
% For example, one can have a rule
% \begin{equation}\label{eqn:ruleIfTrue}
%     \begin{aligned}
%     \texttt{<k>if (} \mathit{true} \texttt{) }P_1\texttt{ else } P_2 \texttt{</k><st>} S \texttt{</st>} 
%     \ \Rightarrow^{\exists}\ \texttt{<k>}P_1 \texttt{</k><st>} S \texttt{</st>}
%     \end{aligned}
% \end{equation}


% \begin{equation}\label{eqn:ruleIfTrue}
%     \begin{aligned}
%     & \texttt{<k> if (} \mathit{true} \texttt{) }P_1\texttt{ else } P_2 \texttt{</k><st>} S \texttt{</st>} \\
%     & \Rightarrow \texttt{<k> }P_1 \texttt{</k><st>} S \texttt{</st>}
%     \end{aligned}
% \end{equation}
% saying that the \texttt{if} construct of the particular language takes the first branch ($P_1$)
% whenever the condition is $\mathit{true}$.
% (Typically, there would be additional rules governing evaluation of the condition.)

% The meaning of reachability rules is the following.
% A reachability system $\mathcal{S} = (\mathcal{T}, S)$ (together with a $\Sigma$-sort $\mathit{Cfg}$)
% induces
% a \emph{transition system}
% $(\Tcfg , \Rightarrow_{\mathcal{S}})$,
% where $\gamma \Rightarrow_{\mathcal{S}} \gamma^\prime$
% for $\gamma, \gamma^\prime \in \Tcfg$
% iff there is some rule $\varphi \Rightarrow^\exists \varphi^\prime \in S$
% and some valuation $\rho : \Var \to \mathcal{T}$ with $(\gamma, \rho) \vDash \varphi$
% and $(\gamma^\prime , \rho) \vDash \varphi^\prime$.
% The intuition is that when taking a transition in the resulting transition system,
% some rule $\varphi \Rightarrow^\exists \varphi^\prime \in S$ is selected,
% then the current configuration is pattern-matched against the rule's left-side pattern $\varphi$,
% resulting in a valuation $\rho$ which is then used to instantiate the right-side $\varphi^\prime$ of the rule,
% forming a new configuration.
% For example, the rule in \Cref{eqn:ruleIfTrue} induces (among others) the transition
% \begin{equation}\label{eqn:ruleIfTrue}
%     \begin{aligned}
%     \texttt{<k>if (} \mathit{true} \texttt{) x++ else x--</k><st>x} \texttt{|->} 3\texttt{</st>}
%     \ \Rightarrow_{\mathcal{S}}\  \texttt{<k>x++</k><st>x} \texttt{|->} 3\texttt{</st>} \, .
%     \end{aligned}
% \end{equation}

% \begin{equation}\label{eqn:ruleIfTrue}
%     \begin{aligned}
%     & \texttt{<k> if (} \mathit{true} \texttt{) x++; else x--; </k><st>x} \texttt{ |-> } 3\texttt{</st>} \\
%     & \Rightarrow_{\mathcal{S}} \texttt{<k> x++; </k><st>x} \texttt{ |-> } 3\texttt{</st>} \, .
%     \end{aligned}
% \end{equation}

On the \emph{partial correctness} side, RL reuses the concept of reachability rules.
For example, one can specify that the program \texttt{while(x > 0) do x--;}
may, if it terminates at all, reach a configuration
where nothing remains to be executed (represented by ``$\cdot$'') and where the program variable \texttt{x} has a non-positive value,
by means of the following reachability rule
\begin{equation*}
  \tap{while (x>0) do x--;}{\mathtt{x}\mapsto V}
  \ \Rightarrow^\exists\  \exists V^\prime.\, \tap{$\cdot$}{\mathtt{x}\mapsto V'} \land (V^\prime \leq_\text{Int} 0 = \mathit{true})
  \end{equation*}
% \begin{equation*}
%     \begin{aligned}
%         & \texttt{<k>while( x > 0 ) x--</k><st>x|->} V \texttt{</st>} \\
%         & \Rightarrow^\exists \exists V^\prime.\, \texttt{<k> . </k> x |-> } V^\prime \texttt{</st>} \land (V^\prime \texttt{ <=Int } 0 = \mathit{true})
%     \end{aligned}
%   \end{equation*}
  
Assuming the language is deterministic, this is equivalent to saying that if the program terminates,
the resulting configuration will have a non-positive value of \texttt{x}.
Formally, we say that a configuration $\gamma \in \Tcfg$ terminates in $(\Tcfg, \Rightarrow_{\mathcal{S}})$
iff there is no infinite chain
$\gamma \Rightarrow_{\mathcal{S}} \gamma_1 \Rightarrow_{\mathcal{S}} \gamma_2 \Rightarrow_{\mathcal{S}} \ldots$.
A rule of the shape $\varphi \Rightarrow^\exists \varphi^\prime$
is \emph{satisfied}
in a reachability system $\mathcal{S} = (\mathcal{T}, S)$,
written $\mathcal{S} \vDash_\RL \varphi \Rightarrow^\exists \varphi^\prime$,
iff for every $\gamma \in \Tcfg$
such that $\gamma$ terminates in $(\Tcfg, \Rightarrow_{\mathcal{S}})$
and for any valuation $\rho : \Var \to \mathcal{T}$
such that $(\gamma, \rho) \vDash \varphi$,
there exists some $\gamma^\prime \in \Tcfg$
such that
$\gamma \Rightarrow^{*}_{\mathcal{S}} \gamma^\prime$
and $(\gamma^\prime, \rho) \vDash \varphi^\prime$.


Reachability logic is equipped with a proof system that derives sequents of the shape
$\mathcal{A}, C \vdash_\RL \varphi \Rightarrow^\exists \varphi^\prime$ (where $\mathcal{A}$ is a
reachability system and $C$ is introduced below). The proof system is sound and complete: an RL claim is satisfied in $\mathcal{S}$
iff $\mathcal{S}, \emptyset \vdash_\RL \varphi \Rightarrow^\exists \varphi^\prime$.
The set $C$, initially empty, contains so-called \emph{circularities},
which are claims postulated to hold but not justified yet.
Circularities, which correspond to the notion of \emph{loop invariants} of Hoare logic,
enable one to reason about repetitive behavior of programs.
The proof system contains a rule
\begin{align*}
    & \prftree[l]{Circularity}
      { \mathcal{A}, C \cup \{ \varphi\Rightarrow^\exists \varphi^\prime \} \vdash_\RL \varphi \Rightarrow^\exists \varphi^\prime }
      { \mathcal{A}, C \vdash_\RL \varphi \Rightarrow^\exists \varphi^\prime }
\end{align*}
which adds the current claim to the set of circularities.  When progress is
made (by means of other rules, essentially performing symbolic execution),
the claim is moved from the set of circularities to $\mathcal{A}$ (using the
\emph{Transitivity} rule -- see Appendix~\ref{app:crlsoundness}) and can be
reused, similarly to the way one assumes a loop invariant in order to prove it
again.  We refer the interested reader to~\cite{RosuS12oopsla} for more
details.


%For illustration purposes, we show one rule of the RL proof system:
%\begin{align*}
%    & \prftree[l]{Case Analysis}
%      %{ \prfStackPremises
%        { A, C \vdash_\RL \varphi_1 \Rightarrow^\exists \varphi }
%        { A, C \vdash_\RL \varphi_2 \Rightarrow^\exists \varphi }
%      %}
%      { A, C \vdash_\RL \varphi_1 \lor \varphi_2 \Rightarrow^\exists \varphi }
%\end{align*}


\begin{remark}\label{rem:noEmptySteps}
  We work only with $\epsilon$-free reachability systems.
  A reachability system $(\mathcal{T}, S)$ is \emph{$\epsilon$-free}
  iff for any two configurations $\sigma, \sigma^\prime \in \mathcal{T}_{\mathit{Cfg}}$, if
  $\sigma \Rightarrow_{\mathcal{S}} \sigma^\prime$, then $\sigma \not = \sigma^\prime$.
  (We are not aware of any practical reachability system that would use these $\epsilon$ steps.)
  \end{remark}

\begin{remark}\label{rem:shapeOfReachabilityRules}
  Following the original reachability logic literature (\cite{RosuS12oopsla,StefanescuCMMSR19}),
  we restrict the class of reachability systems we work with to those whose reachability rules
have the shape
\begin{equation*}
    \phi \land P \Rightarrow^\exists \phi^\prime \land P^\prime
\end{equation*}
where $\phi,\phi^\prime$ are terms-as-predicates, and $P,P^\prime$ contain no terms-as-predicate.
  As argued in these papers, such rules can support various styles of operational semantics,
  including evaluation contexts \cite{PLTRedex}, the chemical abstract machine \cite{CHAM}, and \K{} \cite{KVision}.
  We thus support all the $\epsilon$-free reachability systems supported by reachability logic.
\end{remark}

%%% Local Variables:
%%% mode: latex
%%% TeX-master: "../main"
%%% End:

\section{Cartesian Reachability Logic}
In this section we introduce \emph{Cartesian Reachability logic (CRL)} - a language-parametric logic for reasoning
about $k$-safety hyperproperties.
Our aim with CRL is to make reasoning in the style of \emph{Cartesian Hoare logic (CHL)}~\cite{SousaD16} available for any
deterministic language with given reachability-logic semantics $S$.
For that purpose we define the language of CRL and its semantics, and demonstrate the logic's expressiveness
on a couple of examples.
Then we give CRL a sound proof system, which is the main contribution of this paper.


\subsection{Syntax and Semantics}

Cartesian reachability logic is an extension of (one-path) reachability
logic. In this extension, we replace reachability rules $\varphi
\Rightarrow^{\exists} \varphi$ with  \emph{reachability claims} of the form
\begin{equation*}
  [\varphi_1,\ldots,\varphi_k] \land P
  \Rightarrow^{c\exists} \exists \vec{Y}.\, [\varphi^\prime_1,\ldots,\varphi^\prime_k] \land P^\prime
\end{equation*}

The intuitive meaning of such a claim should be clear enough: There are $k$
programs, the set of source configurations of $i$-th program matching $\varphi_i$
and target configurations matching $\varphi'_i$. Additionally, the (TODO -
matching logic) formula $P$ can relate source configurations, and $P'$ the
target configurations. We call formulas of the form   $ [\varphi_1,\ldots,\varphi_k] \land P$
\emph{existentially-quantified constrained list patterns} (ECLP). 


For example, consider the same program as in \Cref{eqn:CounterProgram}; that is, let
\begin{equation*}
  P \equiv \texttt{while(x > 0)\{ y++; x--;\}} \, .
\end{equation*}
Additionally, let $C(Q, X, Y)$ represent a configuration of a program $Q$
where the program variable $\texttt{x}$ has the value $X$
and the program variable $\texttt{y}$ has the value $Y$;
for example,
\begin{equation}\label{eqn:CQXY}
 C(Q, X, Y) \equiv \texttt{<k>} Q \texttt{</k><st>(x |-> } X \texttt{)(y |-> } Y \texttt{)</st>}    \, .
\end{equation}
Then, the claim $\Omega_{\textit{mono}}$, defined as
\begin{align*}
&[C(P, X_1, Y_1),C(P, X_2, Y_2)] \land X_1 \leq X_2 \land Y_1 \leq Y_2
\\ \Rightarrow^{c\exists} &
\exists X^\prime_1, Y^\prime_1, X^\prime_2, Y^\prime_2.\,  [C(\epsilon, X^\prime_1, Y^\prime_1), C(\epsilon, X^\prime_2, Y^\prime_2)] \land Y^\prime_1 \leq Y^\prime_2   
\end{align*}
(where $\epsilon$ denotes the empty program)
expresses the property that the program $P$ is monotone.
That is, when we start an execution (using the semantics of the particular language)
from some configuration $\gamma_1$ matching $C(P, X_1, Y_1)$
and a second execution from some configuration $\gamma_2$ matching $C(P, X_2, Y_2)$,
if $X_1 \leq X_2$ and $Y_1 \leq Y_2$,
we end up in configurations $\gamma_1^\prime,\gamma_2^\prime$ matching
$C(\epsilon, X^\prime_1, Y^\prime_1)$ and $C(\epsilon, X^\prime_2, Y^\prime_2)$
for some $X^\prime_1,Y^\prime_1,X^\prime_2,Y^\prime_2$
satisfying $Y^\prime_1 \leq Y^\prime_2$.

We formally define the semantics of a CRL claim as follows:
\begin{definition}[CRL semantics]\label{def:opCRLsemantics}
    A claim
    \begin{equation*}
     [\varphi_1,\ldots,\varphi_k] \land P
     \Rightarrow^{c\exists} \exists \vec{Y}.\, [\varphi^\prime_1,\ldots,\varphi^\prime_k] \land P^\prime
    \end{equation*}
    is \emph{valid} in a reachability system $\mathcal{S} = (\mathcal{T}, S)$,
    written
    \begin{equation*}
        \mathcal{S} \vDash_\CRL [\varphi_1,\ldots,\varphi_k] \land P
     \Rightarrow^{c\exists} \exists \vec{Y}.\, [\varphi^\prime_1,\ldots,\varphi^\prime_k] \land P^\prime \, ,
    \end{equation*}
    iff for all configurations $\gamma_1,\ldots,\gamma_k \in \Tcfg$
    which terminate in $(\Tcfg, \Rightarrow_{\mathcal{S}})$
    and any $\mathcal{T}$-valuation $\rho$,
    whenever $(\gamma_1,\rho) \vDash \varphi_1 \land P$ and \ldots
    and $(\gamma_k,\rho) \vDash \varphi_k \land P$,
    then there exist configurations $\gamma_1^\prime,\ldots,\gamma_k^\prime \in \Tcfg$
    such that $\gamma_1 \Rightarrow^{*}_{\mathcal{S}} \gamma_1^\prime$
    and \ldots and $\gamma_k \Rightarrow^{*}_{\mathcal{S}} \gamma_k^\prime$,
    and there also exists an $\mathcal{T}$-valuation $\rho^\prime$
    satisfying $\rho(v) = \rho^\prime(v)$ for any $v \in \mathit{Var} \setminus \vec{Y}$,
    and
    $(\gamma_1^\prime,\rho^\prime) \vDash \varphi^\prime_1 \land P^\prime$ and \ldots and $(\gamma_k^\prime, \rho^\prime) \vDash \varphi^\prime_k \land P^\prime$.
\end{definition}

\subsection{Comparison to CHL}\label{sec:CRLsemanticsComparisonToCHL}
CRL is more verbose then CHL. This is partly because of the need to specify patterns matching the whole program configurations,
and partly because of the need to existentially quantify those variables on the right side whose value is not
determined by the left side.
The second reason is related to the fact that in CRL, as well as in RL, a variable occuring in both sides
has the same value in both sides.
Therefore, in (C)RL, one does not need to introduce ghost variables for remembering values from the precondition.
We inherit a notation commonly used in RL, that variables whose names start with a question mark are implicitly
considered existentially quantified in the right side.
Also, we use underscore for a variable with unknown name, different between all occurences -
that is, to represent a variable whose value we are not interested in.
For example, we can write the claim $\Omega_{\textit{mono}}$ as
\begin{align*}
[C(P, X_1, Y_1),C(P, X_2, Y_2)] \land X_1 \leq X_2 \land Y_1 \leq Y_2
 \Rightarrow^{c\exists} [C(\epsilon, ?\_, ?Y_1), C(\epsilon, ?\_, ?Y_2)] \land ?Y_1 \leq ?Y_2 \, .
\end{align*}

A deeper distinction is that in the CRL semantics we existentially quantify over reachable configurations,
while in CHL, target states are quantified universally.
However, this distinction has no effect when working with deterministic languages.

\subsection{Comparison to Reachability logic}


We note here that in general, having one valid CRL claim is different from having just $k$ valid RL claims,
because in CRL, the variables are shared across components.
%Consider, for example, a reachability system $S_{\mathit{IMP}}$ representing a simple imperative language,
%and the same program as in \Cref{eqn:CounterProgram}; that is, let
%\begin{equation*}
%  P \equiv \texttt{while(x > 0)\{ y++; x--;\}} \, .
%\end{equation*}
%We can let $C(Q, X,Y)$ represent a configuration of a program $Q$ where the program variable $\texttt{x}$ has the value $X$
%and the program variable $\texttt{y}$ has the value $Y$;
%for example,
%\begin{equation*}
% C(Q, X, Y) \equiv \texttt{<k>} Q \texttt{</k><st>(x |-> } X \texttt{)(y |-> } Y \texttt{)</st>}    \, .
%\end{equation*}
%Then, the proposition
%\begin{align*}
% S_{\mathit{IMP}} \vDash_\CRL
%&[C(P, X_1, Y_1),C(P, X_2, Y_2)] \land X_1 \leq X_2
%\\ \Rightarrow^{c\exists} &
%\exists X^\prime_1, Y^\prime_1, X^\prime_2, Y^\prime_2.\,  [C(\epsilon, X^\prime_1, Y^\prime_1), C(\epsilon, X^\prime_2, Y^\prime_2)] \land Y^\prime_1 \leq Y^\prime_2   
%\end{align*}
%holds iff the program $P$ is monotone (when considering the variable $x$ to be an input, and $y$ to be an output; $\epsilon$ represents an empty program).
In CRL, one can localize the ``global'' constraints; for example, the claim $\Omega_{\textit{mono}}$
is equivalent to
\begin{align*}
[C(P, X_1, Y_1),C(P, X_2, Y_2) \land X_1 \leq X_2] 
\Rightarrow^{c\exists}
[C(\epsilon, ?\_, ?Y_1), C(\epsilon, ?\_, ?Y_2) \land ?Y_1 \leq ?Y_2] \, .
\end{align*}
However, if one were to ``split'' the CRL claim into two, the resulting claims might express
a different property than monotonicity.
For example, the two claims
\begin{align}
& C(Q, X_1, Y_1) & \Rightarrow^{\exists} \quad & C(\epsilon, ?X_1, ?Y_1) \\
& C(Q, X_2, Y_2) \land X_1 \leq X_2 & \Rightarrow^{\exists} \quad  & C(\epsilon, ?X_2, ?Y_2) \land ?Y_1 \leq ?Y_2
\end{align}
hold for any reasonable program $Q$ (meaning that $Q$ either executes fully or diverges),
because $?Y_1$ in the second claim is unrelated to $?Y_1$ in the first claim and thus one can set it to
the value of $?Y_2$.
On the other hand, if in the second claim we renamed $?Y_1$ into $Y_1^\prime$ (without a question mark),
the claim would require that $?Y_2$ is greater than or equal to \emph{any} integer (because $Y_1^\prime$ is not present in the left side),
which clearly cannot hold.

Still, reachability logic is a special case of CRL.
\begin{remark}
%\Cref{def:opCRLsemantics} extends \Cref{def:oprlSemantics} of \Cref{def:basics}:
%if we fix $k=1$, then
$
    (\mathcal{T}, S) \vDash_\CRL
    [\varphi] \land \top  \Rightarrow^{c\exists}
    [\varphi^\prime] \land \top
    \iff
    (\mathcal{T}, S) \vDash_\RL \varphi \Rightarrow^{\exists} \varphi^\prime \, .
$
\end{remark}

%\begin{definition}[All-Path Cartesian Reachability Rule]\label{def:apCRLsemantics}
%An all-path Cartesian reachability rule
%$(\varphi_1,\ldots,\varphi_k) \land \varphi \Rightarrow^{c\forall} (\psi_1,\ldots,\psi_k) \land \psi$
%of arity $k$
%is \emph{valid} in a reachability system $\mathcal{S} = (\mathcal{T}, S)$,
%written
%$\mathcal{S} \vDash_\CRL (\varphi_1,\ldots,\varphi_k) \land \varphi \Rightarrow^{c\forall}
%(\psi_1,\ldots,\psi_k) \land \psi$,
%iff for all configurations $\sigma_1,\ldots,\sigma_k \in \Tcfg$ \traian{Why $\sigma$ instead of $\gamma$?}
%and any $\mathcal{T}$-valuation $\rho$,
%whenever $(\sigma_1, \rho) \vDash \varphi_1 \land \varphi$ and \ldots
%and $(\sigma_k, \rho) \vDash \varphi_k \land \varphi$,
%then for every $k$-tuple of complete paths $(\pi_1, \ldots, \pi_k)$
%such that
%$\sigma_1 = \pi_1(0) \land \ldots \land \sigma_k = \pi_k(0)$,
%there exist indices $i_1, \ldots, i_k$ such that
%$(\pi_1(i_1), \rho) \vDash \varphi_1 \land \psi$ and \ldots and $(\pi_k(i_k), \rho) \vDash \varphi_k \land \psi$.
%\end{definition}

Now we present a novel, general technique called \emph{star extension} that is reminiscent of self-composition~\cite{BartheDR11}.
Self-composition is a technique where a program $P$ together with a $k$-safety hyperproperty is reduced to
a sequential composition of $P$ with itself (with renamed variables) together with a safety property.
This technique allows one to use tools and techniques for verification of safety properties
to perform verification of $k$-safety hyperproperties.
The challenge here is to generalize self-composition to work with any deterministic language,
even if we do not know in advance how the language implements sequential composition, if at all.

The main idea of \emph{star extension} is to transform a CRL claim into a RL claim over an extended reachability system,
where configurations of the extended reachability system are lists of configurations of the original system.
The transformation is quite straightforward but technical, and we refer an interested reader to the \Cref{app:CRLandRLcorrespondence};
here we only present the main theorem.
\begin{theorem}\label{thm:CRLandRLcorrespondence}
  There exist a function $\_^*$ on matching logic signatures,
  a (equally-named) function $\_^*$ from reachability systems over $\Sigma$ to reachability systems over $\Sigma^*$,
  and a function $\mathit{flatten}$ from ECLPs over $\Sigma$ to matching logic $\Sigma^*$-formulas,
  such that
  \begin{equation*}
  \mathcal{S} \vDash_{\CRL} \Psi \Rightarrow^{c\exists} \Psi^\prime
    \iff \mathcal{S}^* \vDash_\RL \mathit{flatten}(\Psi) \Rightarrow^{c\exists} \mathit{flatten}(\Psi^\prime)
  \end{equation*}
\end{theorem}

The price paid for self-composition is that the property of the self-composed program is often hard to reason about.
Therefore, in~\cite{SousaD16}, the authors do not apply self-composition directly, but only use its soundness to justify
their technique - namely, the soundness of their proof system, which avoids explicit construction
of the self-composed programs.
We use the star extension for the same purpose.



\subsection{Proof System}




%We give Cartesian Reachability Logic a proof system to faciliate mechanical reasoning.
%One may ask the question, ``why give CRL a new proof system if one can perform the same reasoning
%by means of \Cref{thm:CRLandRLcorrespondence} and the existing proof system of reachability logic?''.
%The answer is the following.
%When one reduces a CRL goal into RL using \Cref{thm:CRLandRLcorrespondence},
%the function $\mathit{flatten}$ appears in the goal.
%To perform reasoning using the RL proof system, one then has to either simplify the goal
%by unfolding the definition of $\mathit{flatten}$,
%or use (and prove) some helper lemmas about the effect of applying RL proof rules
%%and other frequently used steps
%on RL goals containing $\mathit{flatten}$.
%In the first case one lowers the abstraction level and have to reason about matching logic formulas
%containing translations of other matching logic formulas into FOL, which then makes RL reasoning more complex.
%In the second case, however, one may end up proving lemmas which, when combined, result in a proof system for CRL.
%Indeed, the soundness of our proof system (\Cref{fig:CRLproofsystem})
%is established by a (meta-)proof which constructs a RL proof from a CRL one (\Cref{lem:CRLalmostSoundness}).


We give Cartesian Reachability Logic a proof system to faciliate mechanical reasoning.
While the intuition behind the semantics of CRL is similar to that of CHL, with only a few minor differences,
giving CRL a proof system is not straightforward.
One could attempt to reuse the proof system of CHL and modify it \emph{somehow} to be independent
of the particular programming language.
Since many CHL rules (e.g., its \texttt{If} rule shown in \Cref{chlRule:If}) are simply Hoare logic rules acting on a particular component of
a tuple of formulas (that is, symbolic states),
one could for example lift rules of reachability logic (RL) to the tuple-context and be done.
That would indeed work, and the resulting proof system might even be complete.

However, the distinguishing feature of Cartesian Hoare logic is \emph{not} its completeness,
but its ability to simplify reasoning by performing lock-step execution of loops,
because that can \textquote[\cite{SousaD16}]{greatly simplify the verification task (e.g., by requiring simpler invariants)}.
Simply (re)using RL rules lifted to the tuple-context does not provide any support for that.
And it is not entirely obvious how to support lock-step reasoning about construct with repetetive behavior:
in order to support while-with-breaks, Cartesian Hoare logic itself uses additionally (besides the lifted Hoare logic rules)
five fairly complex rules that need to understand the precise semantics of this construct.
The task for us is even harder, since we do not know in advance what constructs with repetetive behavior does
the supplied language support: \texttt{goto}s? Mutual recursion? 

Yet, we provide a single proof system (consisting of only eight rules) which enables lockstep reasoning about such constructs.
The proof system derives claims of the shape
\begin{equation*}
(\mathcal{T}, S) \vdash_\CRL \Phi \Downarrow_{C,E} \Psi
\end{equation*}
where $\Phi$ is of the shape
$(\varphi_1, \ldots, \varphi_k) \land P$
and $\Psi$ is of the shape
$\exists \vec{Y}.\, (\varphi^\prime_1, \ldots, \varphi^\prime_k) \land P^\prime$.
One can think about $\Phi$ as representing a \emph{premise}, while $\Psi$, which propagates through
the proof rules unchanged, as representing a \emph{conclusion}.
Every $\varphi_i, \varphi^\prime_i$ is a matching logic pattern representing a particual component,
and $P$ and $P^\prime$ are FOL formulas (\emph{global constraints}) relating variables from different components.
The sets $C$ and $E$ contain \emph{cutpoints} and \emph{enabled cutpoints}, respectively.
Ttogether, they implement the concept of a \emph{relational invariant}.
In particular, the set $C$ represents the invariants that were postulated \emph{right now},
while the set $E$ represents those that were postulated in past and are ready to be used.
Initially, the proof search starts with $E = C = \emptyset$.

\begin{figure}
    \centering
    \begin{align*}
    & \prftree[l]{Reflexivity}{(\mathcal{T}, S) \vdash_\CRL \Psi \Downarrow_{\emptyset,E} \Psi}
    \end{align*}
    \begin{align*}
    & \prftree[l]{Axiom}{\Psi \in E}{(\mathcal{T}, S) \vdash_\CRL \Psi \Downarrow_{C,E} \Psi^\prime}
    \end{align*}
    \begin{align*}
    & \prftree[l]{Reduce}
      {(\mathcal{T}^*, S^* \cup \mathit{flatten}^\exists(E, \Psi^\prime)), \emptyset \vdash_\RL
        \mathit{flatten}^\exists(\Psi, \Psi^\prime) }
      {(\mathcal{T}, S) \vdash_\CRL \Psi \Downarrow_{C,E} \Psi^\prime}
    \end{align*}
    \begin{align*}
    & \prftree[l]{Case}
    { \prfStackPremises
      {(\mathcal{T}, S) \vdash_\CRL [\varphi_1, \ldots, \varphi_{i-1}, \varphi_i, \varphi_{i+1}, \ldots, \varphi_k] \land P^\prime \Downarrow_{C, E} \Psi^\prime }
      {(\mathcal{T}, S) \vdash_\CRL [\varphi_1, \ldots, \varphi_{i-1}, \psi_i, \varphi_{i+1}, \ldots, \varphi_k] \land P^\prime \Downarrow_{C, E} \Psi^\prime }
    }
    {(\mathcal{T}, S) \vdash_\CRL [\varphi_1, \ldots, \varphi_{i-1}, (\varphi_i \lor \psi_i), \varphi_{i+1}, \ldots, \varphi_k] \land P^\prime \Downarrow_{C, E} \Psi^\prime}
    \end{align*}
    \begin{align*}
    & \prftree[l]{Step}
    { \prfStackPremises
       {\varphi \Rightarrow^\exists \varphi^\prime \in S}
       {\mathcal{T} \vDash_\ML \varphi_i \leftrightarrow \varphi \land P^\prime}
       {P^\prime \mbox{ is a FOL formula}}
       {  (\mathcal{T}, S) \vdash_\CRL [\varphi_1, \ldots, \varphi_{i-1}, \varphi^\prime \land P^\prime, \varphi_{i+1}, \ldots, \varphi_k]
          \land P
          \Downarrow_{\emptyset, (C \cup E)} \Psi^\prime
      }
    }
    {(\mathcal{T}, S) \vdash_\CRL [\varphi_1, \ldots, \varphi_{i-1}, \varphi_i, \varphi_{i+1}, \ldots, \varphi_k] \land P \Downarrow_{C, E} \Psi^\prime}
    \end{align*}
    \begin{align*}
    & \prftree[l]{Circularity}
      { (\mathcal{T}, S) \vdash_\CRL \Psi \Downarrow_{C \cup \{ \Psi \} , E} \Psi^\prime}
      { (\mathcal{T}, S) \vdash_\CRL \Psi \Downarrow_{C, E} \Psi^\prime}
    \end{align*}
    \begin{align*}
    & \prftree[l]{Conseq}
      { \prfStackPremises
        { (\mathcal{T}, S) \vdash_\CRL \Phi^\prime \Downarrow_{C, E} \Psi^\prime}
        { \mathcal{T}^* \vDash_\ML \mathit{flatten}(\Phi) \rightarrow \mathit{flatten}(\Phi^\prime) }
      }
      { (\mathcal{T}, S) \vdash_\CRL \Phi \Downarrow_{C, E} \Psi^\prime}
    \end{align*}
%    \begin{align*}
%    & \prftree[l]{Conseq2}
%      { \prfStackPremises
%        { (\mathcal{T}, S) \vdash_\CRL (\varphi_1, \ldots, \varphi_k) \land P \Downarrow_{C, E} \Psi^\prime}
%        { \mathcal{T} \vDash_\ML \varphi_i \leftrightarrow \psi_i }
%      }
%      { (\mathcal{T}, S) \vdash_\CRL (\varphi_1^\prime, \ldots, \varphi_{i-1}, \varphi_{i}, \varphi_{i+1}, \ldots, \varphi_k^\prime) \land P^\prime \Downarrow_{C, E} \Psi^\prime}
%    \end{align*}

    %\begin{align*}
    %& \prftree[l]{Abstract}
      %{ \prfStackPremises
        %{ X \not\in \mathit{FV}(\Psi^\prime, \varphi_1, \ldots, \varphi_{i-1}, \varphi_{i+1}, \ldots,\varphi_k)
        %}
        %{(\mathcal{T}, S) \vdash_\CRL (\varphi_1, \ldots, \varphi_{i-1}, \varphi_i, \varphi_{i+1}, \ldots, \varphi_k) \land P \Downarrow_{C, E} \Psi^\prime}
      %}
      %{(\mathcal{T}, S) \vdash_\CRL (\varphi_1, \ldots, \varphi_{i-1}, \exists X.\, \varphi_i, \varphi_{i+1}, \ldots, \varphi_k) \land P \Downarrow_{C, E} \Psi^\prime}
    %\end{align*}
    
    \begin{align*}
    & \prftree[l]{Abstract}
      { \prfStackPremises
        { X \not\in \mathit{FV}(\Psi^\prime)
        }
        {(\mathcal{T}, S) \vdash_\CRL \exists \vec{Y}.\, [\varphi_1, \ldots, \varphi_k] \land P \Downarrow_{C, E} \Psi^\prime}
      }
      {(\mathcal{T}, S) \vdash_\CRL \exists X, \vec{Y}.\, [\varphi_1, \ldots, \varphi_k] \land P \Downarrow_{C, E} \Psi^\prime}
    \end{align*}
    \caption{Proof System}
    \label{fig:CRLproofsystem}
\end{figure}

The rules itself are fairly simple, and none of them has the semantics of any particular language construct
hard-wired into it.
We now explain the proof rules of CRL (shown in~\Cref{fig:CRLproofsystem}) one by one.
\begin{itemize}
  \item The Circularity rule is the key that allows lockstep reasoning about arbitrary program constructs.
        It allows the user to postulate validity of the current claim by means of adding the \emph{current premise}
        into the set of \emph{cutpoints}, from which a $k$-tuple of program configurations satisfying the postcondition
        is claimed to be reachable. Once progress is made (by means of the Step rule), the added cutpoints are
        \emph{enabled} and can be used to finish the proof using the Axiom rule.
  \item The Step rule performs symbolic execution on a selected component $i$ (represented by $\varphi_i$)
        using the semantic rule $\varphi \Rightarrow^\exists \varphi^\prime \in S$.
        For this rule to apply, its left side ($\varphi$) has to match all the program configurations matching $\varphi_i$.
        Therefore, the rule decomposes $\varphi_i$ into $\varphi$ and an additional constraint $P^\prime$,
        which can be thought of as a part of \emph{path condition} that is \emph{local} to the component $i$.
        This local path condition $P^\prime$ is then used to constrain the right-side $\varphi^\prime$ of the selected rule.
        This proof rule also enables the cutpoints from $C$ by adding them to $E$.
  \item The Axiom rule uses an enabled cutpoint to finish the proof.
  \item The Reflexivity rule can be used to finish a proof when the premise corresponds to the conclusion.
  \item The Case rule implements case analysis on a selected component $i$.
  \item The Conseq rule is used to weaken (or generalize) the premise. It can also be used
        to propagate information between components and the global constraint.
        TODO explain flatten
  \item The Abstract rule can be used to remove existential quantifiers from the premise.
        Intuitively, this corresponds to a proof step in firstorder logic that replaces
        an existential quantifier on the left side of an implication
        with a universal quantifier over the implication, assuming that the variable
        bound by the existential quantifier does not occur free in the right side.
        In our setting, the typical way of \emph{obtaining} a toplevel existential quantifier in the premise
        is by means of the Conseq rule.
  \item The Reduce rule is a way to get completeness into our proof system:
        it reduces the goal to reachability logic reasoning.
        This rule also provides a way to prove properties that do not benefit
        from lockstep reasoning.

\end{itemize}


The following lemma says that one can generate an RL proof on a star-extended system
from a CRL proof.
This lemma is a major component of the soundness proof of CRL and can be found in \Cref{app:crlsoundness}.
\begin{lemma}\label{lem:CRLalmostSoundness}
    \begin{align*}
        & (\mathcal{T}, S) \vdash_\CRL \Psi \Downarrow_{C,E} \Psi^\prime \implies \\
        &
        (\mathcal{T}^*, S^* \cup \mathit{flatten}^\exists(E, \Psi^\prime)), \mathit{flatten}^\exists(C, \Psi^\prime) \vdash_\RL
          \mathit{flatten}^\exists(\Psi, \Psi^\prime) 
    \end{align*}
\end{lemma}

Now we can state and prove the soundness theorem, which is the main result of this paper.
\begin{theorem}[Proof system soundness]\label{thm:proofsystemSoundness}
\begin{equation*}
    (\mathcal{T}, S) \vdash_\CRL \Psi \Downarrow_{\emptyset,\emptyset} \Psi^\prime \implies
    (\mathcal{T}, S) \vDash_{\CRL} \Psi \Rightarrow^{c\exists} \Psi^\prime
\end{equation*}
\end{theorem}
\begin{proof}[Proof of \Cref{thm:proofsystemSoundness}]
Assume $(\mathcal{T}, S) \vdash_\CRL \Psi \Downarrow_{\emptyset,\emptyset} \Psi^\prime$.
By \Cref{lem:CRLalmostSoundness}, we have $(\mathcal{T}, S) \vdash_\RL \mathit{flatten}^\exists(\Psi, \Psi^\prime)$.
By soundness of reachability logic, we have $(\mathcal{T}, S) \vDash_\RL \mathit{flatten}^\exists(\Psi, \Psi^\prime)$.
By \Cref{thm:CRLandRLcorrespondence}, 
we have $(\mathcal{T}, S) \vDash_{\CRL} \Psi \Rightarrow^{c\exists} \Psi^\prime$ and we are done.
\end{proof}

% \begin{figure}
%   \begin{align*}
%     & \prftree[l]{GenWithCirc}
%       { \prfStackPremises
%         { \vec{X} = \mathit{FV}(\Phi) \setminus \mathit{FV}(\Psi^\prime)
%         }
%         {(\mathcal{T}, S) \vdash_\CRL \Phi \Downarrow_{C \cup \{ \exists \vec{X}.\, \Phi  \}, E} \Psi^\prime}
%       }
%       {(\mathcal{T}, S) \vdash_\CRL \Phi \Downarrow_{C, E} \Psi^\prime}
%   \end{align*}

%   \begin{align*}
%     & \prftree[l]{ImplConclusion}
%       {\mathcal{T}^* \vDash_\ML \mathit{flatten}(\Phi) \rightarrow \mathit{flatten}(\Psi^\prime)}
%       {(\mathcal{T}, S) \vdash_\CRL \Phi \Downarrow_{C, E} \Psi^\prime}
%   \end{align*}

%   \begin{align*}
%     & \prftree[l]{ImplEnabled}
%       { \prfStackPremises
%         { \Phi^\prime \in E }
%         {\mathcal{T}^* \vDash_\ML \mathit{flatten}(\Phi) \rightarrow \mathit{flatten}(\Phi^\prime)}
%       }
%       {(\mathcal{T}, S) \vdash_\CRL \Phi \Downarrow_{C, E} \Psi^\prime}
%   \end{align*}

%   \begin{align*}
%     & \prftree[l]{Contradiction}
%       {\mathcal{T}^* \vDash_\ML \mathit{flatten}(\Phi) \rightarrow \bot }
%       {(\mathcal{T}, S) \vdash_\CRL \Phi \Downarrow_{C, E} \Psi^\prime}
%   \end{align*}

%   \caption{Selected derived rules}
%   \label{fig:CRLderivedRules}
% \end{figure}

The proof system is also relatively complete (with respect to an oracle for deciding validity
in the underlying matching logic model $\mathcal{T}$).
\begin{theorem}[Relative completeness]\label{thm:relativeCompleteness}
  \begin{equation*}
      (\mathcal{T}, S) \vDash_{\CRL} \Psi \Rightarrow^{c\exists} \Psi^\prime \implies
      (\mathcal{T}, S) \vdash_\CRL \Psi \Downarrow_{\emptyset,\emptyset} \Psi^\prime
  \end{equation*}
  \end{theorem}
  \begin{proof}[Proof of \Cref{thm:relativeCompleteness}]
  Assume $(\mathcal{T}, S) \vDash_{\CRL} \Psi \Rightarrow^{c\exists} \Psi^\prime$.
  By \Cref{thm:CRLandRLcorrespondence}, 
  we obtain
  \begin{equation*}
    (\mathcal{T}^*, S^*) \vDash_\RL
    \mathit{flatten}^\exists(\Psi, \Psi^\prime) \, .
  \end{equation*}
  By relative completeness of reachability logic, we obtain
  \begin{equation*}
    (\mathcal{T}^*, S^*) \vdash_\RL
    \mathit{flatten}^\exists(\Psi, \Psi^\prime) \, ,
  \end{equation*}
  and we conclude the proof using the Inherit rule.
  Note that to apply relative completeness of RL, we need have an oracle for deciding validity in the extended model.
  A construction of such oracle from the oracle for deciding validity in $\mathcal{T}$ is in \Cref{app:completeness}.
  \end{proof}
Our completeness result is similar to the completeness result of CHL in the sense that the completeness
does not involve features used for lockstep reasoning.
It would be interesting to investigate whether a proof system without the Reduce rule would still be complete;
we leave this for a future work.

\section{An example proof involving lockstep reasoning}

We now present an example proof using our proof system; the proof involves lockstep reasoning.
Consider again the claim $\Omega_{\mathit{mono}}$ from \Cref{sec:CRLsemanticsComparisonToCHL}:
\begin{align*}
  [C(P, X_1, Y_1),C(P, X_2, Y_2)] \land X_1 \leq X_2 \land Y_1 \leq Y_2
   \Rightarrow^{c\exists} [C(\epsilon, ?\_, ?Y_1), C(\epsilon, ?\_, ?Y_2)] \land ?Y_1 \leq ?Y_2 \, .
\end{align*}
Let $\Psi^\prime_{\mathit{mono}}$ denote the right side of the $\Rightarrow^{c\exists}$ above.
TODO


%%% Local Variables:
%%% mode: latex
%%% TeX-master: "../main"
%%% End:
%\section{Implementation}


\section{Discussion}\label{sec:discussion}

\subsection{Relation to Cartesian Hoare logic}

\subsubsection{(Non)determinism}
We base our work on the \emph{one-path} variant of reachability logic.
Consequently, CRL inherits a known limitation of one-path reachability logic: that the tight correspondence between
one-path RL and Hoare logics is limited to deterministic languages.
However, we would still want to prove the following theorem.
\begin{theorem}
CRL extends CHL on the deterministic fragment of the CHL-supported language.
That is, given any program $P$ in the deterministic fragment of the CHL's imperative language,
a sound reachability-logic formalization (that is, a reachability system) $\mathcal{S}_{\mathit{IMP}}$ of the CHL's imperative language,
and firstorder formulas $\Phi, \Psi$ over variables $\vec{x}_1,\ldots,\vec{x}_k$,
\begin{equation*}
    \vDash_{\mathit{CHL}} ||\Phi||\ P\ ||\Psi||
    \iff
    \mathcal{S}_{\mathit{IMP}} \vDash_\CRL \mathit{tr}(P, \Phi) \Rightarrow^{c\exists} \mathit{tr}(P, \Psi) \, .
\end{equation*}
\end{theorem}
\begin{proof}
Admitted. Use the relationship between one-path CRL and all-path CRL on deterministic languages (TODO).
\end{proof}

\subsubsection{Main Idea and Proof Technique}

Our understanding of the inner workings of CHL is based on the extended, unpublished version (\cite{SousaD16Extended})
of \cite{SousaD16}.
There, the authors define a \emph{linearization operation} on lists of programs, which roughly corresponds to our
\emph{star extension} of the language's semantics.
Then, the authors prove lemmas saying that a CHL triple with a list of programs inside is derivable
in the CHL proof system
if and only if 
a Hoare triple having the same list of programs, but linearized, inside, is derivable;
the "only if" implication corresponds to our \Cref{lem:CRLalmostSoundness}, where we construct
an RL proof from a CRL proof,
while the "if" implication corresponds to our Reduce rule.
Furthermore, in the proof of soundness of CHL, the authors assume soundness of the self-composition technique;
self-composition corresponds to our \emph{star extension} and its soundness to our \Cref{thm:correspondence}.
Finally \cite{SousaD16Extended} assumes soundness and relative completeness of the underling Hoare logic;
similarly, we assume soundness and relative completeness of reachability logic.
However, for completeness, we have to prove that \emph{star extension} preserves decidability.

\subsubsection{Differences}
There are also differences between the CHL and CRL techniques.
First, our proof system has only 8 rules, and they do not mention any programming language construct,
while CHL has 17 rules (not counting the Expand rule), half of which are specific to the underlying language.

Second, there is a redundancy between the language-specific CHL rules and the Hoare logic rules of the
programming language: for example, conditional statement ("if") has (1) a rule in the formal semantics of the language,
(2) a rule in the Hoare logic (not shown in the paper), and (3) a rule in CHL.
When considering that the three rules have to play nice together (that is, CHL and Hoare logic rules have to be sound with respect to the semantics), someone had to think about the conditional statement at least five times .
We consider this to be highly uneconomical
- and the situation
is even worse for the looping construct ("while-with-breaks"), which is supported using additional CHL rules.
In contrast, in the CRL/RL framework, it is enough to design each language construct once - when giving
semantics to it.


Third, in CHL the support for lock-step reasoning is hard-wired into the rules for loops,
while in our framework, lock-step reasoning is not limited to loops, but can support arbitrary sources
of circular behavior - including loops, recursion, gotos.
Therefore, while we are inspired by the \emph{idea} of lock-step execution from \cite{SousaD16},
for the \emph{realization} of this idea we turn to the literature on \emph{equivalence checking}.

\subsection{Relation to semantic-based program equivalence}

TODO

\subsection{Other Related Work}


Also, a game-based technique for verifying software hyperproperties beyond $k$-safety
has been developed in \cite{BeutnerF22}.
This technique works with \emph{symbolic transition systems},
so it already is language-independent \emph{in some sense}. However, it is not clear how to use the technique
with an arbitrary language $L$, without writing a compiler from $L$ to symbolic transition systems first.


\section{Future Work and Conclusion}

We have presented Cartesian Reachability logic - a logic for reasoning about $k$-safety hyperproperties
in any deterministic language equipped with a RL-based operational semantics.
The logic has a simple, sound and complete proof system and allows lock-step reasoning
similarly to Cartesian Hoare logic.
Instantiating CRL with a new language does not require any changes to the soundness proof;
therefore, CRL has the potential to greatly reduce costs of development of tools and techniques for $k$-safety verification.

In future, we want to develop a variant of CRL which does not require the language to be deterministic.
We believe this to be viable, because (1) CHL has some support for nondeterminism,
and because (2) reachability logic, on which we base our work, has a newer variant that supports nondeterminism, too.
On the theoretical side, we would like to know whether our proof system would be complete even in the absence of
the Reduce rule.
An orthogonal line of future research is compositionality: we would like to enable compositional reasoning
using the technique developed in \cite{DOsualdoFD22}.
Finally, we plan to develop a practical, language-parametric tool implementing CRL, using the \K{} semantic framework.


\bibliography{bibliography}
\bibliographystyle{plain}

\appendix

%\section{Appendix}
\section{Matching and Reachability logic}\label{app:MLandRL}

\begin{definition}[Matching logic syntax and semantics]\label{def:matchinglogic}
We define the matching logic syntax and semantics as follows.
\begin{enumerate}
    \item A matching logic \emph{signature} $\mathbf{\Sigma} = (\Sigma, \mathit{Var})$ is a many-sorted algebraic signature $\Sigma$ together with a sort-wise infinite set of variables $\mathit{Var}$.
    \item Let $T_{\Sigma}(\mathit{Var})$ denote the free $\Sigma$-algebra of terms with variables 
          in $\mathit{Var}$.
          Let $T_{\Sigma, s}(\mathit{Var})$ denote the set of $\Sigma$-terms (with variables in $\mathit{Var})$ of sort $s$.
    \item Let $\mathcal{T}$ be a $\Sigma$-algebra, and $f$ be a symbol from $\Sigma$.
          We use the notation $\mathcal{T}_f$ to denote the interpretation of $f$ in $\mathcal{T}$.
    \item A function $\rho : \mathit{Var} \to \mathcal{T}$, where $\mathcal{T}$ is a $\Sigma$-algebra,
          extends uniquely (in the usual way) to a $\Sigma$-algebra morphism
          $\rho : T_{\Sigma}(\mathit{Var}) \to \mathcal{T}$.
    \item The set of \emph{nullary predicate symbols} $P_\epsilon((\Sigma, \mathit{Var}))$
          (or just $P_\epsilon$ if $(\Sigma, \mathit{Var})$ is known from the context) is
          defined to contain exactly terms $\phi \in T_{\Sigma, s}(\mathit{Var})$.
    \item A matching logic $(\Sigma, \mathit{Var})$-formula (aka $(\Sigma, \mathit{Var})$-\emph{pattern})
          of sort $s$
          is a $(\Sigma, P_\epsilon)-\FOLeq$ formula (that is, a $\FOLeq$ formula where function symbols are
          exactly symbols from $\Sigma$, and where nullary predicate symbols are exactly
          terms $\phi \in T_{\Sigma, s}(\mathit{Var})$, without any $k$-ary predicate symbols for $k > 0$).
          We let $\Pattern(\mathbf{\Sigma})$ (or just $\Pattern$ when $\mathbf{\Sigma}$ is known from the context)
          denote the set of all $\mathbf{\Sigma}$-patterns.
    \item Let $\mathit{FV}(\varphi)$ denote the set of all free variables of $\varphi$.
    \item A matching logic pattern is \emph{structureless} if it contains
          no terms $\phi \in T_{\Sigma, s}(\mathit{Var})$ used as predicates.
    \item A \emph{constrained term} is a pattern of the form $\phi \land P$,
          where $\phi \in T_{\Sigma, s}(\mathit{Var})$ and $P$ a structureless pattern.
    \item A matching logic $\Sigma$-\emph{model} $\mathcal{T}$ is a $\Sigma$-algebra with non-empty carrier sets.
    \item A matching logic \emph{semantics} is given by means of
          the satisfaction relation $\vDash_{\ML}$ between a matching logic $\Sigma$-model,
          a model element, and a valuation,
          defined inductively as
          \begin{itemize}
              \item $(\mathcal{T}, \gamma, \rho) \vDash \phi$ iff $\gamma = \rho(\phi)$, where $\phi \in T_{\Sigma, s}(\mathit{Var})$;
              \item $(\mathcal{T}, \gamma, \rho) \vDash \varphi_1 \land \varphi_2$ iff
                $(\mathcal{T}, \gamma, \rho) \vDash \varphi_1$ and
                $(\mathcal{T}, \gamma, \rho) \vDash \varphi_2$;
              \item $(\mathcal{T}, \gamma, \rho) \vDash \neg \varphi$ iff
                $(\mathcal{T}, \gamma, \rho) \not\vDash \varphi$;
              \item $(\mathcal{T}, \gamma, \rho) \vDash \exists x : s.\, \varphi$ iff
                $(\mathcal{T}, \gamma, \rho^\prime) \vDash \varphi$ for some $\rho^\prime$ such that
                $\rho^\prime(y) = \rho(y)$ for all $y \in \mathit{Var} \setminus \{ x \}$
                and $\rho^\prime(x) \in \mathcal{T}_s$;
          \end{itemize}
          
\end{enumerate}

\end{definition}


\begin{lemma}\label{lem:structurelessSemantics}
Let $\mathcal{T}$ be a matching logic $\Sigma$-model, $\gamma,\gamma^\prime \in \mathcal{T}$ model elements,
and $\rho$ a $\mathcal{T}$-valuation. Then for any structureless pattern $P$,
\begin{equation*}
    (\mathcal{T}, \gamma, \rho) \vDash P \iff (\mathcal{T}, \gamma^\prime, \rho) \vDash P \, .
\end{equation*}
Therefore, when $P$ is structureless, we may sometimes write $(\mathcal{T}, \rho) \vDash P$ to mean that 
$(\mathcal{T}, \gamma, \rho) \vDash P$ for \emph{some} $\gamma \in \mathcal{T}$.
\end{lemma}
\begin{proof}[Proof of \Cref{lem:structurelessSemantics}]
Let $P$ be a structureless pattern.
We perform induction on $P$.
If $P = \phi$, we get contradiction with the assumption that $P$ is structureless;
therefore, the conclusion holds by \emph{ex falso quodlibet}.
Other cases follow from the induction hypotheses.
\end{proof}

\begin{lemma}\label{lem:unusedVariables}
    Let $\mathcal{T}$ be a matching logic model, $\gamma$ an element of this model,
    and $\varphi$ a pattern.
    Then for any two $\mathcal{T}$-valuations $\rho,\rho^\prime$
    satisfying $\rho^\prime(y) = \rho(y)$ for any $y \in \mathit{FV}(\varphi)$,
    \begin{equation*}
        (\mathcal{T}, \gamma, \rho) \vDash \varphi \iff (\mathcal{T}, \gamma, \rho^\prime) \vDash \varphi \, .
    \end{equation*}
\end{lemma}
\begin{proof}
Admitted.
\end{proof}

\begin{definition}[\cite{StefanescuCMMSR19}]
    Given a  matching logic $(\Sigma, \mathit{Var})$-formula $\varphi$ of sort $\mathit{Cfg}$,
    and a fresh (with respect to $\varphi$) variable $\square$ of sort $\mathit{Cfg}$,
    we let $\varphi^\square$ denote the $\FOLeq$ formula formed from $\varphi$ by replacing
    nullary predicate symbols $\phi \in T_{\Sigma, \mathit{Cfg}}(\mathit{Var})$
    with equalities $\square = \phi$.
    Given a matching logic $(\Sigma, \mathit{Var})$-model $\mathcal{T}$, a $\mathcal{T}$-valuation $\rho$,
    and an element $\gamma \in \Tcfg$,
    we let the $\mathcal{T}$-valuation $\rho^\gamma$ be such that $\rho^\gamma(\square) = \gamma$,
    and $\rho^\gamma(x) = \rho(x)$ for $x \not = \square$.
\end{definition}

\begin{lemma}[\cite{StefanescuCMMSR19}]\label{lem:patternToFOLSemantics}
    Whenever $\square$ is fresh in $\varphi$, we have
    \begin{equation*}
        (\mathcal{T}, \gamma, \rho) \vDash \varphi \iff (\mathcal{T}, \rho^\gamma) \vDash \varphi^\square
    \end{equation*}    
\end{lemma}

\begin{lemma}[On equivalence and FOL translation]\label{lem:equivFOLtransl}
    Let $\varphi_1, \varphi_2$ be two matching logic formulas such that $\vDash \varphi_1 \leftrightarrow \varphi_2$.
    Then $\vDash (\varphi_1^\square)[X/\square] \leftrightarrow (\varphi_2^\square)[X/\square]$.
\end{lemma}
\begin{proof}
    Let $M$ be any matching logic model, $\gamma$ an element of $M$, and $\rho$ an $M$-valuation.
    We have to prove that $(M, \gamma, \rho) \vDash (\varphi_1^\square)[X/\square] \leftrightarrow (\varphi_2^\square)[X/\square]$,
    which is (by definition of the squaring function and substitution) equivalent to
    $(M, \gamma, \rho) \vDash ((\varphi_1 \leftrightarrow \varphi_2)^\square)[X/\square]$,
    which is
    (by \Cref{lem:varrenamesem}, because $\rho(X) = \rho[\square := \rho(X)](\square)$)
    equivalent to
    $(M, \gamma, \rho[\square := \rho(X)]) \vDash (\varphi_1 \leftrightarrow \varphi_2)^\square$,
    which is  (by \Cref{lem:patternToFOLSemantics}, because $\rho[\square := \rho(X)] = \rho^{\rho(X)}$) equivalent to
    $(M, \rho(X), \rho) \vDash \varphi_1 \leftrightarrow \varphi_2$,
    which holds by the assumption.
\end{proof}

\begin{definition}[\cite{StefanescuCMMSR19,RosuS12oopsla}]\label{def:basics}
We define reachability-logic signatures, rules, and systems as follows.
\begin{enumerate}
    \item A reachability-logic \emph{signature} is a pair $(\mathbf{\Sigma}, \mathit{Cfg})$,
          where $\mathbf{\Sigma}$ is a matching logic signature and $\mathit{Cfg}$ is a sort from $\mathbf{\Sigma}$.
          
    \item A \emph{one-path reachability rule} over reachability logic signature $(\mathbf{\Sigma}, \mathit{Cfg})$        is a pair $\varphi \Rightarrow^\exists \varphi^\prime$,
          where $\varphi$ and $\varphi^\prime$
          are $\mathbf{\Sigma}$-patterns (which can have free variables) of sort $\mathit{Cfg}$.
          
    \item A \emph{reachability system} over a reachability-logic signature $((\Sigma, \mathit{Var}), \mathit{Cfg})$
          is a pair $\mathcal{S} = (\mathcal{T}, S)$, where $\mathcal{T}$ is a $\Sigma$-algebra
          and $S$ is a set of reachability rules over $((\Sigma, \mathit{Var}), \mathit{Cfg})$.
          
    \item A rule $\varphi \Rightarrow^\exists \varphi^\prime$ over $((\Sigma, \mathit{Var}), \mathit{Cfg})$
          is \emph{weakly well-defined}
          with respect to $\Sigma$-algebra $\mathcal{T}$
          iff
          for any $\gamma \in \Tcfg$ and $\rho : \Var \to \mathcal{T}$
          with $(\gamma, \rho) \vDash \varphi$,
          there exists $\gamma^\prime \in \Tcfg$ with $(\gamma^\prime , \rho) \vDash \varphi^\prime$.
          
    \item A reachability system $\mathcal{S}$ is \emph{weakly well-defined} iff each its rule is weakly     
          well-defined.
          
    \item A reachability system $\mathcal{S} = (\mathcal{T}, S)$ over $((\Sigma, \mathit{Var}), \mathit{Cfg})$ induces
          a \emph{transition system} \\
          $(\Tcfg , \Rightarrow_{\mathcal{S}})$,
          where $\gamma \Rightarrow_{\mathcal{S}} \gamma^\prime$
          for $\gamma, \gamma^\prime \in \Tcfg$
          iff there is some rule \\ $\varphi \Rightarrow^\exists \varphi^\prime \in S$
          and some valuation $\rho : \Var \to \mathcal{T}$ with $(\gamma, \rho) \vDash \varphi$
          and $(\gamma^\prime , \rho) \vDash \varphi^\prime$.

    \item A reachability system $(\mathcal{T}, S)$ is \emph{deterministic} iff the induced transition system
          is deterministic.
          
    \item A reachability system $(\mathcal{T}, S)$ is \emph{$\epsilon$-free}
          iff for any two configurations $\sigma, \sigma^\prime \in \mathcal{T}_{\mathit{Cfg}}$, if
          $\sigma \Rightarrow_{\mathcal{S}} \sigma^\prime$, then $\sigma \not = \sigma^\prime$.

    \item A configuration $\gamma \in \Tcfg$ \emph{terminates} in $(\Tcfg , \Rightarrow_{\mathcal{S}})$
          iff there is no infinite $\Rightarrow_{\mathcal{S}})$-sequence starting with $\gamma$.
          
    \item A \emph{$\Rightarrow_{\mathcal{S}}$-path} is a finite
          sequence $\gamma_0 \Rightarrow_{\mathcal{S}} \gamma_1 \Rightarrow_{\mathcal{S}} \ldots
          \Rightarrow_{\mathcal{S}} \gamma_n$
          with $\gamma_0,\ldots,\gamma_n \in \Tcfg$.
          
    \item A $\Rightarrow_{\mathcal{S}}$-path is \emph{complete}
          iff it is not a strict prefix of any
          other $\Rightarrow_{\mathcal{S}}$-path.

    \item \label{def:oprlSemantics}
          A one-path reachability rule $\varphi \Rightarrow^\exists \varphi^\prime$ is \emph{satisfied}
          in a reachability system $\mathcal{S} = (\mathcal{T}, S)$,
          written $\mathcal{S} \vDash_\RL \varphi \Rightarrow^\exists \varphi^\prime$,
          iff for every $\gamma \in \Tcfg$
          such that $\gamma$ terminates in $(\Tcfg, \Rightarrow_{\mathcal{S}})$
          and for any valuation $\rho : \Var \to \mathcal{T}$
          such that $(\gamma, \rho) \vDash \varphi$,
          there exists some $\gamma^\prime \in \Tcfg$
          such that
          $\gamma \Rightarrow^{*}_{\mathcal{S}} \gamma^\prime$
          and $(\gamma^\prime, \rho) \vDash \varphi^\prime$.
          
          
%    \item An \emph{all-path reachability rule} is a pair
%        $\varphi \Rightarrow^\forall \varphi^\prime$ of patterns $\varphi$ and $\varphi^\prime$.
%    
%    \item \label{def:aprlSemantics}
%          An all-path reachability rule $\varphi \Rightarrow^\forall \varphi^\prime$ is \emph{satisfied},
%          written $\mathcal{S} \vDash_\RL \varphi \Rightarrow^\forall \varphi^\prime$,
%          iff for all complete $\Rightarrow_{\mathcal{S}}$-paths $\tau$
%          starting with $\gamma \in \Tcfg$ and for all $\rho : \Var \to \mathcal{T}$
%          such that $(\gamma, \rho) \vDash \varphi$,
%          there exists some $\gamma^\prime \in \tau$
%          such that $(\gamma^\prime, \rho) \vDash \varphi^\prime$.
\end{enumerate}

\end{definition}

\section{Cartesian Reachability logic}\label{app:CRL}

\begin{definition}[CRL Syntax]\label{def:opCRLSyntax}
    We define the syntax of Cartesian Reachability logic as follows:
    \begin{itemize}
        \item 
    A \emph{list-pattern} has the shape $[\varphi_1,\ldots,\varphi_k]$,
              where each $\varphi_j$ (for $j \in \{ 1, \ldots, k \} $) is a matching logic pattern.
        \item
              A \emph{constrained list-pattern (CLP)} is a conjunction $\Psi_0 \land P$
              of a list-pattern $\Psi_0$ and a structureless pattern $P$.
        \item
              An \emph{existentially-quantified constrained list-pattern (ECLP)} has the form
              $\exists \vec{Y}.\, \Psi$, where $\Psi$ is a CLP and $\vec{Y}$ is a (possibly empty) list of variables.
        \item
              A \emph{One-Path Cartesian reachability claim of arity $k$} has the shape
              $\Phi \Rightarrow^{c\exists} \Psi$,
              where $\Phi$ is a CLP and $\Psi$ is an ECLP.
    \end{itemize}
\end{definition}    

\begin{proof}[Proof of \Cref{prop:opCRLopRL}]
    Follows by firstorder reasoning from the definitions of semantics of CRL (\Cref{def:opCRLsemantics}) and RL
    (\Cref{def:basics}).
\end{proof}

\subsection{Proof of~\Cref{thm:CRLandRLcorrespondence}}\label{app:CRLandRLcorrespondence}

\begin{definition}\label{def:starextension}
We translate a language semantics into a semantics for lists of configurations as follows.
\begin{enumerate}
    \item Let $((\Sigma, \mathit{Var}), \mathit{Cfg})$ be a reachability-logic signature.
          Then $((\Sigma, \mathit{Var}^*), \mathit{Cfg})^*$ = $((\Sigma^*, \mathit{Var}^*), \mathit{Cfg}^*)$,
          where
          \begin{enumerate}
              \item $\Sigma^* = \Sigma \cup \{ \mathit{cfgitem}, \mathit{cfgconcat}, \mathit{cfgheat}, \mathit{cfgnil} \}$
              \item $\mathit{Cfg}^*$ is a fresh sort (representing the sort of lists of configurations);
              \item $\mathit{Var}^* = \mathit{Var} \cup \mathit{Var}_{\mathit{Cfg}^*}$,
              where $\mathit{Var}_{\mathit{Cfg}^*}$ is an infinite set of variables of sort $\mathit{Cfg}^*$,
              distinct from varibles in $\mathit{Var}$;
              \item $\mathit{cfgitem}$ a fresh symbol of sort $\mathit{Cfg} \to \mathit{Cfg}^*$;
              \item $\mathit{cfgnil}$ a fresh symbol of sort $\mathit{Cfg}^*$;
              \item $\mathit{cfgconcat}$ a fresh symbol of sort $\mathit{Cfg}^* \times \mathit{Cfg}^* \to \mathit{Cfg}^*$; and
              \item $\mathit{cfgheat}$ is a fresh symbol of sort $\mathit{Cfg}^* \times \mathit{Cfg} \times \mathit{Cfg}^* \to \mathit{Cfg}^*$.
          \end{enumerate}
    \item Let $S$ be a set of reachability rules over $(\mathbf{\Sigma}, \mathit{Cfg})$.
          We generate a set of reachability rules $S^*$ over $(\mathbf{\Sigma}, \mathit{Cfg})^*$
          by
          \begin{enumerate}
              \item defining a function $\mathit{heat} : \mathit{Var} \times \Pattern \times \mathit{Var} \to \Pattern$ by
              \begin{align*}
                  \mathit{heat}(L, \phi \land P, R)
                  = \mathit{cfgheat}(L, \phi, R) \land P
              \end{align*}
              \item setting
              \begin{align*}
              S^* = \{ \mathit{heat}(L, \varphi, R) \Rightarrow^\exists \mathit{heat}(L, \varphi^\prime, R)
              \mid  ( \varphi \Rightarrow^\exists \varphi^\prime ) \in S \} \, ,
            \end{align*}
                          where $L,R$ are distinct fresh variables (not occurring in any rule in $S$).
          \end{enumerate}
    \item Let $(\Sigma, \mathit{Var})$ be a matching logic signature, and let $\mathcal{T}$ be a configuration model; that is, a $\Sigma$-algebra.
          We generate a $\Sigma^*$-algebra $\mathcal{T}*$, which interprets all sorts and symbols from
          $\Sigma$ as in $\mathcal{T}$, and in addition interprets
          \begin{enumerate}
              \item the sort $\mathit{Cfg}*$ as the set of all finite lists
              $[c_1;\ldots;c_n]$ for $n \in \mathbb{N}$, where $c_i$ is an element of sort $\mathit{Cfg}$
              for any $0 \leq i \leq n$;
              \item the symbol $\mathit{cfgitem}$ as the function $\lambda c.\, [c]$;
              \item the symbol $\mathit{cfgnil}$ as the empty list ($[]$);
              \item the symbol $\mathit{cfgconcat}$ as the function $\lambda l_1,l_2.\, l_1 \texttt{++} l_2$,
                where $\texttt{++}$ is list concatenation; and
              \item the symbol $\mathit{cfgheat}$ as the function
                $\lambda l_1, c, l_2.\, l_1 \texttt{++} [c] \texttt{++} l_2$.
          \end{enumerate}

    \item Let $(\Sigma, \mathit{Var})$ be a matching logic signature, let $\mathcal{T}$ be a configuration model,
        and let $\rho$ be a $\mathcal{T}$-valuation.
        We define a $\mathcal{T}^*$-valuation $\rho^*$ by letting $\rho^*(v) = \rho(v)$ for any $v \in \mathit{Var}$,
        and letting $\rho^*(v) = a$, where $a$ is some arbitrary element of sort $\mathit{Cfg}^*$, for
        any variable $v$ of sort $\mathit{Cfg}^*$.
          
    \item Let $\mathcal{S} = (\mathcal{T}, S)$ be a reachability system over $(\Sigma, \mathit{Cfg})$.
          We generate a reachability system $\mathcal{S}^*$ over $(\Sigma, \mathit{Cfg})^*$
          by setting $\mathcal{S}^* = (\mathcal{T}^*, S^*)$.
\end{enumerate}
\end{definition}

\begin{lemma}\label{lem:rhoStarOfPi}
    Let $(\Sigma, \mathit{Var})$ be a matching logic signature, $\mathcal{T}$ be a configuration model,
    and $\rho$ be a $\mathcal{T}$-valuation. Then for any basic $(\Sigma, \mathit{Var})$-pattern $\pi$,
    \begin{equation}
        \rho^*(\pi) = \rho(\pi) \, .
    \end{equation}
\end{lemma}

\begin{proof}[Proof of \Cref{lem:rhoStarOfPi}]
    By induction on the term $\pi$.
    \begin{itemize}
        \item $\pi = v$ for $v \in \mathit{Var}$ - follows from the definition of $\rho^*$.
        \item $\pi = f(\pi_1, \ldots, \pi_k)$ - we have $\rho(\pi_i) = \rho^*(\pi_i)$ for any $i \in \{ 1, \ldots, k \}$
              by the induction hypothesis.
              Then
              \begin{align*}
                  \rho^*(f(\pi_1, \ldots, \pi_k)) 
                  = & {\mathcal{T}^*}_f(\rho^*(\pi_1), \ldots, \rho^*(\pi_k)) \\
                  = & {\mathcal{T}^*}_f(\rho(\pi_1), \ldots, \rho(\pi_k)) \\
                  = & \mathcal{T}_f(\rho(\pi_1), \ldots, \rho(\pi_k)) \\
                  = & \rho(f(\pi_1, \ldots, \pi_k))
              \end{align*}
              where the second-to-last equality holds by definition of $\mathcal{T}^*$.
    \end{itemize}
\end{proof}

\begin{lemma}\label{lem:starConservative}
    The star extension on matching logic models is conservative, in the following sense.
    For any $\Sigma$-model $\mathcal{T}$, any $\mathcal{T}$-valuation $\rho$,
    any $\Sigma$-sort $s$,
    any $\gamma \in \mathcal{T}_s$,
    and any matching logic $s$-pattern $\varphi$,
    \begin{equation*}
        (\mathcal{T}, \gamma, \rho) \vDash \varphi \iff (\mathcal{T}^*, \gamma, \rho^*) \vDash \varphi
    \end{equation*}
\end{lemma}

\begin{proof}[Proof of \Cref{lem:starConservative}]
By induction on $\varphi$.
\begin{itemize}
    \item $\varphi \equiv \pi$, where $\pi$ is a basic pattern (of sort $s$) - follows from \Cref{lem:rhoStarOfPi}.
    \item $\varphi \equiv \varphi_1 \land \varphi_2$ - follows from the induction hypothesis.
    \item $\varphi \equiv \neg \varphi^\prime$ - follows from the induction hypothesis.
    \item $\varphi \equiv \exists x : s^\prime.\, \varphi^\prime$.
    % We have
    % $ (\mathcal{T}^*, \gamma, \rho^\prime) \vDash \exists x : s^\prime.\, \varphi^\prime $
    % if and only if (by definition of $\vDash$)
    % there exists some valuation ${\rho^*}^\prime : \mathit{Var} \to \mathcal{T}^*$ such that
    % $(\mathcal{T}^*, \gamma, {\rho^*}^\prime) \vDash \varphi^\prime$
    % and ${\rho^*}^\prime(x) \in \mathcal{T}^*_{s^\prime}$\traian{you need to use $s'$ here and in the following.}
    % and ${\rho^*}^\prime(y) = \rho^*(y)$ for all $y \in \mathit{Var} \setminus \{ x \}$.
    % This holds if and only if
    % there exists some valuation $\rho^{\prime} : \mathit{Var} \to \mathcal{T}$ such that
    % $(\mathcal{T}^*, \gamma, {\rho^{\prime}}^*) \vDash \varphi^\prime$
    % and ${\rho^{\prime}}^*(x) \in \mathcal{T}^*_s$
    % and ${\rho^{\prime}}^*(y) = \rho^*(y)$ for all $y \in \mathit{Var} \setminus \{ x \}$:
    % one implication follows by letting ${\rho^*}^\prime := \rho^{\prime}$;
    % for the other implication, we let $\rho^{\prime}(v) := {\rho^*}^\prime(v)$ for all $v \in \mathit{Var}$,
    % and by the definition of star, we have ${\rho^{\prime}}^*(v) = {\rho^*}^\prime(v)$, from which the rest follows.
    % Next, by the induction hypothesis, this holds if and only if
    % there exists some valuation $\rho^{\prime} : \mathit{Var} \to \mathcal{T}$ such that
    % $(\mathcal{T}, \gamma, {\rho^{\prime}}) \vDash \varphi^\prime$
    % and ${\rho^{\prime}}^*(x) \in \mathcal{T}^*_s$
    % and ${\rho^{\prime}}^*(y) = \rho^*(y)$ for all $y \in \mathit{Var} \setminus \{ x \}$.
    % Next, by the definition of ${\rho^\prime}^*$ and $\rho^*$, this holds if and only if
    % there exists some valuation $\rho^{\prime} : \mathit{Var} \to \mathcal{T}$ such that
    % $(\mathcal{T}, \gamma, {\rho^{\prime}}) \vDash \varphi^\prime$
    % and ${\rho^{\prime}}(x) \in \mathcal{T}^*_s$
    % and ${\rho^{\prime}}(y) = \rho(y)$ for all $y \in \mathit{Var} \setminus \{ x \}$.
    % But since the star extensions interprets all sorts from $\Sigma$ as in the original model
    % (that is, $\mathcal{T}^*_s = \mathcal{T}_s$),
    % this is equivalent to the semantics of $(\mathcal{T}, \gamma, \rho) \vDash \exists x:s^\prime.\, \varphi^\prime$,
    % which is what we wanted to prove.

    % Alternate attempt:
    The induction hypothesis is: for any $\mathcal{T}$, $\gamma$, $\rho$, 
        $$(\mathcal{T}, \gamma, \rho) \vDash \varphi' \iff (\mathcal{T}^*, \gamma, \rho^\ast) \vDash \varphi'$$
    We want to prove that for any $\mathcal{T}$, $\gamma$, $\rho$, 
        $$(\mathcal{T}, \gamma, \rho) \vDash \exists x:s'. \varphi' \iff (\mathcal{T}^*, \gamma, \rho^\ast) \vDash \exists x:s'.\varphi'$$
    
    First, let us prove the left-to-right implication.
    The left-hand-side of the claim is equivalent with
    $$\exists \rho'. (\forall y. y \neq x \to \rho'(y) = \rho(y)) \wedge (T, \gamma, \rho') \vDash \varphi'$$
    From the induction hypothesis, this is further equivalent with
    $$\exists \rho'. (\forall y. y \neq x \to \rho'(y) = \rho(y)) \wedge (T^\ast, \gamma, {\rho'}^\ast) \vDash \varphi'$$
    
    Since $\forall y. y \neq x \to {\rho'}^\ast(y) = \rho'(y) = \rho(y) = \rho^\ast(y)$, we deduce $(\mathcal{T}^*, \gamma, \rho^\ast) \vDash \exists x:s'.\varphi'$.
    
    Conversely, the right-hand-side of the claim is equivalent with
    $$\exists \rho''. (\forall y. y \neq x \to \rho'(y) = \rho\ast(y)) \wedge (T^\ast, \gamma, \rho'') \vDash \varphi'$$
    Let $\rho'$ be defined by $\rho'(y) = \rho(y)$ if $y\neq x$ and $\rho'(x) = \rho''(x)$. Then it is
    easy to see that ${\rho'}^\ast = \rho''$, whence by the induction hypothesis we obtain that
    $(T, \gamma, \rho') \vDash \varphi'$, and by the definition of $\rho'$, $(T, \gamma, \rho) \vDash \exists x:s'. \varphi'$.
 \end{itemize}
\end{proof}

\begin{lemma}\label{lem:simplifyComposite}
We have
\begin{proofenv}
    \begin{equation*}
        C \Rightarrow_{\mathcal{S}^*} C^\prime
    \end{equation*}
\end{proofenv}
    if and only if
\begin{proofenv}
    there exists a rule $\phi \land P \Rightarrow^\exists \phi^\prime \land P^\prime \in S$
    and valuation $\rho : \mathit{Var}^* \to \mathcal{T}^*$ such that
    \begin{itemize}
        \item $(\mathcal{T}^*, \rho) \vDash P$; and
        \item $(\mathcal{T}^*, \rho) \vDash P^\prime$; and
        \item $C = \rho(L) \texttt{++} [\rho(\phi)] \texttt{++} \rho(R)$; and
        \item $C^\prime = \rho(L) \texttt{++} [\rho(\phi^\prime)] 
        \texttt{++} \rho(R)$,
    \end{itemize}
\end{proofenv}
\end{lemma}
\begin{proof}
We have
\begin{proofenv}
    \begin{equation*}
        C \Rightarrow_{\mathcal{S}^*} C^\prime
    \end{equation*}
\end{proofenv}
iff (by \Cref{def:basics})
\begin{proofenv}
    there exists a rule $\varphi \Rightarrow^\exists \varphi^\prime \in S^*$
    and valuation $\rho : \mathit{Var}^* \to \mathcal{T}^*$ such that
    $(\mathcal{T}^*, C, \rho) \vDash \varphi$
    and $(\mathcal{T}^*, C^\prime, \rho) \vDash \varphi^\prime$,
\end{proofenv}
iff (by
%\Cref{def:matchinglogic} and
\Cref{rem:shapeOfReachabilityRules} and \Cref{def:starextension})
\begin{proofenv}
    there exists a rule $\phi \land P \Rightarrow^\exists \phi^\prime \land P^\prime \in S$
    and valuation $\rho : \mathit{Var}^* \to \mathcal{T}^*$ such that
    \begin{equation*}
    (\mathcal{T}^*, C, \rho) \vDash \mathit{cfgheat}(L, \phi, R) \land P
    \end{equation*}
    and
    \begin{equation*}
        (\mathcal{T}^*, C^\prime, \rho) \vDash
        \mathit{cfgheat}(L, \phi^\prime, R) \land P^\prime \, ,
    \end{equation*}
\end{proofenv}
iff (by \Cref{def:matchinglogic} and \Cref{lem:structurelessSemantics})
\begin{proofenv}
    there exists a rule $\phi \land P \Rightarrow^\exists \phi^\prime \land P^\prime \in S$
    and valuation $\rho : \mathit{Var}^* \to \mathcal{T}^*$ such that
    $(\mathcal{T}^*, \rho) \vDash P$ and $(\mathcal{T}^*, \rho) \vDash P^\prime$ and
    \begin{equation*}
        (\mathcal{T}^*, C, \rho) \vDash \mathit{cfgheat}(L, \phi_l, R)
    \end{equation*}
    and
    \begin{equation*}
        (\mathcal{T}^*, C^\prime, \rho) \vDash
        \mathit{cfgheat}(L, \phi^\prime_j, R) \, ,
    \end{equation*}
\end{proofenv}
iff (by \Cref{def:matchinglogic} and \Cref{def:starextension})
\begin{proofenv}
    there exists a rule $\phi \land P \Rightarrow^\exists \phi^\prime \land P^\prime \in S$
    and valuation $\rho : \mathit{Var}^* \to \mathcal{T}^*$ such that
    \begin{itemize}
        \item $(\mathcal{T}^*, \rho) \vDash P$; and
        \item $(\mathcal{T}^*, \rho) \vDash P^\prime$; and
        \item $C = \rho(L) \texttt{++} [\rho(\phi)] \texttt{++} \rho(R)$; and
        \item $C^\prime = \rho(L)
        \texttt{++} [\rho(\phi^\prime)] \texttt{++} \rho(R)$.
    \end{itemize}
\end{proofenv}
That proves the desired equivalence.
\end{proof}


\begin{lemma}\label{lem:compositeStep}
    Let $\mathcal{S} = (\mathcal{T}, S)$ be a reachability system over $(\Sigma, \mathit{Cfg})$.
    For any $k \geq 1$, any configurations $c_1,\ldots,c_k, c^\prime \in \Tcfg$, and any $1 \leq i \leq k$,
    we have
    \begin{equation*}
        c_i \Rightarrow_{\mathcal{S}} c^\prime
                    \iff
        [c_1,\ldots,c_k] \Rightarrow_{\mathcal{S}^*} [c_1, \ldots, c_{i-1}, c^\prime, c_{i+1}, \ldots, c_k]
    \end{equation*}
\end{lemma}
\begin{proof}[Proof of \Cref{lem:compositeStep}]
We have
\begin{proofenv}
\begin{equation*}
[c_1,\ldots,c_k] \Rightarrow_{\mathcal{S}^*} [c_1, \ldots, c_{i-1}, c^\prime, c_{i+1}, \ldots, c_k]    
\end{equation*}
\end{proofenv}
iff (by \Cref{lem:simplifyComposite})
\begin{proofenv}
there exists a rule $\phi \land P \Rightarrow^\exists \phi^\prime \land P^\prime \in S$
and valuation $\rho : \mathit{Var}^* \to \mathcal{T}^*$ such that
\begin{itemize}
    \item $(\mathcal{T}^*, \rho) \vDash P$; and
    \item $(\mathcal{T}^*, \rho) \vDash P^\prime$; and
    \item $[c_1,\ldots,c_k] = \rho(L) \texttt{++} [\rho(\phi_l)] \texttt{++} \rho(R)$; and
    \item $[c_1, \ldots, c_{i-1}, c^\prime, c_{i+1}, \ldots, c_k] = \rho(L) \texttt{++} [\rho(\phi_j)] 
    \texttt{++} \rho(R)$.
\end{itemize}
\end{proofenv}
Suppose we have such valuation $\rho$.
We can surely construct valuation $\rho_0 : \mathit{Var} \to \mathcal{T}$ by letting
\begin{equation*}
\rho_0(v)=
    \begin{cases}
        \rho(v) & \text{if } \rho(v) \in \mathcal{T}\\
        a & \text{if } \rho(v) \not\in \mathcal{T}
    \end{cases}
\end{equation*}
(where $a \in \mathcal{T}$ is some arbitrary element).
Now, for any $v \in \mathit{FV}(\phi) \cup \mathit{FV}(\phi^\prime) \cup \mathit{FV}(P) \cup \mathit{FV}(P^\prime)$ it holds that
$((\rho_0)^*)(v) = \rho(v)$.
Why? Because $v$ has some sort $s$ from $\Sigma$
(that is, $s \not = \mathit{Cfg}^*$).
Therefore, we can use \Cref{lem:unusedVariables} to change the goal to one saying that
\begin{proofenv}
there exists a rule $\phi \land P \Rightarrow^\exists \phi^\prime \land P^\prime \in S$
and valuation $\rho_0 : \mathit{Var} \to \mathcal{T}$ such that
\begin{itemize}
    \item $(\mathcal{T}^*, (\rho_0)^*) \vDash P$; and
    \item $(\mathcal{T}^*, (\rho_0)^*) \vDash P^\prime$; and
    \item $[c_1,\ldots,c_k] = ((\rho_0)^*)(L) \texttt{++} [((\rho_0)^*)(\phi_l)] \texttt{++} ((\rho_0)^*)(R)$; and
    \item $[c_1, \ldots, c_{i-1}, c^\prime, c_{i+1}, \ldots, c_k] = ((\rho_0)^*)(L)
    \texttt{++} [((\rho_0)^*)(\phi_j)] 
    \texttt{++} ((\rho_0)^*)(R)$
\end{itemize}
\end{proofenv}
(where the opposite implication follows by choice $\rho := (\rho_0)^*$).
Now, we use \Cref{lem:starConservative} and definition of starred valuation to change the goal to one saying that
\begin{proofenv}
there exists a rule $\phi \land P \Rightarrow^\exists \phi^\prime \land P^\prime \in S$
and valuation $\rho_0 : \mathit{Var} \to \mathcal{T}$ such that
\begin{itemize}
    \item $(\mathcal{T}, \rho_0) \vDash P$; and
    \item $(\mathcal{T}, \rho_0) \vDash P^\prime$; and
    \item $[c_1,\ldots,c_k] = \rho_0(L) \texttt{++} [\rho_0(\phi_l)] \texttt{++} \rho_0(R)$; and
    \item $[c_1, \ldots, c_{i-1}, c^\prime, c_{i+1}, \ldots, c_k] = \rho_0(L)
    \texttt{++} [\rho_0(\phi_j)] 
    \texttt{++} \rho_0(R)$.
\end{itemize}
\end{proofenv}
Now, by list reasoning, this is equivalent to
saying that
\begin{proofenv}
there exists a rule $\phi \land P \Rightarrow^\exists \phi^\prime \land P^\prime \in S$
and valuation $\rho_0 : \mathit{Var} \to \mathcal{T}$ such that
there exists some $i^\prime$ satisfying $1 \leq i^\prime \leq k$
such that
\begin{itemize}
    \item $(\mathcal{T}, \rho_0) \vDash P_l$; and
    \item $(\mathcal{T}, \rho_0) \vDash P_j$; and
    \item $[c_1,\ldots, c_{i^\prime-1}] = \rho_0(L)$; and
    \item $c_{i^\prime} = \rho_0(\phi_l)$; and
    \item $[c_{i^\prime+1},\ldots,c_k] = \rho_0(R)$; and
    \item $[c_1, \ldots, c_{i-1}, c^\prime, c_{i+1}, \ldots, c_k] = \rho_0(L)
    \texttt{++} [\rho_0(\phi_j)] 
    \texttt{++} \rho_0(R)$.
\end{itemize}
\end{proofenv}
Now, let us define $c^\prime_{z}$ by
\begin{equation*}
c^\prime_{z} =
    \begin{cases}
        c^\prime & \text{if } z = i \\
        c_z & \text{if } z \not = i
    \end{cases}
\end{equation*}
after which the goal is equivalent to saying that
\begin{proofenv}
there exists a rule $\phi \land P \Rightarrow^\exists \phi^\prime \land P^\prime \in S$
and valuation $\rho_0 : \mathit{Var} \to \mathcal{T}$ such that
there exists some $i^\prime$ satisfying $1 \leq i^\prime \leq k$ such that
\begin{itemize}
    \item $(\mathcal{T}, \rho_0) \vDash P$; and
    \item $(\mathcal{T}, \rho_0) \vDash P^\prime$; and
    \item $[c_1,\ldots, c_{i^\prime-1}] = \rho_0(L)$; and
    \item $c_{i^\prime} = \rho_0(\phi)$; and
    \item $[c_{i^\prime+1},\ldots,c_k] = \rho_0(R)$; and
    \item $[c^\prime_1,\ldots, c^\prime_{i^\prime-1}] = \rho_0(L)$; and
    \item $c^\prime_{i^\prime} = \rho_0(\phi^\prime)$; and
    \item $[c^\prime_{i^\prime+1},\ldots,c^\prime_k] = \rho_0(R)$.
\end{itemize}
\end{proofenv}
Since $L,R$ were fresh, they do not occur in $\phi$ nor in $\phi^\prime$.
Therefore, using \Cref{lem:unusedVariables}, we can equivalently say that
\begin{proofenv}
there exists a rule $\phi \land P \Rightarrow^\exists \phi^\prime \land P^\prime \in S$
and valuation $\rho_0 : \mathit{Var} \to \mathcal{T}$ such that
there exists some $i^\prime$ satisfying $1 \leq i^\prime \leq k$
such that
\begin{itemize}
    \item $(\mathcal{T}, \rho_0) \vDash P$; and
    \item $(\mathcal{T}, \rho_0) \vDash P^\prime$; and
    \item $c_{i^\prime} = \rho_0(\phi_l)$; and
    \item $c^\prime_{i^\prime} = \rho_0(\phi_j)$.
\end{itemize}
\end{proofenv}
(The downwards implication is trivial, as it is only removing constraints; the upwards implication
is from the fact that we can always choose a valuation $\rho_0$ satisfying the constraints.)
But that is equivalent (\Cref{def:matchinglogic}) to saying that 
\begin{proofenv}
there exists a rule $\phi \land P \Rightarrow^\exists \phi^\prime \land P^\prime \in S$
and there exists some $i^\prime$ satisfying $1 \leq i^\prime \leq k$
and valuation $\rho_0 : \mathit{Var} \to \mathcal{T}$ such that
\begin{itemize}
    \item $(\mathcal{T}, c_{i^\prime}, \rho_0) \vDash \phi \land P$; and
    \item $(\mathcal{T}, c^\prime_{i^\prime}, \rho_0) \vDash \phi^\prime \land P^\prime$
    .
\end{itemize}
\end{proofenv}
But that is
equivalent to saying that
\begin{proofenv}
there exists some $i^\prime$ satisfying $1 \leq i^\prime \leq k$
such that $c_{i^\prime} \Rightarrow_{\mathcal{S}} c^\prime_{i^\prime}$,
\end{proofenv}
which is almost equivalent to the left side of the equivalence we want to prove:
that
\begin{proofenv}
$c_{i} \Rightarrow_{\mathcal{S}} c^\prime_{i}$.
\end{proofenv}
The upwards implication is trivial; the downwards is as follows. If $i = i^\prime$, we are done.
But otherwise, it would follow (by definition of $c^\prime_{i^\prime}$) that $c_{i^\prime} \Rightarrow_{\mathcal{S}} c_{i^\prime}$,
which contradicts \Cref{rem:noEmptySteps}.
\end{proof}

\begin{lemma}\label{lem:mkListSemantics}
$(\mathcal{T}^*, C, \rho) \vDash \mathit{mkList}(\phi_1,\ldots,\phi_k)$
iff there exists $c_1, \ldots, c_k \in \Tcfg$ such that $C = [c_1, \ldots, c_k]$ and for every $\rho^\prime : \mathit{Var} \to \mathcal{T}$ satisfying
$\rho^\prime(v) = \rho(v)$ for any \\
$v \in \mathit{FV}(\mathit{mkList}(\phi_1, \ldots, \phi_k))$,
it holds that 
$(\mathcal{T}, c_1, \rho^\prime) \vDash \phi_1$ and \ldots and $(\mathcal{T}, c_k, \rho^\prime) \vDash \phi_k$.
\end{lemma}
\begin{proof}
By induction on $k$.
\begin{itemize}
    \item If $k = 1$, then we have to prove that
    \begin{proofenv}
    $(\mathcal{T}^*, C, \rho) \vDash \mathit{cfgitem}(\phi_1)$
    iff there exists $c_1 \in \Tcfg$ such that $C = [c_1]$ and for every $\rho^\prime : \mathit{Var} \to \mathcal{T}$ satisfying
    $\rho^\prime(v) = \rho(v)$ for any $v \in \mathit{FV}(\mathit{cfgitem}(\phi_1))$, it holds that
    $(\mathcal{T}, c, \rho^\prime) \vDash \phi_1$.
    \end{proofenv}
    By \cref{def:matchinglogic}, this is equivalent to
    \begin{proofenv}
    $C = \rho(\mathit{cfgitem}(\phi_1))$
    iff there exists $c_1 \in \Tcfg$ such that $C = [c_1]$ and for every $\rho^\prime : \mathit{Var} \to \mathcal{T}$ satisfying
    $\rho^\prime(v) = \rho(v)$ for any $v \in \mathit{FV}(\mathit{cfgitem}(\phi_1))$, it holds that
    $c_1 = \rho^\prime(\phi_1)$.
    \end{proofenv}
    By \Cref{def:starextension}, this is equivalent to
    \begin{proofenv}
    $C = [\rho(\phi_1)]$
    iff there exists $c_1 \in \Tcfg$ such that $C = [c_1]$ and for every $\rho^\prime : \mathit{Var} \to \mathcal{T}$ satisfying
    $\rho^\prime(v) = \rho(v)$ for any $v \in \mathit{FV}(\mathit{cfgitem}(\phi_1))$, it holds that
    $c_1 = \rho^\prime(\phi_1)$.
    \end{proofenv}
    We prove each implication separately.
    For the left-to-right implication, we let $c_1 := \rho(\phi_1)$
    and have to prove that $\rho(\phi_1) = \rho^\prime(\phi_1)$, which follows from \Cref{lem:unusedVariables}.
    The right-to-left implication also follows from  \Cref{lem:unusedVariables}.
    
    \item If $k = k^\prime + 1$, we assume the induction hypothesis saying that
    \begin{proofenv}
    for every $C, \phi_1, \ldots, \phi_{k^\prime}$,
    $(\mathcal{T}^*, C, \rho) \vDash \mathit{mkList}(\phi_1,\ldots,\phi_{k^\prime})$
    iff there exists $c_1, \ldots, c_{k^\prime} \in \Tcfg$ such that $C = [c_1, \ldots, c_{k^\prime}]$
    and for every $\rho^\prime : \mathit{Var} \to \mathcal{T}$ satisfying
    $\rho^\prime(v) = \rho(v)$ for any
    $v \in \mathit{FV}(\mathit{mkList}(\phi_1, \ldots, \phi_{k^\prime}))$,
    it holds that
    $(\mathcal{T}, c_1, \rho^\prime) \vDash \phi_1$ and \ldots and $(\mathcal{T}, c_{k^\prime}, \rho^\prime) \vDash \phi_k$,
    \end{proofenv}
    and have to prove that
    \begin{proofenv}
    $(\mathcal{T}^*, C, \rho) \vDash \mathit{mkList}(\phi_1,\ldots,\phi_{k^\prime + 1})$
    iff there exists $c_1, \ldots, c_{k^\prime + 1} \in \Tcfg$ such that $C = [c_1, \ldots, c_{k^\prime + 1}]$ and 
    for every $\rho^\prime : \mathit{Var} \to \mathcal{T}$ satisfying
    $\rho^\prime(v) = \rho(v)$ for any
    $v \in \mathit{FV}(\mathit{mkList}(\phi_1, \ldots, \phi_{k^\prime + 1}))$,
    it holds that
    $(\mathcal{T}, c_1, \rho^\prime) \vDash \phi_1$ and \ldots and $(\mathcal{T}, c_{k^\prime + 1}, \rho^\prime) \vDash \phi_{k^\prime + 1}$,
    \end{proofenv}
    which is (by \Cref{def:matchinglogic}) equivalent to
    \begin{proofenv}
    $C = \rho(\mathit{mkList}(\phi_1,\ldots,\phi_{k^\prime + 1}))$
    iff there exists $c_1, \ldots, c_{k^\prime + 1} \in \Tcfg$ such that $C = [c_1, \ldots, c_{k^\prime + 1}]$
    and for every $\rho^\prime : \mathit{Var} \to \mathcal{T}$ satisfying
    $\rho^\prime(v) = \rho(v)$ for any
    $v \in \mathit{FV}(\mathit{mkList}(\phi_1, \ldots, \phi_{k^\prime + 1}))$,
    it holds that
    $c_1 = \rho^\prime(\phi_1)$ and \ldots and $c_{k^\prime + 1} = \rho^\prime(\phi_{k^\prime + 1})$,
    \end{proofenv}
    which is (by \Cref{def:starextension}) equivalent to
    \begin{proofenv}
    $C = [\rho(\phi_1)] \texttt{++} C^\prime$ and $C^\prime = \rho(\mathit{mkList}(\phi_2,\ldots,\phi_{k^\prime + 1}))$
    iff there exists $c_1, \ldots, c_{k^\prime + 1} \in \Tcfg$ such that $C = [c_1, \ldots, c_{k^\prime + 1}]$
    and for every $\rho^\prime : \mathit{Var} \to \mathcal{T}$ satisfying
    $\rho^\prime(v) = \rho(v)$ for any
    $v \in \mathit{FV}(\mathit{mkList}(\phi_1, \ldots, \phi_{k^\prime + 1}))$,
    it holds that
    $c_1 = \rho^\prime(\phi_1)$ and \ldots and $c_{k^\prime + 1} = \rho^\prime(\phi_{k^\prime + 1})$,
    \end{proofenv}
    which is by the induction hypothesis with $\phi_1 := \phi_2,\ldots,\phi_k := \phi_{k^\prime + 1}$
    and $\alpha$-renaming
    equivalent to
    \begin{proofenv}
    $C = [\rho(\phi_1)] \texttt{++} C^\prime$ and
    there exists $c_2, \ldots, c_{k^\prime + 1} \in \Tcfg$ such that $C^\prime = [c_2, \ldots, c_{k^\prime+1}]$ 
    and for every $\rho^\prime : \mathit{Var} \to \mathcal{T}$ satisfying
    $\rho^\prime(v) = \rho(v)$ for any
    $v \in \mathit{FV}(\mathit{mkList}(\phi_2, \ldots, \phi_{k^\prime+1}))$,
    it holds that
    $(\mathcal{T}, c_2, \rho^\prime) \vDash \phi_2$ and \ldots and $(\mathcal{T}, c_{k^\prime+1}, \rho^\prime) \vDash \phi_{k^\prime + 1}$,
    iff there exists $c_1, \ldots, c_{k^\prime + 1} \in \Tcfg$ such that $C = [c_1, \ldots, c_{k^\prime + 1}]$
    and for every $\rho^\prime : \mathit{Var} \to \mathcal{T}$ satisfying
    $\rho^\prime(v) = \rho(v)$ for any
    $v \in \mathit{FV}(\mathit{mkList}(\phi_1, \ldots, \phi_{k^\prime + 1}))$,
    it holds that
    $c_1 = \rho^\prime(\phi_1)$ and \ldots and $c_{k^\prime + 1} = \rho^\prime(\phi_{k^\prime + 1})$,
    \end{proofenv}
    which is (by firstorder reasoning and simplification of list append) equivalent to
    \begin{proofenv}
    there exists $c_2, \ldots, c_{k^\prime + 1} \in \Tcfg$ such that
    $C = [\rho(\phi_1), c_2, \ldots, c_{k^\prime+1}]$
    and for every $\rho^\prime : \mathit{Var} \to \mathcal{T}$ satisfying
    $\rho^\prime(v) = \rho(v)$ for any
    $v \in \mathit{FV}(\mathit{mkList}(\phi_2, \ldots, \phi_{k^\prime+1}))$,
    it holds that
    $(\mathcal{T}, c_2, \rho^\prime) \vDash \phi_2$ and \ldots and $(\mathcal{T}, c_{k^\prime+1}, \rho^\prime) \vDash \phi_{k^\prime+1}$,
    iff there exists $c_1, \ldots, c_{k^\prime + 1} \in \Tcfg$ such that $C = [c_1, \ldots, c_{k^\prime + 1}]$
    and for every $\rho^\prime : \mathit{Var} \to \mathcal{T}$ satisfying
    $\rho^\prime(v) = \rho(v)$ for any
    $v \in \mathit{FV}(\mathit{mkList}(\phi_1, \ldots, \phi_{k^\prime + 1}))$,
    it holds that
    $c_1 = \rho^\prime(\phi_1)$ and \ldots and $c_{k^\prime + 1} = \rho^\prime(\phi_{k^\prime + 1})$.
    \end{proofenv}
    We simplify the goal using \Cref{def:matchinglogic} to
    \begin{proofenv}
    there exists $c_2, \ldots, c_{k^\prime + 1} \in \Tcfg$ such that
    $C = [\rho(\phi_1), c_2, \ldots, c_{k^\prime+1}]$
    and for every $\rho^\prime : \mathit{Var} \to \mathcal{T}$ satisfying
    $\rho^\prime(v) = \rho(v)$ for any
    $v \in \mathit{FV}(\mathit{mkList}(\phi_2, \ldots, \phi_{k^\prime+1}))$,
    it holds that
    $c_2 = \rho^\prime(\phi_2)$ and \ldots and $c_{k^\prime+1} = \rho^\prime(\phi_{k^\prime+1})$,
    iff there exists $c_1, \ldots, c_{k^\prime + 1} \in \Tcfg$ such that $C = [c_1, \ldots, c_{k^\prime + 1}]$
    and for every $\rho^\prime : \mathit{Var} \to \mathcal{T}$ satisfying
    $\rho^\prime(v) = \rho(v)$ for any
    $v \in \mathit{FV}(\mathit{mkList}(\phi_1, \ldots, \phi_{k^\prime + 1}))$,
    it holds that
    $c_1 = \rho^\prime(\phi_1)$ and \ldots and $c_{k^\prime + 1} = \rho^\prime(\phi_{k^\prime + 1})$.
    \end{proofenv}
    We prove each implication separately.
    \begin{itemize}
        \item Assuming
        \begin{proofenv}
        there exists $c_2, \ldots, c_{k^\prime + 1} \in \Tcfg$ such that
        $C = [\rho(\phi_1), c_2, \ldots, c_{k^\prime+1}]$
        and for every $\rho^\prime : \mathit{Var} \to \mathcal{T}$ satisfying
        $\rho^\prime(v) = \rho(v)$ for any
        $v \in \mathit{FV}(\mathit{mkList}(\phi_2, \ldots, \phi_{k^\prime+1}))$,
        it holds that
        $c_2 = \rho^\prime(\phi_1)$ and \ldots and $c_{k^\prime+1} = \rho^\prime(\phi_k)$,
        \end{proofenv}
        we prove that
        \begin{proofenv}
        there exists $c_1, \ldots, c_{k^\prime + 1} \in \Tcfg$ such that $C = [c_1, \ldots, c_{k^\prime + 1}]$
        and for every $\rho^\prime : \mathit{Var} \to \mathcal{T}$ satisfying
        $\rho^\prime(v) = \rho(v)$ for any
        $v \in \mathit{FV}(\mathit{mkList}(\phi_1, \ldots, \phi_{k^\prime + 1}))$,
        it holds that
        $c_1 = \rho^\prime(\phi_1)$ and \ldots and $c_{k^\prime + 1} = \rho^\prime(\phi_{k^\prime + 1})$.
        \end{proofenv}
        by choosing $c_1 := \rho^\prime(\phi_1)$ and using \Cref{lem:unusedVariables}\\
        (note that $\mathit{FV}(\mathit{mkList}(\phi_2,\ldots,\phi_{k^\prime+1})) \subseteq \mathit{FV}(\mathit{mkList}(\phi_1,\ldots,\phi_{k^\prime+1}))$).
        \item Assuming
        \begin{proofenv}
        there exists $c_1, \ldots, c_{k^\prime + 1} \in \Tcfg$ such that $C = [c_1, \ldots, c_{k^\prime + 1}]$
        and for every $\rho^\prime : \mathit{Var} \to \mathcal{T}$ satisfying
        $\rho^\prime(v) = \rho(v)$ for any
        $v \in \mathit{FV}(\mathit{mkList}(\phi_1, \ldots, \phi_{k^\prime + 1}))$,
        it holds that
        $c_1 = \rho^\prime(\phi_1)$ and \ldots and $c_{k^\prime + 1} = \rho^\prime(\phi_{k^\prime + 1})$,
        \end{proofenv}
        we prove that
        \begin{proofenv}
        there exists $c_2, \ldots, c_{k^\prime + 1} \in \Tcfg$ such that
        $C = [\rho(\phi_1), c_2, \ldots, c_{k^\prime+1}]$
        and for every $\rho^\prime : \mathit{Var} \to \mathcal{T}$ satisfying
        $\rho^\prime(v) = \rho(v)$ for any
        $v \in \mathit{FV}(\mathit{mkList}(\phi_2, \ldots, \phi_{k^\prime+1}))$,
        it holds that
        $c_2 = \rho^\prime(\phi_1)$ and \ldots and $c_{k^\prime+1} = \rho^\prime(\phi_k)$
        \end{proofenv}
        by setting $c_i := c_i$
        (and again noting that $\mathit{FV}(\mathit{mkList}(\phi_2,\ldots,\phi_{k^\prime+1})) \subseteq \mathit{FV}(\mathit{mkList}(\phi_1,\ldots,\phi_{k^\prime+1}))$).
    \end{itemize}
\end{itemize}
\end{proof}

\begin{lemma}\label{lem:transitionOnlyBetweenListsOfSameLength}
    Let $\mathcal{S} = (\mathcal{T}, S)$ be a reachability system over $(\Sigma, \mathit{Cfg})$.
    Then for any $C,C^\prime \in \mathcal{T}^*_{\mathit{Cfg}^*}$,
    if $C \Rightarrow_{\mathcal{S}^*} C^\prime$,
    then the length of $C$ (it is a list) is the same as the length of $C^\prime$.
\end{lemma}
\begin{proof}
Assume $C \Rightarrow_{\mathcal{S}^*} C^\prime$.
Then by \Cref{lem:simplifyComposite},
\begin{proofenv}
    there exists a rule $\phi \land P_l \Rightarrow^\exists \phi^\prime \land P^\prime \in S$
    and valuation $\rho : \mathit{Var}^* \to \mathcal{T}^*$ such that
    \begin{itemize}
        \item $(\mathcal{T}^*, \rho) \vDash P$; and
        \item $(\mathcal{T}^*, \rho) \vDash P^\prime$; and
        \item $C = \rho(L) \texttt{++} [\rho(\phi)] \texttt{++} \rho(R)$; and
        \item $C^\prime = \rho(L) \texttt{++} [\rho(\phi^\prime)] 
        \texttt{++} \rho(R)$.
    \end{itemize}
\end{proofenv}
But then $C$ and $C^\prime$ have the same length.
\end{proof}

\begin{lemma}[At most one component changes]\label{lem:atMostOneComponentChanges}
    Let $\mathcal{S} = (\mathcal{T}, S)$ be a reachability system over $(\Sigma, \mathit{Cfg})$.
    Then for any $C,C^\prime \in \mathcal{T}^*{\mathit{Cfg}^*}$ satisfying $C \Rightarrow_{\mathcal{S}^*} C^\prime$
    there exists some $i \in \mathbb{N}$ such that
    for every $i^\prime \in \mathbb{N}$ such that $i^\prime \not = i$,
    we have $C[i^\prime] = C^\prime[i^\prime]$ if both are defined.
\end{lemma}
\begin{proof}
Assume $C \Rightarrow_{\mathcal{S}^*} C^\prime$.
Then by \Cref{lem:simplifyComposite},
\begin{proofenv}
    there exists a rule $\phi \land P \Rightarrow^\exists \phi^\prime \land P^\prime \in S$
    and valuation $\rho : \mathit{Var}^* \to \mathcal{T}^*$ such that
    \begin{itemize}
        \item $(\mathcal{T}^*, \rho) \vDash P_l$; and
        \item $(\mathcal{T}^*, \rho) \vDash P_j$; and
        \item $C = \rho(L) \texttt{++} [\rho(\phi)] \texttt{++} \rho(R)$; and
        \item $C^\prime = \rho(L) \texttt{++} [\rho(\phi^\prime)] 
        \texttt{++} \rho(R)$.
    \end{itemize}
\end{proofenv}
But then we can let $i := |\rho(L)|$, and the rest follows.
\end{proof}


The following definition and theorem on filtering infinite sequences
are based on the Coq development of \cite{ZamfirVLSM}
(specifically, on \url{https://github.com/runtimeverification/vlsm/blob/d6c8cee56708c7be2431b9743fe80ca6a7a29a58/theories/VLSM/Lib/StreamFilters.v}).
\begin{definition}[Filtering subsequence]\label{def:filteringSubsequence}
Given a set $A$, a subset $P \subseteq A$ and a function $s : \mathbb{N} \to A$,
a function $\mathit{ns} : \mathbb{N} \to \mathbb{N}$ is called a filtering subsequence for $P$ on $s$,
iff
\begin{enumerate}
    \item $\mathit{ns}$ is monotone;
    \item $s(x) \not \in P$ for any $x < ns(0)$;
    \item $s(\mathit{ns}(j)) \in P$ for any $j \in \mathbb{N}$; and
    \item for every $j \in \mathbb{N}$ and every $x$ such that $\mathit{ns}(j) < x < \mathit{ns}(j+1)$,
          $s(x) \not\in P$.
\end{enumerate}
Intuitively, the last condition says that $ns$ does not skip any $P$-element in $s$.
\end{definition}

\begin{lemma}[Existence of filtering sequence for infinite occurrences]\label{lem:filteringSubsequenceExistsForInfinite}
    Let $A$ be a set, let $P \subseteq A$, and let $s : \mathbb{N} \to A$ be a function whose output
    falls to $P$ infinitely often (that is, $s(i) \in P$ for infinitely many $i$).
    Then there exists a filtering subsequence for $P$ on $s$.
\end{lemma}

\begin{lemma}\label{lem:terminationComposite}
    For any reachability system $\mathcal{S} = (\mathcal{T}, S)$, any $C \in \mathcal{T}^*_{\mathit{Cfg}^*}$,
    and any $c_1,\ldots,c_k \in \Tcfg$ such that
    $C = [c_1,\ldots,c_k]$, $C$ is terminating in $(\mathcal{T}^*_{\mathit{Cfg}^*}, \Rightarrow_{S^*})$
    iff for every $j \in \{ 1, \ldots, k \}$, $c_j$ is terminating in $(\Tcfg, \Rightarrow_S)$.
\end{lemma}
\begin{proof}[Proof of \Cref{lem:terminationComposite}]
We prove both implications separately, by contraposition.
\begin{itemize}
    \item Suppose some $c_j$ is not terminating in $(\Tcfg, \Rightarrow_S)$.
    In other words, there exists some infinite $\Rightarrow_{\mathcal{S}}$-sequence
    $c_j = d(0) \Rightarrow_{\mathcal{S}} d(1) \Rightarrow_{\mathcal{S}} d(2) \Rightarrow_{\mathcal{S}} \ldots$.
    Then $C = [c_1,\ldots,c_{j-1}, d(0), c_{j+1}, \ldots, c_k] \Rightarrow_{\mathcal{S}^*}
    [c_1,\ldots,c_{j-1}, d(1), c_{j+1}, \ldots, c_k] \Rightarrow_{\mathcal{S}^*} \ldots$
    is (by \Cref{lem:compositeStep}) an infinite $\Rightarrow_{\mathcal{S}^*}$-sequence.
    Therefore, $C$ is not terminating in $(\mathcal{T}^*_{\mathit{Cfg}^*}, \Rightarrow_{S^*})$.
    \item Suppose $C$ is not terminating in $(\mathcal{T}^*_{\mathit{Cfg}^*}, \Rightarrow_{S^*})$.
    In other words, there exists an infinite sequence $C = D(0) \Rightarrow_{\mathcal{S}^*} D(1) \Rightarrow_{\mathcal{S}^*} \ldots$.
    Then there exists a component $j$ of the sequence which changes infinitely often in the sequence,
    because we have only $k$ components.
    Now, consider the function $s : \mathbb{N} \to \mathcal{T}^*_{\mathit{Cfg}^*} \times \mathcal{T}^*_{\mathit{Cfg}^*}$
    defined by $s(i) = (D(i), D(i+1))$, and let $P \subseteq \mathcal{T}^*_{\mathit{Cfg}^*} \times \mathcal{T}^*_{\mathit{Cfg}^*}$
    be defined by $(X, X^\prime) \in P$ iff $X[j] \not = X^\prime[j]$.
    By \Cref{lem:atMostOneComponentChanges}, we know that whenever $(X, X^\prime) \in P$,
    then for any $j^\prime$ satisfying $1 \leq j^\prime \leq k$ and $j^\prime \not = j$,
    we have $X[j^\prime] = X^\prime[j^\prime]$.
    Then, $s(i) \in P$ iff in the sequence $C$, on position $i$, it is exactly the $j$th component (and no other)
    which makes step.
    Now, by \Cref{lem:filteringSubsequenceExistsForInfinite}, there exists a filtering subsequence $\mathit{ns}$
    for $P$ on $s$.
    But then
    \begin{equation*}
        D(\mathit{ns}(0))[j] \Rightarrow_S D(\mathit{ns}(1))[j] \Rightarrow_S D(\mathit{ns}(2))[j] \Rightarrow_S \ldots
    \end{equation*}
    is a $(\Tcfg, \Rightarrow_S)$ sequence witnessing the non-termination of $D(0)[j] = c_j$.
    Indeed, we have
    \begin{itemize}
        \item $D(\mathit{ns}(0))[j] = D(0)[j]$, by (2) of \Cref{def:filteringSubsequence}, the definition of $P$,
        and transitivity of equality;
        \item for any $i \in \mathbb{N}$, $D(\mathit{ns}(i))[j] \Rightarrow_{\mathcal{S}} D(\mathit{ns}(i+1))[j]$.
        We prove this as follows. By
        (4) of \Cref{def:filteringSubsequence} and definition of $P$ we have
        $D(\mathit{ns}(i+1))[j] = D(\mathit{ns}(i)+1)[j]$.
        Therefore, it is enough to show that
        \begin{equation*}
            D(\mathit{ns}(i))[j] \Rightarrow_{\mathcal{S}} D(\mathit{ns}(i)+1)[j] \, .
        \end{equation*}
        By (3) of \Cref{def:filteringSubsequence} and definition of $P$ we have
        $D(\mathit{ns}(i))[j] \not = D(\mathit{ns}(i)+1)[j]$.
        By \Cref{lem:compositeStep}, it is enough to show that there exists $k \geq 1$,
        $c_1,\ldots,c_k,c^\prime \in \Tcfg$, and some $i$ satisfying $1 \leq i \leq k$,
        such that
        $[c_1,\ldots,c_k] = D(\mathit{ns}(i))$
        and
        $[c_1,\ldots,c_{i-1},c^\prime,c_{i+1},c_k] = D(\mathit{ns}(i) + 1)$.
        But that follows from the fact that $D(\mathit{ns}(i)) \Rightarrow_{\mathcal{S}^*} D(\mathit{ns}(i) + 1)$
        and that $D(\mathit{ns}(i))[j] \not = D(\mathit{ns}(i)+1)[j]$
        by \Cref{lem:transitionOnlyBetweenListsOfSameLength} and \Cref{lem:atMostOneComponentChanges}.
    \end{itemize}
\end{itemize}
\end{proof}


\begin{lemma}\label{lem:reachComposite}
    For any reachability system $\mathcal{S} = (\mathcal{T}, S)$, any $C,C^\prime \in \mathcal{T}^*_{\mathit{Cfg}^*}$,
    and any $c_1,\ldots,c_k,c_1^\prime,\ldots,c_k^\prime \in \Tcfg$ such that
    $C = [c_1,\ldots,c_k]$ and $C^\prime = [c_1^\prime,\ldots,c_k^\prime]$,
    $C \Rightarrow^*_{\mathcal{S}^*} C^\prime$ iff for every $i \in \{ 1, \ldots, k \}$,
    $c_i \Rightarrow^*_{\mathcal{S}} c_i^\prime$.
\end{lemma}
\begin{proof}[Proof of \Cref{lem:reachComposite}]
We prove each implication separately.
\begin{itemize}
    \item For the "if" implication, we assume that $c_i \Rightarrow_{\mathcal{S}}^* c_i^\prime$
          for any $i \in \{ 1, \ldots, k \}$,
          and have to prove that $[c_1,\ldots,c_k] \Rightarrow_{\mathcal{S}^*}^* [c_1^\prime,\ldots,c_k^\prime]$.
          We will prove that for any $j \in \{ 1, \ldots, k \}$, we it holds that
          \begin{equation*}
           [c^\prime_1,\ldots,c^\prime_{j-1}, c_j, c_{j+1}, \ldots, c_k] \Rightarrow_{\mathcal{S}^*}^* [c^\prime_1,\ldots,c^\prime_{j-1}, c_j^\prime, c_{j+1}, \ldots, c_k]    \, ,
          \end{equation*}
          from which the goal follows by transitivity.
          Ok then, let $j \in \{ 1, \ldots, k \}$.
          By \Cref{lem:compositeStep}, it is enough to prove that $c_j \Rightarrow^*_{\mathcal{S}} c_j^\prime$.
          But that holds by the assumption.
    \item For the "only if" implication, we assume $C \Rightarrow^*_{\mathcal{S}^*} C^\prime$,
          $i \in \{ 1,\ldots,k \}$, and have to prove that $c_i \Rightarrow^*_{\mathcal{S}} c_i^\prime$.
          Let $C_1,\ldots,C_l \in \mathcal{T}^*_{\mathit{Cfg}^*}$ be a sequence witnessing $C \Rightarrow^*_{\mathcal{S}^*} C^\prime$;
          that is, we have $C = C_1$, $C_l = C^\prime$, and $C_j \Rightarrow_{\mathcal{S}^*} C_{j+1}$ for any $j \in \{ 1, \ldots, l-1 \}$.
          Let $i_1,\ldots,i_m \in \{ 1,\ldots,l-1 \}$ be a strictly increasing sequence of maximal length such that
          $C_{i_j}[i] \not = C_{i_j + 1}[i]$ for any $j \in \{ 1,\ldots,m \}$;
          that is, the sequence of positions in the witnessing sequence when the component $i$ changes.
          Then clearly, $C_1[i] = C_{i_1}[i]$ (otherwise we could create a longer sequence).
          Similarly, $C_l[i] = C_{i_m}[i]$.
          Now we claim that $C_{i_1}[i] \Rightarrow_{\mathcal{S}} \ldots \Rightarrow_{\mathcal{S}} C_{i_m}[i]$,
          from which the conclusion easily follows.
          We have to prove that for any $o \in \{ 1, \ldots, m \}$, it holds that
          $C_{i_o}[i] \Rightarrow_{\mathcal{S}} C_{i_{o+1}}[i]$.
          Let $d := i_{o+1} - i_{o}$; clearly, we have $d > 0$.
          By the definition of $d$, we have $C_{i_{o+1}}[i] = C_{i_{o} + d}[i]$.
          By definition of the sequence, in particular by maximality, we have $C_{i_{o} + d}[i] = C_{i_{o} + 1}[i]$
          (because there can be no change of the component $i$ between the change at the position $i_o$ and the change at the position $i_{o+1}$).
          Therefore, it is enough to show that
          $C_{i_o}[i] \Rightarrow_{\mathcal{S}} C_{i_{o}+1}[i]$.
          By \Cref{lem:compositeStep} (using also \Cref{lem:atMostOneComponentChanges} and \Cref{lem:transitionOnlyBetweenListsOfSameLength}), it is enough to show that
          $C_{i_{o}} \Rightarrow_{\mathcal{S}^*} C_{i_{o}+1}$, but that is trivial and we are done.
\end{itemize}
\end{proof}

\begin{lemma}[A semantic property of variable renaming]\label{lem:varrenamesem}
  For any two variables $X,Y$ of the same sort, and for any two $\mathcal{T}$-valuations $\rho_1, \rho_2$
  which agree on all variables other than $X,Y$, if $\rho_1(X) = \rho_2(Y)$, then for any matching logic formula
  $\varphi$ in which $Y$ does not occur free,
  \begin{equation*}
      (\mathcal{T}, \gamma, \rho_1) \vDash \varphi \iff (\mathcal{T}, \gamma, \rho_2) \vDash \varphi[Y/X] \, .
  \end{equation*}
\end{lemma}
\begin{proof}[Proof of \Cref{lem:varrenamesem}]
    We first prove that for any term $t \in T_{\Sigma, s}(\mathit{Var})$,
    $\rho_1(t) = \rho_2(t[Y/X])$, by induction in $t$:
    if $t$ is a variable, then we perform case analysis on whether the variable is $X$ or not, and we are done;
    if $t$ is a function application, then we use the induction hypothesis and the fact that function application preserves equality.
    Now we prove the lemma by induction on the size of $\varphi$, generalizing over $\rho_1$ and $\rho_2$.
    \begin{itemize}
        \item If $\varphi \equiv \phi$ for some $\phi \in T_{\Sigma, s}(\mathit{Var})$, we apply the above property.
        \item If $\varphi$ is a conjunction or negation, the desired property follows directly from the induction hypothesis.
        \item If $\varphi \equiv \exists Z : s.\, \varphi^\prime$.
        \begin{itemize}
            \item If $X = Z$, then we have to prove that $(\mathcal{T}, \gamma, \rho^\prime) \vDash \varphi^\prime$
            for some $\rho^\prime$ satisfying $\rho^\prime(y) = \rho_1(y)$ for all $y \in \mathit{Var} \setminus \{ Z \}$, iff $(\mathcal{T}, \gamma, \rho^{\prime\prime}) \vDash \varphi^\prime$
            for some $\rho^{\prime\prime}$ satisfying $\rho^{\prime\prime}(y) = \rho_2(y)$ for all $y \in \mathit{Var} \setminus \{ Z \}$ - which is easy.
            \item If $X \not = Z$, then we have to prove that
            $(\mathcal{T}, \gamma, \rho^\prime) \vDash \varphi^\prime$
            for some $\rho^\prime$ satisfying $\rho^\prime(y) = \rho_1(y)$ for all $y \in \mathit{Var} \setminus \{ Z \}$, iff $(\mathcal{T}, \gamma, \rho^{\prime\prime}) \vDash \varphi^\prime[Y/X]$ for some $\rho^{\prime\prime}$ satisfying $\rho^{\prime\prime}(y) = \rho_2(y)$ for all $y \in \mathit{Var} \setminus \{ Z \}$.
            Since the size of $\varphi^\prime$ is smaller then the size of $\varphi$,
            and since $\rho^{\prime}(X) = \rho_1(X) = \rho_2(Y) = \rho^{\prime\prime}(Y)$,
            we can use the induction hypothesis and finish the proof using firstorder reasoning.
        \end{itemize}
    \end{itemize}
\end{proof}



\begin{definition}\label{def:mkList}
We define $\mathit{mkList}$ by letting
\begin{itemize}
    \item $\mathit{mkList}(\phi) = \mathit{cfgitem}(\phi)$; and
    \item $\mathit{mkList}(\phi_1, \ldots, \phi_k) = \mathit{cfgconcat}(\mathit{cfgitem}(\phi), \mathit{mkList}(\phi_2, \ldots, \phi_k))$ whenever $k > 1$.
\end{itemize}
\end{definition}

\begin{definition}\label{def:flatten}
    We define a function $\mathit{flatten}$ from (potentially existentially-quantified) constrained list patterns
    to matching logic patterns over a star-extended signature by
    \begin{align*}
        & \mathit{flatten}(\exists \vec{X}.\, [\varphi_1, \ldots, \varphi_k) \land P] \equiv \\
        & \exists \vec{X}.\, \mathit{mkList}(Y_1, \ldots, Y_k)
        \land (\varphi_1^\square)[Y_1/\square] \land \ldots
        \land (\varphi_k^\square)[Y_k/\square] \land P \, ,
    \end{align*}
    where $Y_1,\ldots,Y_k,\square$ are fresh.
    Furthermore, we let
    \begin{equation*}
        \mathit{flatten}^\exists(\Psi,\Psi^\prime) \equiv \mathit{flatten}(\Psi) \Rightarrow^\exists \mathit{flatten}(\Psi^\prime) \, .
    \end{equation*}
\end{definition}


\begin{lemma}[On Flattening]\label{lem:flatten}
    For any matching logic $\Sigma$-model $\mathcal{T}$, any $C \in \mathcal{T}^*_{\mathit{Cfg}^*}$,
    and any $\mathcal{T}^*$-valuation $\rho$, we have
    %\begin{proofenv}
        \begin{equation*}
            (\mathcal{T}^*, C, \rho) \vDash \mathit{flatten}(\exists \vec{X}.\, [\varphi_1,\ldots,\varphi_k] \land P)
        \end{equation*}
    %\end{proofenv}
    if and only if
    %\begin{proofenv}
        there exist configurations $c_1, \ldots, c_k \in \mathcal{T}_{\mathit{Cfg}}$ such that
        $C = [c_1, \ldots, c_k]$ and there exists a $\mathcal{T}$-valuation $\rho_0$
        satisfying $\rho_0(v) = \rho(v)$ for any $v \in \mathit{Var} \setminus \vec{X}$
        such that for any $j \in \{ 1, \ldots, k \}$,
        $(\mathcal{T}, c_j, \rho_0) \vDash \varphi_j \land P$.
    %\end{proofenv}
\end{lemma}


\begin{proof}[Proof of \Cref{lem:flatten}]
    We have
    \begin{proofenv}
        \begin{equation*}
            (\mathcal{T}^*, C, \rho) \vDash \mathit{flatten}(\exists \vec{X}.\, (\varphi_1,\ldots,\varphi_k) \land P)
        \end{equation*}
    \end{proofenv}
    if and only if (by unfolding the definition of $\mathit{flatten}$ and \Cref{def:matchinglogic})
    \begin{proofenv}
        there exists a $\mathcal{T}^*$-valuation $\rho^\prime$ satisfying $\rho^\prime(v) = \rho(v)$
        for any $v \in \mathit{Var}^* \setminus \vec{X}$ such that
        $(\mathcal{T}^*, C, \rho^\prime) \vDash \mathit{mkList}(Y_1,\ldots,Y_k)$
        and for any $j \in \{ 1, \ldots, k \}$,
        $(\mathcal{T}^*, C, \rho^\prime) \vDash (\varphi_j^{\square})[Y_j/\square] \land P$ \, ,
    \end{proofenv}
    if and only if (by \Cref{lem:mkListSemantics})
    \begin{proofenv}
        there exists a $\mathcal{T}^*$-valuation $\rho^\prime$ satisfying $\rho^\prime(v) = \rho(v)$
        for any $v \in \mathit{Var}^* \setminus \vec{X}$ such that
        \begin{itemize}
            \item there exist configurations $c_1, \ldots, c_k \in \mathcal{T}_{\mathit{Cfg}}$ such that
                    $C = [c_1, \ldots, c_k]$
                    and for every $\mathcal{T}$-valuation $\rho^{\prime\prime}$
                    satisfying $\rho^{\prime\prime}(Y_j) = \rho^\prime(Y_j)$ for any $j \in \{ 1, \ldots, k \}$,
                    it holds that for any $j \in \{ 1, \ldots, k \}$,
                    $(\mathcal{T}, c_j, \rho^{\prime\prime}) \vDash Y_j$; and
            \item for any $j \in \{ 1, \ldots, k \}$,
                    $(\mathcal{T}^*, C, \rho^\prime) \vDash (\varphi_j^{\square})[Y_j/\square] \land P$ ,
        \end{itemize}
    \end{proofenv}
    if and only if (by \Cref{def:matchinglogic} and firstorder reasoning)
    \begin{proofenv}
        there exists a $\mathcal{T}^*$-valuation $\rho^\prime$ satisfying $\rho^\prime(v) = \rho(v)$
        for any $v \in \mathit{Var}^* \setminus \vec{X}$ such that
        \begin{itemize}
            \item there exist configurations $c_1, \ldots, c_k \in \mathcal{T}_{\mathit{Cfg}}$ such that
                    $C = [c_1, \ldots, c_k]$ and for any $j \in \{ 1, \ldots, k \}$,
                    $\rho^{\prime}(Y_j) = c_j$; and
            \item for any $j \in \{ 1, \ldots, k \}$,
                    $(\mathcal{T}^*, C, \rho^\prime) \vDash (\varphi_j^{\square})[Y_j/\square] \land P$ ,
        \end{itemize}
    \end{proofenv}
    if and only if (by firstorder reasoning, \Cref{def:matchinglogic}, and \Cref{lem:structurelessSemantics})
    \begin{proofenv}
        there exist configurations $c_1, \ldots, c_k \in \mathcal{T}_{\mathit{Cfg}}$ such that $C = [c_1, \ldots, c_k]$,
        and there exists a $\mathcal{T}^*$-valuation $\rho^\prime$ satisfying $\rho^\prime(v) = \rho(v)$
        for any $v \in \mathit{Var}^* \setminus \vec{X}$ such that for any $j \in \{ 1, \ldots, k \}$,
        \begin{itemize}
            \item $\rho^{\prime}(Y_j) = c_j$;
            \item $(\mathcal{T}^*, \rho^\prime) \vDash (\varphi_j^{\square})[Y_j/\square]$ ; and
            \item $(\mathcal{T}^*, \rho^\prime) \vDash P$,
        \end{itemize}
    \end{proofenv}
    if and only if (by \Cref{lem:starConservative} and firstorder reasoning)
    \begin{proofenv}
        there exist configurations $c_1, \ldots, c_k \in \mathcal{T}_{\mathit{Cfg}}$ such that $C = [c_1, \ldots, c_k]$,
        and there exists a $\mathcal{T}$-valuation $\rho^\prime$ satisfying $\rho^\prime(v) = \rho(v)$
        for any $v \in \mathit{Var} \setminus \vec{X}$ such that for any $j \in \{ 1, \ldots, k \}$,
        \begin{itemize}
            \item $\rho^{\prime}(Y_j) = c_j$;
            \item $(\mathcal{T}, \rho^\prime) \vDash (\varphi_j^{\square})[Y_j/\square]$ ; and
            \item $(\mathcal{T}, \rho^\prime) \vDash P$,
        \end{itemize}
    \end{proofenv}
    if and only if (by \Cref{lem:varrenamesem},
    since we have $\rho^\prime(Y_j) = c_j$
    on one side and $\rho_0^{c_j}(\square) = c_j$ on the other; for the implication from bottom to top,
    we also need the assumption that $Y_j$ was fresh and \Cref{lem:unusedVariables} -
    in order to choose the valuation $\rho^\prime$ satisfying $\rho^\prime(Y_j) = c_j$)
    \begin{proofenv}
        there exist configurations $c_1, \ldots, c_k \in \mathcal{T}_{\mathit{Cfg}}$ such that
        $C = [c_1, \ldots, c_k]$ and there exists a $\mathcal{T}$-valuation $\rho_0$
        satisfying $\rho_0(v) = \rho(v)$ for any $v \in \mathit{Var} \setminus \vec{X}$
        such that for any $j \in \{ 1, \ldots, k \}$,
        \begin{itemize}
            \item $(\mathcal{T}, \rho_0^{c_j}) \vDash \varphi_j^{\square}$ ; and
            \item $(\mathcal{T}, \rho_0) \vDash P$ ,
        \end{itemize}
    \end{proofenv}
    if and only if (by \Cref{lem:patternToFOLSemantics}, \Cref{def:matchinglogic}, and \Cref{lem:structurelessSemantics})
    \begin{proofenv}
        there exist configurations $c_1, \ldots, c_k \in \mathcal{T}_{\mathit{Cfg}}$ such that
        $C = [c_1, \ldots, c_k]$ and there exists a $\mathcal{T}$-valuation $\rho_0$
        satisfying $\rho_0(v) = \rho(v)$ for any $v \in \mathit{Var} \setminus \vec{X}$
        such that for any $j \in \{ 1, \ldots, k \}$,
        $(\mathcal{T}, c_j, \rho_0) \vDash \varphi_j \land P$ \, ,
    \end{proofenv}
    which is what we wanted to prove.
\end{proof}


\begin{lemma}\label{thm:correspondence}
\begin{equation*}
\mathcal{S} \vDash_{\CRL} \Psi \Rightarrow^{c\exists} \Psi^\prime
  \iff \mathcal{S}^* \vDash_\RL \mathit{flatten}(\Psi) \Rightarrow^{c\exists} \mathit{flatten}(\Psi^\prime)
\end{equation*}
\end{lemma}

\begin{proof}[Proof of \Cref{thm:correspondence}]
We prove each implication separately.
\begin{enumerate}
    \item For the left-to-right implication, we
    let $\mathcal{S} = (\mathcal{T}, S)$
    and $\Psi \equiv (\varphi_1,\ldots,\varphi_k) \land P$
    and $\Psi^\prime \equiv \exists \vec{Y}.\, (\varphi_1^\prime,\ldots,\varphi_k^\prime) \land P^\prime$,
    %where $\varphi_j \equiv \phi_j \land P_j$,
    %$\varphi^\prime_j \equiv \phi^\prime_j \land P^\prime_j$,
    and assume that
    \begin{proofenv}
        \begin{equation*}
            \mathcal{S} \vDash_\CRL \Psi \Rightarrow^{c\exists} \Psi^\prime \, ;
        \end{equation*}
    \end{proofenv}
    that is, (i)
    \begin{proofenv}
        for all configurations $c_1,\ldots,c_k \in \Tcfg$
        which terminate in $(\Tcfg, \Rightarrow_{\mathcal{S}})$
        and any $\mathcal{T}$-valuation $\rho_1$,
        whenever $(\mathcal{T}, c_1,\rho_1) \vDash \varphi_1 \land P$ and \ldots
        and $(\mathcal{T}, c_k,\rho_1) \vDash \varphi_k \land P$,
        then there exist configurations $c_1^\prime,\ldots,c_k^\prime \in \Tcfg$
        such that $c_1 \Rightarrow^{*}_{\mathcal{S}} c_1^\prime$
        and \ldots and $c_k \Rightarrow^{*}_{\mathcal{S}} c_k^\prime$,
        and there also exists an $\mathcal{T}$-valuation $\rho_2$
        satisfying $\rho_1(v) = \rho_2(v)$ for any $v \in \mathit{Var} \setminus \vec{Y}$,
        and
        $(\mathcal{T}, c_1^\prime,\rho_2) \vDash \varphi^\prime_1 \land P^\prime$ and \ldots
        and $(\mathcal{T}, c_k^\prime, \rho_2) \vDash \varphi^\prime_k \land P^\prime$.
    \end{proofenv}
    We have to prove that
    \begin{proofenv}
    for every $C \in \mathcal{T}^*_{\mathit{Cfg}^*}$
    such that $C$ terminates in $(\mathcal{T}^*_{\mathit{Cfg}^*}, \Rightarrow_{\mathcal{S}^*})$
    and for any valuation $\rho : \Var^* \to \mathcal{T}^*$
    such that $(\mathcal{T}^*, C, \rho) \vDash \mathit{flatten}(\Psi)$,
    there exists some $C^\prime \in \mathcal{T}^*_{\mathit{Cfg}^*}$
    such that
    $C \Rightarrow^{*}_{\mathcal{S}^*} C^\prime$
    and $(\mathcal{T}^*, C^\prime, \rho) \vDash \mathit{flatten}(\Psi^\prime)$.
    \end{proofenv}
    Let us then have some $C \in \mathcal{T}^*_{\mathit{Cfg}^*}$
    such that $C$ terminates in $(\mathcal{T}^*_{\mathit{Cfg}^*}, \Rightarrow_{\mathcal{S}^*})$,
    and a valuation $\rho : \Var^* \to \mathcal{T}^*$
    such that $(\mathcal{T}^*, C, \rho) \vDash \mathit{flatten}(\Psi)$.
    We have to prove that
    \begin{proofenv}
    there exists some $C^\prime \in \mathcal{T}^*_{\mathit{Cfg}^*}$
    such that
    $C \Rightarrow^{*}_{\mathcal{S}^*} C^\prime$
    and $(\mathcal{T}^*, C^\prime, \rho) \vDash \mathit{flatten}(\Psi^\prime)$.
    \end{proofenv}
    We will proceed in five steps:
    \begin{enumerate}
        \item \label{item:corr:step1} We prove that there exists $c_1,\ldots,c_k$ such that $C = [c_1,\ldots, c_k]$.
        \item \label{item:corr:step2} We prove that $c_1,\ldots,c_k$ are terminating.
        \item \label{item:corr:step3} We find appropriate valuation $\rho_1 : \mathit{Var} \to \mathcal{T}$
              and prove the premise of the assumption (i): that $(\mathcal{T}, c_1, \rho_1) \vDash \varphi_1 \land P$
        and \ldots and $(\mathcal{T}, c_k, \rho_1) \vDash \varphi_k \land P$.
        \item \label{item:corr:step4} From the assumption (i) we get the appropriate $c_1^\prime,\ldots,c_k^\prime$,
        as well as a valuation $\rho_2 : \mathit{Var} \to \mathcal{T}$ satisfying $\rho_1(v) = \rho_2(v)$ for any $v \in \mathit{Var} \setminus \vec{Y}$, and 
        \item \label{item:corr:step5} We let $C^\prime := [c_1^\prime,\ldots,c_k^\prime]$ and prove that it is reachable from $C$,
        as well as that it satisfies the flattened $\Psi^\prime$ in $\rho$.
    \end{enumerate}
    We have
    \begin{proofenv}
    $(\mathcal{T}^*, C, \rho) \vDash \mathit{flatten}(\Psi)$;
    \end{proofenv}
    that is,
    \begin{proofenv}
    $(\mathcal{T}^*, C, \rho) \vDash \mathit{flatten}((\varphi_1, \ldots, \varphi_k) \land P)$;
    \end{proofenv}
    which is by \Cref{lem:flatten} equivalent to (ii)
    \begin{proofenv}
        there exist configurations $c_1, \ldots, c_k \in \mathcal{T}_{\mathit{Cfg}}$ such that
        $C = [c_1, \ldots, c_k ]$, and
        there exists a $\mathcal{T}$-valuation $\rho_0$ satisfying $\rho_0(v) = \rho(v)$
        for any $v \in \mathit{Var}$,
        such that for any $j \in \{ 1, \ldots, k \}$, $(\mathcal{T} , c_j , \rho_0 ) \vDash  \varphi_j \land P$.
    \end{proofenv}
    Now, let $\rho_0$ be such valuation,
    and let $c_1,\ldots,c_k$ be such configurations. We have just proved Item~\ref{item:corr:step1}.
    To prove Item~\ref{item:corr:step2}, saying that the configurations $c_1, \ldots, c_k$ are terminating,
    we simply use \Cref{lem:terminationComposite}.
    To prove Item~\ref{item:corr:step3}, we let $\rho_1 := \rho_0$, and apply the assumption (ii).
    Now it follows that (iii)
    \begin{proofenv}
        there exist configurations $c_1^\prime,\ldots,c_k^\prime \in \Tcfg$
        such that $c_1 \Rightarrow^{*}_{\mathcal{S}} c_1^\prime$
        and \ldots and $c_k \Rightarrow^{*}_{\mathcal{S}} c_k^\prime$,
        and there also exists an $\mathcal{T}$-valuation $\rho_2$
        satisfying $\rho_1(v) = \rho_2(v)$ for any $v \in \mathit{Var} \setminus \vec{Y}$,
        and
        $(\mathcal{T}, c_1^\prime,\rho_2) \vDash \varphi^\prime_1 \land P^\prime$ and \ldots
        and $(\mathcal{T}, c_k^\prime, \rho_2) \vDash \varphi^\prime_k \land P^\prime$.
    \end{proofenv}
    Let us have such configurations $c_1^\prime,\ldots,c_k^\prime$ and valuation $\rho_2$.
    We choose $C^\prime := [ c_1^\prime,\ldots,c_k^\prime ]$,
    and it remains to be proven that
    \begin{proofenv}
        $C \Rightarrow^*_{\mathcal{S}^*} C^\prime$ and $(\mathcal{T}^*, C^\prime, \rho) \vDash \mathit{flatten}(\Psi^\prime)$ \,
    \end{proofenv}
    (where $\rho$ is the valuation that we started with).
    The part saying that $C \Rightarrow^*_{\mathcal{S}^*} C^\prime$ holds
    follows by \Cref{lem:reachComposite}. The other part can be changed using \Cref{lem:flatten} into
    \begin{proofenv}
        there exist configurations $c^\prime_1, \ldots, c^\prime_k \in \mathcal{T}_{\mathit{Cfg}}$ such that
        $C^\prime = [c^\prime_1, \ldots, c^\prime_k]$ and there exists a $\mathcal{T}$-valuation $\rho^\prime_0$
        satisfying $\rho^\prime_0(v) = \rho(v)$ for any $v \in \mathit{Var} \setminus \vec{Y}$
        such that for any $j \in \{ 1, \ldots, k \}$,
        $(\mathcal{T}, c^\prime_j, \rho^\prime_0) \vDash \varphi^\prime_j \land P$.
    \end{proofenv}
    Since we have constructed $C^\prime$ as a list of smaller configurations,
    it remains to be proven that
    \begin{proofenv}
        there exists a $\mathcal{T}$-valuation $\rho^\prime_0$
        satisfying $\rho^\prime_0(v) = \rho(v)$ for any $v \in \mathit{Var} \setminus \vec{Y}$
        such that for any $j \in \{ 1, \ldots, k \}$,
        $(\mathcal{T}, c^\prime_j, \rho^\prime_0) \vDash \varphi^\prime_j \land P$.
    \end{proofenv}
    Let us choose $\rho^\prime_0$ defined by
    $\rho_0^\prime(v) = \rho_2(v)$ for any $v \in \mathit{Var}$.
    We verify that $\rho_0^\prime(v) = \rho_2(v) = \rho_1(v) = \rho_0(v) = \rho(v)$
    for any $v \in \mathit{Var} \setminus \vec{Y}$,
    and the rest follows from the assumption (iii) by \Cref{lem:unusedVariables}.
    This concludes the proof of the first implication.
    \item For the opposite implication, 
    we again assume that $\Psi \equiv (\varphi_1,\ldots,\varphi_k) \land P$,
    $\Psi^\prime \equiv \exists \vec{Y}.\, (\varphi_1^\prime,\ldots,\varphi_k^\prime) \land P^\prime$,
    $\varphi_j = \phi_j \land P_j$ and $\varphi^\prime_j = \phi^\prime_j \land P^\prime_j$ for any $j \in \{ 1, \ldots, k \}$,
    and assume that
    \begin{proofenv}
        \begin{equation*}
            \mathcal{S}^* \vDash_\RL \mathit{flatten}(\Psi) \Rightarrow^\exists \mathit{flatten}(\Psi^\prime) \, ;
        \end{equation*}
    \end{proofenv}
    that is (i),
    \begin{proofenv}
        for every $C \in \mathcal{T}^*_{\mathit{Cfg}^*}$ such that $C$ terminates in
        $(\mathcal{T}_{\mathit{Cfg}^*}, \Rightarrow_{\mathcal{S}^*})$
        and for any valuation $\rho : \mathit{Var}^* \to \mathcal{T}^*$ such that
        $(\mathcal{T}^*, C, \rho) \vDash \mathit{flatten}(\Psi)$,
        there exists some $C^\prime \in \mathcal{T}^*_{\mathit{Cfg}^*}$ such that
        $C \Rightarrow_{\mathcal{S}^*}^* C^\prime$
        and $(\mathcal{T}^*, C^\prime, \rho) \vDash \mathit{flatten}(\Psi^\prime)$;
    \end{proofenv}
    we have to prove that
    \begin{proofenv}
        \begin{equation*}
            \mathcal{S} \vDash_\CRL \Psi \Rightarrow^{c\exists} \Psi^\prime \, ;
        \end{equation*}
    \end{proofenv}
    that is,
    \begin{proofenv}
        for all configurations $c_1,\ldots,c_k \in \Tcfg$
        which terminate in $(\Tcfg, \Rightarrow_{\mathcal{S}})$
        and any $\mathcal{T}$-valuation $\rho_1$,
        whenever $(\mathcal{T}, c_1,\rho_1) \vDash \varphi_1 \land P$ and \ldots
        and $(\mathcal{T}, c_k,\rho_1) \vDash \varphi_k \land P$,
        then there exist configurations $c_1^\prime,\ldots,c_k^\prime \in \Tcfg$
        such that $c_1 \Rightarrow^{*}_{\mathcal{S}} c_1^\prime$
        and \ldots and $c_k \Rightarrow^{*}_{\mathcal{S}} c_k^\prime$,
        and there also exists an $\mathcal{T}$-valuation $\rho_2$
        satisfying $\rho_1(v) = \rho_2(v)$ for any $v \in \mathit{Var} \setminus \vec{Y}$,
        and
        $(\mathcal{T}, c_1^\prime,\rho_2) \vDash \varphi^\prime_1 \land P^\prime$ and \ldots
        and $(\mathcal{T}, c_k^\prime, \rho_2) \vDash \varphi^\prime_k \land P^\prime$.
    \end{proofenv}
    Let us then have such terminating configurations $c_1,\ldots,c_k \in \mathcal{T}_{\mathit{Cfg}}$
    and such valuation $\rho_1 : \mathit{Var} \to \mathcal{T}$.
    We have to show that
    \begin{proofenv}
        there exist configurations $c_1^\prime,\ldots,c_k^\prime \in \Tcfg$
        such that $c_1 \Rightarrow^{*}_{\mathcal{S}} c_1^\prime$
        and \ldots and $c_k \Rightarrow^{*}_{\mathcal{S}} c_k^\prime$,
        and there also exists an $\mathcal{T}$-valuation $\rho_2$
        satisfying $\rho_1(v) = \rho_2(v)$ for any $v \in \mathit{Var} \setminus \vec{Y}$,
        and
        $(\mathcal{T}, c_1^\prime,\rho_2) \vDash \psi_1 \land P^\prime$ and \ldots
        and $(\mathcal{T}, c_k^\prime, \rho_2) \vDash \psi_k \land P^\prime$.
    \end{proofenv}
    We will proceed in the following steps.
    \begin{enumerate}
        \item We prove the premise of (i) for $C := [c_1,\ldots,c_k]$, that is:
        \begin{enumerate}
            \item $[c_1,\ldots,c_k]$ terminates in $(\mathcal{T}_{\mathit{Cfg}^*}, \Rightarrow_{\mathcal{S}^*})$; and
            \item $(\mathcal{T}^*, [c_1,\ldots,c_k], \rho) \vDash \mathit{flatten}(\Psi)$ for some constructed valuation $\rho$.
        \end{enumerate}
        \item We ``destruct'' the obtained $C^\prime$ into $[c^\prime_1,\ldots,c^\prime_k]$;
        \item We prove the desired properties of $c^\prime_j$ from the properties of $C^\prime$.
    \end{enumerate}
    First, $[c_1,\ldots,c_k]$ is terminating by \Cref{lem:terminationComposite}.
    Next, we have to show that
    \begin{proofenv}
        \begin{equation*}
            (\mathcal{T}^*, [c_1,\ldots,c_k], \rho) \vDash \mathit{flatten}((\varphi_1,\ldots,\varphi_k) \land P) \, ;
        \end{equation*}
    \end{proofenv}
    where $\rho(v) = \rho_1(v)$ for any $v \in \mathit{Var}$ (and $\rho(v)$ has arbitrary value for $v$ outside of $\mathit{Var}$).
    By \Cref{lem:flatten}, this is equivalent to showing that
    \begin{proofenv}
        there exist configurations $c_1, \ldots, c_k \in \mathcal{T}_{\mathit{Cfg}}$ such that
        $[c_1, \ldots, c_k] = [c_1, \ldots, c_k]$ and there exists a $\mathcal{T}$-valuation $\rho_0$
        satisfying $\rho_0(v) = \rho(v)$ for any $v \in \mathit{Var}$
        such that for any $j \in \{ 1, \ldots, k \}$, $(\mathcal{T}, c_j, \rho_0) \vDash \varphi_j \land P$.
    \end{proofenv}
    We choose $c_j := c_j$ and $\rho_0 := \rho_1$; it remains to be proven that
    \begin{proofenv}
        $(\mathcal{T}, c_j, \rho_1) \vDash \varphi_j \land P$.    
    \end{proofenv}
    which holds by assumption.
    Now we have obtained the following:
    \begin{proofenv}
        there exists some $C^\prime \in \mathcal{T}_{\mathit{Cfg}}^*$ such that
        $[c_1,\ldots,c_k] \Rightarrow_{\mathcal{S}^*}^* C^\prime$
        and $(\mathcal{T}^*, C^\prime, \rho) \vDash \mathit{flaten}(\Psi^\prime)$.
    \end{proofenv}
    Now, by \Cref{lem:transitionOnlyBetweenListsOfSameLength} (and using induction on the length of the
    sequence witnessing the reachability),
    we get some $c^\prime_1,\ldots,c^\prime_k \in \Tcfg$ such that
    \begin{proofenv}
        $[c_1,\ldots,c_k] \Rightarrow_{\mathcal{S}^*}^* [c^\prime_1,\ldots,c^\prime_k]$
        and $(\mathcal{T}^*, [c^\prime_1,\ldots,c^\prime_k], \rho) \vDash \mathit{flaten}(\Psi^\prime)$.
    \end{proofenv}
    Our goal is to prove that
    \begin{proofenv}
        there exist configurations $c_1^\prime,\ldots,c_k^\prime \in \Tcfg$
        such that $c_1 \Rightarrow^{*}_{\mathcal{S}} c_1^\prime$
        and \ldots and $c_k \Rightarrow^{*}_{\mathcal{S}} c_k^\prime$,
        and there also exists an $\mathcal{T}$-valuation $\rho_2$
        satisfying $\rho_1(v) = \rho_2(v)$ for any $v \in \mathit{Var} \setminus \vec{Y}$,
        and
        $(\mathcal{T}, c_1^\prime,\rho_2) \vDash \psi_1 \land P^\prime$ and \ldots
        and $(\mathcal{T}, c_k^\prime, \rho_2) \vDash \psi_k \land P^\prime$,
    \end{proofenv}
    so we choose $c^\prime_j := c^\prime_j$ and have to prove that
    \begin{proofenv}
        $c_1 \Rightarrow^{*}_{\mathcal{S}} c_1^\prime$
        and \ldots and $c_k \Rightarrow^{*}_{\mathcal{S}} c_k^\prime$,
        and there also exists an $\mathcal{T}$-valuation $\rho_2$
        satisfying $\rho_1(v) = \rho_2(v)$ for any $v \in \mathit{Var} \setminus \vec{Y}$,
        and
        $(\mathcal{T}, c_1^\prime,\rho_2) \vDash \psi_1 \land P^\prime$ and \ldots
        and $(\mathcal{T}, c_k^\prime, \rho_2) \vDash \psi_k \land P^\prime$.
    \end{proofenv}
    The first part follows from \Cref{lem:reachComposite};
    it remains to be proven that
    \begin{proofenv}
        there also exists an $\mathcal{T}$-valuation $\rho_2$
        satisfying $\rho_1(v) = \rho_2(v)$ for any $v \in \mathit{Var} \setminus \vec{Y}$,
        and
        $(\mathcal{T}, c_1^\prime,\rho_2) \vDash \psi_1 \land P^\prime$ and \ldots
        and $(\mathcal{T}, c_k^\prime, \rho_2) \vDash \psi_k \land P^\prime$.
    \end{proofenv}
    and we already have
    \begin{proofenv}
        $(\mathcal{T}^*, [c^\prime_1,\ldots,c^\prime_k], \rho) \vDash \mathit{flaten}(\Psi^\prime)$ \, ;
    \end{proofenv}
    that is, by \Cref{lem:flatten} we know that
    \begin{proofenv}
        there exists a $\mathcal{T}$-valuation $\rho_0$ satisfying $\rho_0(v) = \rho(v)$
        for any $v \in \mathit{Var} \setminus \vec{Y}$ such that for any $j \in \{ 1, \ldots, k \}$,
        $(\mathcal{T}, c^\prime_j, \rho_0) \vDash \varphi^\prime$.
    \end{proofenv}
    Let $\rho_0^\prime$ be such valuation.
    In the goal, we let $\rho_2(v) := \rho_0^\prime(v)$ for any $v \in \mathit{Var}$;
    we then note that $\rho_2(v) = \rho_0^\prime(v) =  \rho(v) =  \rho_1(v)$ for any $v \in \mathit{Var} \setminus \vec{Y}$ by definitions.
    The rest of the goal follows from the assumption by \Cref{lem:unusedVariables}.
    This concludes the proof.
\end{enumerate}
\end{proof}

\subsection{Proof of~\Cref{thm:proofsystemSoundness}}\label{app:crlsoundness}

\begin{figure}
    \centering
    \begin{align*}
        & \prftree[l]{Axiom}
          { \varphi \Rightarrow^{\exists} \varphi^\prime \in \mathcal{A} }
          { \mathcal{A}, C \vdash_\RL \varphi \Rightarrow^{\exists} \varphi^\prime }
    \end{align*}
    \begin{align*}
        & \prftree[l]{Reflexivity}
          { \mathcal{A}, \emptyset \vdash_\RL \varphi \Rightarrow^{\exists} \varphi }
    \end{align*}
    \begin{align*}
        & \prftree[l]{Transitivity}
          { (\mathcal{T}, A), C \vdash_\RL \varphi_1 \Rightarrow^{+ \exists} \varphi_2 }
          { (\mathcal{T}, A \cup C), \emptyset \vdash_\RL \varphi_2 \Rightarrow^{\exists} \varphi_3 }
          { (\mathcal{T}, A), C \vdash_\RL \varphi_1 \Rightarrow^{\exists} \varphi_3 }
    \end{align*}
    \begin{align*}
        & \prftree[l]{Logic Framing}
          { \mathcal{A}, C \vdash_\RL \varphi \Rightarrow^{\exists} \varphi^\prime }
          { \psi \mbox{ is a FOL formula} }
          { \mathcal{A}, C \vdash_\RL \varphi \land \psi \Rightarrow^{\exists} \varphi^\prime \land \psi }
    \end{align*}
    \begin{align*}
        & \prftree[l]{Consequence}
          { \mathcal{T} \vDash \varphi_1 \rightarrow \varphi_1^\prime }
          { (\mathcal{T}, A), C \vdash_\RL \varphi_1^\prime \Rightarrow^{\exists} \varphi_2^\prime }
          { \mathcal{T} \vDash \varphi_2^\prime \rightarrow \varphi_2 }
          { (\mathcal{T}, A), C \vdash_\RL \varphi_1 \Rightarrow^{\exists} \varphi_2 }
    \end{align*}
    \begin{align*}
        & \prftree[l]{Case Analysis}
          { \mathcal{A}, C \vdash_\RL \varphi_1 \Rightarrow^{\exists} \varphi }
          { \mathcal{A}, C \vdash_\RL \varphi_2 \Rightarrow^{\exists} \varphi }
          { \mathcal{A}, C \vdash_\RL \varphi_1 \lor \varphi_2 \Rightarrow^{\exists} \varphi }
    \end{align*}
    \begin{align*}
        & \prftree[l]{Abstraction}
          { \mathcal{A}, C \vdash_\RL \varphi \Rightarrow^{\exists} \varphi^\prime }
          { X \not\in \mathit{FV}(\varphi^\prime) }
          { \mathcal{A}, C \vdash_\RL \exists X.\, \varphi \Rightarrow^{\exists} \varphi^\prime }
    \end{align*}
    \begin{align*}
        & \prftree[l]{Circularity}
          { A, C \cup \{ \varphi\Rightarrow^\exists \varphi^\prime \} \vdash_\RL \varphi \Rightarrow^\exists \varphi^\prime }
          { A, C \vdash_\RL \varphi \Rightarrow^\exists \varphi^\prime }
    \end{align*}

    \caption{One-path reachability-logic proof system.
    The use of $\Rightarrow^{+ \exists}$ in a sequent means that it was derived without Reflexivity.}
    \label{fig:RLproofsystem}
\end{figure}

\begin{proof}[Proof of \Cref{lem:CRLalmostSoundness}]
By induction on the structure of the CRL proof.
\begin{enumerate}
    \item If the proof ends with \emph{Reduce}, then we are done, since $\mathit{flatten}^\exists(\emptyset, \psi^\prime) = \emptyset$.
    
    \item If the proof ends with \emph{Reflexivity}, then we need to prove
    \begin{equation*}
        \mathcal{S}^* \cup \mathit{flatten}^\exists(E, \psi), \emptyset \vdash_\RL
          \mathit{flatten}^\exists(\psi, \psi) 
    \end{equation*}
    which we do by applying the Reflexivity proof rule.
    
    \item If the proof ends with \emph{Axiom}, then $\psi \in E$,
          and we have to prove that
          \begin{equation*}
            \mathcal{S}^* \cup \mathit{flatten}^\prime(E, \psi^\prime), \mathit{flatten}^\prime(C, \psi^\prime) \vdash_\RL
            \mathit{flatten}^\prime(\psi, \psi^\prime)               \, .
          \end{equation*}
          By applying the Axiom proof rule of RL, it is enough to show that
          \begin{equation*}
              \mathit{flatten}^\prime(\psi, \psi^\prime) \in \mathit{flatten^\prime}(E, \psi^\prime) \, ,
          \end{equation*}
          which follows from $\psi \in E$.
          
    \item If the proof ends with \emph{Case}, then we have
        \begin{equation*}
            \mathcal{S}^* \cup \Bar{E}, \Bar{C} \vdash_\RL
            \mathit{flatten}^\exists((\varphi_1, \ldots, \varphi_{i-1}, \varphi_i, \varphi_{i+1}, \ldots, \varphi_k) \land P^\prime, \Psi^\prime)
        \end{equation*}
        and
        \begin{equation*}
            \mathcal{S}^* \cup \Bar{E}, \Bar{C} \vdash_\RL
            \mathit{flatten}^\exists((\varphi_1, \ldots, \varphi_{i-1}, \psi_i, \varphi_{i+1}, \ldots, \varphi_k) \land P^\prime, \Psi^\prime) 
        \end{equation*}
        as hypotheses, and we have to prove
        \begin{equation*}
            \mathcal{S}^* \cup \Bar{E}, \Bar{C} \vdash_\RL
            \mathit{flatten}^\exists((\varphi_1, \ldots, \varphi_{i-1}, (\varphi_i \lor \psi_i), \varphi_{i+1}, \ldots, \varphi_k) \land P^\prime, \Psi^\prime)               \, .
        \end{equation*}
        (where $\Bar{E} = \mathit{flatten}^\exists(E, \psi^\prime)$
         and $\Bar{C} = \mathit{flatten}^\exists(C, \psi^\prime)$
        ).
        After simplifications, we get
        \begin{align*}
            \mathcal{S}^* \cup \Bar{E}, \Bar{C} \vdash_\RL
            &
            \mathit{mkList}(X_1, \ldots, X_k) \land \left( \bigwedge_{j=1}^{k} (\varphi_j^\square)[X_j/\square] \right) \land P^\prime
            \\ & \Rightarrow^\exists
            \mathit{flatten}(\Psi^\prime)
        \end{align*}
        and
        \begin{align*}
            \mathcal{S}^* \cup \Bar{E}, \Bar{C} \vdash_\RL
            &
            \mathit{mkList}(Y_1, \ldots, Y_k) \land \left(\bigwedge_{j=1, j \not = i}^{k} (\varphi_j^\square)[Y_j/\square] \right)
            \land (\psi_i^\square)[Y_i/\square] \land P^\prime
            \\ & \Rightarrow^\exists
            \mathit{flatten}(\Psi^\prime)
        \end{align*}
        as hypotheses,
        and have to prove
        \begin{align*}
            \mathcal{S}^* \cup \Bar{E}, \Bar{C} \vdash_\RL
            &
            \mathit{mkList}(Z_1, \ldots, Z_k) \land \left(\bigwedge_{j=1,j \not = i}^{k} (\varphi_j^\square)[Z_j/\square]\right)
            \land ((\varphi_i \lor \psi_i)^\square)[Z_i/\square]
            \\ & \Rightarrow^\exists
            \mathit{flatten}(\Psi^\prime)
        \end{align*}
        (where $X_1,\ldots,X_k,Y_1,\ldots,Y_k,Z_1,\ldots,Z_k$ are fresh variables).
        We first apply the Consequence RL rule to the goal to distribute the $\varphi_i \lor \psi_i$ disjunction
        to the top, changing the goal to
        \begin{align*}
            \mathcal{S}^* \cup \Bar{E}, \Bar{C} \vdash_\RL
            &
            \left( \mathit{mkList}(Z_1, \ldots, Z_k) \land \left(\bigwedge_{j=1,j \not = i}^{k} (\varphi_j^\square)[Z_j/\square]\right)
            \land (\varphi_i^\square)[Z_i/\square] \right)
            \\ \lor &
            \left(
            \mathit{mkList}(Z_1, \ldots, Z_k) \land \left(\bigwedge_{j=1,j \not = i}^{k} (\varphi_j^\square)[Z_j/\square]\right)
            \land (\psi_i^\square)[Z_i/\square]
            \right)
            \\ & \Rightarrow^\exists
            \mathit{flatten}(\Psi^\prime) \, .
        \end{align*}
        Now we apply the Case Analysis rule.
        Then we transform the hypotheses to the respective goals by existentially quantifying the $X_j$s (and $Y_j$s, respectively)
        in the hypotheses
        using the Abstraction RL rule, alpha-renaming (using the Consequence rule) the $X_j$s (and $Y_j$s, respectively)
        into $Z_j$s, and stripping the existential quantifiers (using the Consequence rule, again), and we are done.
        
    \item If the proof ends with \emph{Step},
      we can assume a structureless FOL formula $P^\prime$, a rule $\varphi \Rightarrow^\exists \varphi^\prime \in S$ such that
      $\mathcal{T} \vDash_\ML \varphi_i \leftrightarrow \varphi \land P^\prime$,
      and an induction hypothesis
      \begin{align*}
        (&\mathcal{T}^*, S^* \cup \mathit{flatten}^\exists(C \cup E, \Psi^\prime)), \emptyset \vdash_\RL
          \\ &
          \mathit{flatten}([\varphi_1, \ldots, \varphi_{i-1}, \varphi^\prime \land P^\prime, \varphi_{i+1}, \ldots, \varphi_k] \land P) \Rightarrow^\exists \mathit{flatten}(\Psi^\prime)     
      \end{align*}
      and have to construct
      \begin{align*}
      & (\mathcal{T}^*, S^* \cup \mathit{flatten}^\exists(E, \Psi^\prime)), \mathit{flatten}^\exists(C, \Psi^\prime) \vdash_\RL \\
          & \mathit{flatten}([\varphi_1, \ldots, \varphi_{i-1}, \varphi_i, \varphi_{i+1}, \ldots, \varphi_k] \land P) \Rightarrow^\exists \mathit{flatten}(\Psi^\prime)    \, .
      \end{align*}
        By definition of $S^*$, we also have
        \begin{align*}
            (\mathit{heat}(L, \varphi, R) \Rightarrow^\exists \mathit{heat}(L, \varphi^\prime, R)) \in S^* \, .
        \end{align*}
    We apply the Transitivity rule with the second premise being our first inductive hypothesis, and it remains to prove the second premise, which is
    \begin{align*}
        & (\mathcal{T}, S)^*, \mathit{flatten}^\exists(E, \psi^\prime), \mathit{flatten}^\exists(C, \psi^\prime)
        \\& \vdash_\RL
        \mathit{flatten}([\varphi_1, \ldots, \varphi_{i-1}, \varphi_i, \varphi_{i+1}, \ldots, \varphi_k] \land P)
        \\&\quad \Rightarrow^\exists
        \mathit{flatten}([\varphi_1, \ldots, \varphi_{i-1}, \varphi^\prime \land P^\prime, \varphi_{i+1}, \ldots, \varphi_k] \land P) \, .
    \end{align*}
    that is (after simplification, assuming a reasonable choice of fresh variables)
    \begin{align*}
        & (\mathcal{T}, S)^*, \mathit{flatten}^\exists(E, \psi^\prime), \mathit{flatten}^\exists(C, \psi^\prime)
        \\& \vdash_\RL
        \mathit{mkList}(X_1, \ldots, X_k) \land \left( \bigwedge_{j=1,j\not = i}^{k} (\varphi_j^\square)[X_j/\square] \right)
        \land (\varphi_i^\square)[X_i/\square] \land P
        \\&\quad \Rightarrow^\exists
        \mathit{mkList}(X_1, \ldots, X_k) \land \left( \bigwedge_{j=1, j \not = i}^{k} (\varphi_j^\square)[X_j/\square] \right) \land ((\varphi^\prime \land P^\prime)^\square)[X_i/\square] \land P
        \, .
    \end{align*}
    By \Cref{lem:equivFOLtransl}, our assumption that $\mathcal{T} \vDash_\ML \varphi_i \leftrightarrow (\varphi \land P^\prime)$,
    and conservativeness,
    we can apply the Consequence rule, and the goal changes to
    \begin{align*}
        & (\mathcal{T}, S)^*, \mathit{flatten}^\exists(E, \psi^\prime), \mathit{flatten}^\exists(C, \psi^\prime)
        \\& \vdash_\RL
        \mathit{mkList}(X_1, \ldots, X_k) \land \left( \bigwedge_{j=1,j\not = i}^{k} (\varphi_j^\square)[X_j/\square] \right)
        \land ((\varphi \land P^\prime)^\square)[X_i/\square] \land P
        \\&\quad \Rightarrow^\exists
        \mathit{mkList}(X_1, \ldots, X_k) \land \left( \bigwedge_{j=1, j \not = i}^{k} (\varphi_j^\square)[X_j/\square] \right) \land ((\varphi^\prime \land P^\prime)^\square)[X_i/\square] \land P
        \, .
    \end{align*}
    We apply the Consequence rule again, changing the goal to
    \begin{align*}
        & (\mathcal{T}, S)^*, \mathit{flatten}^\exists(E, \psi^\prime), \mathit{flatten}^\exists(C, \psi^\prime)
        \\& \vdash_\RL
        (\varphi^\square)[X_i/\square] \land (
        \mathit{mkList}(X_1, \ldots, X_k) \land \left( \bigwedge_{j=1,j\not = i}^{k} (\varphi_j^\square)[X_j/\square] \right)
        \land ((P^\prime)^\square)[X_i/\square] \land P)
        \\&\quad \Rightarrow^\exists
        ((\varphi^\prime)^\square)[X_i/\square] \land (
        \mathit{mkList}(X_1, \ldots, X_k) \land \left( \bigwedge_{j=1, j \not = i}^{k} (\varphi_j^\square)[X_j/\square] \right) \land ((P^\prime)^\square)[X_i/\square] \land P)
        \, .
    \end{align*}
    Now we apply Logic Framing to remove the structureless parts that are the same in both the left and right sides,
    resulting in the goal
    \begin{align*}
        & (\mathcal{T}, S)^*, \mathit{flatten}^\exists(E, \psi^\prime), \mathit{flatten}^\exists(C, \psi^\prime)
        \\& \vdash_\RL
        (\varphi^\square)[X_i/\square] \land
        \mathit{mkList}(X_1, \ldots, X_k)
        \\&\quad \Rightarrow^\exists
        ((\varphi^\prime)^\square)[X_i/\square] \land
        \mathit{mkList}(X_1, \ldots, X_k)
        \, .
    \end{align*}
    Now, from the assumption that $\varphi \Rightarrow^\exists \varphi^\prime \in S$ and the construction of $S^*$
    it follows that
    $\mathit{heat}(L, \varphi, R) \Rightarrow^\exists \mathit{heat}(L, \varphi^\prime, R) \in S^*$;
    and therefore
    $\mathit{cfgheat}(L, \phi, R) \land Q \Rightarrow^\exists \mathit{cfgheat}(L, \phi^\prime, R) \land Q^\prime \in S^*$
    where $\varphi \equiv \phi \land Q$ and $\varphi^\prime \equiv \phi^\prime \land Q^\prime$.
    By semantic reasoning we can prove that
    \begin{align*}
        \mathcal{T}^* \vDash & ((\varphi^\square)[X_i/\square] \land \mathit{mkList}(X_1, \ldots, X_k))
        \\ \leftrightarrow  &
        \mathit{cfgheat}(L, X_i, R)
        \\ & \land L = \mathit{mkList}(X_1, \ldots, X_{i-1})
        \\ & \land R = \mathit{mkList}(X_{i+1}, \ldots, X_k)
        \\ & \land (\phi^\square)[X_i/\square] \land Q
    \end{align*}
    and that
    \begin{align*}
        \mathcal{T}^* \vDash & (((\varphi^\prime)^\square)[X_i/\square] \land \mathit{mkList}(X_1, \ldots, X_k))
        \\ \leftrightarrow  &
        \mathit{cfgheat}(L, X_i, R)
        \\ & \land L = \mathit{mkList}(X_1, \ldots, X_{i-1})
        \\ & \land R = \mathit{mkList}(X_{i+1}, \ldots, X_k)
        \\ & \land ((\phi^\prime)^\square)[X_i/\square] \land Q^\prime \, .
    \end{align*}
    Now we apply Consequence and subsequently strip the $L$,$R$ equalities using Logic Framing, thus getting
    \begin{align*}
        & (\mathcal{T}, S)^*, \mathit{flatten}^\exists(E, \psi^\prime), \mathit{flatten}^\exists(C, \psi^\prime)
        \\& \vdash_\RL
        \mathit{cfgheat}(L, X_i, R) \land (\phi^\square)[X_i/\square] \land Q
        \\&\quad \Rightarrow^\exists
        \mathit{cfgheat}(L, X_i, R) \land ((\phi^\prime)^\square)[X_i/\square] \land Q^\prime
        \, .
    \end{align*}
    Now we use Consequence to expand $\phi^\square$ and $(\phi^\prime)^\square$ into equalities,
    perform the substitution, and use the equalities to replace the $X_i$ subterm of $\mathit{cfgheat}$ with
    $\phi$ and $\phi^\prime$, respectively; this way the goal becomes
    \begin{align*}
        & (\mathcal{T}, S)^*, \mathit{flatten}^\exists(E, \psi^\prime), \mathit{flatten}^\exists(C, \psi^\prime)
        \\& \vdash_\RL
        \mathit{cfgheat}(L, \phi, R) \land Q
        \\&\quad \Rightarrow^\exists
        \mathit{cfgheat}(L, \phi^\prime, R) \land Q^\prime
        \, .
    \end{align*}
    We finish the proof of this case using the Axiom rule.
    
    \item If the proof ends with \emph{Circularity}, we can assume
        \begin{align*}
            (\mathcal{T}^*, S^* \cup \mathit{flatten}^\exists(E, \Psi^\prime)),
            \mathit{flatten}^\exists(C \cup \{ \Psi \}, \Psi^\prime) \vdash_\RL
            \mathit{flatten}^\exists(\Psi, \Psi^\prime)
        \end{align*}
        which simplifies to
        \begin{align*}
            (\mathcal{T}^*, S^* \cup \mathit{flatten}^\exists(E, \Psi^\prime)),
            \mathit{flatten}^\exists(C, \Psi^\prime) \cup \mathit{flatten}^\exists(\{ \Psi \}, \Psi^\prime) \vdash_\RL
            \mathit{flatten}^\exists(\Psi, \Psi^\prime)
        \end{align*}
        and have to prove
        \begin{align*}
            (\mathcal{T}^*, S^* \cup \mathit{flatten}^\exists(E, \Psi^\prime)),
            \mathit{flatten}^\exists(C, \Psi^\prime) \vdash_\RL
            \mathit{flatten}^\exists(\Psi, \Psi^\prime)
        \end{align*}
        which follows from the assumption by Circularity.
        
    \item If the proof ends with \emph{Conseq}, we can assume
    \begin{align*}
        \mathcal{T}^* \vDash_\ML \mathit{flatten}(\Phi) \rightarrow \mathit{flatten}(\Phi^\prime)
    \end{align*}
    and
    \begin{align*}
        (\mathcal{T}^*, S^* \cup \mathit{flatten}^\exists(E, \Psi^\prime)), \mathit{flatten}^\exists(C, \Psi^\prime) \vdash_\RL
          \mathit{flatten}^\exists(\Phi^\prime, \Psi^\prime) \, ,
    \end{align*}
    and have to prove
    \begin{align*}
        (\mathcal{T}^*, S^* \cup \mathit{flatten}^\exists(E, \Psi^\prime)), \mathit{flatten}^\exists(C, \Psi^\prime) \vdash_\RL
          \mathit{flatten}^\exists(\Phi, \Psi^\prime)  \, .
    \end{align*}
    The second assumption simplifies to
    \begin{align*}
        (\mathcal{T}^*, S^* \cup \mathit{flatten}^\exists(E, \Psi^\prime)), \mathit{flatten}^\exists(C, \Psi^\prime) \vdash_\RL
          \mathit{flatten}(\Phi^\prime) \Rightarrow^\exists \mathit{flatten}(\Psi^\prime) \, ,
    \end{align*}
    while the goal to
    \begin{align*}
        (\mathcal{T}^*, S^* \cup \mathit{flatten}^\exists(E, \Psi^\prime)), \mathit{flatten}^\exists(C, \Psi^\prime) \vdash_\RL
          \mathit{flatten}(\Phi) \Rightarrow^\exists \mathit{flatten}(\Psi^\prime) \, ;
    \end{align*}
    therefore, we can apply the \emph{Consequence} rule.        
        
        
    \item If the proof ends with \emph{Abstract},
    we assume
    \begin{align*}
        X \not\in \mathit{FV}(\Psi^\prime)
    \end{align*}
    and
    \begin{align*}
                (\mathcal{T}^*, S^* \cup \mathit{flatten}^\exists(E, \Psi^\prime)), \mathit{flatten}^\exists(C, \Psi^\prime) \vdash_\RL
          \mathit{flatten}^\exists(\exists \vec{Y}.\, (\varphi_1, \ldots, \varphi_k) \land P, \Psi^\prime)
    \end{align*}
    and have to prove that
    \begin{align*}
                (\mathcal{T}^*, S^* \cup \mathit{flatten}^\exists(E, \Psi^\prime)), \mathit{flatten}^\exists(C, \Psi^\prime) \vdash_\RL
          \mathit{flatten}^\exists(\exists X,\vec{Y}.\, (\varphi_1, \ldots, \varphi_k) \land P, \Psi^\prime) \, .
    \end{align*}
    After simplifications, the second premise becomes
    \begin{align*}
            &(\mathcal{T}^*, S^* \cup \mathit{flatten}^\exists(E, \Psi^\prime)), \mathit{flatten}^\exists(C, \Psi^\prime) \vdash_\RL
          \\& \exists \vec{Y}.\, (\mathit{mkList}(Z_1, \ldots, Z_k) \land (\varphi_1^\square)[Z_1/\square] \land \ldots (\varphi_1^\square)[Z_k/\square]) \land P)
          \\&
          \Rightarrow^\exists \mathit{flatten}(\Psi^\prime) \, ,
    \end{align*}
    while the goal becomes
    \begin{align*}
          &(\mathcal{T}^*, S^* \cup \mathit{flatten}^\exists(E, \Psi^\prime)), \mathit{flatten}^\exists(C, \Psi^\prime) \vdash_\RL
          \\&\exists X. \exists \vec{Y}.\, (\mathit{mkList}(Z_1, \ldots, Z_k) \land (\varphi_1^\square)[Z_1/\square] \land \ldots (\varphi_1^\square)[Z_k/\square]) \land P) \, .
    \end{align*}
    We prove the goal using the Abstraction rule
    (note that $X \not\in \mathit{FV}(\Psi^\prime)$ implies $X \not\in \mathit{FV}(\mathit{flatten}(\Psi^\prime))$
    because we are free to choose the fresh variables inside the $\mathit{flatten}$ such).
    \end{enumerate}
    This concludes the proof.
\end{proof}


\subsection{Completeness}\label{app:completeness}

For completeness of our proof system, we assume a framework similar to that of \cite{StefanescuCMMSR19}.
Specifically, we assume that the model $\mathcal{T}$ interprets
\begin{itemize}
    \item the symbol $\alpha$ as an injective function from configurations into integers;
    \item the symbol $<$ as the usual ordering on integers.
\end{itemize}
\begin{theorem}\label{thm:oracleLifting}
For every $\Sigma$-model $\mathcal{T}$
satisfying the conditions above
there exists a function $\mathit{red}$ from matching logic patterns over $\Sigma^*$
to matching logic patterns over $\Sigma$ such that
$\mathcal{T} \vDash_\ML \mathit{red}(\varphi) \iff \mathcal{T^*} \vDash_\ML \varphi$.
\end{theorem}
\begin{proof}
Intuitively, we perform Gödelization of the formula $\varphi^\square$, using Gödel's $\beta$ predicate.
The idea is that the only construct appearing in FOL formulas over $\Sigma^*$ and not in FOL formulas over $\Sigma$
is comparison of lists of configuration for equality, and we reduce this construct to equality of the corresponding
elements of the list.
For this purpose, we introduce a function $\mathit{val}(M, l, n, y)$ representing a predicate saying that
the term $l$ representing a list of configurations has at position $n$ the configuration $y$.
%Equalities between lists are then traslated as the list having the same elements on the same indices.

We define functions $\mathit{val}$ and $\mathit{length}$ by mutual structural recursion on its second parameter:
\begin{itemize}
    \item $\mathit{val}(M, l, n, y) = \beta(a_{l}, b_{l}, n, \alpha(y))$ whenever $(l, a_{l}, b_{l}) \in M$;
    \item $\mathit{val}(\_, \mathit{cfgnil}, \_, \_) = \bot$.
    \item $\mathit{val}(M, \mathit{cfgconcat}(t_1, t_2), n, y) =$ \\
    $\forall i:\mathit{Int}. \, \mathit{length}(M, t_1, i) \rightarrow \mathit{ite}(n < i, \mathit{val}(M, t_1, n, y), \mathit{val}(M, t_2, n-i, y))$
    \item $\mathit{val}(M, \mathit{cfgheat}(l_1, c, l_2), n, y) =$
    \begin{align*}
        & \forall i:\mathit{Int}. \, \mathit{length}(M, t_1, i) \rightarrow \\
        & \mathit{ite}(n < i, \mathit{val}(M, t_1, n, y), \mathit{ite}(n = i, y = c, \mathit{val}(M, t_2, n-i-1, y)))
    \end{align*}    
    \item $\mathit{length}(M, l, n) = \forall i:\mathit{Int}.\, (0 \leq i < n) \leftrightarrow (\exists v : \mathit{Cfg}. \mathit{val}(M, l, i, v))$
\end{itemize}
Intuitively, $\mathit{val}(M, l, n, y)$ is a predicate saying that the list $l$ of configurations
contains the configuration $y$ on the position $n$,
where $M$ is a mapping from list variables into the two variables representing a list in the Godel encoding.
The predicate $\mathit{length}(M, l, n)$ holds if the lenght of $l$ is exactly $n$.


Next, we define a function $\mathit{tr}$ which performs basic recursion on a given FOL $\Sigma^*$ formula,
except that quantification over lists and equality of lists is handled as follows:
\begin{itemize}
    \item $\mathit{tr}(M, \forall (l : \mathit{Cfg}^*).\, \varphi) = \forall a_l : \mathit{Int}, b_l : \mathit{Int}.\, \mathit{tr}(M \cup \{ (l, a_l, b_l) \}, \varphi)$
    for some fresh $a_l,b_l$;
    \item $\mathit{tr}(M, l_1 = l_2) = \forall c : \mathit{Int}, d:\mathit{Cfg}.\, ( \mathit{val}(M, l_1, c, d) \leftrightarrow \mathit{val}(M, l_2, c, d))$
\end{itemize}

Let use consider the universal closure $\varphi_c$ of $\varphi^\square$.
Finally, we define $\mathit{red}(\varphi) = \mathit{tr}(\varphi_c)$; the desired equivalence holds by
properties of the Godel $\beta$ predicate and properties lists in the extended model $\mathcal{T}^*$.

\end{proof}

\subsection{Relation to CHL}\label{app:reltoCHL}

In order to tie (one-path) CRL to CHL, we first define an all-path variant of CRL.
\begin{definition}[All-Path CRL semantics]\label{def:apCRLsemantics}
    A claim
    \begin{equation*}
     [\varphi_1,\ldots,\varphi_k] \land P
     \Rightarrow^{c\forall} \exists \vec{Y}.\, [\varphi^\prime_1,\ldots,\varphi^\prime_k] \land P^\prime
    \end{equation*}
    is \emph{valid} in a reachability system $\mathcal{S} = (\mathcal{T}, S)$,
    written
    \begin{equation*}
        (\mathcal{T}, S) \vDash_\CRL [\varphi_1,\ldots,\varphi_k] \land P
     \Rightarrow^{c\forall} \exists \vec{Y}.\, [\varphi^\prime_1,\ldots,\varphi^\prime_k] \land P^\prime \, ,
    \end{equation*}
    iff for all configurations $\gamma_1,\ldots,\gamma_k \in \Tcfg$
    and any $\mathcal{T}$-valuation $\rho$,
    whenever $(\gamma_1,\rho) \vDash \varphi_1 \land P$ and \ldots
    and $(\gamma_k,\rho) \vDash \varphi_k \land P$,
    then for all complete $\Rightarrow_{\mathcal{S}}$-paths (that is, finite paths which cannot be extended further)
    $\pi_1,\ldots,\pi_k$
    satisfying $\pi_i[0] = \gamma_i$ for any $i \in \{ 1, \ldots, k \}$
    there exist natural numbers $j_1, \ldots, j_k$
    such that $\gamma_1 \Rightarrow^{*}_{\mathcal{S}} \pi_1[j_1]$
    and \ldots and $\gamma_k \Rightarrow^{*}_{\mathcal{S}} \pi_k[j_k]$,
    and $(\pi_1[j_1], \rho^\prime) \vDash \varphi^\prime_1 \land P^\prime$
    and \ldots and $(\pi_k[j_k], \rho^\prime) \vDash \varphi^\prime_k \land P^\prime$
    for some $\rho^\prime$ that agrees with $\rho$ on all variables except $\vec{Y}$.
\end{definition}

All-path CRL and one-path CRL have the same semantics on deterministic languages, assuming the RHS of the claim is terminal.

\begin{lemma}[One-path / All-path CRL correspondence]\label{crlOPAPcorrespondence}
    Let $\Phi$ be a CLP and $\Phi^\prime$ an ECLP such that $\Phi^\prime$ is terminal
    (that is, configurations matching $\Phi^\prime$ cannot take a step).
    Then $\mathcal{S} \vDash \Phi \Rightarrow^{c\exists} \Phi^\prime$
    if and only if $\mathcal{S} \vDash \Phi \Rightarrow^{c\forall} \Phi^\prime$.
\end{lemma}
\begin{proof}
Assume $\Phi \equiv [\varphi_1,\ldots,\varphi_k] \land P$
and $\Phi^\prime \equiv \exists \vec{Y}.\, [\varphi^\prime_1,\ldots,\varphi^\prime_k] \land P^\prime$.
For the ``if'' part, assume $\mathcal{S} \vDash \Phi \Rightarrow^{c\forall} \Phi^\prime$.
Assume some configurations $\gamma_1,\ldots,\gamma_k$ which terminate, and some valuation $\rho$
satisfying $(\gamma_1,\rho) \vDash \varphi_1 \land P$ and \ldots and $(\gamma_k,\rho) \vDash \varphi_k \land P$.
Let $\pi_1$ be some complete path $\gamma_1 = \gamma_1^1,\gamma_1^2,\ldots,\gamma_1^{l_1}$
and \ldots and $pi_k$ be some complete path $\gamma_k = \gamma_k^1,\gamma_k^2,\ldots,\gamma_k^{l_k}$.
Such paths exist, because $\gamma_i$ (for $i \in \{1, \ldots, k \}$) are terminating
- we can start with a path consisting of $\gamma_j$ only and repeatedly extend the path until the last element has no successor.
Now, since every $\phi_i^\prime$ is terminal, all the $j_i$s that exist by \Cref{def:apCRLsemantics}
refer to the last configurations in the paths, because only those are terminal.
That is, we have $j_i = l_i$ for any $i \in \{ 1, \ldots, k \}$.
Therefore, we have some $\rho^\prime$ and $\gamma_1^{l_1}, \ldots, \gamma_k^{l_k}$ satisfying
$(\gamma_1^{l_1}, \rho^\prime) \vDash \varphi^\prime_1 \land P$ and \ldots
and $(\gamma_k^{l_k}, \rho^\prime) \vDash \varphi^\prime_k \land P$,
as required by \Cref{def:opCRLsemantics}.

For the ``only if'' part, assume $\mathcal{S} \vDash \Phi \Rightarrow^{c\exists} \Phi^\prime$.
Assume some configurations $\gamma_1,\ldots,\gamma_k$ and some valuation $\rho$
satisfying $(\gamma_1,\rho) \vDash \varphi_1 \land P$ and \ldots and $(\gamma_k,\rho) \vDash \varphi_k \land P$.
Let $\pi_1, \ldots, \pi_k$ be complete paths
$\gamma_1^1,\gamma_1^2,\ldots,\gamma_1^{l_1}$ and \ldots and $\gamma_k^1,\gamma_k^2,\ldots,\gamma_k^{l_k}$
starting with $\gamma_1,\ldots,\gamma_k$, respectively.
By \Cref{def:opCRLsemantics}, there exist configurations $\gamma_1^\prime,\ldots,\gamma_k^\prime$
such that $\gamma_1 \Rightarrow_{\mathcal{S}}^* \gamma^\prime_1$
and \ldots
and $\gamma_k \Rightarrow_{\mathcal{S}}^* \gamma^\prime_k$,
and there also exists a valuation $\rho^\prime$
that agrees with $\rho$ on all variables outside $\vec{Y}$
such that
$(\gamma^\prime_1, \rho^\prime) \vDash \varphi^\prime_1 \land P$
and \ldots and
$(\gamma^\prime_k, \rho^\prime) \vDash \varphi^\prime_k \land P$.
Now by determinism and the fact that $\gamma^\prime_i$ are terminating (because $\varphi^\prime_i$ are terminating), we have
$\gamma^{l_i}_i = \gamma^\prime_i$.
Thus,
$(\pi_1[j_1], \rho^\prime) \vDash \varphi^\prime_1 \land P^\prime$
and \ldots and $(\pi_k[j_k], \rho^\prime) \vDash \varphi^\prime_k \land P^\prime$,
which concludes the proof.
\end{proof}

\begin{definition}
    Let $L_{\mathit{CHL}}$ denote the CHL's imperative language
    and $\Sigma_{L_{\mathit{CHL}}}$ denote the matching logic signature of a RL-based formalization of $L_{\mathit{CHL}}$
    that that has a distinct constant symbol $\mathit{sym}_\mathit{var}(x)$
    and a distinct variable $\mathit{var}_\mathit{inj}(x)$ in the signature
    for ever program variable $x$ of $L_{\mathit{CHL}}$,
    and that subsumues the syntax of the codomain of CHL state.
    We define a function $\mathit{tr}_{\mathit{con}}$ by
    \begin{equation*}
        \mathit{tr}_{\mathit{con}}(P, \sigma) \quad \equiv \quad
        \impConfig{P}{
            \mathit{sym}_\mathit{var}(x_1) \mapsto \sigma(x_1),
            \ldots,
            \mathit{sym}_\mathit{var}(x_n) \mapsto \sigma(x_n)
        }
    \end{equation*}
    and a function $\mathit{end}_{\mathit{con}}$ by
    \begin{equation*}
        \mathit{end}_{\mathit{con}}(P, \sigma) \quad \equiv \quad
        \impConfig{\texttt{skip}}{
            \mathit{sym}_\mathit{var}(x_1) \mapsto \sigma(x_1),
            \ldots,
            \mathit{sym}_\mathit{var}(x_n) \mapsto \sigma(x_n)
        }
    \end{equation*}
    (where $x_1,\ldots,x_n$ are program variables occurring in $P$).
\end{definition}

\begin{assumption}\label{a:sf}
    We assume a sound formalization of $L_{\mathit{CHL}}$ in the form of a reachability system
    $S_{L_{\mathit{CHL}}}$ that for any $P,\sigma$ and any terminal $\gamma^\prime$ satisfies
    \begin{equation*}
        \mathit{tr}_{\mathit{con}}(P, \sigma) \Rightarrow^*_{S_{L_{\mathit{CHL}}}} \gamma^\prime
        \iff
        (\sigma, S \Downarrow \sigma^\prime) \land (\gamma^\prime = \mathit{end}_{\mathit{con}}(P, \sigma^\prime))
    \end{equation*}
    (where $\Downarrow$ denotes the relation defined by the original big-step semantics of $L_{\mathit{CHL}}$).
    Furthermore, we assume that $S_{L_{\mathit{CHL}}}$ ignores the state of program variables not mentioned in the program
    being executed - a property that is easy to satisfy by the implementation.
    Formally, for any finite maps $m_1,m_2,m_f$ such that $m_1$ and $m_f$ are disjoint, and for any program $P$,
    we have
    \begin{equation*}
        \impConfig{P}{m_1} \quad \Rightarrow^*_{S_{L_{\mathit{CHL}}}} \quad \impConfig{P}{m_2} \, ,
    \end{equation*}
    iff
    \begin{equation*}
        \impConfig{P}{m_1 \cup m_f} \quad \Rightarrow^*_{S_{L_{\mathit{CHL}}}} \quad \impConfig{P}{m_2 \cup m_f} \, .
    \end{equation*}
\end{assumption}

\begin{definition}
    Let $f$ be an injective function on program variables of $L_{\mathit{CHL}}$.
    We define a function $\mathit{tr}_{\mathit{sym}}$ by
    \begin{equation*}
        \mathit{tr}_{\mathit{sym}}(P, f_{\mathit{inj}}) \quad \equiv \quad
        \impConfig{P}{
            \mathit{sym}_\mathit{var}(x_1) \mapsto \mathit{var}_{\mathit{inj}}(f(x_1)),
            \ldots,
            \mathit{sym}_\mathit{var}(x_n) \mapsto \mathit{var}_{\mathit{inj}}(f(x_n))
        }% \land f_{\mathit{inj}}(\varphi)
    \end{equation*}
    and a function $\mathit{end}_{\mathit{sym}}$ by
    \begin{equation*}
       \mathit{end}_{\mathit{sym}}(P, f_{\mathit{inj}}) \quad \equiv \quad
       \impConfig{\texttt{skip}}{
           \mathit{sym}_\mathit{var}(x_1) \mapsto \mathit{var}_{\mathit{inj}}(f(x_1)),
           \ldots,
           \mathit{sym}_\mathit{var}(x_n) \mapsto \mathit{var}_{\mathit{inj}}(f(x_n))
       }
    \end{equation*}
    where $P$ is a statement of $L_{\mathit{CHL}}$ over variables $x_1,\ldots,x_n$.
    %and $f_{\mathit{inj}}$ is homomorphically extended to FOL formulas.
\end{definition}

\begin{lemma}[Symbolic and concrete match]\label{lem:symConcreteTranslationMatch}
    For any statement $P$, any injective function $f$ on program variables of $L_{\mathit{CHL}}$,
    any finite map $\sigma$ and any configuration $\gamma$,
    we have
    \begin{equation*}
        (\gamma, \rho) \vDash_{\ML} \mathit{tr}_{\mathit{sym}}(P, f)
    \end{equation*}
    iff there exists a finite map $h$ that is disjoint with $\sigma$ such that
    \begin{equation*}
        (\gamma = \mathit{tr}_{\mathit{con}}(P, \sigma[f] \cup h[f])) \land (\forall j \in \{ 1, \ldots, n \}.\, \rho(\mathit{var}_{\mathit{inj}}(f(x_j))) = (\sigma[f])(x_j)) \, ;
    \end{equation*}
    similarly, we have
    \begin{equation*}
        (\gamma, \rho) \vDash_{\ML} \mathit{end}_{\mathit{sym}}(P, f)
    \end{equation*}
    iff there exists a finite map $h$ that is disjoint with $\sigma$ such that
    \begin{equation*}
        (\gamma = \mathit{end}_{\mathit{con}}(P, \sigma[f] \cup h[f])) \land (\forall j \in \{ 1, \ldots, n \}.\, \rho(\mathit{var}_{\mathit{inj}}(f(x_j))) = (\sigma[f])(x_j))
    \end{equation*}
    (where $x_1,\ldots,x_n$ are program variables occurring in $P$).
\end{lemma}
\begin{proof}
    Follows by simple pattern matching.
\end{proof}

\begin{lemma}[Static reasoning in CHL vs CRL]\label{lem:chlCRLstatic}
    Let $\sigma_1,\ldots,\sigma_k$ be program states over $\vec{x}$.
    Let $r_i(x) = x_i$ for any $x \in \vec{x}$ and any $i \in \{ 1, \ldots, k \}$.
    Let $P$ be a statement over variables $r_1(\vec{x}), \ldots, r_k(\vec{x})$.
    %Let $f_{r_i}()$
    Let $\sigma = \biguplus_{1 \leq i \leq k} \sigma_i[r_i]$.
    Let $\rho_\sigma$ be a matching logic valuation sending matching logic variables
    $\mathit{var}_{\mathit{inj}}(x)$ to $\sigma(x)$.
    Then:
    \begin{align*}
        \sigma \vDash_{\FOL} \varphi
        & \iff \forall P.\, \forall i \in \{ 1, \ldots, k \} . \,
        (\mathit{tr}_{\mathit{con}}(P, \sigma_i[r_i]), \rho_\sigma)
        \vDash_{\ML} \mathit{tr}_{\mathit{sym}}(P, r_i) \land \varphi \\
        & \iff \forall P.\, \forall i \in \{ 1, \ldots, k \} . \,
        (\mathit{end}_{\mathit{con}}(P, \sigma_i[r_i]), \rho_\sigma)
        \vDash_{\ML} \mathit{end}_{\mathit{sym}}(P, r_i) \land \varphi
    \end{align*}
\end{lemma}
\begin{proof}
    We show proof of the first equivalence only, since the second is similar.
    By \Cref{lem:symConcreteTranslationMatch} (with $h$ being empty) and properties of matching logic conjunction and structureless formulas,
    the RHS is equivalent to
    \begin{proofenv}
        \begin{itemize}
            \item for any $j \in \{ 1,\ldots,n \}$,
            $\rho_{\sigma}(\mathit{var}_{\mathit{inj}}(r_i(x_j))) = (\sigma_i[r_i])(x_j)$; and
            \item $\rho_\sigma \vDash_{\ML} \varphi$.
        \end{itemize}
    \end{proofenv}
    The first item is always true: by the defining property of $\rho_{\sigma}$,
    it is enough to show that $\sigma(r_i(x_j)) = (\sigma_i[r_i])(x_j)$, which is trivially true.
    The second item is then equivalent to the LHS.
\end{proof}



\begin{lemma}[All-path CRL vs CHL]\label{thm:chlAPCRLrelation}
    Let $P$ be a statement of $L_{\mathit{CHL}}$ over variables $\vec{x} = x_1,\ldots,x_n$,
    and let $\Phi,\Psi$ be FOL formulas over variables
    $\vec{x}_1,\ldots,\vec{x}_k$.
    Let $\vec{Y} = \mathit{var}_{\mathit{inj}}(\vec{x}_1, \ldots, \vec{x}_k)$.
    Then $||\varphi||\ P\ ||\psi||$ if and only if
    \begin{equation*}
        [\mathit{tr}_{\mathit{sym}}(P, r_1),\ldots,\mathit{tr}_{\mathit{sym}}(P, r_k)] \land \varphi
        \Rightarrow^{c\forall} \exists \vec{Y}.\, [\mathit{end}_{\mathit{sym}}(P, r_1),\ldots,\mathit{end}_{\mathit{sym}}(P, r_k)] \land \psi \, .
    \end{equation*}
\end{lemma}
\begin{proof}
    The left side is true iff (by definition)
    \begin{proofenv}
        for every set of valuation pairs $\{ (\sigma_1,\sigma_1^\prime),\ldots,(\sigma_k,\sigma_k^\prime) \}$
        satisfying
        \begin{equation*}
            \left( \biguplus_{1 \leq i \leq k} \sigma_i[r_i] \vDash_{\FOL} \varphi \right)
        \end{equation*}
        and
        \begin{equation*}
            \forall i \in \{ 1, \ldots, k \}.\, \sigma_i, P  \Downarrow \sigma^\prime_i
        \end{equation*}
        we also have
        \begin{equation*}
            \left( \biguplus_{1 \leq i \leq k} \sigma^\prime_i[r_i] \vDash_{\FOL} \psi \right)
        \end{equation*}
    \end{proofenv}
    which is (by \Cref{lem:chlCRLstatic} and Assumption~\ref{a:sf}) equivalent to
    \begin{proofenv}
        for every set of valuation pairs $\{ (\sigma_1,\sigma_1^\prime),\ldots,(\sigma_k,\sigma_k^\prime) \}$
        satisfying
        \begin{equation*}
            \forall P.\, \forall i \in \{ 1, \ldots, k \} . \,
                (\mathit{tr}_{\mathit{con}}(P, \sigma_i[r_i]), \rho_\sigma)
                \vDash_{\ML} \mathit{tr}_{\mathit{sym}}(P, r_i) \land \varphi
        \end{equation*}
        and
        \begin{equation*}
            \forall i \in \{ 1, \ldots, k \}.\,
            \mathit{tr}_{\mathit{con}}(P, \sigma_i) \Rightarrow^*_{S_{L_{\mathit{CHL}}}} \mathit{end}_{\mathit{con}}(P, \sigma^\prime_i)
        \end{equation*}
        we also have
        \begin{equation*}
            \forall P.\, \forall i \in \{ 1, \ldots, k \} . \,
            (\mathit{end}_{\mathit{con}}(P, \sigma_i[r_i]), \rho_{\sigma^\prime})
            \vDash_{\ML} \mathit{end}_{\mathit{sym}}(P, r_i) \land \psi
        \end{equation*}
        (where $\sigma = \biguplus_{1 \leq i \leq k} \sigma_i[r_i]$
        and $\sigma^\prime = \biguplus_{1 \leq i \leq k} \sigma^\prime_i[r_i]$)
    \end{proofenv}

    We want to show this to be equivalent with
    \begin{proofenv}
        for all configurations $\gamma_1,\ldots,\gamma_k \in \Tcfg$
        and any $\mathcal{T}$-valuation $\rho$,
        whenever $(\gamma_1,\rho) \vDash \mathit{tr}_{\mathit{sym}}(P, r_1) \land \varphi$ and \ldots
        and $(\gamma_k,\rho) \vDash \mathit{tr}_{\mathit{sym}}(P, r_k) \land \varphi$,
        then for all complete $\Rightarrow_{\mathcal{S}}$-paths (that is, finite paths which cannot be extended further)
        $\pi_1,\ldots,\pi_k$
        satisfying $\pi_i[0] = \gamma_i$ for any $i \in \{ 1, \ldots, k \}$
        there exist natural numbers $j_1, \ldots, j_k$
        such that $\gamma_1 \Rightarrow^{*}_{S_{L_{\mathit{CHL}}}} \pi_1[j_1]$
        and \ldots and $\gamma_k \Rightarrow^{*}_{S_{L_{\mathit{CHL}}}} \pi_k[j_k]$,
        and $(\pi_1[j_1], \rho^\prime) \vDash \mathit{end}_{\mathit{sym}}(P, r_1) \land \psi$
        and \ldots and $(\pi_k[j_k], \rho^\prime) \vDash \mathit{end}_{\mathit{sym}}(P, r_k) \land \psi$
        for some $\rho^\prime$ that agrees with $\rho$ on all variables except $\vec{Y}$.
    \end{proofenv}
    \begin{enumerate}
        \item For the top-down implication, assume such configurations and such complete paths.
            Let $l_1,\ldots,l_k$ denote the lengths of the paths $\pi_1,\ldots,\pi_k$.
            By \Cref{lem:symConcreteTranslationMatch},
            we have for every $j \in \{ 1, \ldots, k \} $ and some $h_j$:
            \begin{equation*}
                \gamma_j = \mathit{tr}_{\mathit{con}}(P, \sigma[r_j] \cup h_j[r_j])
            \end{equation*}
            and therefore
            \begin{equation*}
                \mathit{tr}_{\mathit{con}}(P, \sigma[r_j] \cup h_j[r_j])
                \Rightarrow^*_{S_{L_{\mathit{CHL}}}}
                \pi_j[l_j] \, .
            \end{equation*}
            %Let us destruct $\pi_j[l_j]$ into some $\impConfig{P^\prime_i}{}$
            By Assumption~\ref{a:sf}, we have
            \begin{equation*}
                \pi_j[l_j] = \mathit{end}_{\mathit{con}}(P, \sigma[r_j]) \, .
            \end{equation*}
            Because we have
            \begin{equation*}
                \forall P.\, \forall i \in \{ 1, \ldots, k \} . \,
                (\mathit{end}_{\mathit{con}}(P, \sigma_i[r_i]), \rho_{\sigma^\prime})
                \vDash_{\ML} \mathit{end}_{\mathit{sym}}(P, r_i) \land \psi \, ,
            \end{equation*}
            it follows that
            \begin{equation*}
                (\pi_j[l_j], \rho_{\sigma^\prime}) \vDash_{\ML} \mathit{end}_{\mathit{sym}}(P, r_i) \land \psi \, ,
            \end{equation*}
            which is what we needed to prove.
        \item For the bottom-up implication,
            assume a set of valuation pairs $\{ (\sigma_1,\sigma_1^\prime),\ldots,(\sigma_k,\sigma_k^\prime) \}$
            satisfying
            \begin{equation*}
                \forall P.\, \forall i \in \{ 1, \ldots, k \} . \,
                    (\mathit{tr}_{\mathit{con}}(P, \sigma_i[r_i]), \rho_\sigma)
                    \vDash_{\ML} \mathit{tr}_{\mathit{sym}}(P, r_i) \land \varphi
            \end{equation*}
            and
            \begin{equation*}
                \forall i \in \{ 1, \ldots, k \}.\,
                \mathit{tr}_{\mathit{con}}(P, \sigma_i) \Rightarrow^*_{S_{L_{\mathit{CHL}}}} \mathit{end}_{\mathit{con}}(P, \sigma^\prime_i)
            \end{equation*}
            The second means we have for every $i$ some complete trace $\pi_i$ of length $l_i$
            starting in $\mathit{tr}_{\mathit{con}}(P, \sigma_i)$ and ending in $\mathit{end}_{\mathit{con}}(P, \sigma^\prime_i)$.
            We need to prove:
            \begin{equation*}
                \forall P.\, \forall i \in \{ 1, \ldots, k \} . \,
                (\mathit{end}_{\mathit{con}}(P, \sigma_i[r_i]), \rho_{\sigma^\prime})
                \vDash_{\ML} \mathit{end}_{\mathit{sym}}(P, r_i) \land \psi
            \end{equation*}
            (where $\sigma = \biguplus_{1 \leq i \leq k} \sigma_i[r_i]$
            and $\sigma^\prime = \biguplus_{1 \leq i \leq k} \sigma^\prime_i[r_i]$).
            Let $P^\prime$ be some program and let $i$ be in $\{ 1, \ldots, k \}$.
            We need to show that 
            \begin{equation*}
                (\mathit{end}_{\mathit{con}}(P^\prime, \sigma_i[r_i]), \rho_{\sigma^\prime})
                \vDash_{\ML} \mathit{end}_{\mathit{sym}}(P^\prime, r_i) \land \psi \, .
            \end{equation*}
            By \Cref{lem:symConcreteTranslationMatch}, it is enough to show that
            \begin{equation*}
                (\forall j \in \{ 1, \ldots, n \}.\, \rho_{\sigma^\prime}(\mathit{var}_{\mathit{inj}}(f(x_j))) = (\sigma^\prime[f])(x_j))
            \end{equation*}
            (which holds by definition of $\rho_{\sigma^\prime}$) and that
            \begin{equation*}
                \rho_{\sigma^\prime} \vDash_{\ML} \psi \, .
            \end{equation*}
            It is enough to show that
            $(\pi_i[l_i], \rho^\prime) \vDash \mathit{end}_{\mathit{sym}}(P, r_i) \land \psi$
            which follows from the assumptions by firstorder reasoning.
    \end{enumerate}

\end{proof}

    Now we can combine the above into the following result:

    \begin{theorem}[One-path CRL vs CHL]\label{thm:chlOPCRLrelation}
        Let $P$ be a \emph{deterministic} statement of $L_{\mathit{CHL}}$ over variables $\vec{x} = x_1,\ldots,x_n$,
        and let $\varphi,\psi$ be FOL formulas over variables
        $\vec{x}_1,\ldots,\vec{x}_k$.
        Let $\vec{Y} = \mathit{var}_{\mathit{inj}}(\vec{x}_1, \ldots, \vec{x}_k)$.
        Then $||\varphi||\ P\ ||\psi||$ if and only if
        \begin{equation*}
            [\mathit{tr}_{\mathit{sym}}(P, r_1),\ldots,\mathit{tr}_{\mathit{sym}}(P, r_k)] \land \varphi
            \Rightarrow^{c\exists} \exists \vec{Y}.\, [\mathit{end}_{\mathit{sym}}(P, r_1),\ldots,\mathit{end}_{\mathit{sym}}(P, r_k)] \land \psi \, .
        \end{equation*}
    \end{theorem}
    \begin{proof}
        Use \Cref{thm:chlAPCRLrelation} and \Cref{crlOPAPcorrespondence}.
    \end{proof}


    \begin{proof}[Proof of \Cref{thm:chlCRLrelation}]
        We define
        \begin{align*}
            & \mathit{tr}(P, \varphi) \equiv [\mathit{tr}_{\mathit{sym}}(P, r_1),\ldots,\mathit{tr}_{\mathit{sym}}(P, r_k)] \land \varphi \\
            & \mathit{end}(P, \psi) \equiv \exists \vec{Y}.\, [\mathit{end}_{\mathit{sym}}(P, r_1),\ldots,\mathit{end}_{\mathit{sym}}(P, r_k)] \land \psi \,
        \end{align*}
        (where $P$ is a deterministic statement over $\vec{x} = x_1,\ldots,x_n$, 
        and $\varphi,\psi$ are FOL formulas over variables
        $\vec{x}_1,\ldots,\vec{x}_k$,
        and $\vec{Y} = \mathit{var}_{\mathit{inj}}(\vec{x}_1, \ldots, \vec{x}_k)$
        )
        and apply \Cref{thm:chlOPCRLrelation}.
    \end{proof}



\end{document}
