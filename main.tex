\documentclass{article}
\usepackage[utf8]{inputenc}
\usepackage{amsmath,amssymb,mathtools}
\usepackage{thmtools}
\usepackage{hyperref}
% `cleveref` has to be loaded after `hyperref`
\usepackage{cleveref}
\usepackage{stackengine}
\usepackage{minted}
\usepackage{tcolorbox}
\usepackage{prftree}
\usepackage{fancyvrb}
\usepackage{csquotes}

\title{Cartesian Reachability Logic}
\author{Jan Tušil }
\date{February 2022}

\declaretheorem[]{example}
\declaretheorem[]{definition}
\declaretheorem[]{lemma}
\declaretheorem[]{theorem}
\declaretheorem[]{proof}
\declaretheorem[]{remark}

\newcommand{\CRL}{\mathsf{CRL}}

\newcommand\oast{\stackMath\mathbin{\stackinset{c}{0ex}{c}{0ex}{\ast}{\bigcirc}}}

\begin{document}

\maketitle

\section{Introduction}

An option would be to come up with a language-agnostic way of composing two programs in such a way
that the composition is still executable.
In particular, given a matching logic signature $\Sigma$ and a reachability logic theory $S$, one may
want to construct a signature $\Sigma^\prime = \Sigma \cup \{ \mathit{prod} \}$
(where $\mathit{prod}$ is a fresh symbol)
and a theory $S^\prime$ such that for any $\mu$-free patterns $\varphi_1,\varphi_2,\psi_1,\psi_2$
it holds that $S \vdash \varphi_1 \Rightarrow^\exists \psi_1$ and
$S \vdash \varphi_2 \Rightarrow^\exists \psi_2$ if and only if
\begin{equation*}
S^\prime \vdash \mathit{prod}\ \varphi_1\ \varphi_2 \Rightarrow^\exists \mathit{prod}\ \psi_1\ \psi_2    \, .
\end{equation*}
%We may also want $S^\prime$ to be deterministic whenever $S$ is.

Next, we need to find a generalization of Cartesian Hoare triples.
Something like the following might suffice:
\begin{definition}\label{def:CRLsemantics}
A Cartesian reachability rule
$(\varphi_1,\ldots,\varphi_k) \land \varphi \Rightarrow^{c\exists} (\psi_1,\ldots,\psi_k) \land \psi$
of arity $k$
is \emph{valid},
written
$\vDash_\CRL (\varphi_1,\ldots,\varphi_k) \land \varphi \Rightarrow^{c\exists}
(\psi_1,\ldots,\psi_k) \land \psi$,
iff for all configurations $\sigma_1,\ldots,\sigma_k$ and any valuation $\rho$,
whenever $(\sigma_1, \ldots, \sigma_k) \in |(\varphi_1, \ldots, \varphi_k) \land \varphi|_\rho$,
then there exist configurations $\sigma_1^\prime,\ldots,\sigma_k^\prime$ such that
$(\sigma_1,\ldots,\sigma_k) \in |(\psi_1,\ldots,\psi_k) \land \psi|_\rho$ and for any $i \in \{ 1,\ldots,k \}$,
$\sigma_i \Rightarrow^*_S \sigma_i^\prime$.
\end{definition}
The idea behind the above definition is that $\varphi,\psi$ are predicate patterns
that constraints the relationship between the configurations.
I would then want to come up with a (language-parametric) proof system
for deriving claims of the shape
$\vdash_\CRL (\varphi_1,\ldots,\varphi_k) \land \varphi \Rightarrow^{c\exists}
(\psi_1,\ldots,\psi_k) \land \psi$;
similarly to CHL, the proof system should have a \emph{linearization} property,
meaning that a claim could be derived if and only if
a linearized claim (using the composition with $\textit{prod}$ described above) could be derived in reachability logic.

Then I would need to overcome the following three challenges:
\begin{enumerate}
    \item to extend the above definition with axioms and circularities, as in reachability logic;
    \item to create a rule (or set of rules) playing the role of the Fusion rule from CHL; and
    \item to support non-deterministic operations and go all-path.
\end{enumerate}

\section{Questions}

The CHL paper has a lemma saying that sequential composition commutes (assuming the two statements use non-overlapping sets of variables).
We need something similar. What would be our equivalent? Basically, that
$\mathit{prod} c_1 c_2$ satisfies the same set of reachability claims as $\mathit{prod} c_2 c_1$?
Also, what is the relationship between the product program in product semantics and the original program? What is the semantics of the CRL claim - is it defined using only the semantics of the original program in original language? (I think so.) So we first need to define language independent product programs/semantics and prove properties about them.

Can we verify some property of a heap manipulating program? Say, 'monotonicity' of sum with list append?


\Cref{def:CRLsemantis} uses pairs inside matching logic. Maybe we should rephrase it so that it talks about valuations instead.

\section{Product programs and their semantics}
We represent a programming language semantics as a reachability system.
\begin{definition}\label{def:RLsystem}
A \emph{reachability system} $\mathcal{S}$ is a triple $(\Sigma, \mathcal{M}, S)$, where
\begin{enumerate}
    \item $\Sigma$ is a matching logic signature;
    \item $\mathcal{M}$ is a matching logic $\Sigma$-model; and
    \item $S$ is a set of \emph{reachability rules} of the shape
    b$\varphi_1 \Rightarrow^\exists \varphi_2$, where $\varphi_1,\varphi_2$ are matching logic $\Sigma$-patterns.
\end{enumerate}
\end{definition}

\begin{definition}\label{def:RLproduct}
Given a reachability system $\mathcal{S} = (\Sigma, \mathcal{M}, S)$
where $\mathcal{M} = (M, \{\sigma_M\}_{\sigma \in \Sigma}, @_M)$,
we define a \emph{self-product system}
$\mathcal{S}\oast{}\mathcal{S} = (\Sigma\oast{}\Sigma, \mathcal{M}\oast{}\mathcal{M}, S\oast{}S)$ as follows.
\begin{enumerate}
    \item $\Sigma\oast{}\Sigma = \Sigma \cup \{ \mathit{prod} \}$, where $\mathit{prod}$ is a fresh symbol not occurring in $\Sigma$;
    \item $\mathcal{M}\oast{}\mathcal{M} = (M\oast{}M, \{\sigma_{M\oast{}M}\}_{\sigma \in \Sigma\oast{}\Sigma}, @_{M\oast{}M})$, where
    \begin{enumerate}
        \item $M\oast{}M = M \cup \{ \mathrm{\#prod} \} \cup (\{ \mathrm{\#prod} \} \times M) \cup (\{ \mathrm{\#prod} \} \times M \times M)$, where $\mathrm{\#prod}$ is a fresh element not occurring in $M$; 
        \item $\sigma_{\mathcal{M} \oast \mathcal{M}} = \sigma_M$ for any $\sigma \in \Sigma$;
        \item $\sigma_{\mathit{prod}}$ = $\mathrm{\#prod}$;
        \item 
    \end{enumerate}
    \item 
\end{enumerate}
\end{definition}

\end{document}
