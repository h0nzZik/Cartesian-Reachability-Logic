\documentclass{easychair}
\usepackage[utf8]{inputenc}
\usepackage{amsmath,amssymb,mathtools}
\usepackage{amsthm}
\usepackage{thmtools}
\usepackage{hyperref}
% `cleveref` has to be loaded after `hyperref`
\usepackage{cleveref}
\usepackage{stackengine}
%\usepackage{minted}
\usepackage{tcolorbox}
\usepackage{multicol}
\usepackage{prftree}
\usepackage{fancyvrb}
\usepackage{csquotes}
\usepackage{appendix}
\usepackage{enumitem}
\usepackage{todonotes}


\usepackage{amssymb}

\newcommand{\traian}[1]{\todo[author=Traian]{#1}}
\newcommand{\jt}[1]{\todo[author=Jan]{#1}}

\title{Cartesian Reachability Logic: A Language-Parametric Logic for Verifying $k$-Safety Properties}
\author{
  Jan Tu\v{s}il \inst{1}
  \and Traian Florin Șerbănuță  \inst{2}
  \and Jan Obdržálek \inst{1}
}

\institute{
  Masaryk University,
  Brno, Czech Republic\\
  \email{jan.tusil@mail.muni.cz,obdrzalek@fi.muni.cz}
\and
   University of Bucharest,
   Bucharest, Romania\\
   \email{traian.serbanuta@unibuc.ro}\\
 }

\authorrunning{Tušil, Șerbănuță, and Obdržálek}
\titlerunning{CRL: A Language-Independent Logic for $k$-Safety}

%\date{\today}

\declaretheorem[]{example}
\declaretheorem[]{definition}
\declaretheorem[]{lemma}
\declaretheorem[]{theorem}
\declaretheorem[]{remark}

\newcommand{\K}{$\mathbb{K}$}
\newcommand{\RL}{\mathsf{RL}}
\newcommand{\ML}{\mathsf{ML}}
\newcommand{\CRL}{\mathsf{CRL}}
\newcommand{\FOL}{\mathsf{FOL}}
\newcommand{\FOLeq}{\FOL_{=}}
\newcommand{\Var}{\mathit{Var}}
\newcommand{\Tcfg}{\mathcal{T}_{\mathit{Cfg}}}
\newcommand{\Pattern}{\mathsf{Pattern}}

\newcommand\oast{\stackMath\mathbin{\stackinset{c}{0ex}{c}{0ex}{\ast}{\bigcirc}}}

\newenvironment{proofenv}
  {
    \VerbatimEnvironment\begin{tcolorbox}[colback=black!0!white] % 5 is the default
  }
  {
   \end{tcolorbox}
  }


% CHL macros
\newcommand{\chl}[3]{\langle #1\rangle\ #2\ \langle#3\rangle}
\newcommand{\st}{\circledast}


  
\begin{document}

\maketitle


\begin{abstract}
  We introduce Cartesian Reachability Logic (CRL):
  a language-parametric logic for reasoning about k-safety hyperproperties.

  In recent years, formal verification of hyperproperties has become an important topic
  in the formal methods community.
  An interesting class of hyperproperties is known as $k$-safety:
  a $k$-safety property prohibits the existence of a \emph{bad} $k$-tuple of execution traces.
  Many security policies, such as \emph{noninterference}, 
  and functional properties, such as \emph{commutativity}, are $k$-safety.
  Previously, Cartesian Hoare logic (CHL) has been used for reasoning about $k$-safety hyperproperties
  of software systems.
  CHL has a sound and complete proof system and a performant verification algorithm.
  However, CHL targets only a specific, tiny imperative language, and to use it for sound verification
  of programs in a different language $L$, one needs to extend it with the desired features,
  or to hand-craft a translation.
  Both these approaches require a lot of tedious, error-prone work.
  An even worse alternative is to manually implement the verification algorithm,
  designed to work with the tiny language, with ad-hoc modifications targeting $L$:
  that gives the user of the implementation no guarantee at all.

  In contrast, CRL is language-parametric: it can be instantiated with any deterministic language
  modeled in a particular operational semantics formalism.
  Its soundness theorem is proved once and for all, with no need to adapt or re-prove it
  for different languages or their variants.
  Therefore, this powerful instrument can significantly reduce the development costs of tools and techniques
  for sound $k$-safety verification.
\end{abstract}

% \begin{abstract}
%   In recent years, formal verification of hyperproperties has become an important topic
%   in the formal methods community.
%   An interesting class of hyperproperties is known as $k$-safety:
%   a $k$-safety property prohibits the existence of a \emph{bad} $k$-tuple of execution traces.
%   Many security policies, such as \emph{noninterference}, 
%   and functional properties, such as \emph{commutativity}, are $k$-safety.
  
%   One logic for reasoning about $k$-safety hyperproperties of software
%   is Cartesian Hoare logic (CHL).
%   CHL has a sound and complete proof system and a performant verification algorithm.
%   However, CHL targets only a specific, tiny imperative language, and to use it for sound verification
%   of programs in a different language $L$, one needs to either extend it with the desired features,
%   or to hand-craft a translation. Both these approaches require a lot of tedious, error-prone work.
%   An even worse alternative is to manually implement the verification algorithm,
%   designed to work with the tiny language, with ad-hoc modifications targeting $L$:
%   that gives the user of the implementation no guarantee at all.
  
%   Therefore we introduce Cartesian Reachability logic (CRL): a new program logic for reasoning
%   about $k$-safety hyperproperties.
%   The distinguishing feature of CRL is that it can be instantiated with any deterministic language
%   whose model can be specified in a particular "operational semantics"-like formalism.
%   Its soundness theorem is proved once and for all, with no need to adapt or re-prove it
%   for different languages or their variants.
%   This powerful instrument can significantly reduce the development costs of tools and techniques
%   for sound $k$-safety verification.
% \end{abstract}


%\begin{abstract}
%  Crafting a program logic targeting a specific language can be a challenging and time-consuming task.
%  In this paper we introduce a program logic for reasoning about $k$-safety hyperproperties,
%  with the distinguishing feature that our logic works with any deterministic language
%  that has a Reachability logic-based formal semantics.
%  This powerful instrument can significantly reduce development costs of tools and techniques for k-safety verification.
%\end{abstract}

\section{Introduction}
Recent years have witnessed an increased interest in formal verification of
\emph{hyperproperties}~\cite{ClarksonS08}. Unlike properties, whose validity depends on a single
execution trace, hyperproperties can relate multiple program executions. A
particularly interesting class of hyperproperties are \emph{$k$-safety
  (hyper-)properties}~\cite{FinkbeinerHT19CanonicalKsafety,SousaD16,AgrawalB16RuntimeKSafetyHLTL,ClarksonS08}.
A $k$-safety property is a hyperproperty whose violation can be witnessed by a
$k$-tuple of execution traces.  Many \emph{security policies} - for example,
\emph{noninterference} (requiring that sensitive or privileged data do not influence
insensitive or unprivileged computations)
or \emph{observational determinism}
- are $k$-safety hyperproperties~\cite{}.
Similarly, many functional correctness properties are actually $k$-safety
hyperproperties; for example, \emph{transitivity} (which needs to by satisfied
e.g. by comparators when managing data in collections), associativity
(important in the map/reduce paradigm), or monotonicity~\cite{SousaD16}.
% TODO - give references to literature justifying the preceding paragraph


Techniques and tools for verifying hyperproperties of (finite-state) hardware
\cite{CoenenFST19,FinkbeinerRS15}, as well as (infinite-state) software
systems have been developed.  For verification of software systems in
particular, \emph{Cartesian Hoare logic} (CHL), introduced in \cite{SousaD16},
extends Hoare logic in order to allow reasoning about \emph{$k$-safety
  hyperproperties}. In \cite{SousaD16} the authors give
CHL a sound and relatively complete proof system and use it
to analyse several natural $k$-safety properties of Java
programs. (Additionally, the verification algorithm is implemented in a fully automated
tool.) To formalize Cartesian Hoare Logic, \cite{SousaD16} uses a simple
imperative language, whose looping constructs are while loops with
breaks. However, extending the approach of~\cite{SousaD16} to
other constructs affecting control flow, or indeed other programming
languages, can be both highly non-trivial and time consuming.


On the other hand, there have been recent developments in the area of
\emph{language-paremetric} software verification.  \emph{Reachability logic}
(RL) \cite{RosuS12oopsla,RosuSCM13lics,StefanescuCMMSR19} is a formalism for
reasoning about partial correctness of software, in the spirit of Hoare
logics.  Being implemented in the \K{} framework \cite{KVision}, its biggest
advantage is that reachability logic is \emph{language parametric}: its proof
system can be used unchanged to reason about programs in any language, as long
as the language has a formal semantics in RL.  Therefore, researchers no
longer need to think about a particular language construct three times (once
for the operational semantics, once for axiomatic, and once for the
correspondence); additionally, a single researcher or an architect of a tool does not
need to understand both the precise (and often intricate) semantics of a
programming language, \emph{and} formal verification techniques, which makes
\emph{division of labour} possible.  Through \K{}, reachability logic has been
used to build verifiers for real-world languages, such as C (\cite{RVMatch}),
Java (\cite{StefanescuPYLR16VerifiersForAll}), JavaScript
(\cite{StefanescuPYLR16VerifiersForAll}), and EVM
(\cite{KevmVerificationTool}).

In this paper we argue that we can indeed have the best of both worlds.  We
propose a new logic called \emph{Cartesian Reachability logic (CRL)}, which
properly extends reachability logic to allow reasoning about $k$-safety
hyperproperties. Similarly to CHL, CRL has a sound and relatively complete
proof system. A major advantage of CRL against CHL is that it works with any
deterministic
language for which RL works; that is, with any deterministic language which has a RL-based
formal semantics.  However, CRL does \emph{not} extend CHL, because the two
logics give different semantics to properties of nondeterministic programs;
despite this distinction, CRL extends CHL on the deterministic fragment of the
CHL-supported language.  We elaborate on this relation in
\Cref{sec:discussion}.  We draw our inspiration from the literature on
language-independent verification of partial correctness
(\cite{RosuS12oopsla,RosuSCM13lics,StefanescuCMMSR19}) and program equivalence
(\cite{CiobacaLRR16,CiobacaLRR14}).  


\paragraph{Contributions} The approach of our paper can be summarized as follows:
\begin{itemize}
\item We propose Cartesian Reachability Logic, an extension of reachability
  logic for reasoning about k-safety properties along the lines of Cartesian
  Hoare Logic.
\item We define a \emph{language-agnostic equivalent of self-composition}
  (\cite{BartheDR11}) (Section~\ref{sec:self-composition}) and establish a relation between CRL validity of the
  original and RL validity of the composed system.
\item We give CHL sound and relatively complete \emph{proof system}
  (Section~\ref{sec:proof-system}). The proofs in this proof system can be
  translated to ordinary RL proofs of the composed system (for soundness), while
  also allowing relatively high-level reasoning about
  \emph{circular behaviour} and \emph{lockstep execution} (for ease of
  verification and simplicity of invariants).
\end{itemize}



%%% Local Variables:
%%% mode: latex
%%% TeX-master: "../main"
%%% End:

\section{Preliminaries}

\subsection{($\mu$-free) Matching Logic}

We work with a variant of matching logic described in
\cite{StefanescuCMMSR19,RosuSCM13lics}.
This particular variant of matching logic is used for reasoning about static properties of program configurations.
There exist newer and more expressive variants of matching logic (\cite{MmL, MLexplained});
we used the older variant in order to be compatible with the literature on reachability logic which uses this variant.

Matching logic \emph{formula} (aka \emph{pattern}) is a first-order logic (FOL) formula which allows terms,
over a signature $\Sigma$, with variables, as nullary predicates.
A typical example of a matching logic formula is $\varphi_{\mathit{example}}$, defined as
\begin{equation}\label{eqn:exampleMLPattern}
\texttt{<k> x--; <k><st> x} \texttt{ |-> } X\texttt{ </st>} \land (X \texttt{ >Int } 1 = \mathit{true})
\end{equation}
which, when interpreted in a model of a particular programming language,
denotes the set of program configurations in which the code \texttt{x--} is to be executed
and in which the program variable $\texttt{x}$ has a value $X$ that is greater than $1$.
In this example, the part $\texttt{<k> x--; <k><st> x} \texttt{ |-> } X\texttt{ </st>}$
is the term-as-predicate, with $X$ being the only free FOL variable.
The program variable $\texttt{x}$ is not a FOL variable, but a constant symbol from the signature of the programming language.
The subterm $\texttt{x} \texttt{ |-> } X\texttt{ </st>}$ says that the program variable $\texttt{x}$
has the value $X$, and the $X \texttt{ >Int } 1 = \mathit{true}$ part then says that the realization
of the function symbol $\_ \texttt{ >Int } \_$ returns the boolean value $\mathit{true}$ when given $X$ and $1$
as arguments.

The satisfaction relation $(M, \gamma, \rho) \vDash \varphi$ between a model $M$, a model element $\gamma \in M$,
an $M$-valuation $\rho$, and a pattern $\varphi$, is defined inductively on the structure of $\varphi$.
The definition is as in FOL; the main difference is the semantics of terms-as-predicates, which is defined by
\begin{equation*}
    (M, \gamma, \rho) \vDash t \iff \gamma = \rho(t) \text{ if t is a term}
\end{equation*}
(where $\rho(t)$ is the homomorphic extension of $\rho$ applied to the term $t$).
For example, we might have a matching logic model $M$ containing (concrete) program configurations
of a particular programming language.
One such configuration might be $\gamma_{\mathit{example}}$:
\begin{equation*}
    \texttt{<k> x--; <k><st> x} \texttt{ |-> } 3\texttt{ </st>} \, .
\end{equation*}
Then, we have that $(M, \gamma_{\mathit{example}}, \rho) \vDash \varphi_{\mathit{example}}$
for any valuation $\rho$ satisfying $\rho(X) = 3$, and we say that
$\varphi_{\mathit{example}}$ \emph{matches} $\gamma_{\mathit{example}}$ in $\rho$.


A pattern $\varphi$ is \emph{valid in $M$}, written $M \vDash \varphi$, iff $(M, \gamma, \rho) \vDash \varphi$
for every $\gamma$ and $\rho$.
We observe that validity of a structureless pattern (that is a pattern without terms-as-predicates) does not depend on the selected model element.
Also, validity of any pattern does not depend on those variables which the pattern does not mention.
A more formal treatment of matching logic is to be found in the Appendix.


\subsection{One-path Reachability Logic}
Reachability logic \cite{StefanescuCMMSR19,RosuS12oopsla} (RL) is a formalism for
defining formal semantics of programming languages,
and also for specifying and reasoning about partial correctness properties
of programs in those languages.
On the formal semantics side, a programming language is modelled as a \emph{reachability system}
$\mathcal{S} = (\mathcal{T}, S)$, where $\mathcal{T}$ is a $\Sigma$-algebra
and $S$ is a set of \emph{reachability rules} of the shape $\varphi \Rightarrow^\exists \varphi^\prime$,
where $\varphi$ and $\varphi^\prime$ are matching logic patterns over $\Sigma$.
For example, one can have a rule
\begin{equation}\label{eqn:ruleIfTrue}
    \begin{aligned}
    & \texttt{<k> if (} \mathit{true} \texttt{) }P_1\texttt{ else } P_2 \texttt{</k><st>} S \texttt{</st>} \\
    & \Rightarrow \texttt{<k> }P_1 \texttt{</k><st>} S \texttt{</st>}
    \end{aligned}
\end{equation}
saying that the \texttt{if} construct of the particular language takes the first branch ($P_1$)
whenever the condition is $\mathit{true}$.
(Typically, there would be another rules governing evaluation of the condition.)

The meaning of reachability rules is the following.
A reachability system $\mathcal{S} = (\mathcal{T}, S)$ (together with a $\Sigma$-sort $\mathit{Cfg}$)
induces
a \emph{transition system}
$(\Tcfg , \Rightarrow_{\mathcal{S}})$,
where $\gamma \Rightarrow_{\mathcal{S}} \gamma^\prime$
for $\gamma, \gamma^\prime \in \Tcfg$
iff there is some rule $\varphi \Rightarrow^\exists \varphi^\prime \in S$
and some valuation $\rho : \Var \to \mathcal{T}$ with $(\gamma, \rho) \vDash \varphi$
and $(\gamma^\prime , \rho) \vDash \varphi^\prime$.
The intuition is that when taking a transition in the resulting transition system,
some rule $\varphi \Rightarrow^\exists \varphi^\prime \in S$ is selected,
then the current configuration is pattern-matched against the rule's left-side pattern $\varphi$,
resulting in a valuation $\rho$ which is then used to instantiate the right-side $\varphi^\prime$ of the rule,
forming a new configuration.
For example, the rule in \Cref{eqn:ruleIfTrue} induces (among others) the transition
\begin{equation}\label{eqn:ruleIfTrue}
    \begin{aligned}
    & \texttt{<k> if (} \mathit{true} \texttt{) x++; else x--; </k><st>x} \texttt{ |-> } 3\texttt{</st>} \\
    & \Rightarrow_{\mathcal{S}} \texttt{<k> x++; </k><st>x} \texttt{ |-> } 3\texttt{</st>} \, .
    \end{aligned}
\end{equation}

On the partial correctness side, RL reuses the notion of a \emph{reachability rule}.
For example, one can specify that the program \texttt{while(x > 0) x--;}
can, if it terminates at all, reach a configuration
where the program variable \texttt{x} has a non-positive value
by means of the reachability rule
\begin{equation*}
    \begin{aligned}
        & \texttt{<k> while( x > 0 ) x--; </k> <st> x |-> } V \texttt{</st>} \\
        & \Rightarrow^\exists \exists V^\prime.\, \texttt{<k> . </k> x |-> } V^\prime \texttt{</st>} \land (V^\prime \texttt{ <=Int } 0 = \mathit{true})
    \end{aligned}
\end{equation*}
Assuming the language is deterministic, this is equivalent to saying that if the program terminates,
the resulting configuration will have non-positive value of \texttt{x}.
Formally, a rule of the shape $\varphi \Rightarrow^\exists \varphi^\prime$
is \emph{satisfied}
in a reachability system $\mathcal{S} = (\mathcal{T}, S)$,
written $\mathcal{S} \vDash_\RL \varphi \Rightarrow^\exists \varphi^\prime$,
iff for every $\gamma \in \Tcfg$
such that $\gamma$ terminates in $(\Tcfg, \Rightarrow_{\mathcal{S}})$
and for any valuation $\rho : \Var \to \mathcal{T}$
such that $(\gamma, \rho) \vDash \varphi$,
there exists some $\gamma^\prime \in \Tcfg$
such that
$\gamma \Rightarrow^{*}_{\mathcal{S}} \gamma^\prime$
and $(\gamma^\prime, \rho) \vDash \varphi^\prime$.


Reachability logic is equipped with a proof system which derives sequents of the shape
$\mathcal{S}, A, C \vdash_\RL \varphi \Rightarrow^\exists \varphi^\prime$;
the proof system is sound and complete: a RL claim is satisfied in $\mathcal{S}$
iff $\mathcal{S}, \emptyset, \emptyset \vdash_\RL \varphi \Rightarrow^\exists \varphi^\prime$.


\begin{remark}\label{rem:shapeOfReachabilityRules}
For simplicity, we restrict the class of reachability systems we work with to those whose reachability rules
have the shape
\begin{equation*}
    \phi \land P \Rightarrow^\exists \phi^\prime \land P^\prime
\end{equation*}
where $\phi \land P$ and $\phi^\prime \land P^\prime$ are constrained terms.
\end{remark}

\begin{remark}\label{rem:noEmptySteps}
We work only with $\epsilon$-free reachability systems.
A reachability system $(\mathcal{T}, S)$ is \emph{$\epsilon$-free}
iff for any two configurations $\sigma, \sigma^\prime \in \mathcal{T}_{\mathit{Cfg}}$, if
$\sigma \Rightarrow_{\mathcal{S}} \sigma^\prime$, then $\sigma \not = \sigma^\prime$;
\end{remark}
\section{Cartesian Reachability Logic}
In this section we introduce \emph{Cartesian Reachability logic (CRL)} - a
language-parametric logic for reasoning about $k$-safety hyperproperties.  Our
aim with CRL is to make reasoning in the style of \emph{Cartesian Hoare logic
  (CHL)}~\cite{SousaD16} available for any deterministic language for which a
reachability-logic semantics $S$ is available.  For that purpose we define the
language of CRL and its semantics, and demonstrate the logic's expressiveness
on a couple of examples.  Finally we give CRL a sound proof system, which is the
main contribution of this paper.


\subsection{Syntax and Semantics}
\label{sec:CRLsemanticsComparisonToCHL}

Cartesian reachability logic is an extension of (one-path) reachability
logic. To express k-safety properties we extend reachability rules $\varphi
\Rightarrow^{\exists} \varphi'$ to  \emph{cartesian reachability claims} of the form
\begin{equation*}
  [\varphi_1,\ldots,\varphi_k] \land P
  \Rightarrow^{c\exists} \exists \vec{Y}.\, [\varphi^\prime_1,\ldots,\varphi^\prime_k] \land P^\prime
\end{equation*}

The intuitive meaning of such a claim is as follows: there are $k$ programs
embedded into $k$ source configurations, with $i$-th source configuration matching $\varphi_i$,
and $k$ target configurations matching $\varphi'_i$. Additionally, a FOL
formula $P$ can relate the source configurations, and a FOL formula $P'$ the target
configurations. We call formulas of the form
$ \exists \vec{Y}.\,[\varphi_1,\ldots,\varphi_k] \land P$ (where $\vec{Y}$ may
be an empty vector) \emph{existentially-quantified constrained list patterns}
(ECLP).


For example, let us again consider the program
\begin{equation*}
  P \equiv \texttt{while(x > 0)\{ x--; y++; \}} \, .
\end{equation*}
Additionally, let $\xi(Q, X, Y)$ be a pattern matching those configurations of
program $Q$ where the program variable $\texttt{x}$ has value $X$ and the program
variable $\texttt{y}$ has value $Y$:
\begin{equation}\label{eqn:CQXY}
 \xi(Q, X, Y) \equiv\ \tap{$Q$}{\texttt{x}\mapsto X, \texttt{y}\mapsto Y }  \, .
\end{equation}
Then, the claim $\Omega_{\textit{mono}}$, defined as
\begin{align*}
&[\xi(P, X_1, Y_1),\xi(P, X_2, Y_2)] \land X_1 \leq X_2 \land Y_1 \leq Y_2
\\ \Rightarrow^{c\exists}\ &
\exists X^\prime_1, Y^\prime_1, X^\prime_2, Y^\prime_2.\,  [\xi(\epsilon, X^\prime_1, Y^\prime_1), \xi(\epsilon, X^\prime_2, Y^\prime_2)] \land Y^\prime_1 \leq Y^\prime_2   
\end{align*}
(where $\epsilon$ denotes the empty program)
expresses the property that the program $P$ is \emph{monotone}.
That is, when we start an execution (using the semantics of the particular language)
from some configuration $\gamma_1$ matching $\xi(P, X_1, Y_1)$
and a second execution from some configuration $\gamma_2$ matching $\xi(P, X_2, Y_2)$,
if $X_1 \leq X_2$ and $Y_1 \leq Y_2$,
we end up in configurations $\gamma_1^\prime,\gamma_2^\prime$ matching
$\xi(\epsilon, X^\prime_1, Y^\prime_1)$ and $\xi(\epsilon, X^\prime_2, Y^\prime_2)$
for some $X^\prime_1,Y^\prime_1,X^\prime_2,Y^\prime_2$
satisfying $Y^\prime_1 \leq Y^\prime_2$.
%
Note that we need the existential quantification on the right hand side of
claims to be able to talk about the new values of program variables (whereas
in CHL we have original values in the precondition, and new values in the postcondition).

We formally define the semantics of a CRL claim as follows:
\begin{definition}[CRL semantics]\label{def:opCRLsemantics}
    A claim
    \begin{equation*}
     [\varphi_1,\ldots,\varphi_k] \land P
     \Rightarrow^{c\exists} \exists \vec{Y}.\, [\varphi^\prime_1,\ldots,\varphi^\prime_k] \land P^\prime
    \end{equation*}
    is \emph{valid} in a reachability system $\mathcal{S} = (\mathcal{T}, S)$,
    written
    \begin{equation*}
        (\mathcal{T}, S) \vDash_\CRL [\varphi_1,\ldots,\varphi_k] \land P
     \Rightarrow^{c\exists} \exists \vec{Y}.\, [\varphi^\prime_1,\ldots,\varphi^\prime_k] \land P^\prime \, ,
    \end{equation*}
    iff for all configurations $\gamma_1,\ldots,\gamma_k \in \Tcfg$
    which terminate in $(\Tcfg, \Rightarrow_{\mathcal{S}})$
    and any $\mathcal{T}$-valuation $\rho$,
    whenever $(\gamma_1,\rho) \vDash \varphi_1 \land P$ and \ldots
    and $(\gamma_k,\rho) \vDash \varphi_k \land P$,
    then there exist configurations $\gamma_1^\prime,\ldots,\gamma_k^\prime \in \Tcfg$
    such that $\gamma_1 \Rightarrow^{*}_{\mathcal{S}} \gamma_1^\prime$
    and \ldots and $\gamma_k \Rightarrow^{*}_{\mathcal{S}} \gamma_k^\prime$,
    and there also exists an $\mathcal{T}$-valuation $\rho^\prime$
    satisfying $\rho(v) = \rho^\prime(v)$ for any $v \in \mathit{Var} \setminus \vec{Y}$,
    and
    $(\gamma_1^\prime,\rho^\prime) \vDash \varphi^\prime_1 \land P^\prime$ and \ldots and $(\gamma_k^\prime, \rho^\prime) \vDash \varphi^\prime_k \land P^\prime$.
\end{definition}

%\subsection{Comparison to CHL}\label{sec:CRLsemanticsComparisonToCHL}

As can be seen from the example above, CRL is more verbose then CHL. This is
partly because of the need to specify patterns matching the whole program
configurations, and partly because of the need to existentially quantify those
variables on the right side whose value is not determined by the left side. To
alleviate this problem, we introduce the following notation, which we inherit
from RL:

\paragraph{Notation}
Variables whose names start with a question mark are implicitly considered
existentially quantified on the right side.  Also, underscore is used to
denote anonymous variables, whose values we are not interested in.  For
example, we can write the claim $\Omega_{\textit{mono}}$ as
\begin{align*}
[\xi(P, X_1, Y_1),\xi(P, X_2, Y_2)] \land X_1 \leq X_2 \land Y_1 \leq Y_2
 \Rightarrow^{c\exists} [\xi(\epsilon, ?\_, ?Y_1), \xi(\epsilon, ?\_, ?Y_2)] \land ?Y_1 \leq ?Y_2 \, .
\end{align*}

% TODO move this to a more appropriate place!
A deeper distinction is that in the CRL semantics we existentially quantify
over reachable configurations, while in CHL, target states are quantified
universally. However, this distinction has no effect when working with
deterministic languages.\traian{People might be asking why? I'm also asking why.}

\subsection{CRL as an extension of Reachability Logic}

We would like to point out that one cannot handle k-safety properties directly
in RL, by simply replacing a CRL claim with $k$ components by $k$ RL
claims. The reason is that in CRL formulas, one can relate variables from different components. 


%Consider, for example, a reachability system $S_{\mathit{IMP}}$ representing a simple imperative language,
%and the same program as in \Cref{eqn:CounterProgram}; that is, let
%\begin{equation*}
%  P \equiv \texttt{while(x > 0)\{ y++; x--;\}} \, .
%\end{equation*}
%We can let $C(Q, X,Y)$ represent a configuration of a program $Q$ where the program variable $\texttt{x}$ has the value $X$
%and the program variable $\texttt{y}$ has the value $Y$;
%for example,
%\begin{equation*}
% C(Q, X, Y) \equiv \texttt{<k>} Q \texttt{</k><st>(x |-> } X \texttt{)(y |-> } Y \texttt{)</st>}    \, .
%\end{equation*}
%Then, the proposition
%\begin{align*}
% S_{\mathit{IMP}} \vDash_\CRL
%&[C(P, X_1, Y_1),C(P, X_2, Y_2)] \land X_1 \leq X_2
%\\ \Rightarrow^{c\exists} &
%\exists X^\prime_1, Y^\prime_1, X^\prime_2, Y^\prime_2.\,  [C(\epsilon, X^\prime_1, Y^\prime_1), C(\epsilon, X^\prime_2, Y^\prime_2)] \land Y^\prime_1 \leq Y^\prime_2   
%\end{align*}
%holds iff the program $P$ is monotone (when considering the variable $x$ to be an input, and $y$ to be an output; $\epsilon$ represents an empty program).
In CRL, one can localize the ``global'' constraints; for example, the claim $\Omega_{\textit{mono}}$
is equivalent to\traian{Here and below you seem to have forgotten $Y_1 \leq Y_2$}
\begin{align*}
[\xi(P, X_1, Y_1),\xi(P, X_2, Y_2) \land X_1 \leq X_2] 
\Rightarrow^{c\exists}
[\xi(\epsilon, ?\_, ?Y_1), \xi(\epsilon, ?\_, ?Y_2) \land ?Y_1 \leq ?Y_2] \, .
\end{align*}
However, if one were to ``split'' the CRL claim into two, the resulting claims might express
a different property than monotonicity.
For example, the two claims
\begin{align}
& \xi(Q, X_1, Y_1) & \Rightarrow^{\exists} \quad & \xi(\epsilon, ?X_1, ?Y_1) \\
& \xi(Q, X_2, Y_2) \land X_1 \leq X_2 & \Rightarrow^{\exists} \quad  & \xi(\epsilon, ?X_2, ?Y_2) \land ?Y_1 \leq ?Y_2
\end{align}
hold for any reasonable program $Q$ (meaning that $Q$ either executes fully or diverges),
because $?Y_1$ in the second claim is unrelated to $?Y_1$ in the first claim and thus one can set it to
the value of $?Y_2$.
On the other hand, if in the second claim we renamed $?Y_1$ to $Y_1^\prime$ (without a question mark),
the claim would require that $?Y_2$ is greater than or equal to \emph{any} integer (because $Y_1^\prime$ is not present in the left side),
which clearly cannot hold.

On the other hand, CRL is an extension of RL in the following natural sense:
If we restrict configuration lists in CRL claims to contain a single
configuration each, validity of CRL and RL claims neatly coincide
(the proof of the follwong proposition can be found in Appendix~\ref{app:CRL}):


%\begin{remark}
%\Cref{def:opCRLsemantics} extends \Cref{def:oprlSemantics} of \Cref{def:basics}:
%if we fix $k=1$, then
\begin{proposition}\label{prop:opCRLopRL}
  $ (\mathcal{T}, S) \vDash_\CRL [\varphi] \land \top \Rightarrow^{c\exists}
  [\varphi^\prime] \land \top \iff (\mathcal{T}, S) \vDash_\RL \varphi
  \Rightarrow^{\exists} \varphi^\prime \, .  $
\end{proposition}
%\end{remark}

%\begin{definition}[All-Path Cartesian Reachability Rule]\label{def:apCRLsemantics}
%An all-path Cartesian reachability rule
%$(\varphi_1,\ldots,\varphi_k) \land \varphi \Rightarrow^{c\forall} (\psi_1,\ldots,\psi_k) \land \psi$
%of arity $k$
%is \emph{valid} in a reachability system $\mathcal{S} = (\mathcal{T}, S)$,
%written
%$\mathcal{S} \vDash_\CRL (\varphi_1,\ldots,\varphi_k) \land \varphi \Rightarrow^{c\forall}
%(\psi_1,\ldots,\psi_k) \land \psi$,
%iff for all configurations $\sigma_1,\ldots,\sigma_k \in \Tcfg$ \traian{Why $\sigma$ instead of $\gamma$?}
%and any $\mathcal{T}$-valuation $\rho$,
%whenever $(\sigma_1, \rho) \vDash \varphi_1 \land \varphi$ and \ldots
%and $(\sigma_k, \rho) \vDash \varphi_k \land \varphi$,
%then for every $k$-tuple of complete paths $(\pi_1, \ldots, \pi_k)$
%such that
%$\sigma_1 = \pi_1(0) \land \ldots \land \sigma_k = \pi_k(0)$,
%there exist indices $i_1, \ldots, i_k$ such that
%$(\pi_1(i_1), \rho) \vDash \varphi_1 \land \psi$ and \ldots and $(\pi_k(i_k), \rho) \vDash \varphi_k \land \psi$.
%\end{definition}

\subsection{A language-independent alternative to self-composition}
\label{sec:self-composition}


%Now we present a novel, general technique called \emph{star extension} that is reminiscent of.

% To prove the soundness of our proof system for CRL, it would be advantegous to
% be able to apply the 

Before presenting a proof system for CRL, we need to first discuss a concept
which is crucial to proving its soundness. Self-composition~\cite{BartheDR11}
is a technique where a program $P$ together with a $k$-safety hyperproperty is
reduced to a sequential composition of $P$ with itself (with renamed
variables) together with a safety property.  This technique allows one to use
tools and techniques for verification of safety properties to perform
verification of $k$-safety hyperproperties.  The challenge here is to
generalize self-composition to work with any deterministic language, even if
we do not know in advance how the language implements sequential composition,
if at all.

To address this issue, we present a novel, general technique which we call
\emph{star extension}. The main idea is to transform a CRL claim into a RL
claim over an extended reachability system, whose configurations are lists of
configurations of the original system.  The function $\mathit{flatten}$ then
converts an ECLP (which contains a meta-level list of matching logic patterns)
into a matching logic formula (containing an object-level list) that matches
lists inside the model.  The transformation is quite straightforward but
technical, and we refer an interested reader to the
\Cref{app:CRLandRLcorrespondence}; here we state the main theorem:
\begin{theorem}\label{thm:CRLandRLcorrespondence}
  There exist a function $\_^*$ on matching logic signatures,
  a (equally-named) function $\_^*$ from reachability systems over $\Sigma$ to reachability systems over $\Sigma^*$,
  and a function $\mathit{flatten}$ from ECLPs over $\Sigma$ to matching logic $\Sigma^*$-formulas,
  such that
  \begin{equation*}
  \mathcal{S} \vDash_{\CRL} \Psi \Rightarrow^{c\exists} \Psi^\prime
    \iff \mathcal{S}^* \vDash_\RL \mathit{flatten}(\Psi) \Rightarrow^{c\exists} \mathit{flatten}(\Psi^\prime)
  \end{equation*}
\end{theorem}


The price paid for self-composition is that the property of the self-composed program is often hard to reason about.
Therefore, in~\cite{SousaD16}, the authors do not apply self-composition directly, but only use its soundness to justify
their technique - namely, the soundness of their proof system, which avoids explicit construction
of the self-composed programs.
We use the star extension for the same purpose.



\subsection{Proof System for CRL}
\label{sec:proof-system}

%We give Cartesian Reachability Logic a proof system to faciliate mechanical reasoning.
%One may ask the question, ``why give CRL a new proof system if one can perform the same reasoning
%by means of \Cref{thm:CRLandRLcorrespondence} and the existing proof system of reachability logic?''.
%The answer is the following.
%When one reduces a CRL goal into RL using \Cref{thm:CRLandRLcorrespondence},
%the function $\mathit{flatten}$ appears in the goal.
%To perform reasoning using the RL proof system, one then has to either simplify the goal
%by unfolding the definition of $\mathit{flatten}$,
%or use (and prove) some helper lemmas about the effect of applying RL proof rules
%%and other frequently used steps
%on RL goals containing $\mathit{flatten}$.
%In the first case one lowers the abstraction level and have to reason about matching logic formulas
%containing translations of other matching logic formulas into FOL, which then makes RL reasoning more complex.
%In the second case, however, one may end up proving lemmas which, when combined, result in a proof system for CRL.
%Indeed, the soundness of our proof system (\Cref{fig:CRLproofsystem})
%is established by a (meta-)proof which constructs a RL proof from a CRL one (\Cref{lem:CRLalmostSoundness}).


We are now ready to give Cartesian Reachability Logic a proof system to
facilitate mechanical reasoning.  While the intuition behind the semantics of
CRL is similar to that of CHL, with only a few minor differences, coming up
with a proof system for CRL is not straightforward.  One could attempt to reuse the proof
system of CHL and modify it \emph{somehow} to be independent of the particular
programming language.  Since many CHL rules (e.g., its \texttt{If} rule shown
in \Cref{chlRule:If}) are simply Hoare logic rules acting on a particular
component of a tuple of formulas (that is, symbolic states), one could for
example lift rules of reachability logic (RL) to the tuple-context and be
done.  That would indeed work, and the resulting proof system might even be
complete.

However, the distinguishing feature of Cartesian Hoare Logic is \emph{not} its
completeness, but its ability to simplify reasoning by performing lockstep
execution of loops, because that can \textquote[\cite{SousaD16}]{greatly
  simplify the verification task (e.g., by requiring simpler invariants)}.  We
cannot just lift RL rules to tuple-context. And it is not entirely obvious how
to support lockstep reasoning involving constructs with repetitive behaviour: in
order to support \texttt{while} cycles with \texttt{break}, Cartesian Hoare logic itself uses
five additional fairly complex rules (besides the lifted Hoare logic rules) 
that need to reflect the precise semantics of this construct.  The task for
us is even harder, since we do not know in advance what types of constructs with
repetitive behaviour does the supplied language support.

Yet, we provide a single proof system consisting of only eight rules (Figure~\ref{fig:CRLproofsystem}) which enables lockstep reasoning about such constructs.
The proof system derives claims of the shape
\begin{equation*}
(\mathcal{T}, S) \vdash_\CRL \Phi \Downarrow_{C,E} \Psi
\end{equation*}
where $\Phi$ is of the shape $(\varphi_1, \ldots, \varphi_k) \land P$ and
$\Psi$ is of the shape
$\exists \vec{Y}.\, (\varphi^\prime_1, \ldots, \varphi^\prime_k) \land
P^\prime$.  One can think about $\Phi$ as representing a \emph{premise}, while
$\Psi$, which propagates through the proof rules unchanged, as representing a
\emph{conclusion}.  For each $i$, $\varphi_i$ and $\varphi^\prime_i$ are matching logic
patterns representing a particular component, and $P$ and $P^\prime$ are FOL
formulas (\emph{global constraints}) relating variables from different
components.  The sets $C$ and $E$ contain \emph{cutpoints} and \emph{enabled
  cutpoints}, respectively.  Together, they implement the concept of an
\emph{invariant} relating different components.  In particular, the set $C$
represents the invariants that were postulated \emph{right now}, while the set
$E$ represents those that were postulated in past and are ready to be used.
Initially, the proof search starts with $E = C = \emptyset$.

\begin{figure}
    \centering
    \begin{align*}
    & \prftree[l]{Reflexivity}{(\mathcal{T}, S) \vdash_\CRL \Psi \Downarrow_{\emptyset,E} \Psi}
    \end{align*}
    \begin{align*}
    & \prftree[l]{Axiom}{\Psi \in E}{(\mathcal{T}, S) \vdash_\CRL \Psi \Downarrow_{C,E} \Psi^\prime}
    \end{align*}
    \begin{align*}
    & \prftree[l]{Reduce}
      {(\mathcal{T}^*, S^* \cup \mathit{flatten}^\exists(E, \Psi^\prime)), \emptyset \vdash_\RL
        \mathit{flatten}^\exists(\Psi, \Psi^\prime) }
      {(\mathcal{T}, S) \vdash_\CRL \Psi \Downarrow_{C,E} \Psi^\prime}
    \end{align*}
    \begin{align*}
    & \prftree[l]{Case}
    { \prfStackPremises
      {(\mathcal{T}, S) \vdash_\CRL [\varphi_1, \ldots, \varphi_{i-1}, \varphi_i, \varphi_{i+1}, \ldots, \varphi_k] \land P^\prime \Downarrow_{C, E} \Psi^\prime }
      {(\mathcal{T}, S) \vdash_\CRL [\varphi_1, \ldots, \varphi_{i-1}, \psi_i, \varphi_{i+1}, \ldots, \varphi_k] \land P^\prime \Downarrow_{C, E} \Psi^\prime }
    }
    {(\mathcal{T}, S) \vdash_\CRL [\varphi_1, \ldots, \varphi_{i-1}, (\varphi_i \lor \psi_i), \varphi_{i+1}, \ldots, \varphi_k] \land P^\prime \Downarrow_{C, E} \Psi^\prime}
    \end{align*}
    \begin{align*}
    & \prftree[l]{Step}
    { \prfStackPremises
       {\varphi \Rightarrow^\exists \varphi^\prime \in S}
       {\mathcal{T} \vDash_\ML \varphi_i \leftrightarrow \varphi \land P^\prime}
       {P^\prime \mbox{ is a FOL formula}}
       {  (\mathcal{T}, S) \vdash_\CRL [\varphi_1, \ldots, \varphi_{i-1}, \varphi^\prime \land P^\prime, \varphi_{i+1}, \ldots, \varphi_k]
          \land P
          \Downarrow_{\emptyset, (C \cup E)} \Psi^\prime
      }
    }
    {(\mathcal{T}, S) \vdash_\CRL [\varphi_1, \ldots, \varphi_{i-1}, \varphi_i, \varphi_{i+1}, \ldots, \varphi_k] \land P \Downarrow_{C, E} \Psi^\prime}
    \end{align*}
    \begin{align*}
    & \prftree[l]{Circularity}
      { (\mathcal{T}, S) \vdash_\CRL \Psi \Downarrow_{C \cup \{ \Psi \} , E} \Psi^\prime}
      { (\mathcal{T}, S) \vdash_\CRL \Psi \Downarrow_{C, E} \Psi^\prime}
    \end{align*}
    \begin{align*}
    & \prftree[l]{Conseq}
      { \prfStackPremises
        { (\mathcal{T}, S) \vdash_\CRL \Phi^\prime \Downarrow_{C, E} \Psi^\prime}
        { \mathcal{T}^* \vDash_\ML \mathit{flatten}(\Phi) \rightarrow \mathit{flatten}(\Phi^\prime) }
      }
      { (\mathcal{T}, S) \vdash_\CRL \Phi \Downarrow_{C, E} \Psi^\prime}
    \end{align*}
%    \begin{align*}
%    & \prftree[l]{Conseq2}
%      { \prfStackPremises
%        { (\mathcal{T}, S) \vdash_\CRL (\varphi_1, \ldots, \varphi_k) \land P \Downarrow_{C, E} \Psi^\prime}
%        { \mathcal{T} \vDash_\ML \varphi_i \leftrightarrow \psi_i }
%      }
%      { (\mathcal{T}, S) \vdash_\CRL (\varphi_1^\prime, \ldots, \varphi_{i-1}, \varphi_{i}, \varphi_{i+1}, \ldots, \varphi_k^\prime) \land P^\prime \Downarrow_{C, E} \Psi^\prime}
%    \end{align*}

    %\begin{align*}
    %& \prftree[l]{Abstract}
      %{ \prfStackPremises
        %{ X \not\in \mathit{FV}(\Psi^\prime, \varphi_1, \ldots, \varphi_{i-1}, \varphi_{i+1}, \ldots,\varphi_k)
        %}
        %{(\mathcal{T}, S) \vdash_\CRL (\varphi_1, \ldots, \varphi_{i-1}, \varphi_i, \varphi_{i+1}, \ldots, \varphi_k) \land P \Downarrow_{C, E} \Psi^\prime}
      %}
      %{(\mathcal{T}, S) \vdash_\CRL (\varphi_1, \ldots, \varphi_{i-1}, \exists X.\, \varphi_i, \varphi_{i+1}, \ldots, \varphi_k) \land P \Downarrow_{C, E} \Psi^\prime}
    %\end{align*}
    
    \begin{align*}
    & \prftree[l]{Abstract}
      { \prfStackPremises
        { X \not\in \mathit{FV}(\Psi^\prime)
        }
        {(\mathcal{T}, S) \vdash_\CRL \exists \vec{Y}.\, [\varphi_1, \ldots, \varphi_k] \land P \Downarrow_{C, E} \Psi^\prime}
      }
      {(\mathcal{T}, S) \vdash_\CRL \exists X, \vec{Y}.\, [\varphi_1, \ldots, \varphi_k] \land P \Downarrow_{C, E} \Psi^\prime}
    \end{align*}
    \caption{A proof system for CRL}
    \label{fig:CRLproofsystem}
\end{figure}

The rules of our system are fairly simple, and general in the sense that none of them has the semantics of any particular language construct
hard-wired into it.
We now explain the proof rules of CRL (shown in~\Cref{fig:CRLproofsystem}) one by one.
\begin{itemize}
  \item The Circularity rule is a key rule which allows lockstep reasoning about arbitrary program constructs.
        It allows the user to postulate validity of the current claim by means of adding the \emph{current premise}
        into the set of \emph{cutpoints}, from which a $k$-tuple of program configurations satisfying the postcondition
        is claimed to be reachable. Once progress is made (by means of the Step rule), the added cutpoints are
        \emph{enabled} and can be used to finish the proof using the Axiom rule.
  \item The Step rule performs symbolic execution on a selected component $i$ (represented by $\varphi_i$)
        using the semantic reachability rule $\varphi \Rightarrow^\exists \varphi^\prime \in S$.
        For this rule to apply, its left side ($\varphi$) has to match all the program configurations matching $\varphi_i$.
        Therefore, the rule decomposes $\varphi_i$ into $\varphi$ and an additional constraint $P^\prime$,
        which can be thought of as a part of \emph{path condition} that is \emph{local} to the component $i$.
        This local path condition $P^\prime$ is then used to constrain the right-side $\varphi^\prime$ of the selected rule.
        This proof rule also enables the cutpoints from $C$ by adding them to $E$.
  \item The Axiom rule uses an enabled cutpoint to finish the proof.
  \item The Reflexivity rule can be used to finish a proof when the premise corresponds to the conclusion.
  \item The Case rule implements case analysis on a selected component $i$.
  \item The Conseq rule is used to weaken (or generalize) the premise. It can also be used
        to propagate information between components and the global constraint.
        %TODO explain flatten
  \item The Abstract rule can be used to remove existential quantifiers from the premise.
        Intuitively, this corresponds to a proof step in first-order logic that replaces
        an existential quantifier on the left side of an implication
        with a universal quantifier over the implication, assuming that the variable
        bound by the existential quantifier does not occur free in the right side.
        In our setting, the typical way of \emph{obtaining} a top-level existential quantifier in the premise
        is by means of the Conseq rule.
  \item The Reduce rule is a way to get completeness into our proof system:
        it reduces the goal to reachability logic reasoning.
        This rule also provides a way to prove properties that do not benefit
        from lockstep reasoning.
\end{itemize}

Soundness of the proof system, as stated by the following theorem, is the main
technical result of this paper.

\begin{theorem}[Proof system soundness]\label{thm:proofsystemSoundness}
\begin{equation*}
    (\mathcal{T}, S) \vdash_\CRL \Psi \Downarrow_{\emptyset,\emptyset} \Psi^\prime \implies
    (\mathcal{T}, S) \vDash_{\CRL} \Psi \Rightarrow^{c\exists} \Psi^\prime
\end{equation*}
\end{theorem}

In order to  prove this theorem, we will need (beside
Theorem~\ref{thm:CRLandRLcorrespondence}, which we discuss in previous
section) a technical lemma stating that one can generate an RL proof on a star-extended system
from a CRL proof. This lemma is the second major component of the soundness
proof of CRL and its proof can be found in \Cref{app:crlsoundness}.
\begin{lemma}\label{lem:CRLalmostSoundness}
    \begin{align*}
        & (\mathcal{T}, S) \vdash_\CRL \Psi \Downarrow_{C,E} \Psi^\prime \implies \\
        &
        (\mathcal{T}^*, S^* \cup \mathit{flatten}^\exists(E, \Psi^\prime)), \mathit{flatten}^\exists(C, \Psi^\prime) \vdash_\RL
          \mathit{flatten}^\exists(\Psi, \Psi^\prime) 
    \end{align*}
\end{lemma}

\noindent With all the technical tools in place, the proof itself is a straightforward affair:
\begin{proof}[Proof of \Cref{thm:proofsystemSoundness}]
Assume $(\mathcal{T}, S) \vdash_\CRL \Psi \Downarrow_{\emptyset,\emptyset} \Psi^\prime$.
By \Cref{lem:CRLalmostSoundness}, we have $(\mathcal{T}, S) \vdash_\RL \mathit{flatten}^\exists(\Psi, \Psi^\prime)$.
By soundness of reachability logic, we have $(\mathcal{T}, S) \vDash_\RL \mathit{flatten}^\exists(\Psi, \Psi^\prime)$.
By \Cref{thm:CRLandRLcorrespondence}, 
we have $(\mathcal{T}, S) \vDash_{\CRL} \Psi \Rightarrow^{c\exists} \Psi^\prime$ and we are done.
\end{proof}

% \begin{figure}
%   \begin{align*}
%     & \prftree[l]{GenWithCirc}
%       { \prfStackPremises
%         { \vec{X} = \mathit{FV}(\Phi) \setminus \mathit{FV}(\Psi^\prime)
%         }
%         {(\mathcal{T}, S) \vdash_\CRL \Phi \Downarrow_{C \cup \{ \exists \vec{X}.\, \Phi  \}, E} \Psi^\prime}
%       }
%       {(\mathcal{T}, S) \vdash_\CRL \Phi \Downarrow_{C, E} \Psi^\prime}
%   \end{align*}

%   \begin{align*}
%     & \prftree[l]{ImplConclusion}
%       {\mathcal{T}^* \vDash_\ML \mathit{flatten}(\Phi) \rightarrow \mathit{flatten}(\Psi^\prime)}
%       {(\mathcal{T}, S) \vdash_\CRL \Phi \Downarrow_{C, E} \Psi^\prime}
%   \end{align*}

%   \begin{align*}
%     & \prftree[l]{ImplEnabled}
%       { \prfStackPremises
%         { \Phi^\prime \in E }
%         {\mathcal{T}^* \vDash_\ML \mathit{flatten}(\Phi) \rightarrow \mathit{flatten}(\Phi^\prime)}
%       }
%       {(\mathcal{T}, S) \vdash_\CRL \Phi \Downarrow_{C, E} \Psi^\prime}
%   \end{align*}

%   \begin{align*}
%     & \prftree[l]{Contradiction}
%       {\mathcal{T}^* \vDash_\ML \mathit{flatten}(\Phi) \rightarrow \bot }
%       {(\mathcal{T}, S) \vdash_\CRL \Phi \Downarrow_{C, E} \Psi^\prime}
%   \end{align*}

%   \caption{Selected derived rules}
%   \label{fig:CRLderivedRules}
% \end{figure}

The proof system, beside being sound, is also \emph{relatively complete} (with respect to an oracle deciding validity
in the underlying matching logic model $\mathcal{T}$).
\begin{theorem}[Relative completeness]\label{thm:relativeCompleteness}
  \begin{equation*}
      (\mathcal{T}, S) \vDash_{\CRL} \Psi \Rightarrow^{c\exists} \Psi^\prime \implies
      (\mathcal{T}, S) \vdash_\CRL \Psi \Downarrow_{\emptyset,\emptyset} \Psi^\prime
  \end{equation*}
  \end{theorem}
  \begin{proof}[Proof of \Cref{thm:relativeCompleteness}]
  Assume $(\mathcal{T}, S) \vDash_{\CRL} \Psi \Rightarrow^{c\exists} \Psi^\prime$.
  By \Cref{thm:CRLandRLcorrespondence}, 
  we obtain
  \begin{equation*}
    (\mathcal{T}^*, S^*) \vDash_\RL
    \mathit{flatten}^\exists(\Psi, \Psi^\prime) \, .
  \end{equation*}
  By relative completeness of reachability logic, we obtain
  \begin{equation*}
    (\mathcal{T}^*, S^*) \vdash_\RL
    \mathit{flatten}^\exists(\Psi, \Psi^\prime) \, ,
  \end{equation*}
  and we conclude the proof using the Inherit rule.
  Note that to apply relative completeness of RL, we need have an oracle for deciding validity in the extended model.
  A construction of such oracle from the oracle for deciding validity in $\mathcal{T}$ is in \Cref{app:completeness}.
  \end{proof}
Our completeness result is similar to the completeness result of CHL in the sense that the completeness
does not involve features used for lockstep reasoning.
It would be interesting to investigate whether a proof system without the Reduce rule would still be complete;
we leave this for a future work.

\section{An example proof involving lockstep reasoning}

We now present an example proof sketch using our proof system; the proof involves lockstep reasoning.
To ease the notation, we write simply $t_b$ instead of $t_b = \mathit{true}$ for
boolean-sorted side conditions $t_b$.
Consider again the claim $\Omega_{\mathit{mono}}$ from \Cref{sec:CRLsemanticsComparisonToCHL}:
\begin{align*}
  [\xi(P, X_1, Y_1),\xi(P, X_2, Y_2)] \land X_1 \leq_{\mathit{Int}} X_2 \land Y_1 \leq_{\mathit{Int}} Y_2
   \Rightarrow^{c\exists} [\xi(\epsilon, ?\_, ?Y_1), \xi(\epsilon, ?\_, ?Y_2)] \land ?Y_1 \leq_{\mathit{Int}} ?Y_2 \, .
\end{align*}
Let $\Psi^\prime_{\mathit{mono}}$ denote the right side of the $\Rightarrow^{c\exists}$ above.
We want to prove
\begin{align*}
  \mathcal{S}_{\mathit{imp}} \vdash_\CRL [\xi(P, X_1, Y_1),\xi(P, X_2, Y_2)] \land X_1 \leq_{\mathit{Int}} X_2 \land Y_1 \leq_{\mathit{Int}} Y_2
  \Downarrow_{\emptyset, \emptyset} \Psi^\prime_{\mathit{mono}} \, .
\end{align*}
To do so, we first want to add the current premise as a cutpoint.
However, we need to make the cutpoint more general than the current claim, as we will see later in the proof.
Therefore, we first apply the Conseq rule to change the goal to\traian{Why not using ? variables}
\begin{align*}
  \mathcal{S}_{\mathit{imp}} \vdash_\CRL \exists X_1,Y_1,X_2,Y_2.\, [\xi(P, X_1, Y_1),\xi(P, X_2, Y_2)] \land X_1 \leq_{\mathit{Int}} X_2 \land Y_1 \leq_{\mathit{Int}} Y_2
  \Downarrow_{\emptyset, \emptyset} \Psi^\prime_{\mathit{mono}} \, ,
\end{align*}
then apply the Circularity rule, followed by application of the Abstract rule, which basically changes the goal
back, except that now we have a general cutpoint. The goal is now
\begin{align*}
  \mathcal{S}_{\mathit{imp}} \vdash_\CRL [\xi(P, X_1, Y_1),\xi(P, X_2, Y_2)] \land X_1 \leq X_2 \land Y_1 \leq Y_2
  \Downarrow_{\Phi_{\mathit{circ}}, \emptyset} \Psi^\prime_{\mathit{mono}} \, ,
\end{align*}
where
\begin{align*}
  \Phi_{\mathit{circ}} \equiv \mathcal{S}_{\mathit{imp}} \vdash_\CRL \exists X_1,Y_1,X_2,Y_2.\, [\xi(P, X_1, Y_1),\xi(P, X_2, Y_2)] \land X_1 \leq_{\mathit{Int}} X_2 \land Y_1 \leq_{\mathit{Int}} Y_2 \, .
\end{align*}
Now we perform symbolic execution on both components, using repeated applications of the Step rule,
enabling the circularities.
The exact details depend on the exact rules of $\mathcal{S}_{\mathit{imp}}$, but assuming that the semantics of
the \texttt{while} statement is defined by unrolling into the \texttt{if} statement,
we end up with a goal like\traian{you need something like $\{\texttt{\{y++;x--;\};}P\}$}
\begin{align*}
  & \mathcal{S}_{\mathit{imp}} \vdash_\CRL
  [ \xi(\texttt{if (}X_1 >_{\mathit{Int}} 0 \texttt{)\{y++;x--;\};}P, X_1, Y_1),  
    \xi(\texttt{if (}X_2 >_{\mathit{Int}} 0 \texttt{)\{y++;x--;\};}P, X_2, Y_2) ] \\
  & \land X_1 \leq_{\mathit{Int}} X_2 \land Y_1 \leq_{\mathit{Int}} Y_2
  \Downarrow_{\emptyset, \Phi_{\mathit{circ}}} \Psi^\prime_{\mathit{mono}} \, ,
\end{align*}
Now we want to perform case analysis. To do so, we first use the Consequence rule to
obtain disjunctions of patterns at the respective components, and repeatedly apply the Case rule, leading to four goals
(the differences are shown in {\color{blue}blue}):
\begin{equation}\label{ex:case1}
\begin{aligned}
  & \mathcal{S}_{\mathit{imp}} \vdash_\CRL
  [ \xi(\texttt{if (}{\color{blue}\mathit{true}}\texttt{)\{y++;x--;\};}P, X_1, Y_1),  
    \xi(\texttt{if (}{\color{blue}\mathit{true}}\texttt{)\{y++;x--;\};}P, X_2, Y_2) ] \\
  & \land X_1 \leq_{\mathit{Int}} X_2 \land Y_1 \leq_{\mathit{Int}} Y_2 \land {\color{blue} X_1 >_{\mathit{Int}} 0 \land X_2 >_{\mathit{Int}} 0 }
  \Downarrow_{\emptyset, \Phi_{\mathit{circ}}} \Psi^\prime_{\mathit{mono}} \, ,
\end{aligned}
\end{equation}
\begin{equation}\label{ex:case2}
\begin{aligned}
  & \mathcal{S}_{\mathit{imp}} \vdash_\CRL
  [ \xi(\texttt{if (}{\color{blue}\mathit{true}}\texttt{)\{y++;x--;\};P}, X_1, Y_1),  
    \xi(\texttt{if (}{\color{blue}\mathit{false}}\texttt{)\{y++;x--;\};P}, X_2, Y_2) ] \\
  & \land X_1 \leq_{\mathit{Int}} X_2 \land Y_1 \leq_{\mathit{Int}} Y_2 \land {\color{blue} X_1 >_{\mathit{Int}} 0 \land X_2 \leq_{\mathit{Int}} 0 }
  \Downarrow_{\emptyset, \Phi_{\mathit{circ}}} \Psi^\prime_{\mathit{mono}} \, ,
\end{aligned}
\end{equation}
\begin{equation}\label{ex:case3}
\begin{aligned}
  & \mathcal{S}_{\mathit{imp}} \vdash_\CRL
  [ \xi(\texttt{if (}{\color{blue}\mathit{false}}\texttt{)\{y++;x--;\};P}, X_1, Y_1),  
    \xi(\texttt{if (}{\color{blue}\mathit{true}}\texttt{)\{y++;x--;\};P}, X_2, Y_2) ] \\
  & \land X_1 \leq_{\mathit{Int}} X_2 \land Y_1 \leq_{\mathit{Int}} Y_2 \land {\color{blue} X_1 \leq_{\mathit{Int}} 0 \land X_2 >_{\mathit{Int}} 0 }
  \Downarrow_{\emptyset, \Phi_{\mathit{circ}}} \Psi^\prime_{\mathit{mono}} \, ,
\end{aligned}
\end{equation}
\begin{equation}\label{ex:case4}
\begin{aligned}
  & \mathcal{S}_{\mathit{imp}} \vdash_\CRL
  [ \xi(\texttt{if (}{\color{blue}\mathit{false}}\texttt{)\{y++;x--;\};P}, X_1, Y_1) ,  
    \xi(\texttt{if (}{\color{blue}\mathit{false}}\texttt{)\{y++;x--;\};P}, X_2, Y_2) ] \\
  & \land X_1 \leq_{\mathit{Int}} X_2 \land Y_1 \leq_{\mathit{Int}} Y_2 \land {\color{blue} X_1 \leq_{\mathit{Int}} 0 \land X_2 \leq_{\mathit{Int}} 0 }
  \Downarrow_{\emptyset, \Phi_{\mathit{circ}}} \Psi^\prime_{\mathit{mono}} \, .
\end{aligned}
\end{equation}
\begin{itemize}
\item The case in \Cref{ex:case4} represents the situation when both loops have finished their execution.
We can solve this case by symbolically executing both programs (using the Step rule) to the end.
It is easy to see that then the premise implies the conclusion; therefore, we finish this case
by generalizing the premise (using the Conseq rule) to be exactly the conclusion, and then applying
the Reflexivity rule.
\item The case in \Cref{ex:case2} represents the situation when the first loop continues execution and the second is finished.
      This can never happen - we see that the side condition is contradictory.
      We finish this case using the Conseq rule followed by the Reflexivity rule (since a contradiction implies anything).
\item The case in \Cref{ex:case3} represents the complementary situation - the first loop has finished its execution
      and the second loop continues.
      This requires inventing an invariant capturing the idea that the execution of the second loop can only increase
      the difference between the values of \texttt{y}.
      We can Reduce this subgoal to simple RL reasoning. We could also prove this case without Reduce,
      using the other rules, but lockstep reasoning does not help there.
\item The case in \Cref{ex:case1} is the one when we utilize the ability to perform lockstep reasoning.
      We symbolically execute both components until their program parts become the \texttt{while} loops again;
      that is, $P$.
      The goal is now
      \begin{align*}
        & \mathcal{S}_{\mathit{imp}} \vdash_\CRL
        [ \xi(P, X_1 -_{\mathit{Int}} 1, Y_1 +_{\mathit{Int}} 1 ) ,  
          \xi(P, X_2 -_{\mathit{Int}} 1, Y_2 +_{\mathit{Int}} 1) ] \\
        & \land X_1 \leq_{\mathit{Int}} X_2 \land Y_1 \leq_{\mathit{Int}} Y_2 \land {X_1 \leq_{\mathit{Int}} 0 \land X_2 \leq_{\mathit{Int}} 0 }
        \Downarrow_{\emptyset, \Phi_{\mathit{circ}}} \Psi^\prime_{\mathit{mono}} \, .
      \end{align*}
      which implies the circularity $\Phi_{\mathit{circ}}$ which is already enabled.
      We can therefore conclude the proof using Conseq and Axiom.
      %their program parts
      %become $\xi(P, X_1, Y_1)$ and $\xi(P, X_2, Y_2)$, respectively.
\end{itemize}
We observe that the proof is rather low-level.
On the other hand, the proof system itself is very simple.

%%% Local Variables:
%%% mode: latex
%%% TeX-master: "../main"
%%% End:

%\input{sec/04-implementation}

\section{Discussion}\label{sec:discussion}

\subsection{Relation to Cartesian Hoare logic}

\subsubsection{(Non)determinism}
We base our work on the \emph{one-path} variant of reachability logic.
Consequently, CRL inherits a known limitation of one-path reachability logic: that the tight correspondence between
one-path RL and Hoare logics is limited to deterministic languages.
However, we would still want to prove the following theorem.
\begin{theorem}
CRL extends CHL on the deterministic fragment of the CHL-supported language.
That is, given any program $P$ in the deterministic fragment of the CHL's imperative language,
a sound reachability-logic formalization (that is, a reachability system) $\mathcal{S}_{\mathit{IMP}}$ of the CHL's imperative language,
and firstorder formulas $\Phi, \Psi$ over variables $\vec{x}_1,\ldots,\vec{x}_k$,
\begin{equation*}
    \vDash_{\mathit{CHL}} ||\Phi||\ P\ ||\Psi||
    \iff
    \mathcal{S}_{\mathit{IMP}} \vDash_\CRL \mathit{tr}(P, \Phi) \Rightarrow^{c\exists} \mathit{tr}(\texttt{skip}, \Psi) \, .
\end{equation*}
\end{theorem}
\begin{proof}
Admitted. Use the relationship between one-path CRL and all-path CRL on deterministic languages (TODO).
\end{proof}

\subsubsection{Main Idea and Proof Technique}

Our understanding of the inner workings of CHL is based on the extended, unpublished version (\cite{SousaD16Extended})
of \cite{SousaD16}.
There, the authors define a \emph{linearization operation} on lists of programs, which roughly corresponds to our
\emph{star extension} of the language's semantics.
Then, the authors prove lemmas saying that a CHL triple with a list of programs inside is derivable
in the CHL proof system
if and only if 
a Hoare triple having the same list of programs, but linearized, inside, is derivable;
the "only if" implication corresponds to our \Cref{lem:CRLalmostSoundness}, where we construct
an RL proof from a CRL proof,
while the "if" implication corresponds to our Reduce rule.
Furthermore, in the proof of soundness of CHL, the authors assume soundness of the self-composition technique;
self-composition corresponds to our \emph{star extension} and its soundness to our \Cref{thm:correspondence}.
Finally \cite{SousaD16Extended} assumes soundness and relative completeness of the underling Hoare logic;
similarly, we assume soundness and relative completeness of reachability logic.
However, for completeness, we have to prove that \emph{star extension} preserves decidability.

\subsubsection{Differences}
There are also differences between the CHL and CRL techniques.
First, our proof system has only 8 rules, and they do not mention any programming language construct,
while CHL has 17 rules (not counting the Expand rule), half of which are specific to the underlying language.

Second, there is a redundancy between the language-specific CHL rules and the Hoare logic rules of the
programming language: for example, conditional statement ("if") has (1) a rule in the formal semantics of the language,
(2) a rule in the Hoare logic (not shown in the paper), and (3) a rule in CHL.
When considering that the three rules have to play nice together (that is, CHL and Hoare logic rules have to be sound with respect to the semantics), someone had to think about the conditional statement at least five times .
We consider this to be highly uneconomical
- and the situation
is even worse for the looping construct ("while-with-breaks"), which is supported using additional CHL rules.
In contrast, in the CRL/RL framework, it is enough to design each language construct once - when giving
semantics to it.


Third, in CHL the support for lock-step reasoning is hard-wired into the rules for loops,
while in our framework, lock-step reasoning is not limited to loops, but can support arbitrary sources
of circular behavior - including loops, recursion, gotos.
Therefore, while we are inspired by the \emph{idea} of lock-step execution from \cite{SousaD16},
for the \emph{realization} of this idea we turn to the literature on \emph{equivalence checking}.

\subsection{Relation to semantic-based program equivalence}

TODO

\subsection{Other Related Work}

In \cite{DOsualdoFD22}, the authors develop a \emph{logic for hyper-triple composition (LHC)}
that allows reasoning about $k-safety$ properties compositionally.
Similarly to CHL, LHC targets a particular small imperative language.
We believe that compositionality is ortogonal to language-parametericity,
and thus we would like to generalize their work to language-parameteric settings in future work.

A game-based technique for verifying software hyperproperties beyond $k$-safety
has been developed in \cite{BeutnerF22}.
This technique works with \emph{symbolic transition systems},
so it already is language-independent \emph{in some sense}. However, it is not clear how to use the technique
with an arbitrary language $L$, without writing a \emph{compiler} from $L$ to symbolic transition systems first.
\section{Future Work and Conclusion}

We have presented Cartesian Reachability logic - a logic for reasoning about $k$-safety hyperproperties
in any deterministic language equipped with a RL-based operational semantics.
The logic has a simple, sound and complete proof system and allows lockstep reasoning
similarly to Cartesian Hoare logic.
Instantiating CRL with a new language does not require any changes to the soundness proof;
therefore, CRL has the potential to greatly reduce costs of development of tools and techniques for $k$-safety verification.

In future, we want to develop a variant of CRL which does not require the language to be deterministic.
We believe this to be viable, because (1) CHL has some support for nondeterminism,
and because (2) reachability logic, on which we base our work, has a newer variant that supports nondeterminism, too.
On the theoretical side, we would like to know whether our proof system would be complete even in the absence of
the Reduce rule.
An orthogonal line of future research is compositionality: we would like to enable compositional reasoning
using the technique developed in \cite{DOsualdoFD22}.
Finally, we plan to develop a practical, language-parametric tool implementing CRL, using the \K{} semantic framework.


\bibliography{bibliography}
\bibliographystyle{plain}

\appendix

%\section{Appendix}


\begin{proof}[Proof of \Cref{lem:structurelessSemantics}]
Let $P$ be a structureless pattern.
We perform induction on $P$.
If $P = \phi$, we get contradiction with the assumption that $P$ is structureless;
therefore, the conclusion holds by \emph{ex falso quodlibet}.
Other cases follow from the induction hypotheses.
\end{proof}


\begin{proof}[Proof of \Cref{lem:rhoStarOfPi}]
    By induction on the term $\pi$.
    \begin{itemize}
        \item $\pi = v$ for $v \in \mathit{Var}$ - follows from the definition of $\rho^*$.
        \item $\pi = f(\pi_1, \ldots, \pi_k)$ - we have $\rho(\pi_i) = \rho^*(\pi_i)$ for any $i \in \{ 1, \ldots, k \}$
              by the induction hypothesis.
              Then
              \begin{align*}
                  \rho^*(f(\pi_1, \ldots, \pi_k)) 
                  = & {\mathcal{T}^*}_f(\rho^*(\pi_1), \ldots, \rho^*(\pi_k)) \\
                  = & {\mathcal{T}^*}_f(\rho(\pi_1), \ldots, \rho(\pi_k)) \\
                  = & \mathcal{T}_f(\rho(\pi_1), \ldots, \rho(\pi_k)) \\
                  = & \rho(f(\pi_1, \ldots, \pi_k))
              \end{align*}
              where the second-to-last equality holds by definition of $\mathcal{T}^*$.
    \end{itemize}
\end{proof}


\begin{proof}[Proof of \Cref{lem:starConservative}]
By induction on $\varphi$.
\begin{itemize}
    \item $\varphi \equiv \pi$, where $\pi$ is a basic pattern (of sort $s$) - follows from \Cref{lem:rhoStarOfPi}.
    \item $\varphi \equiv \varphi_1 \land \varphi_2$ - follows from the induction hypothesis.
    \item $\varphi \equiv \neg \varphi^\prime$ - follows from the induction hypothesis.
    \item $\varphi \equiv \exists x : s^\prime.\, \varphi^\prime$. We have
    $ (\mathcal{T}^*, \gamma, \rho^\prime) \vDash \exists x : s^\prime.\, \varphi^\prime $
    if and only if (by definition of $\vDash$)
    there exists some valuation ${\rho^*}^\prime : \mathit{Var} \to \mathcal{T}^*$ such that
    $(\mathcal{T}^*, \gamma, {\rho^*}^\prime) \vDash \varphi^\prime$
    and ${\rho^*}^\prime(x) \in \mathcal{T}^*_{s^\prime}$\traian{you need to use $s'$ here and in the following.}
    and ${\rho^*}^\prime(y) = \rho^*(y)$ for all $y \in \mathit{Var} \setminus \{ x \}$.
    This holds if and only if
    there exists some valuation $\rho^{\prime} : \mathit{Var} \to \mathcal{T}$ such that
    $(\mathcal{T}^*, \gamma, {\rho^{\prime}}^*) \vDash \varphi^\prime$
    and ${\rho^{\prime}}^*(x) \in \mathcal{T}^*_s$
    and ${\rho^{\prime}}^*(y) = \rho^*(y)$ for all $y \in \mathit{Var} \setminus \{ x \}$:
    one implication follows by letting ${\rho^*}^\prime := \rho^{\prime}$;
    for the other implication, we let $\rho^{\prime}(v) := {\rho^*}^\prime(v)$ for all $v \in \mathit{Var}$,
    and by the definition of star, we have ${\rho^{\prime}}^*(v) = {\rho^*}^\prime(v)$, from which the rest follows.
    Next, by the induction hypothesis, this holds if and only if
    there exists some valuation $\rho^{\prime} : \mathit{Var} \to \mathcal{T}$ such that
    $(\mathcal{T}, \gamma, {\rho^{\prime}}) \vDash \varphi^\prime$
    and ${\rho^{\prime}}^*(x) \in \mathcal{T}^*_s$
    and ${\rho^{\prime}}^*(y) = \rho^*(y)$ for all $y \in \mathit{Var} \setminus \{ x \}$.
    Next, by the definition of ${\rho^\prime}^*$ and $\rho^*$, this holds if and only if
    there exists some valuation $\rho^{\prime} : \mathit{Var} \to \mathcal{T}$ such that
    $(\mathcal{T}, \gamma, {\rho^{\prime}}) \vDash \varphi^\prime$
    and ${\rho^{\prime}}(x) \in \mathcal{T}^*_s$
    and ${\rho^{\prime}}(y) = \rho(y)$ for all $y \in \mathit{Var} \setminus \{ x \}$.
    But since the star extensions interprets all sorts from $\Sigma$ as in the original model
    (that is, $\mathcal{T}^*_s = \mathcal{T}_s$),
    this is equivalent to the semantics of $(\mathcal{T}, \gamma, \rho) \vDash \exists x:s^\prime.\, \varphi^\prime$,
    which is what we wanted to prove.

    Alternate attempt:
    The induction hypothesis is: for any $\mathcal{T}$, $\gamma$, $\rho$, 
        $$(\mathcal{T}, \gamma, \rho) \vDash \varphi' \iff (\mathcal{T}^*, \gamma, \rho^\ast) \vDash \varphi'$$
    We want to prove that for any $\mathcal{T}$, $\gamma$, $\rho$, 
        $$(\mathcal{T}, \gamma, \rho) \vDash \exists x:s'. \varphi' \iff (\mathcal{T}^*, \gamma, \rho^\ast) \vDash \exists x:s'.\varphi'$$
    
    First, let us prove the left-to-right implication.
    The left-hand-side of the claim is equivalent with
    $$\exists \rho'. (\forall y. y \neq x \to \rho'(y) = \rho(y)) \wedge (T, \gamma, \rho') \vDash \varphi'$$
    From the induction hypothesis, this is further equivalent with
    $$\exists \rho'. (\forall y. y \neq x \to \rho'(y) = \rho(y)) \wedge (T^\ast, \gamma, {\rho'}^\ast) \vDash \varphi'$$
    
    Since $\forall y. y \neq x \to {\rho'}^\ast(y) = \rho'(y) = \rho(y) = \rho^\ast(y)$, we deduce $(\mathcal{T}^*, \gamma, \rho^\ast) \vDash \exists x:s'.\varphi'$.
    
    Conversely, the right-hand-side of the claim is equivalent with
    $$\exists \rho''. (\forall y. y \neq x \to \rho'(y) = \rho\ast(y)) \wedge (T^\ast, \gamma, \rho'') \vDash \varphi'$$
    Let $\rho'$ be defined by $\rho'(y) = \rho(y)$ if $y\neq x$ and $\rho'(x) = \rho''(x)$. Then it is
    easy to see that ${\rho'}^\ast = \rho''$, whence by the induction hypothesis we obtain that
    $(T, \gamma, \rho') \vDash \varphi'$, and by the definition of $\rho'$, $(T, \gamma, \rho) \vDash \exists x:s'. \varphi'$.
 \end{itemize}
\end{proof}

\begin{lemma}\label{lem:simplifyComposite}
We have
\begin{proofenv}
    \begin{equation*}
        C \Rightarrow_{\mathcal{S}^*} C^\prime
    \end{equation*}
\end{proofenv}
    if and only if
\begin{proofenv}
    there exists a rule $\phi \land P \Rightarrow^\exists \phi^\prime \land P^\prime \in S$
    and valuation $\rho : \mathit{Var}^* \to \mathcal{T}^*$ such that
    \begin{itemize}
        \item $(\mathcal{T}^*, \rho) \vDash P$; and
        \item $(\mathcal{T}^*, \rho) \vDash P^\prime$; and
        \item $C = \rho(L) \texttt{++} [\rho(\phi)] \texttt{++} \rho(R)$; and
        \item $C^\prime = \rho(L) \texttt{++} [\rho(\phi^\prime)] 
        \texttt{++} \rho(R)$,
    \end{itemize}
\end{proofenv}
\end{lemma}
\begin{proof}
We have
\begin{proofenv}
    \begin{equation*}
        C \Rightarrow_{\mathcal{S}^*} C^\prime
    \end{equation*}
\end{proofenv}
iff (by \Cref{def:basics})
\begin{proofenv}
    there exists a rule $\varphi \Rightarrow^\exists \varphi^\prime \in S^*$
    and valuation $\rho : \mathit{Var}^* \to \mathcal{T}^*$ such that
    $(\mathcal{T}^*, C, \rho) \vDash \varphi$
    and $(\mathcal{T}^*, C^\prime, \rho) \vDash \varphi^\prime$,
\end{proofenv}
iff (by
%\Cref{def:matchinglogic} and
\Cref{rem:shapeOfReachabilityRules} and \Cref{def:starextension})
\begin{proofenv}
    there exists a rule $\phi \land P \Rightarrow^\exists \phi^\prime \land P^\prime \in S$
    and valuation $\rho : \mathit{Var}^* \to \mathcal{T}^*$ such that
    \begin{equation*}
    (\mathcal{T}^*, C, \rho) \vDash \mathit{cfgheat}(L, \phi, R) \land P
    \end{equation*}
    and
    \begin{equation*}
        (\mathcal{T}^*, C^\prime, \rho) \vDash
        \mathit{cfgheat}(L, \phi^\prime, R) \land P^\prime \, ,
    \end{equation*}
\end{proofenv}
iff (by \Cref{def:matchinglogic} and \Cref{lem:structurelessSemantics})
\begin{proofenv}
    there exists a rule $\phi \land P \Rightarrow^\exists \phi^\prime \land P^\prime \in S$
    and valuation $\rho : \mathit{Var}^* \to \mathcal{T}^*$ such that
    $(\mathcal{T}^*, \rho) \vDash P$ and $(\mathcal{T}^*, \rho) \vDash P^\prime$ and
    \begin{equation*}
        (\mathcal{T}^*, C, \rho) \vDash \mathit{cfgheat}(L, \phi_l, R)
    \end{equation*}
    and
    \begin{equation*}
        (\mathcal{T}^*, C^\prime, \rho) \vDash
        \mathit{cfgheat}(L, \phi^\prime_j, R) \, ,
    \end{equation*}
\end{proofenv}
iff (by \Cref{def:matchinglogic} and \Cref{def:starextension})
\begin{proofenv}
    there exists a rule $\phi \land P \Rightarrow^\exists \phi^\prime \land P^\prime \in S$
    and valuation $\rho : \mathit{Var}^* \to \mathcal{T}^*$ such that
    \begin{itemize}
        \item $(\mathcal{T}^*, \rho) \vDash P$; and
        \item $(\mathcal{T}^*, \rho) \vDash P^\prime$; and
        \item $C = \rho(L) \texttt{++} [\rho(\phi)] \texttt{++} \rho(R)$; and
        \item $C^\prime = \rho(L)
        \texttt{++} [\rho(\phi^\prime)] \texttt{++} \rho(R)$.
    \end{itemize}
\end{proofenv}
That proves the desired equivalence.
\end{proof}


\begin{proof}[Proof of \Cref{lem:compositeStep}]
We have
\begin{proofenv}
\begin{equation*}
[c_1,\ldots,c_k] \Rightarrow_{\mathcal{S}^*} [c_1, \ldots, c_{i-1}, c^\prime, c_{i+1}, \ldots, c_k]    
\end{equation*}
\end{proofenv}
iff (by \Cref{lem:simplifyComposite})
\begin{proofenv}
there exists a rule $\phi \land P \Rightarrow^\exists \phi^\prime \land P^\prime \in S$
and valuation $\rho : \mathit{Var}^* \to \mathcal{T}^*$ such that
\begin{itemize}
    \item $(\mathcal{T}^*, \rho) \vDash P$; and
    \item $(\mathcal{T}^*, \rho) \vDash P^\prime$; and
    \item $[c_1,\ldots,c_k] = \rho(L) \texttt{++} [\rho(\phi_l)] \texttt{++} \rho(R)$; and
    \item $[c_1, \ldots, c_{i-1}, c^\prime, c_{i+1}, \ldots, c_k] = \rho(L) \texttt{++} [\rho(\phi_j)] 
    \texttt{++} \rho(R)$.
\end{itemize}
\end{proofenv}
Suppose we have such valuation $\rho$.
We can surely construct valuation $\rho_0 : \mathit{Var} \to \mathcal{T}$ by letting
\begin{equation*}
\rho_0(v)=
    \begin{cases}
        \rho(v) & \text{if } \rho(v) \in \mathcal{T}\\
        a & \text{if } \rho(v) \not\in \mathcal{T}
    \end{cases}
\end{equation*}
(where $a \in \mathcal{T}$ is some arbitrary element).
Now, for any $v \in \mathit{FV}(\phi) \cup \mathit{FV}(\phi^\prime) \cup \mathit{FV}(P) \cup \mathit{FV}(P^\prime)$ it holds that
$((\rho_0)^*)(v) = \rho(v)$.
Why? Because $v$ has some sort $s$ from $\Sigma$
(that is, $s \not = \mathit{Cfg}^*$).
Therefore, we can use \Cref{lem:unusedVariables} to change the goal to one saying that
\begin{proofenv}
there exists a rule $\phi \land P \Rightarrow^\exists \phi^\prime \land P^\prime \in S$
and valuation $\rho_0 : \mathit{Var} \to \mathcal{T}$ such that
\begin{itemize}
    \item $(\mathcal{T}^*, (\rho_0)^*) \vDash P$; and
    \item $(\mathcal{T}^*, (\rho_0)^*) \vDash P^\prime$; and
    \item $[c_1,\ldots,c_k] = ((\rho_0)^*)(L) \texttt{++} [((\rho_0)^*)(\phi_l)] \texttt{++} ((\rho_0)^*)(R)$; and
    \item $[c_1, \ldots, c_{i-1}, c^\prime, c_{i+1}, \ldots, c_k] = ((\rho_0)^*)(L)
    \texttt{++} [((\rho_0)^*)(\phi_j)] 
    \texttt{++} ((\rho_0)^*)(R)$
\end{itemize}
\end{proofenv}
(where the opposite implication follows by choice $\rho := (\rho_0)^*$).
Now, we use \Cref{lem:starConservative} and definition of starred valuation to change the goal to one saying that
\begin{proofenv}
there exists a rule $\phi \land P \Rightarrow^\exists \phi^\prime \land P^\prime \in S$
and valuation $\rho_0 : \mathit{Var} \to \mathcal{T}$ such that
\begin{itemize}
    \item $(\mathcal{T}, \rho_0) \vDash P$; and
    \item $(\mathcal{T}, \rho_0) \vDash P^\prime$; and
    \item $[c_1,\ldots,c_k] = \rho_0(L) \texttt{++} [\rho_0(\phi_l)] \texttt{++} \rho_0(R)$; and
    \item $[c_1, \ldots, c_{i-1}, c^\prime, c_{i+1}, \ldots, c_k] = \rho_0(L)
    \texttt{++} [\rho_0(\phi_j)] 
    \texttt{++} \rho_0(R)$.
\end{itemize}
\end{proofenv}
Now, by list reasoning, this is equivalent to
saying that
\begin{proofenv}
there exists a rule $\phi \land P \Rightarrow^\exists \phi^\prime \land P^\prime \in S$
and valuation $\rho_0 : \mathit{Var} \to \mathcal{T}$ such that
there exists some $i^\prime$ satisfying $1 \leq i^\prime \leq k$
such that
\begin{itemize}
    \item $(\mathcal{T}, \rho_0) \vDash P_l$; and
    \item $(\mathcal{T}, \rho_0) \vDash P_j$; and
    \item $[c_1,\ldots, c_{i^\prime-1}] = \rho_0(L)$; and
    \item $c_{i^\prime} = \rho_0(\phi_l)$; and
    \item $[c_{i^\prime+1},\ldots,c_k] = \rho_0(R)$; and
    \item $[c_1, \ldots, c_{i-1}, c^\prime, c_{i+1}, \ldots, c_k] = \rho_0(L)
    \texttt{++} [\rho_0(\phi_j)] 
    \texttt{++} \rho_0(R)$.
\end{itemize}
\end{proofenv}
Now, let us define $c^\prime_{z}$ by
\begin{equation*}
c^\prime_{z} =
    \begin{cases}
        c^\prime & \text{if } z = i \\
        c_z & \text{if } z \not = i
    \end{cases}
\end{equation*}
after which the goal is equivalent to saying that
\begin{proofenv}
there exists a rule $\phi \land P \Rightarrow^\exists \phi^\prime \land P^\prime \in S$
and valuation $\rho_0 : \mathit{Var} \to \mathcal{T}$ such that
there exists some $i^\prime$ satisfying $1 \leq i^\prime \leq k$ such that
\begin{itemize}
    \item $(\mathcal{T}, \rho_0) \vDash P$; and
    \item $(\mathcal{T}, \rho_0) \vDash P^\prime$; and
    \item $[c_1,\ldots, c_{i^\prime-1}] = \rho_0(L)$; and
    \item $c_{i^\prime} = \rho_0(\phi)$; and
    \item $[c_{i^\prime+1},\ldots,c_k] = \rho_0(R)$; and
    \item $[c^\prime_1,\ldots, c^\prime_{i^\prime-1}] = \rho_0(L)$; and
    \item $c^\prime_{i^\prime} = \rho_0(\phi^\prime)$; and
    \item $[c^\prime_{i^\prime+1},\ldots,c^\prime_k] = \rho_0(R)$.
\end{itemize}
\end{proofenv}
Since $L,R$ were fresh, they do not occur in $\phi$ nor in $\phi^\prime$.
Therefore, using \Cref{lem:unusedVariables}, we can equivalently say that
\begin{proofenv}
there exists a rule $\phi \land P \Rightarrow^\exists \phi^\prime \land P^\prime \in S$
and valuation $\rho_0 : \mathit{Var} \to \mathcal{T}$ such that
there exists some $i^\prime$ satisfying $1 \leq i^\prime \leq k$
such that
\begin{itemize}
    \item $(\mathcal{T}, \rho_0) \vDash P$; and
    \item $(\mathcal{T}, \rho_0) \vDash P^\prime$; and
    \item $c_{i^\prime} = \rho_0(\phi_l)$; and
    \item $c^\prime_{i^\prime} = \rho_0(\phi_j)$.
\end{itemize}
\end{proofenv}
(The downwards implication is trivial, as it is only removing constraints; the upwards implication
is from the fact that we can always choose a valuation $\rho_0$ satisfying the constraints.)
But that is equivalent (\Cref{def:matchinglogic}) to saying that 
\begin{proofenv}
there exists a rule $\phi \land P \Rightarrow^\exists \phi^\prime \land P^\prime \in S$
and there exists some $i^\prime$ satisfying $1 \leq i^\prime \leq k$
and valuation $\rho_0 : \mathit{Var} \to \mathcal{T}$ such that
\begin{itemize}
    \item $(\mathcal{T}, c_{i^\prime}, \rho_0) \vDash \phi \land P$; and
    \item $(\mathcal{T}, c^\prime_{i^\prime}, \rho_0) \vDash \phi^\prime \land P^\prime$
    .
\end{itemize}
\end{proofenv}
But that is
equivalent to saying that
\begin{proofenv}
there exists some $i^\prime$ satisfying $1 \leq i^\prime \leq k$
such that $c_{i^\prime} \Rightarrow_{\mathcal{S}} c^\prime_{i^\prime}$,
\end{proofenv}
which is almost equivalent to the left side of the equivalence we want to prove:
that
\begin{proofenv}
$c_{i} \Rightarrow_{\mathcal{S}} c^\prime_{i}$.
\end{proofenv}
The upwards implication is trivial; the downwards is as follows. If $i = i^\prime$, we are done.
But otherwise, it would follow (by definition of $c^\prime_{i^\prime}$) that $c_{i^\prime} \Rightarrow_{\mathcal{S}} c_{i^\prime}$,
which contradicts \Cref{rem:noEmptySteps}.
\end{proof}

\begin{lemma}\label{lem:mkListSemantics}
$(\mathcal{T}^*, C, \rho) \vDash \mathit{mkList}(\phi_1,\ldots,\phi_k)$
iff there exists $c_1, \ldots, c_k \in \Tcfg$ such that $C = [c_1, \ldots, c_k]$ and for every $\rho^\prime : \mathit{Var} \to \mathcal{T}$ satisfying
$\rho^\prime(v) = \rho(v)$ for any \\
$v \in \mathit{FV}(\mathit{mkList}(\phi_1, \ldots, \phi_k))$,
it holds that 
$(\mathcal{T}, c_1, \rho^\prime) \vDash \phi_1$ and \ldots and $(\mathcal{T}, c_k, \rho^\prime) \vDash \phi_k$.
\end{lemma}
\begin{proof}
By induction on $k$.
\begin{itemize}
    \item If $k = 1$, then we have to prove that
    \begin{proofenv}
    $(\mathcal{T}^*, C, \rho) \vDash \mathit{cfgitem}(\phi_1)$
    iff there exists $c_1 \in \Tcfg$ such that $C = [c_1]$ and for every $\rho^\prime : \mathit{Var} \to \mathcal{T}$ satisfying
    $\rho^\prime(v) = \rho(v)$ for any $v \in \mathit{FV}(\mathit{cfgitem}(\phi_1))$, it holds that
    $(\mathcal{T}, c, \rho^\prime) \vDash \phi_1$.
    \end{proofenv}
    By \cref{def:matchinglogic}, this is equivalent to
    \begin{proofenv}
    $C = \rho(\mathit{cfgitem}(\phi_1))$
    iff there exists $c_1 \in \Tcfg$ such that $C = [c_1]$ and for every $\rho^\prime : \mathit{Var} \to \mathcal{T}$ satisfying
    $\rho^\prime(v) = \rho(v)$ for any $v \in \mathit{FV}(\mathit{cfgitem}(\phi_1))$, it holds that
    $c_1 = \rho^\prime(\phi_1)$.
    \end{proofenv}
    By \Cref{def:starextension}, this is equivalent to
    \begin{proofenv}
    $C = [\rho(\phi_1)]$
    iff there exists $c_1 \in \Tcfg$ such that $C = [c_1]$ and for every $\rho^\prime : \mathit{Var} \to \mathcal{T}$ satisfying
    $\rho^\prime(v) = \rho(v)$ for any $v \in \mathit{FV}(\mathit{cfgitem}(\phi_1))$, it holds that
    $c_1 = \rho^\prime(\phi_1)$.
    \end{proofenv}
    We prove each implication separately.
    For the left-to-right implication, we let $c_1 := \rho(\phi_1)$
    and have to prove that $\rho(\phi_1) = \rho^\prime(\phi_1)$, which follows from \Cref{lem:unusedVariables}.
    The right-to-left implication also follows from  \Cref{lem:unusedVariables}.
    
    \item If $k = k^\prime + 1$, we assume the induction hypothesis saying that
    \begin{proofenv}
    for every $C, \phi_1, \ldots, \phi_{k^\prime}$,
    $(\mathcal{T}^*, C, \rho) \vDash \mathit{mkList}(\phi_1,\ldots,\phi_{k^\prime})$
    iff there exists $c_1, \ldots, c_{k^\prime} \in \Tcfg$ such that $C = [c_1, \ldots, c_{k^\prime}]$
    and for every $\rho^\prime : \mathit{Var} \to \mathcal{T}$ satisfying
    $\rho^\prime(v) = \rho(v)$ for any
    $v \in \mathit{FV}(\mathit{mkList}(\phi_1, \ldots, \phi_{k^\prime}))$,
    it holds that
    $(\mathcal{T}, c_1, \rho^\prime) \vDash \phi_1$ and \ldots and $(\mathcal{T}, c_{k^\prime}, \rho^\prime) \vDash \phi_k$,
    \end{proofenv}
    and have to prove that
    \begin{proofenv}
    $(\mathcal{T}^*, C, \rho) \vDash \mathit{mkList}(\phi_1,\ldots,\phi_{k^\prime + 1})$
    iff there exists $c_1, \ldots, c_{k^\prime + 1} \in \Tcfg$ such that $C = [c_1, \ldots, c_{k^\prime + 1}]$ and 
    for every $\rho^\prime : \mathit{Var} \to \mathcal{T}$ satisfying
    $\rho^\prime(v) = \rho(v)$ for any
    $v \in \mathit{FV}(\mathit{mkList}(\phi_1, \ldots, \phi_{k^\prime + 1}))$,
    it holds that
    $(\mathcal{T}, c_1, \rho^\prime) \vDash \phi_1$ and \ldots and $(\mathcal{T}, c_{k^\prime + 1}, \rho^\prime) \vDash \phi_{k^\prime + 1}$,
    \end{proofenv}
    which is (by \Cref{def:matchinglogic}) equivalent to
    \begin{proofenv}
    $C = \rho(\mathit{mkList}(\phi_1,\ldots,\phi_{k^\prime + 1}))$
    iff there exists $c_1, \ldots, c_{k^\prime + 1} \in \Tcfg$ such that $C = [c_1, \ldots, c_{k^\prime + 1}]$
    and for every $\rho^\prime : \mathit{Var} \to \mathcal{T}$ satisfying
    $\rho^\prime(v) = \rho(v)$ for any
    $v \in \mathit{FV}(\mathit{mkList}(\phi_1, \ldots, \phi_{k^\prime + 1}))$,
    it holds that
    $c_1 = \rho^\prime(\phi_1)$ and \ldots and $c_{k^\prime + 1} = \rho^\prime(\phi_{k^\prime + 1})$,
    \end{proofenv}
    which is (by \Cref{def:starextension}) equivalent to
    \begin{proofenv}
    $C = [\rho(\phi_1)] \texttt{++} C^\prime$ and $C^\prime = \rho(\mathit{mkList}(\phi_2,\ldots,\phi_{k^\prime + 1}))$
    iff there exists $c_1, \ldots, c_{k^\prime + 1} \in \Tcfg$ such that $C = [c_1, \ldots, c_{k^\prime + 1}]$
    and for every $\rho^\prime : \mathit{Var} \to \mathcal{T}$ satisfying
    $\rho^\prime(v) = \rho(v)$ for any
    $v \in \mathit{FV}(\mathit{mkList}(\phi_1, \ldots, \phi_{k^\prime + 1}))$,
    it holds that
    $c_1 = \rho^\prime(\phi_1)$ and \ldots and $c_{k^\prime + 1} = \rho^\prime(\phi_{k^\prime + 1})$,
    \end{proofenv}
    which is by the induction hypothesis with $\phi_1 := \phi_2,\ldots,\phi_k := \phi_{k^\prime + 1}$
    and $\alpha$-renaming
    equivalent to
    \begin{proofenv}
    $C = [\rho(\phi_1)] \texttt{++} C^\prime$ and
    there exists $c_2, \ldots, c_{k^\prime + 1} \in \Tcfg$ such that $C^\prime = [c_2, \ldots, c_{k^\prime+1}]$ 
    and for every $\rho^\prime : \mathit{Var} \to \mathcal{T}$ satisfying
    $\rho^\prime(v) = \rho(v)$ for any
    $v \in \mathit{FV}(\mathit{mkList}(\phi_2, \ldots, \phi_{k^\prime+1}))$,
    it holds that
    $(\mathcal{T}, c_2, \rho^\prime) \vDash \phi_2$ and \ldots and $(\mathcal{T}, c_{k^\prime+1}, \rho^\prime) \vDash \phi_{k^\prime + 1}$,
    iff there exists $c_1, \ldots, c_{k^\prime + 1} \in \Tcfg$ such that $C = [c_1, \ldots, c_{k^\prime + 1}]$
    and for every $\rho^\prime : \mathit{Var} \to \mathcal{T}$ satisfying
    $\rho^\prime(v) = \rho(v)$ for any
    $v \in \mathit{FV}(\mathit{mkList}(\phi_1, \ldots, \phi_{k^\prime + 1}))$,
    it holds that
    $c_1 = \rho^\prime(\phi_1)$ and \ldots and $c_{k^\prime + 1} = \rho^\prime(\phi_{k^\prime + 1})$,
    \end{proofenv}
    which is (by firstorder reasoning and simplification of list append) equivalent to
    \begin{proofenv}
    there exists $c_2, \ldots, c_{k^\prime + 1} \in \Tcfg$ such that
    $C = [\rho(\phi_1), c_2, \ldots, c_{k^\prime+1}]$
    and for every $\rho^\prime : \mathit{Var} \to \mathcal{T}$ satisfying
    $\rho^\prime(v) = \rho(v)$ for any
    $v \in \mathit{FV}(\mathit{mkList}(\phi_2, \ldots, \phi_{k^\prime+1}))$,
    it holds that
    $(\mathcal{T}, c_2, \rho^\prime) \vDash \phi_2$ and \ldots and $(\mathcal{T}, c_{k^\prime+1}, \rho^\prime) \vDash \phi_{k^\prime+1}$,
    iff there exists $c_1, \ldots, c_{k^\prime + 1} \in \Tcfg$ such that $C = [c_1, \ldots, c_{k^\prime + 1}]$
    and for every $\rho^\prime : \mathit{Var} \to \mathcal{T}$ satisfying
    $\rho^\prime(v) = \rho(v)$ for any
    $v \in \mathit{FV}(\mathit{mkList}(\phi_1, \ldots, \phi_{k^\prime + 1}))$,
    it holds that
    $c_1 = \rho^\prime(\phi_1)$ and \ldots and $c_{k^\prime + 1} = \rho^\prime(\phi_{k^\prime + 1})$.
    \end{proofenv}
    We simplify the goal using \Cref{def:matchinglogic} to
    \begin{proofenv}
    there exists $c_2, \ldots, c_{k^\prime + 1} \in \Tcfg$ such that
    $C = [\rho(\phi_1), c_2, \ldots, c_{k^\prime+1}]$
    and for every $\rho^\prime : \mathit{Var} \to \mathcal{T}$ satisfying
    $\rho^\prime(v) = \rho(v)$ for any
    $v \in \mathit{FV}(\mathit{mkList}(\phi_2, \ldots, \phi_{k^\prime+1}))$,
    it holds that
    $c_2 = \rho^\prime(\phi_2)$ and \ldots and $c_{k^\prime+1} = \rho^\prime(\phi_{k^\prime+1})$,
    iff there exists $c_1, \ldots, c_{k^\prime + 1} \in \Tcfg$ such that $C = [c_1, \ldots, c_{k^\prime + 1}]$
    and for every $\rho^\prime : \mathit{Var} \to \mathcal{T}$ satisfying
    $\rho^\prime(v) = \rho(v)$ for any
    $v \in \mathit{FV}(\mathit{mkList}(\phi_1, \ldots, \phi_{k^\prime + 1}))$,
    it holds that
    $c_1 = \rho^\prime(\phi_1)$ and \ldots and $c_{k^\prime + 1} = \rho^\prime(\phi_{k^\prime + 1})$.
    \end{proofenv}
    We prove each implication separately.
    \begin{itemize}
        \item Assuming
        \begin{proofenv}
        there exists $c_2, \ldots, c_{k^\prime + 1} \in \Tcfg$ such that
        $C = [\rho(\phi_1), c_2, \ldots, c_{k^\prime+1}]$
        and for every $\rho^\prime : \mathit{Var} \to \mathcal{T}$ satisfying
        $\rho^\prime(v) = \rho(v)$ for any
        $v \in \mathit{FV}(\mathit{mkList}(\phi_2, \ldots, \phi_{k^\prime+1}))$,
        it holds that
        $c_2 = \rho^\prime(\phi_1)$ and \ldots and $c_{k^\prime+1} = \rho^\prime(\phi_k)$,
        \end{proofenv}
        we prove that
        \begin{proofenv}
        there exists $c_1, \ldots, c_{k^\prime + 1} \in \Tcfg$ such that $C = [c_1, \ldots, c_{k^\prime + 1}]$
        and for every $\rho^\prime : \mathit{Var} \to \mathcal{T}$ satisfying
        $\rho^\prime(v) = \rho(v)$ for any
        $v \in \mathit{FV}(\mathit{mkList}(\phi_1, \ldots, \phi_{k^\prime + 1}))$,
        it holds that
        $c_1 = \rho^\prime(\phi_1)$ and \ldots and $c_{k^\prime + 1} = \rho^\prime(\phi_{k^\prime + 1})$.
        \end{proofenv}
        by choosing $c_1 := \rho^\prime(\phi_1)$ and using \Cref{lem:unusedVariables}\\
        (note that $\mathit{FV}(\mathit{mkList}(\phi_2,\ldots,\phi_{k^\prime+1})) \subseteq \mathit{FV}(\mathit{mkList}(\phi_1,\ldots,\phi_{k^\prime+1}))$).
        \item Assuming
        \begin{proofenv}
        there exists $c_1, \ldots, c_{k^\prime + 1} \in \Tcfg$ such that $C = [c_1, \ldots, c_{k^\prime + 1}]$
        and for every $\rho^\prime : \mathit{Var} \to \mathcal{T}$ satisfying
        $\rho^\prime(v) = \rho(v)$ for any
        $v \in \mathit{FV}(\mathit{mkList}(\phi_1, \ldots, \phi_{k^\prime + 1}))$,
        it holds that
        $c_1 = \rho^\prime(\phi_1)$ and \ldots and $c_{k^\prime + 1} = \rho^\prime(\phi_{k^\prime + 1})$,
        \end{proofenv}
        we prove that
        \begin{proofenv}
        there exists $c_2, \ldots, c_{k^\prime + 1} \in \Tcfg$ such that
        $C = [\rho(\phi_1), c_2, \ldots, c_{k^\prime+1}]$
        and for every $\rho^\prime : \mathit{Var} \to \mathcal{T}$ satisfying
        $\rho^\prime(v) = \rho(v)$ for any
        $v \in \mathit{FV}(\mathit{mkList}(\phi_2, \ldots, \phi_{k^\prime+1}))$,
        it holds that
        $c_2 = \rho^\prime(\phi_1)$ and \ldots and $c_{k^\prime+1} = \rho^\prime(\phi_k)$
        \end{proofenv}
        by setting $c_i := c_i$
        (and again noting that $\mathit{FV}(\mathit{mkList}(\phi_2,\ldots,\phi_{k^\prime+1})) \subseteq \mathit{FV}(\mathit{mkList}(\phi_1,\ldots,\phi_{k^\prime+1}))$).
    \end{itemize}
\end{itemize}
\end{proof}

\begin{lemma}\label{lem:transitionOnlyBetweenListsOfSameLength}
    Let $\mathcal{S} = (\mathcal{T}, S)$ be a reachability system over $(\Sigma, \mathit{Cfg})$.
    Then for any $C,C^\prime \in \mathcal{T}^*_{\mathit{Cfg}^*}$,
    if $C \Rightarrow_{\mathcal{S}^*} C^\prime$,
    then the length of $C$ (it is a list) is the same as the length of $C^\prime$.
\end{lemma}
\begin{proof}
Assume $C \Rightarrow_{\mathcal{S}^*} C^\prime$.
Then by \Cref{lem:simplifyComposite},
\begin{proofenv}
    there exists a rule $\phi \land P_l \Rightarrow^\exists \phi^\prime \land P^\prime \in S$
    and valuation $\rho : \mathit{Var}^* \to \mathcal{T}^*$ such that
    \begin{itemize}
        \item $(\mathcal{T}^*, \rho) \vDash P$; and
        \item $(\mathcal{T}^*, \rho) \vDash P^\prime$; and
        \item $C = \rho(L) \texttt{++} [\rho(\phi)] \texttt{++} \rho(R)$; and
        \item $C^\prime = \rho(L) \texttt{++} [\rho(\phi^\prime)] 
        \texttt{++} \rho(R)$.
    \end{itemize}
\end{proofenv}
But then $C$ and $C^\prime$ have the same length.
\end{proof}

\begin{lemma}[At most one component changes]\label{lem:atMostOneComponentChanges}
    Let $\mathcal{S} = (\mathcal{T}, S)$ be a reachability system over $(\Sigma, \mathit{Cfg})$.
    Then for any $C,C^\prime \in \mathcal{T}^*{\mathit{Cfg}^*}$ satisfying $C \Rightarrow_{\mathcal{S}^*} C^\prime$
    there exists some $i \in \mathbb{N}$ such that
    for every $i^\prime \in \mathbb{N}$ such that $i^\prime \not = i$,
    we have $C[i^\prime] = C^\prime[i^\prime]$ if both are defined.
\end{lemma}
\begin{proof}
Assume $C \Rightarrow_{\mathcal{S}^*} C^\prime$.
Then by \Cref{lem:simplifyComposite},
\begin{proofenv}
    there exists a rule $\phi \land P \Rightarrow^\exists \phi^\prime \land P^\prime \in S$
    and valuation $\rho : \mathit{Var}^* \to \mathcal{T}^*$ such that
    \begin{itemize}
        \item $(\mathcal{T}^*, \rho) \vDash P_l$; and
        \item $(\mathcal{T}^*, \rho) \vDash P_j$; and
        \item $C = \rho(L) \texttt{++} [\rho(\phi)] \texttt{++} \rho(R)$; and
        \item $C^\prime = \rho(L) \texttt{++} [\rho(\phi^\prime)] 
        \texttt{++} \rho(R)$.
    \end{itemize}
\end{proofenv}
But then we can let $i := |\rho(L)|$, and the rest follows.
\end{proof}


The following definition and theorem on filtering infinite sequences
are based on the Coq development of \cite{ZamfirVLSM}
(specifically, on \url{https://github.com/runtimeverification/vlsm/blob/d6c8cee56708c7be2431b9743fe80ca6a7a29a58/theories/VLSM/Lib/StreamFilters.v}).
\begin{definition}[Filtering subsequence]\label{def:filteringSubsequence}
Given a set $A$, a subset $P \subseteq A$ and a function $s : \mathbb{N} \to A$,
a function $\mathit{ns} : \mathbb{N} \to \mathbb{N}$ is called a filtering subsequence for $P$ on $s$,
iff
\begin{enumerate}
    \item $\mathit{ns}$ is monotone;
    \item $s(x) \not \in P$ for any $x < ns(0)$;
    \item $s(\mathit{ns}(j)) \in P$ for any $j \in \mathbb{N}$; and
    \item for every $j \in \mathbb{N}$ and every $x$ such that $\mathit{ns}(j) < x < \mathit{ns}(j+1)$,
          $s(x) \not\in P$.
\end{enumerate}
Intuitively, the last condition says that $ns$ does not skip any $P$-element in $s$.
\end{definition}

\begin{lemma}[Existence of filtering sequence for infinite occurrences]\label{lem:filteringSubsequenceExistsForInfinite}
    Let $A$ be a set, let $P \subseteq A$, and let $s : \mathbb{N} \to A$ be a function whose output
    falls to $P$ infinitely often (that is, $s(i) \in P$ for infinitely many $i$).
    Then there exists a filtering subsequence for $P$ on $s$.
\end{lemma}

\begin{lemma}\label{lem:terminationComposite}
    For any reachability system $\mathcal{S} = (\mathcal{T}, S)$, any $C \in \mathcal{T}^*_{\mathit{Cfg}^*}$,
    and any $c_1,\ldots,c_k \in \Tcfg$ such that
    $C = [c_1,\ldots,c_k]$, $C$ is terminating in $(\mathcal{T}^*_{\mathit{Cfg}^*}, \Rightarrow_{S^*})$
    iff for every $j \in \{ 1, \ldots, k \}$, $c_j$ is terminating in $(\Tcfg, \Rightarrow_S)$.
\end{lemma}
\begin{proof}[Proof of \Cref{lem:terminationComposite}]
We prove both implications separately, by contraposition.
\begin{itemize}
    \item Suppose some $c_j$ is not terminating in $(\Tcfg, \Rightarrow_S)$.
    In other words, there exists some infinite $\Rightarrow_{\mathcal{S}}$-sequence
    $c_j = d(0) \Rightarrow_{\mathcal{S}} d(1) \Rightarrow_{\mathcal{S}} d(2) \Rightarrow_{\mathcal{S}} \ldots$.
    Then $C = [c_1,\ldots,c_{j-1}, d(0), c_{j+1}, \ldots, c_k] \Rightarrow_{\mathcal{S}^*}
    [c_1,\ldots,c_{j-1}, d(1), c_{j+1}, \ldots, c_k] \Rightarrow_{\mathcal{S}^*} \ldots$
    is (by \Cref{lem:compositeStep}) an infinite $\Rightarrow_{\mathcal{S}^*}$-sequence.
    Therefore, $C$ is not terminating in $(\mathcal{T}^*_{\mathit{Cfg}^*}, \Rightarrow_{S^*})$.
    \item Suppose $C$ is not terminating in $(\mathcal{T}^*_{\mathit{Cfg}^*}, \Rightarrow_{S^*})$.
    In other words, there exists an infinite sequence $C = D(0) \Rightarrow_{\mathcal{S}^*} D(1) \Rightarrow_{\mathcal{S}^*} \ldots$.
    Then there exists a component $j$ of the sequence which changes infinitely often in the sequence,
    because we have only $k$ components.
    Now, consider the function $s : \mathbb{N} \to \mathcal{T}^*_{\mathit{Cfg}^*} \times \mathcal{T}^*_{\mathit{Cfg}^*}$
    defined by $s(i) = (D(i), D(i+1))$, and let $P \subseteq \mathcal{T}^*_{\mathit{Cfg}^*} \times \mathcal{T}^*_{\mathit{Cfg}^*}$
    be defined by $(X, X^\prime) \in P$ iff $X[j] \not = X^\prime[j]$.
    By \Cref{lem:atMostOneComponentChanges}, we know that whenever $(X, X^\prime) \in P$,
    then for any $j^\prime$ satisfying $1 \leq j^\prime \leq k$ and $j^\prime \not = j$,
    we have $X[j^\prime] = X^\prime[j^\prime]$.
    Then, $s(i) \in P$ iff in the sequence $C$, on position $i$, it is exactly the $j$th component (and no other)
    which makes step.
    Now, by \Cref{lem:filteringSubsequenceExistsForInfinite}, there exists a filtering subsequence $\mathit{ns}$
    for $P$ on $s$.
    But then
    \begin{equation*}
        D(\mathit{ns}(0))[j] \Rightarrow_S D(\mathit{ns}(1))[j] \Rightarrow_S D(\mathit{ns}(2))[j] \Rightarrow_S \ldots
    \end{equation*}
    is a $(\Tcfg, \Rightarrow_S)$ sequence witnessing the non-termination of $D(0)[j] = c_j$.
    Indeed, we have
    \begin{itemize}
        \item $D(\mathit{ns}(0))[j] = D(0)[j]$, by (2) of \Cref{def:filteringSubsequence}, the definition of $P$,
        and transitivity of equality;
        \item for any $i \in \mathbb{N}$, $D(\mathit{ns}(i))[j] \Rightarrow_{\mathcal{S}} D(\mathit{ns}(i+1))[j]$.
        We prove this as follows. By
        (4) of \Cref{def:filteringSubsequence} and definition of $P$ we have
        $D(\mathit{ns}(i+1))[j] = D(\mathit{ns}(i)+1)[j]$.
        Therefore, it is enough to show that
        \begin{equation*}
            D(\mathit{ns}(i))[j] \Rightarrow_{\mathcal{S}} D(\mathit{ns}(i)+1)[j] \, .
        \end{equation*}
        By (3) of \Cref{def:filteringSubsequence} and definition of $P$ we have
        $D(\mathit{ns}(i))[j] \not = D(\mathit{ns}(i)+1)[j]$.
        By \Cref{lem:compositeStep}, it is enough to show that there exists $k \geq 1$,
        $c_1,\ldots,c_k,c^\prime \in \Tcfg$, and some $i$ satisfying $1 \leq i \leq k$,
        such that
        $[c_1,\ldots,c_k] = D(\mathit{ns}(i))$
        and
        $[c_1,\ldots,c_{i-1},c^\prime,c_{i+1},c_k] = D(\mathit{ns}(i) + 1)$.
        But that follows from the fact that $D(\mathit{ns}(i)) \Rightarrow_{\mathcal{S}^*} D(\mathit{ns}(i) + 1)$
        and that $D(\mathit{ns}(i))[j] \not = D(\mathit{ns}(i)+1)[j]$
        by \Cref{lem:transitionOnlyBetweenListsOfSameLength} and \Cref{lem:atMostOneComponentChanges}.
    \end{itemize}
\end{itemize}
\end{proof}


\begin{lemma}\label{lem:reachComposite}
    For any reachability system $\mathcal{S} = (\mathcal{T}, S)$, any $C,C^\prime \in \mathcal{T}^*_{\mathit{Cfg}^*}$,
    and any $c_1,\ldots,c_k,c_1^\prime,\ldots,c_k^\prime \in \Tcfg$ such that
    $C = [c_1,\ldots,c_k]$ and $C^\prime = [c_1^\prime,\ldots,c_k^\prime]$,
    $C \Rightarrow^*_{\mathcal{S}^*} C^\prime$ iff for every $i \in \{ 1, \ldots, k \}$,
    $c_i \Rightarrow^*_{\mathcal{S}} c_i^\prime$.
\end{lemma}
\begin{proof}[Proof of \Cref{lem:reachComposite}]
We prove each implication separately.
\begin{itemize}
    \item For the "if" implication, we assume that $c_i \Rightarrow_{\mathcal{S}}^* c_i^\prime$
          for any $i \in \{ 1, \ldots, k \}$,
          and have to prove that $[c_1,\ldots,c_k] \Rightarrow_{\mathcal{S}^*}^* [c_1^\prime,\ldots,c_k^\prime]$.
          We will prove that for any $j \in \{ 1, \ldots, k \}$, we it holds that
          \begin{equation*}
           [c^\prime_1,\ldots,c^\prime_{j-1}, c_j, c_{j+1}, \ldots, c_k] \Rightarrow_{\mathcal{S}^*}^* [c^\prime_1,\ldots,c^\prime_{j-1}, c_j^\prime, c_{j+1}, \ldots, c_k]    \, ,
          \end{equation*}
          from which the goal follows by transitivity.
          Ok then, let $j \in \{ 1, \ldots, k \}$.
          By \Cref{lem:compositeStep}, it is enough to prove that $c_j \Rightarrow^*_{\mathcal{S}} c_j^\prime$.
          But that holds by the assumption.
    \item For the "only if" implication, we assume $C \Rightarrow^*_{\mathcal{S}^*} C^\prime$,
          $i \in \{ 1,\ldots,k \}$, and have to prove that $c_i \Rightarrow^*_{\mathcal{S}} c_i^\prime$.
          Let $C_1,\ldots,C_l \in \mathcal{T}^*_{\mathit{Cfg}^*}$ be a sequence witnessing $C \Rightarrow^*_{\mathcal{S}^*} C^\prime$;
          that is, we have $C = C_1$, $C_l = C^\prime$, and $C_j \Rightarrow_{\mathcal{S}^*} C_{j+1}$ for any $j \in \{ 1, \ldots, l-1 \}$.
          Let $i_1,\ldots,i_m \in \{ 1,\ldots,l-1 \}$ be a strictly increasing sequence of maximal length such that
          $C_{i_j}[i] \not = C_{i_j + 1}[i]$ for any $j \in \{ 1,\ldots,m \}$;
          that is, the sequence of positions in the witnessing sequence when the component $i$ changes.
          Then clearly, $C_1[i] = C_{i_1}[i]$ (otherwise we could create a longer sequence).
          Similarly, $C_l[i] = C_{i_m}[i]$.
          Now we claim that $C_{i_1}[i] \Rightarrow_{\mathcal{S}} \ldots \Rightarrow_{\mathcal{S}} C_{i_m}[i]$,
          from which the conclusion easily follows.
          We have to prove that for any $o \in \{ 1, \ldots, m \}$, it holds that
          $C_{i_o}[i] \Rightarrow_{\mathcal{S}} C_{i_{o+1}}[i]$.
          Let $d := i_{o+1} - i_{o}$; clearly, we have $d > 0$.
          By the definition of $d$, we have $C_{i_{o+1}}[i] = C_{i_{o} + d}[i]$.
          By definition of the sequence, in particular by maximality, we have $C_{i_{o} + d}[i] = C_{i_{o} + 1}[i]$
          (because there can be no change of the component $i$ between the change at the position $i_o$ and the change at the position $i_{o+1}$).
          Therefore, it is enough to show that
          $C_{i_o}[i] \Rightarrow_{\mathcal{S}} C_{i_{o}+1}[i]$.
          By \Cref{lem:compositeStep} (using also \Cref{lem:atMostOneComponentChanges} and \Cref{lem:transitionOnlyBetweenListsOfSameLength}), it is enough to show that
          $C_{i_{o}} \Rightarrow_{\mathcal{S}^*} C_{i_{o}+1}$, but that is trivial and we are done.
\end{itemize}
\end{proof}

\begin{lemma}[A semantic property of variable renaming]\label{lem:varrenamesem}
  For any two variables $X,Y$ of the same sort, and for any two $\mathcal{T}$-valuations $\rho_1, \rho_2$
  which agree on all variables other than $X,Y$, if $\rho_1(X) = \rho_2(Y)$, then for any matching logic formula
  $\varphi$,
  \begin{equation*}
      (\mathcal{T}, \gamma, \rho_1) \vDash \varphi \iff (\mathcal{T}, \gamma, \rho_2) \vDash \varphi[Y/X] \, .
  \end{equation*}
\end{lemma}
\begin{proof}
    Admitted.
\end{proof}



\begin{definition}\label{def:mkList}
We define $\mathit{mkList}$ by letting
\begin{itemize}
    \item $\mathit{mkList}(\phi) = \mathit{cfgitem}(\phi)$; and
    \item $\mathit{mkList}(\phi_1, \ldots, \phi_k) = \mathit{cfgconcat}(\mathit{cfgitem}(\phi), \mathit{mkList}(\phi_2, \ldots, \phi_k))$ whenever $k > 1$.
\end{itemize}
\end{definition}

\begin{definition}\label{def:flatten}
    We define a function $\mathit{flatten}$ from (potentially existentially-quantified) constrained list patterns
    to matching logic patterns over a star-extended signature by
    \begin{align*}
        & \mathit{flatten}(\exists \vec{X}.\, [\varphi_1, \ldots, \varphi_k) \land P] \equiv \\
        & \exists \vec{X}.\, \mathit{mkList}(Y_1, \ldots, Y_k)
        \land (\varphi_1^\square)[Y_1/\square] \land \ldots
        \land (\varphi_k^\square)[Y_k/\square] \land P \, ,
    \end{align*}
    where $Y_1,\ldots,Y_k,\square$ are fresh.
    Furthermore, we let
    \begin{equation*}
        \mathit{flatten}^\exists(\Psi,\Psi^\prime) \equiv \mathit{flatten}(\Psi) \Rightarrow^\exists \mathit{flatten}(\Psi^\prime) \, .
    \end{equation*}
\end{definition}


\begin{lemma}[On Flattening]\label{lem:flatten}
    For any matching logic $\Sigma$-model $\mathcal{T}$, any $C \in \mathcal{T}^*_{\mathit{Cfg}^*}$,
    and any $\mathcal{T}^*$-valuation $\rho$, we have
    %\begin{proofenv}
        \begin{equation*}
            (\mathcal{T}^*, C, \rho) \vDash \mathit{flatten}(\exists \vec{X}.\, [\varphi_1,\ldots,\varphi_k] \land P)
        \end{equation*}
    %\end{proofenv}
    if and only if
    %\begin{proofenv}
        there exist configurations $c_1, \ldots, c_k \in \mathcal{T}_{\mathit{Cfg}}$ such that
        $C = [c_1, \ldots, c_k]$ and there exists a $\mathcal{T}$-valuation $\rho_0$
        satisfying $\rho_0(v) = \rho(v)$ for any $v \in \mathit{Var} \setminus \vec{X}$
        such that for any $j \in \{ 1, \ldots, k \}$,
        $(\mathcal{T}, c_j, \rho_0) \vDash \varphi_j \land P$.
    %\end{proofenv}
\end{lemma}


\begin{proof}[Proof of \Cref{lem:flatten}]
    We have
    \begin{proofenv}
        \begin{equation*}
            (\mathcal{T}^*, C, \rho) \vDash \mathit{flatten}(\exists \vec{X}.\, (\varphi_1,\ldots,\varphi_k) \land P)
        \end{equation*}
    \end{proofenv}
    if and only if (by unfolding the definition of $\mathit{flatten}$ and \Cref{def:matchinglogic})
    \begin{proofenv}
        there exists a $\mathcal{T}^*$-valuation $\rho^\prime$ satisfying $\rho^\prime(v) = \rho(v)$
        for any $v \in \mathit{Var}^* \setminus \vec{X}$ such that
        $(\mathcal{T}^*, C, \rho^\prime) \vDash \mathit{mkList}(Y_1,\ldots,Y_k)$
        and for any $j \in \{ 1, \ldots, k \}$,
        $(\mathcal{T}^*, C, \rho^\prime) \vDash (\varphi_j^{\square})[Y_j/\square] \land P$ \, ,
    \end{proofenv}
    if and only if (by \Cref{lem:mkListSemantics})
    \begin{proofenv}
        there exists a $\mathcal{T}^*$-valuation $\rho^\prime$ satisfying $\rho^\prime(v) = \rho(v)$
        for any $v \in \mathit{Var}^* \setminus \vec{X}$ such that
        \begin{itemize}
            \item there exist configurations $c_1, \ldots, c_k \in \mathcal{T}_{\mathit{Cfg}}$ such that
                    $C = [c_1, \ldots, c_k]$
                    and for every $\mathcal{T}$-valuation $\rho^{\prime\prime}$
                    satisfying $\rho^{\prime\prime}(Y_j) = \rho^\prime(Y_j)$ for any $j \in \{ 1, \ldots, k \}$,
                    it holds that for any $j \in \{ 1, \ldots, k \}$,
                    $(\mathcal{T}, c_j, \rho^{\prime\prime}) \vDash Y_j$; and
            \item for any $j \in \{ 1, \ldots, k \}$,
                    $(\mathcal{T}^*, C, \rho^\prime) \vDash (\varphi_j^{\square})[Y_j/\square] \land P$ ,
        \end{itemize}
    \end{proofenv}
    if and only if (by \Cref{def:matchinglogic} and firstorder reasoning)
    \begin{proofenv}
        there exists a $\mathcal{T}^*$-valuation $\rho^\prime$ satisfying $\rho^\prime(v) = \rho(v)$
        for any $v \in \mathit{Var}^* \setminus \vec{X}$ such that
        \begin{itemize}
            \item there exist configurations $c_1, \ldots, c_k \in \mathcal{T}_{\mathit{Cfg}}$ such that
                    $C = [c_1, \ldots, c_k]$ and for any $j \in \{ 1, \ldots, k \}$,
                    $\rho^{\prime}(Y_j) = c_j$; and
            \item for any $j \in \{ 1, \ldots, k \}$,
                    $(\mathcal{T}^*, C, \rho^\prime) \vDash (\varphi_j^{\square})[Y_j/\square] \land P$ ,
        \end{itemize}
    \end{proofenv}
    if and only if (by firstorder reasoning, \Cref{def:matchinglogic}, and \Cref{lem:structurelessSemantics})
    \begin{proofenv}
        there exist configurations $c_1, \ldots, c_k \in \mathcal{T}_{\mathit{Cfg}}$ such that $C = [c_1, \ldots, c_k]$,
        and there exists a $\mathcal{T}^*$-valuation $\rho^\prime$ satisfying $\rho^\prime(v) = \rho(v)$
        for any $v \in \mathit{Var}^* \setminus \vec{X}$ such that for any $j \in \{ 1, \ldots, k \}$,
        \begin{itemize}
            \item $\rho^{\prime}(Y_j) = c_j$;
            \item $(\mathcal{T}^*, \rho^\prime) \vDash (\varphi_j^{\square})[Y_j/\square]$ ; and
            \item $(\mathcal{T}^*, \rho^\prime) \vDash P$,
        \end{itemize}
    \end{proofenv}
    if and only if (by \Cref{lem:starConservative} and firstorder reasoning)
    \begin{proofenv}
        there exist configurations $c_1, \ldots, c_k \in \mathcal{T}_{\mathit{Cfg}}$ such that $C = [c_1, \ldots, c_k]$,
        and there exists a $\mathcal{T}$-valuation $\rho^\prime$ satisfying $\rho^\prime(v) = \rho(v)$
        for any $v \in \mathit{Var} \setminus \vec{X}$ such that for any $j \in \{ 1, \ldots, k \}$,
        \begin{itemize}
            \item $\rho^{\prime}(Y_j) = c_j$;
            \item $(\mathcal{T}, \rho^\prime) \vDash (\varphi_j^{\square})[Y_j/\square]$ ; and
            \item $(\mathcal{T}, \rho^\prime) \vDash P$,
        \end{itemize}
    \end{proofenv}
    if and only if (by \Cref{lem:varrenamesem},
    since we have $\rho^\prime(Y_j) = c_j$
    on one side and $\rho_0^{c_j}(\square) = c_j$ on the other; for the implication from bottom to top,
    we also need the assumption that $Y_j$ was fresh and \Cref{lem:unusedVariables} -
    in order to choose the valuation $\rho^\prime$ satisfying $\rho^\prime(Y_j) = c_j$)
    \begin{proofenv}
        there exist configurations $c_1, \ldots, c_k \in \mathcal{T}_{\mathit{Cfg}}$ such that
        $C = [c_1, \ldots, c_k]$ and there exists a $\mathcal{T}$-valuation $\rho_0$
        satisfying $\rho_0(v) = \rho(v)$ for any $v \in \mathit{Var} \setminus \vec{X}$
        such that for any $j \in \{ 1, \ldots, k \}$,
        \begin{itemize}
            \item $(\mathcal{T}, \rho_0^{c_j}) \vDash \varphi_j^{\square}$ ; and
            \item $(\mathcal{T}, \rho_0) \vDash P$ ,
        \end{itemize}
    \end{proofenv}
    if and only if (by \Cref{lem:patternToFOLSemantics}, \Cref{def:matchinglogic}, and \Cref{lem:structurelessSemantics})
    \begin{proofenv}
        there exist configurations $c_1, \ldots, c_k \in \mathcal{T}_{\mathit{Cfg}}$ such that
        $C = [c_1, \ldots, c_k]$ and there exists a $\mathcal{T}$-valuation $\rho_0$
        satisfying $\rho_0(v) = \rho(v)$ for any $v \in \mathit{Var} \setminus \vec{X}$
        such that for any $j \in \{ 1, \ldots, k \}$,
        $(\mathcal{T}, c_j, \rho_0) \vDash \varphi_j \land P$ \, ,
    \end{proofenv}
    which is what we wanted to prove.
\end{proof}

\begin{proof}[Proof of \Cref{thm:correspondence}]
We prove each implication separately.
\begin{enumerate}
    \item For the left-to-right implication, we
    let $\mathcal{S} = (\mathcal{T}, S)$
    and $\Psi \equiv (\varphi_1,\ldots,\varphi_k) \land P$
    and $\Psi^\prime \equiv \exists \vec{Y}.\, (\varphi_1^\prime,\ldots,\varphi_k^\prime) \land P^\prime$,
    %where $\varphi_j \equiv \phi_j \land P_j$,
    %$\varphi^\prime_j \equiv \phi^\prime_j \land P^\prime_j$,
    and assume that
    \begin{proofenv}
        \begin{equation*}
            \mathcal{S} \vDash_\CRL \Psi \Rightarrow^{c\exists} \Psi^\prime \, ;
        \end{equation*}
    \end{proofenv}
    that is, (i)
    \begin{proofenv}
        for all configurations $c_1,\ldots,c_k \in \Tcfg$
        which terminate in $(\Tcfg, \Rightarrow_{\mathcal{S}})$
        and any $\mathcal{T}$-valuation $\rho_1$,
        whenever $(\mathcal{T}, c_1,\rho_1) \vDash \varphi_1 \land P$ and \ldots
        and $(\mathcal{T}, c_k,\rho_1) \vDash \varphi_k \land P$,
        then there exist configurations $c_1^\prime,\ldots,c_k^\prime \in \Tcfg$
        such that $c_1 \Rightarrow^{*}_{\mathcal{S}} c_1^\prime$
        and \ldots and $c_k \Rightarrow^{*}_{\mathcal{S}} c_k^\prime$,
        and there also exists an $\mathcal{T}$-valuation $\rho_2$
        satisfying $\rho_1(v) = \rho_2(v)$ for any $v \in \mathit{Var} \setminus \vec{Y}$,
        and
        $(\mathcal{T}, c_1^\prime,\rho_2) \vDash \varphi^\prime_1 \land P^\prime$ and \ldots
        and $(\mathcal{T}, c_k^\prime, \rho_2) \vDash \varphi^\prime_k \land P^\prime$.
    \end{proofenv}
    We have to prove that
    \begin{proofenv}
    for every $C \in \mathcal{T}^*_{\mathit{Cfg}^*}$
    such that $C$ terminates in $(\mathcal{T}^*_{\mathit{Cfg}^*}, \Rightarrow_{\mathcal{S}^*})$
    and for any valuation $\rho : \Var^* \to \mathcal{T}^*$
    such that $(\mathcal{T}^*, C, \rho) \vDash \mathit{flatten}(\Psi)$,
    there exists some $C^\prime \in \mathcal{T}^*_{\mathit{Cfg}^*}$
    such that
    $C \Rightarrow^{*}_{\mathcal{S}^*} C^\prime$
    and $(\mathcal{T}^*, C^\prime, \rho) \vDash \mathit{flatten}(\Psi^\prime)$.
    \end{proofenv}
    Let us then have some $C \in \mathcal{T}^*_{\mathit{Cfg}^*}$
    such that $C$ terminates in $(\mathcal{T}^*_{\mathit{Cfg}^*}, \Rightarrow_{\mathcal{S}^*})$,
    and a valuation $\rho : \Var^* \to \mathcal{T}^*$
    such that $(\mathcal{T}^*, C, \rho) \vDash \mathit{flatten}(\Psi)$.
    We have to prove that
    \begin{proofenv}
    there exists some $C^\prime \in \mathcal{T}^*_{\mathit{Cfg}^*}$
    such that
    $C \Rightarrow^{*}_{\mathcal{S}^*} C^\prime$
    and $(\mathcal{T}^*, C^\prime, \rho) \vDash \mathit{flatten}(\Psi^\prime)$.
    \end{proofenv}
    We will proceed in five steps:
    \begin{enumerate}
        \item \label{item:corr:step1} We prove that there exists $c_1,\ldots,c_k$ such that $C = [c_1,\ldots, c_k]$.
        \item \label{item:corr:step2} We prove that $c_1,\ldots,c_k$ are terminating.
        \item \label{item:corr:step3} We find appropriate valuation $\rho_1 : \mathit{Var} \to \mathcal{T}$
              and prove the premise of the assumption (i): that $(\mathcal{T}, c_1, \rho_1) \vDash \varphi_1 \land P$
        and \ldots and $(\mathcal{T}, c_k, \rho_1) \vDash \varphi_k \land P$.
        \item \label{item:corr:step4} From the assumption (i) we get the appropriate $c_1^\prime,\ldots,c_k^\prime$,
        as well as a valuation $\rho_2 : \mathit{Var} \to \mathcal{T}$ satisfying $\rho_1(v) = \rho_2(v)$ for any $v \in \mathit{Var} \setminus \vec{Y}$, and 
        \item \label{item:corr:step5} We let $C^\prime := [c_1^\prime,\ldots,c_k^\prime]$ and prove that it is reachable from $C$,
        as well as that it satisfies the flattened $\Psi^\prime$ in $\rho$.
    \end{enumerate}
    We have
    \begin{proofenv}
    $(\mathcal{T}^*, C, \rho) \vDash \mathit{flatten}(\Psi)$;
    \end{proofenv}
    that is,
    \begin{proofenv}
    $(\mathcal{T}^*, C, \rho) \vDash \mathit{flatten}((\varphi_1, \ldots, \varphi_k) \land P)$;
    \end{proofenv}
    which is by \Cref{lem:flatten} equivalent to (ii)
    \begin{proofenv}
        there exist configurations $c_1, \ldots, c_k \in \mathcal{T}_{\mathit{Cfg}}$ such that
        $C = [c_1, \ldots, c_k ]$, and
        there exists a $\mathcal{T}$-valuation $\rho_0$ satisfying $\rho_0(v) = \rho(v)$
        for any $v \in \mathit{Var}$,
        such that for any $j \in \{ 1, \ldots, k \}$, $(\mathcal{T} , c_j , \rho_0 ) \vDash  \varphi_j \land P$.
    \end{proofenv}
    Now, let $\rho_0$ be such valuation,
    and let $c_1,\ldots,c_k$ be such configurations. We have just proved Item~\ref{item:corr:step1}.
    To prove Item~\ref{item:corr:step2}, saying that the configurations $c_1, \ldots, c_k$ are terminating,
    we simply use \Cref{lem:terminationComposite}.
    To prove Item~\ref{item:corr:step3}, we let $\rho_1 := \rho_0$, and apply the assumption (ii).
    Now it follows that (iii)
    \begin{proofenv}
        there exist configurations $c_1^\prime,\ldots,c_k^\prime \in \Tcfg$
        such that $c_1 \Rightarrow^{*}_{\mathcal{S}} c_1^\prime$
        and \ldots and $c_k \Rightarrow^{*}_{\mathcal{S}} c_k^\prime$,
        and there also exists an $\mathcal{T}$-valuation $\rho_2$
        satisfying $\rho_1(v) = \rho_2(v)$ for any $v \in \mathit{Var} \setminus \vec{Y}$,
        and
        $(\mathcal{T}, c_1^\prime,\rho_2) \vDash \varphi^\prime_1 \land P^\prime$ and \ldots
        and $(\mathcal{T}, c_k^\prime, \rho_2) \vDash \varphi^\prime_k \land P^\prime$.
    \end{proofenv}
    Let us have such configurations $c_1^\prime,\ldots,c_k^\prime$ and valuation $\rho_2$.
    We choose $C^\prime := [ c_1^\prime,\ldots,c_k^\prime ]$,
    and it remains to be proven that
    \begin{proofenv}
        $C \Rightarrow^*_{\mathcal{S}^*} C^\prime$ and $(\mathcal{T}^*, C^\prime, \rho) \vDash \mathit{flatten}(\Psi^\prime)$ \,
    \end{proofenv}
    (where $\rho$ is the valuation that we started with).
    The part saying that $C \Rightarrow^*_{\mathcal{S}^*} C^\prime$ holds
    follows by \Cref{lem:reachComposite}. The other part can be changed using \Cref{lem:flatten} into
    \begin{proofenv}
        there exist configurations $c^\prime_1, \ldots, c^\prime_k \in \mathcal{T}_{\mathit{Cfg}}$ such that
        $C^\prime = [c^\prime_1, \ldots, c^\prime_k]$ and there exists a $\mathcal{T}$-valuation $\rho^\prime_0$
        satisfying $\rho^\prime_0(v) = \rho(v)$ for any $v \in \mathit{Var} \setminus \vec{Y}$
        such that for any $j \in \{ 1, \ldots, k \}$,
        $(\mathcal{T}, c^\prime_j, \rho^\prime_0) \vDash \varphi^\prime_j \land P$.
    \end{proofenv}
    Since we have constructed $C^\prime$ as a list of smaller configurations,
    it remains to be proven that
    \begin{proofenv}
        there exists a $\mathcal{T}$-valuation $\rho^\prime_0$
        satisfying $\rho^\prime_0(v) = \rho(v)$ for any $v \in \mathit{Var} \setminus \vec{Y}$
        such that for any $j \in \{ 1, \ldots, k \}$,
        $(\mathcal{T}, c^\prime_j, \rho^\prime_0) \vDash \varphi^\prime_j \land P$.
    \end{proofenv}
    Let us choose $\rho^\prime_0$ defined by
    $\rho_0^\prime(v) = \rho_2(v)$ for any $v \in \mathit{Var}$.
    We verify that $\rho_0^\prime(v) = \rho_2(v) = \rho_1(v) = \rho_0(v) = \rho(v)$
    for any $v \in \mathit{Var} \setminus \vec{Y}$,
    and the rest follows from the assumption (iii) by \Cref{lem:unusedVariables}.
    This concludes the proof of the first implication.
    \item For the opposite implication, 
    we again assume that $\Psi \equiv (\varphi_1,\ldots,\varphi_k) \land P$,
    $\Psi^\prime \equiv \exists \vec{Y}.\, (\varphi_1^\prime,\ldots,\varphi_k^\prime) \land P^\prime$,
    $\varphi_j = \phi_j \land P_j$ and $\varphi^\prime_j = \phi^\prime_j \land P^\prime_j$ for any $j \in \{ 1, \ldots, k \}$,
    and assume that
    \begin{proofenv}
        \begin{equation*}
            \mathcal{S}^* \vDash_\RL \mathit{flatten}(\Psi) \Rightarrow^\exists \mathit{flatten}(\Psi^\prime) \, ;
        \end{equation*}
    \end{proofenv}
    that is (i),
    \begin{proofenv}
        for every $C \in \mathcal{T}^*_{\mathit{Cfg}^*}$ such that $C$ terminates in
        $(\mathcal{T}_{\mathit{Cfg}^*}, \Rightarrow_{\mathcal{S}^*})$
        and for any valuation $\rho : \mathit{Var}^* \to \mathcal{T}^*$ such that
        $(\mathcal{T}^*, C, \rho) \vDash \mathit{flatten}(\Psi)$,
        there exists some $C^\prime \in \mathcal{T}^*_{\mathit{Cfg}^*}$ such that
        $C \Rightarrow_{\mathcal{S}^*}^* C^\prime$
        and $(\mathcal{T}^*, C^\prime, \rho) \vDash \mathit{flatten}(\Psi^\prime)$;
    \end{proofenv}
    we have to prove that
    \begin{proofenv}
        \begin{equation*}
            \mathcal{S} \vDash_\CRL \Psi \Rightarrow^{c\exists} \Psi^\prime \, ;
        \end{equation*}
    \end{proofenv}
    that is,
    \begin{proofenv}
        for all configurations $c_1,\ldots,c_k \in \Tcfg$
        which terminate in $(\Tcfg, \Rightarrow_{\mathcal{S}})$
        and any $\mathcal{T}$-valuation $\rho_1$,
        whenever $(\mathcal{T}, c_1,\rho_1) \vDash \varphi_1 \land P$ and \ldots
        and $(\mathcal{T}, c_k,\rho_1) \vDash \varphi_k \land P$,
        then there exist configurations $c_1^\prime,\ldots,c_k^\prime \in \Tcfg$
        such that $c_1 \Rightarrow^{*}_{\mathcal{S}} c_1^\prime$
        and \ldots and $c_k \Rightarrow^{*}_{\mathcal{S}} c_k^\prime$,
        and there also exists an $\mathcal{T}$-valuation $\rho_2$
        satisfying $\rho_1(v) = \rho_2(v)$ for any $v \in \mathit{Var} \setminus \vec{Y}$,
        and
        $(\mathcal{T}, c_1^\prime,\rho_2) \vDash \varphi^\prime_1 \land P^\prime$ and \ldots
        and $(\mathcal{T}, c_k^\prime, \rho_2) \vDash \varphi^\prime_k \land P^\prime$.
    \end{proofenv}
    Let us then have such terminating configurations $c_1,\ldots,c_k \in \mathcal{T}_{\mathit{Cfg}}$
    and such valuation $\rho_1 : \mathit{Var} \to \mathcal{T}$.
    We have to show that
    \begin{proofenv}
        there exist configurations $c_1^\prime,\ldots,c_k^\prime \in \Tcfg$
        such that $c_1 \Rightarrow^{*}_{\mathcal{S}} c_1^\prime$
        and \ldots and $c_k \Rightarrow^{*}_{\mathcal{S}} c_k^\prime$,
        and there also exists an $\mathcal{T}$-valuation $\rho_2$
        satisfying $\rho_1(v) = \rho_2(v)$ for any $v \in \mathit{Var} \setminus \vec{Y}$,
        and
        $(\mathcal{T}, c_1^\prime,\rho_2) \vDash \psi_1 \land P^\prime$ and \ldots
        and $(\mathcal{T}, c_k^\prime, \rho_2) \vDash \psi_k \land P^\prime$.
    \end{proofenv}
    We will proceed in the following steps.
    \begin{enumerate}
        \item We prove the premise of (i) for $C := [c_1,\ldots,c_k]$, that is:
        \begin{enumerate}
            \item $[c_1,\ldots,c_k]$ terminates in $(\mathcal{T}_{\mathit{Cfg}^*}, \Rightarrow_{\mathcal{S}^*})$; and
            \item $(\mathcal{T}^*, [c_1,\ldots,c_k], \rho) \vDash \mathit{flatten}(\Psi)$ for some constructed valuation $\rho$.
        \end{enumerate}
        \item We ``destruct'' the obtained $C^\prime$ into $[c^\prime_1,\ldots,c^\prime_k]$;
        \item We prove the desired properties of $c^\prime_j$ from the properties of $C^\prime$.
    \end{enumerate}
    First, $[c_1,\ldots,c_k]$ is terminating by \Cref{lem:terminationComposite}.
    Next, we have to show that
    \begin{proofenv}
        \begin{equation*}
            (\mathcal{T}^*, [c_1,\ldots,c_k], \rho) \vDash \mathit{flatten}((\varphi_1,\ldots,\varphi_k) \land P) \, ;
        \end{equation*}
    \end{proofenv}
    where $\rho(v) = \rho_1(v)$ for any $v \in \mathit{Var}$ (and $\rho(v)$ has arbitrary value for $v$ outside of $\mathit{Var}$).
    By \Cref{lem:flatten}, this is equivalent to showing that
    \begin{proofenv}
        there exist configurations $c_1, \ldots, c_k \in \mathcal{T}_{\mathit{Cfg}}$ such that
        $[c_1, \ldots, c_k] = [c_1, \ldots, c_k]$ and there exists a $\mathcal{T}$-valuation $\rho_0$
        satisfying $\rho_0(v) = \rho(v)$ for any $v \in \mathit{Var}$
        such that for any $j \in \{ 1, \ldots, k \}$, $(\mathcal{T}, c_j, \rho_0) \vDash \varphi_j \land P$.
    \end{proofenv}
    We choose $c_j := c_j$ and $\rho_0 := \rho_1$; it remains to be proven that
    \begin{proofenv}
        $(\mathcal{T}, c_j, \rho_1) \vDash \varphi_j \land P$.    
    \end{proofenv}
    which holds by assumption.
    Now we have obtained the following:
    \begin{proofenv}
        there exists some $C^\prime \in \mathcal{T}_{\mathit{Cfg}}^*$ such that
        $[c_1,\ldots,c_k] \Rightarrow_{\mathcal{S}^*}^* C^\prime$
        and $(\mathcal{T}^*, C^\prime, \rho) \vDash \mathit{flaten}(\Psi^\prime)$.
    \end{proofenv}
    Now, by \Cref{lem:transitionOnlyBetweenListsOfSameLength} (and using induction on the length of the
    sequence witnessing the reachability),
    we get some $c^\prime_1,\ldots,c^\prime_k \in \Tcfg$ such that
    \begin{proofenv}
        $[c_1,\ldots,c_k] \Rightarrow_{\mathcal{S}^*}^* [c^\prime_1,\ldots,c^\prime_k]$
        and $(\mathcal{T}^*, [c^\prime_1,\ldots,c^\prime_k], \rho) \vDash \mathit{flaten}(\Psi^\prime)$.
    \end{proofenv}
    Our goal is to prove that
    \begin{proofenv}
        there exist configurations $c_1^\prime,\ldots,c_k^\prime \in \Tcfg$
        such that $c_1 \Rightarrow^{*}_{\mathcal{S}} c_1^\prime$
        and \ldots and $c_k \Rightarrow^{*}_{\mathcal{S}} c_k^\prime$,
        and there also exists an $\mathcal{T}$-valuation $\rho_2$
        satisfying $\rho_1(v) = \rho_2(v)$ for any $v \in \mathit{Var} \setminus \vec{Y}$,
        and
        $(\mathcal{T}, c_1^\prime,\rho_2) \vDash \psi_1 \land P^\prime$ and \ldots
        and $(\mathcal{T}, c_k^\prime, \rho_2) \vDash \psi_k \land P^\prime$,
    \end{proofenv}
    so we choose $c^\prime_j := c^\prime_j$ and have to prove that
    \begin{proofenv}
        $c_1 \Rightarrow^{*}_{\mathcal{S}} c_1^\prime$
        and \ldots and $c_k \Rightarrow^{*}_{\mathcal{S}} c_k^\prime$,
        and there also exists an $\mathcal{T}$-valuation $\rho_2$
        satisfying $\rho_1(v) = \rho_2(v)$ for any $v \in \mathit{Var} \setminus \vec{Y}$,
        and
        $(\mathcal{T}, c_1^\prime,\rho_2) \vDash \psi_1 \land P^\prime$ and \ldots
        and $(\mathcal{T}, c_k^\prime, \rho_2) \vDash \psi_k \land P^\prime$.
    \end{proofenv}
    The first part follows from \Cref{lem:reachComposite};
    it remains to be proven that
    \begin{proofenv}
        there also exists an $\mathcal{T}$-valuation $\rho_2$
        satisfying $\rho_1(v) = \rho_2(v)$ for any $v \in \mathit{Var} \setminus \vec{Y}$,
        and
        $(\mathcal{T}, c_1^\prime,\rho_2) \vDash \psi_1 \land P^\prime$ and \ldots
        and $(\mathcal{T}, c_k^\prime, \rho_2) \vDash \psi_k \land P^\prime$.
    \end{proofenv}
    and we already have
    \begin{proofenv}
        $(\mathcal{T}^*, [c^\prime_1,\ldots,c^\prime_k], \rho) \vDash \mathit{flaten}(\Psi^\prime)$ \, ;
    \end{proofenv}
    that is, by \Cref{lem:flatten} we know that
    \begin{proofenv}
        there exists a $\mathcal{T}$-valuation $\rho_0$ satisfying $\rho_0(v) = \rho(v)$
        for any $v \in \mathit{Var} \setminus \vec{Y}$ such that for any $j \in \{ 1, \ldots, k \}$,
        $(\mathcal{T}, c^\prime_j, \rho_0) \vDash \varphi^\prime$.
    \end{proofenv}
    Let $\rho_0^\prime$ be such valuation.
    In the goal, we let $\rho_2(v) := \rho_0^\prime(v)$ for any $v \in \mathit{Var}$;
    we then note that $\rho_2(v) = \rho_0^\prime(v) =  \rho(v) =  \rho_1(v)$ for any $v \in \mathit{Var} \setminus \vec{Y}$ by definitions.
    The rest of the goal follows from the assumption by \Cref{lem:unusedVariables}.
    This concludes the proof.
\end{enumerate}
\end{proof}


\begin{figure}
    \centering
    \includegraphics[width=0.5\textwidth]{img/onepath-rl.png}
    \caption{One-path reachability-logic proof system.
    The use of $\Rightarrow^+$ in sequent means that it was derived without Reflexivity.
    TODO retypeset}
    \label{fig:RLproofsystem}
\end{figure}

\begin{lemma}[On equivalence and FOL translation]\label{lem:equivFOLtransl}
    Let $\varphi_1, \varphi_2$ be two matching logic formulas such that $\vDash \varphi_1 \leftrightarrow \varphi_2$.
    Then $\vDash (\varphi_1^\square)[X/\square] \leftrightarrow (\varphi_2^\square)[X/\square]$.
\end{lemma}
\begin{proof}
    Let $M$ be any matching logic model, $\gamma$ an element of $M$, and $\rho$ an $M$-valuation.
    We have to prove that $(M, \gamma, \rho) \vDash (\varphi_1^\square)[X/\square] \leftrightarrow (\varphi_2^\square)[X/\square]$,
    which is (by definition of the squaring function and substitution) equivalent to
    $(M, \gamma, \rho) \vDash ((\varphi_1 \leftrightarrow \varphi_2)^\square)[X/\square]$,
    which is
    (by \Cref{lem:varrenamesem}, because $\rho(X) = \rho[\square := \rho(X)](\square)$)
    equivalent to
    $(M, \gamma, \rho[\square := \rho(X)]) \vDash (\varphi_1 \leftrightarrow \varphi_2)^\square$,
    which is  (by \Cref{lem:patternToFOLSemantics}, because $\rho[\square := \rho(X)] = \rho^{\rho(X)}$) equivalent to
    $(M, \rho(X), \rho) \vDash \varphi_1 \leftrightarrow \varphi_2$,
    which holds by the assumption.
\end{proof}

\begin{proof}[Proof of \Cref{lem:CRLalmostSoundness}]
By induction on the structure of the CRL proof.
\begin{enumerate}
    \item If the proof ends with \emph{Reduce}, then we are done, since $\mathit{flatten}^\exists(\emptyset, \psi^\prime) = \emptyset$.
    
    \item If the proof ends with \emph{Reflexivity}, then we need to prove
    \begin{equation*}
        \mathcal{S}^* \cup \mathit{flatten}^\exists(E, \psi), \emptyset \vdash_\RL
          \mathit{flatten}^\exists(\psi, \psi) 
    \end{equation*}
    which we do by applying the Reflexivity proof rule.
    
    \item If the proof ends with \emph{Axiom}, then $\psi \in E$,
          and we have to prove that
          \begin{equation*}
            \mathcal{S}^* \cup \mathit{flatten}^\prime(E, \psi^\prime), \mathit{flatten}^\prime(C, \psi^\prime) \vdash_\RL
            \mathit{flatten}^\prime(\psi, \psi^\prime)               \, .
          \end{equation*}
          By applying the Axiom proof rule of RL, it is enough to show that
          \begin{equation*}
              \mathit{flatten}^\prime(\psi, \psi^\prime) \in \mathit{flatten^\prime}(E, \psi^\prime) \, ,
          \end{equation*}
          which follows from $\psi \in E$.
          
    \item If the proof ends with \emph{Case}, then we have
        \begin{equation*}
            \mathcal{S}^* \cup \Bar{E}, \Bar{C} \vdash_\RL
            \mathit{flatten}^\exists((\varphi_1, \ldots, \varphi_{i-1}, \varphi_i, \varphi_{i+1}, \ldots, \varphi_k) \land P^\prime, \Psi^\prime)
        \end{equation*}
        and
        \begin{equation*}
            \mathcal{S}^* \cup \Bar{E}, \Bar{C} \vdash_\RL
            \mathit{flatten}^\exists((\varphi_1, \ldots, \varphi_{i-1}, \psi_i, \varphi_{i+1}, \ldots, \varphi_k) \land P^\prime, \Psi^\prime) 
        \end{equation*}
        as hypotheses, and we have to prove
        \begin{equation*}
            \mathcal{S}^* \cup \Bar{E}, \Bar{C} \vdash_\RL
            \mathit{flatten}^\exists((\varphi_1, \ldots, \varphi_{i-1}, (\varphi_i \lor \psi_i), \varphi_{i+1}, \ldots, \varphi_k) \land P^\prime, \Psi^\prime)               \, .
        \end{equation*}
        (where $\Bar{E} = \mathit{flatten}^\exists(E, \psi^\prime)$
         and $\Bar{C} = \mathit{flatten}^\exists(C, \psi^\prime)$
        ).
        After simplifications, we get
        \begin{align*}
            \mathcal{S}^* \cup \Bar{E}, \Bar{C} \vdash_\RL
            &
            \mathit{mkList}(X_1, \ldots, X_k) \land \left( \bigwedge_{j=1}^{k} (\varphi_j^\square)[X_j/\square] \right) \land P^\prime
            \\ & \Rightarrow^\exists
            \mathit{flatten}(\Psi^\prime)
        \end{align*}
        and
        \begin{align*}
            \mathcal{S}^* \cup \Bar{E}, \Bar{C} \vdash_\RL
            &
            \mathit{mkList}(Y_1, \ldots, Y_k) \land \left(\bigwedge_{j=1, j \not = i}^{k} (\varphi_j^\square)[Y_j/\square] \right)
            \land (\psi_i^\square)[Y_i/\square] \land P^\prime
            \\ & \Rightarrow^\exists
            \mathit{flatten}(\Psi^\prime)
        \end{align*}
        as hypotheses,
        and have to prove
        \begin{align*}
            \mathcal{S}^* \cup \Bar{E}, \Bar{C} \vdash_\RL
            &
            \mathit{mkList}(Z_1, \ldots, Z_k) \land \left(\bigwedge_{j=1,j \not = i}^{k} (\varphi_j^\square)[Z_j/\square]\right)
            \land ((\varphi_i \lor \psi_i)^\square)[Z_i/\square]
            \\ & \Rightarrow^\exists
            \mathit{flatten}(\Psi^\prime)
        \end{align*}
        (where $X_1,\ldots,X_k,Y_1,\ldots,Y_k,Z_1,\ldots,Z_k$ are fresh variables).
        We first apply the Consequence RL rule to the goal to distribute the $\varphi_i \lor \psi_i$ disjunction
        to the top, changing the goal to
        \begin{align*}
            \mathcal{S}^* \cup \Bar{E}, \Bar{C} \vdash_\RL
            &
            \left( \mathit{mkList}(Z_1, \ldots, Z_k) \land \left(\bigwedge_{j=1,j \not = i}^{k} (\varphi_j^\square)[Z_j/\square]\right)
            \land (\varphi_i^\square)[Z_i/\square] \right)
            \\ \lor &
            \left(
            \mathit{mkList}(Z_1, \ldots, Z_k) \land \left(\bigwedge_{j=1,j \not = i}^{k} (\varphi_j^\square)[Z_j/\square]\right)
            \land (\psi_i^\square)[Z_i/\square]
            \right)
            \\ & \Rightarrow^\exists
            \mathit{flatten}(\Psi^\prime) \, .
        \end{align*}
        Now we apply the Case Analysis rule.
        Then we transform the hypotheses to the respective goals by existentially quantifying the $X_j$s (and $Y_j$s, respectively)
        in the hypotheses
        using the Abstraction RL rule, alpha-renaming (using the Consequence rule) the $X_j$s (and $Y_j$s, respectively)
        into $Z_j$s, and stripping the existential quantifiers (using the Consequence rule, again), and we are done.
        
    \item If the proof ends with \emph{Step},
      we can assume a structureless FOL formula $P^\prime$, a rule $\varphi \Rightarrow^\exists \varphi^\prime \in S$ such that
      $\mathcal{T} \vDash_\ML \varphi_i \leftrightarrow \varphi \land P^\prime$,
      and an induction hypothesis
      \begin{align*}
        (&\mathcal{T}^*, S^* \cup \mathit{flatten}^\exists(C \cup E, \Psi^\prime)), \emptyset \vdash_\RL
          \\ &
          \mathit{flatten}([\varphi_1, \ldots, \varphi_{i-1}, \varphi^\prime \land P^\prime, \varphi_{i+1}, \ldots, \varphi_k] \land P) \Rightarrow^\exists \mathit{flatten}(\Psi^\prime)     
      \end{align*}
      and have to construct
      \begin{align*}
      & (\mathcal{T}^*, S^* \cup \mathit{flatten}^\exists(E, \Psi^\prime)), \mathit{flatten}^\exists(C, \Psi^\prime) \vdash_\RL \\
          & \mathit{flatten}([\varphi_1, \ldots, \varphi_{i-1}, \varphi_i, \varphi_{i+1}, \ldots, \varphi_k] \land P) \Rightarrow^\exists \mathit{flatten}(\Psi^\prime)    \, .
      \end{align*}
        By definition of $S^*$, we also have
        \begin{align*}
            (\mathit{heat}(L, \varphi, R) \Rightarrow^\exists \mathit{heat}(L, \varphi^\prime, R)) \in S^* \, .
        \end{align*}
    We apply the Transitivity rule with the second premise being our first inductive hypothesis, and it remains to prove the second premise, which is
    \begin{align*}
        & (\mathcal{T}, S)^*, \mathit{flatten}^\exists(E, \psi^\prime), \mathit{flatten}^\exists(C, \psi^\prime)
        \\& \vdash_\RL
        \mathit{flatten}([\varphi_1, \ldots, \varphi_{i-1}, \varphi_i, \varphi_{i+1}, \ldots, \varphi_k] \land P)
        \\&\quad \Rightarrow^\exists
        \mathit{flatten}([\varphi_1, \ldots, \varphi_{i-1}, \varphi^\prime \land P^\prime, \varphi_{i+1}, \ldots, \varphi_k] \land P) \, .
    \end{align*}
    that is (after simplification, assuming a reasonable choice of fresh variables)
    \begin{align*}
        & (\mathcal{T}, S)^*, \mathit{flatten}^\exists(E, \psi^\prime), \mathit{flatten}^\exists(C, \psi^\prime)
        \\& \vdash_\RL
        \mathit{mkList}(X_1, \ldots, X_k) \land \left( \bigwedge_{j=1,j\not = i}^{k} (\varphi_j^\square)[X_j/\square] \right)
        \land (\varphi_i^\square)[X_i/\square] \land P
        \\&\quad \Rightarrow^\exists
        \mathit{mkList}(X_1, \ldots, X_k) \land \left( \bigwedge_{j=1, j \not = i}^{k} (\varphi_j^\square)[X_j/\square] \right) \land ((\varphi^\prime \land P^\prime)^\square)[X_i/\square] \land P
        \, .
    \end{align*}
    By \Cref{lem:equivFOLtransl}, our assumption that $\mathcal{T} \vDash_\ML \varphi_i \leftrightarrow (\varphi \land P^\prime)$,
    and conservativeness,
    we can apply the Consequence rule, and the goal changes to
    \begin{align*}
        & (\mathcal{T}, S)^*, \mathit{flatten}^\exists(E, \psi^\prime), \mathit{flatten}^\exists(C, \psi^\prime)
        \\& \vdash_\RL
        \mathit{mkList}(X_1, \ldots, X_k) \land \left( \bigwedge_{j=1,j\not = i}^{k} (\varphi_j^\square)[X_j/\square] \right)
        \land ((\varphi \land P^\prime)^\square)[X_i/\square] \land P
        \\&\quad \Rightarrow^\exists
        \mathit{mkList}(X_1, \ldots, X_k) \land \left( \bigwedge_{j=1, j \not = i}^{k} (\varphi_j^\square)[X_j/\square] \right) \land ((\varphi^\prime \land P^\prime)^\square)[X_i/\square] \land P
        \, .
    \end{align*}
    We apply the Consequence rule again, changing the goal to
    \begin{align*}
        & (\mathcal{T}, S)^*, \mathit{flatten}^\exists(E, \psi^\prime), \mathit{flatten}^\exists(C, \psi^\prime)
        \\& \vdash_\RL
        (\varphi^\square)[X_i/\square] \land (
        \mathit{mkList}(X_1, \ldots, X_k) \land \left( \bigwedge_{j=1,j\not = i}^{k} (\varphi_j^\square)[X_j/\square] \right)
        \land ((P^\prime)^\square)[X_i/\square] \land P)
        \\&\quad \Rightarrow^\exists
        ((\varphi^\prime)^\square)[X_i/\square] \land (
        \mathit{mkList}(X_1, \ldots, X_k) \land \left( \bigwedge_{j=1, j \not = i}^{k} (\varphi_j^\square)[X_j/\square] \right) \land ((P^\prime)^\square)[X_i/\square] \land P)
        \, .
    \end{align*}
    Now we apply Logic Framing to remove the structureless parts that are the same in both the left and right sides,
    resulting in the goal
    \begin{align*}
        & (\mathcal{T}, S)^*, \mathit{flatten}^\exists(E, \psi^\prime), \mathit{flatten}^\exists(C, \psi^\prime)
        \\& \vdash_\RL
        (\varphi^\square)[X_i/\square] \land
        \mathit{mkList}(X_1, \ldots, X_k)
        \\&\quad \Rightarrow^\exists
        ((\varphi^\prime)^\square)[X_i/\square] \land
        \mathit{mkList}(X_1, \ldots, X_k)
        \, .
    \end{align*}
    Now, from the assumption that $\varphi \Rightarrow^\exists \varphi^\prime \in S$ and the construction of $S^*$
    it follows that
    $\mathit{heat}(L, \varphi, R) \Rightarrow^\exists \mathit{heat}(L, \varphi^\prime, R) \in S^*$;
    and therefore
    $\mathit{cfgheat}(L, \phi, R) \land Q \Rightarrow^\exists \mathit{cfgheat}(L, \phi^\prime, R) \land Q^\prime \in S^*$
    where $\varphi \equiv \phi \land Q$ and $\varphi^\prime \equiv \phi^\prime \land Q^\prime$.
    By semantic reasoning we can prove that
    \begin{align*}
        \mathcal{T}^* \vDash & ((\varphi^\square)[X_i/\square] \land \mathit{mkList}(X_1, \ldots, X_k))
        \\ \leftrightarrow  &
        \mathit{cfgheat}(L, X_i, R)
        \\ & \land L = \mathit{mkList}(X_1, \ldots, X_{i-1})
        \\ & \land R = \mathit{mkList}(X_{i+1}, \ldots, X_k)
        \\ & \land (\phi^\square)[X_i/\square] \land Q
    \end{align*}
    and that
    \begin{align*}
        \mathcal{T}^* \vDash & (((\varphi^\prime)^\square)[X_i/\square] \land \mathit{mkList}(X_1, \ldots, X_k))
        \\ \leftrightarrow  &
        \mathit{cfgheat}(L, X_i, R)
        \\ & \land L = \mathit{mkList}(X_1, \ldots, X_{i-1})
        \\ & \land R = \mathit{mkList}(X_{i+1}, \ldots, X_k)
        \\ & \land ((\phi^\prime)^\square)[X_i/\square] \land Q^\prime \, .
    \end{align*}
    Now we apply Consequence and subsequently strip the $L$,$R$ equalities using Logic Framing, thus getting
    \begin{align*}
        & (\mathcal{T}, S)^*, \mathit{flatten}^\exists(E, \psi^\prime), \mathit{flatten}^\exists(C, \psi^\prime)
        \\& \vdash_\RL
        \mathit{cfgheat}(L, X_i, R) \land (\phi^\square)[X_i/\square] \land Q
        \\&\quad \Rightarrow^\exists
        \mathit{cfgheat}(L, X_i, R) \land ((\phi^\prime)^\square)[X_i/\square] \land Q^\prime
        \, .
    \end{align*}
    Now we use Consequence to expand $\phi^\square$ and $(\phi^\prime)^\square$ into equalities,
    perform the substitution, and use the equalities to replace the $X_i$ subterm of $\mathit{cfgheat}$ with
    $\phi$ and $\phi^\prime$, respectively; this way the goal becomes
    \begin{align*}
        & (\mathcal{T}, S)^*, \mathit{flatten}^\exists(E, \psi^\prime), \mathit{flatten}^\exists(C, \psi^\prime)
        \\& \vdash_\RL
        \mathit{cfgheat}(L, \phi, R) \land Q
        \\&\quad \Rightarrow^\exists
        \mathit{cfgheat}(L, \phi^\prime, R) \land Q^\prime
        \, .
    \end{align*}
    We finish the proof of this case using the Axiom rule.
    
    \item If the proof ends with \emph{Circularity}, we can assume
        \begin{align*}
            (\mathcal{T}^*, S^* \cup \mathit{flatten}^\exists(E, \Psi^\prime)),
            \mathit{flatten}^\exists(C \cup \{ \Psi \}, \Psi^\prime) \vdash_\RL
            \mathit{flatten}^\exists(\Psi, \Psi^\prime)
        \end{align*}
        which simplifies to
        \begin{align*}
            (\mathcal{T}^*, S^* \cup \mathit{flatten}^\exists(E, \Psi^\prime)),
            \mathit{flatten}^\exists(C, \Psi^\prime) \cup \mathit{flatten}^\exists(\{ \Psi \}, \Psi^\prime) \vdash_\RL
            \mathit{flatten}^\exists(\Psi, \Psi^\prime)
        \end{align*}
        and have to prove
        \begin{align*}
            (\mathcal{T}^*, S^* \cup \mathit{flatten}^\exists(E, \Psi^\prime)),
            \mathit{flatten}^\exists(C, \Psi^\prime) \vdash_\RL
            \mathit{flatten}^\exists(\Psi, \Psi^\prime)
        \end{align*}
        which follows from the assumption by Circularity.
        
    \item If the proof ends with \emph{Conseq}, we can assume
    \begin{align*}
        \mathcal{T}^* \vDash_\ML \mathit{flatten}(\Phi) \rightarrow \mathit{flatten}(\Phi^\prime)
    \end{align*}
    and
    \begin{align*}
        (\mathcal{T}^*, S^* \cup \mathit{flatten}^\exists(E, \Psi^\prime)), \mathit{flatten}^\exists(C, \Psi^\prime) \vdash_\RL
          \mathit{flatten}^\exists(\Phi^\prime, \Psi^\prime) \, ,
    \end{align*}
    and have to prove
    \begin{align*}
        (\mathcal{T}^*, S^* \cup \mathit{flatten}^\exists(E, \Psi^\prime)), \mathit{flatten}^\exists(C, \Psi^\prime) \vdash_\RL
          \mathit{flatten}^\exists(\Phi, \Psi^\prime)  \, .
    \end{align*}
    The second assumption simplifies to
    \begin{align*}
        (\mathcal{T}^*, S^* \cup \mathit{flatten}^\exists(E, \Psi^\prime)), \mathit{flatten}^\exists(C, \Psi^\prime) \vdash_\RL
          \mathit{flatten}(\Phi^\prime) \Rightarrow^\exists \mathit{flatten}(\Psi^\prime) \, ,
    \end{align*}
    while the goal to
    \begin{align*}
        (\mathcal{T}^*, S^* \cup \mathit{flatten}^\exists(E, \Psi^\prime)), \mathit{flatten}^\exists(C, \Psi^\prime) \vdash_\RL
          \mathit{flatten}(\Phi) \Rightarrow^\exists \mathit{flatten}(\Psi^\prime) \, ;
    \end{align*}
    therefore, we can apply the \emph{Consequence} rule.        
        
        
    \item If the proof ends with \emph{Abstract},
    we assume
    \begin{align*}
        X \not\in \mathit{FV}(\Psi^\prime)
    \end{align*}
    and
    \begin{align*}
                (\mathcal{T}^*, S^* \cup \mathit{flatten}^\exists(E, \Psi^\prime)), \mathit{flatten}^\exists(C, \Psi^\prime) \vdash_\RL
          \mathit{flatten}^\exists(\exists \vec{Y}.\, (\varphi_1, \ldots, \varphi_k) \land P, \Psi^\prime)
    \end{align*}
    and have to prove that
    \begin{align*}
                (\mathcal{T}^*, S^* \cup \mathit{flatten}^\exists(E, \Psi^\prime)), \mathit{flatten}^\exists(C, \Psi^\prime) \vdash_\RL
          \mathit{flatten}^\exists(\exists X,\vec{Y}.\, (\varphi_1, \ldots, \varphi_k) \land P, \Psi^\prime) \, .
    \end{align*}
    After simplifications, the second premise becomes
    \begin{align*}
            &(\mathcal{T}^*, S^* \cup \mathit{flatten}^\exists(E, \Psi^\prime)), \mathit{flatten}^\exists(C, \Psi^\prime) \vdash_\RL
          \\& \exists \vec{Y}.\, (\mathit{mkList}(Z_1, \ldots, Z_k) \land (\varphi_1^\square)[Z_1/\square] \land \ldots (\varphi_1^\square)[Z_k/\square]) \land P)
          \\&
          \Rightarrow^\exists \mathit{flatten}(\Psi^\prime) \, ,
    \end{align*}
    while the goal becomes
    \begin{align*}
          &(\mathcal{T}^*, S^* \cup \mathit{flatten}^\exists(E, \Psi^\prime)), \mathit{flatten}^\exists(C, \Psi^\prime) \vdash_\RL
          \\&\exists X. \exists \vec{Y}.\, (\mathit{mkList}(Z_1, \ldots, Z_k) \land (\varphi_1^\square)[Z_1/\square] \land \ldots (\varphi_1^\square)[Z_k/\square]) \land P) \, .
    \end{align*}
    We prove the goal using the Abstraction rule
    (note that $X \not\in \mathit{FV}(\Psi^\prime)$ implies $X \not\in \mathit{FV}(\mathit{flatten}(\Psi^\prime))$
    because we are free to choose the fresh variables inside the $\mathit{flatten}$ such).
    \end{enumerate}
    This concludes the proof.
\end{proof}


\end{document}
